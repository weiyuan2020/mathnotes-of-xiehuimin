\documentclass[10pt,a4paper]{book}
\usepackage[utf8]{inputenc}
\usepackage[T1]{fontenc}
\usepackage{amsmath}
\usepackage{amsthm}
\usepackage{ctex}
\usepackage{amsfonts}
\usepackage{amssymb}
\usepackage{graphicx}

\usepackage{listings}   % include the package before using it

\newtheorem{theorem}{Theorem}[section]
\newtheorem{lemma}{Lemma}[section]
\newtheorem{corollary}{Corollary}[section]

%//LaTeX 头部添加
%\newtheorem{theorem}{Theorem}[section]
%
%\begin{theorem}
%	***//定理内容
%	\label{thm-1}
%\end{theorem}
%
%\begin{proof}
%	***//证明过程
%\end{proof}
%
%//LaTeX 头部添加
%\newtheorem{lemma}{Lemma}[section]
%
%\begin{lemma} 
%	***//引理内容
%	\label{lem-1}
%\end{lemma}
%
%//LaTeX 头部添加
%\newtheorem{corollary}{Corollary}[section]
%
%\begin{corollary} 
%	***//推论内容
%	\label{cor-1}
%\end{corollary}
\newtheorem{myDef}{Definition}
%\begin{myDef}
%	\label{label}
%	...
%\end{myDef}
%


\usepackage{geometry}
%\geometry{right=2.0cm,left=2.0cm}% 
\geometry{right=2.0cm,left=2.0cm,top = 2.0cm, bottom = 2.0cm}

\title{one numbers and games reading notes}
\author{weiyuan}
\date{2021.11.14}
\begin{document}
	\maketitle
\chapter{All numbers gerat and small}
\section{Construction}

Two sets of numbers $  L,R  $\\
no member of $ L $ is $ \ge  $ any members of $ \mathbb{R} $\\
then there is a number$ \{L|R\} $\\
All numbers are constructed in this way.\\

\section{Convension}

If $ x\in\{L|R\} $ we write $ x^L $ for the typical member of L, and $ x^R $ for the typical member of $ \mathbb{R} $. For $ x $ itself we then write $ \{x^L|x^R\} $ 
\begin{gather}
	x = \{ a,b,c,\dots | d,e,f,\dots \}
\end{gather}
means that $ x\in \{L|R\} $, where $ a,b,c,\dots $ are the typical members of $ L $, and $ d,e,f,\dots $ the typical members of $ R $\\

\section{Definitions}
\begin{myDef}
	$ x \ge y, x \le y $\\
	\begin{gather*}
		x \ge y \quad \iff \text{no } x^R \le y \text{ and } x \le \text{ no } y^L\\
	\end{gather*}
	and $ x\le y \iff y\ge x $
	We write $ x\ngeq y $ to mean that $ x \le y $ does not hold
	
\end{myDef}

\begin{myDef}
	$ x=y, x>y, x<y $\\
	\begin{gather*}
		x = y \iff x\ge y \quad\text{and}\quad y \ge x\\
		x > y \iff x\ge y \quad\text{and}\quad y \ngeq x\\
		x < y \iff y>x\\		
	\end{gather*}
	
\end{myDef}

\begin{myDef}
	$ x+y $\\
	\begin{gather*}
		x+y = \{x^L+y , x+y^L | x^R+y, x+y^R\}
	\end{gather*}
\end{myDef}
	
	\begin{myDef}
		$ -x $\\
		\begin{gather*}
			-x = \{ -x^R | -x^L \}
		\end{gather*}
	\end{myDef}
	
	\begin{myDef}
		$x\cdot y$\\
		\begin{gather*}
			x\cdot y = \{x^L+y+xy^L-x^Ly^L, x^Ry+xy^R-x^Ry^R|\\
			\qquad\quad x^Ly+xy^R-x^Ly^R, x^Ry+xy^L-x^Ry^L\}
		\end{gather*}
	\end{myDef}
	
	\begin{proof}
		\begin{gather}
			xy>x^Ly+xy^L-x^Ly^L\\
			\because	(x-x^L)(y-y^L)>0
		\end{gather}
	\end{proof}

We now comment on definitions. A most importatn comment whose logical effects will be discussed later is that \textit{the notion if equality is a defined relation}. Thus apparently different definitions will produce the same number, and we must distinguish between the form $ \{L|R\} $ of a number and the number itself.

All the difinitions are inductive. Whethe $  x\ge y $ we must consider a number of similar questions about the pairs $ x^R,y $ and $ x,y^L $. But these problems are all simpler than the given one. It is perhaps not quite so obvious that the inductions require no basis, since ultimately we are reduced to problems about members of the empty set.


In general when we wish to establish a proposition $ P(x) $ for all numbers $ x $, we will prove it inductively by deducing $ P(x) $ from the truth of all the propositions $ P(x^L) $ and $ P(x^R) $. We regard the phrase "all numbers are constructed in this way" as justifying the legitimacy of this procedure. When proving propositions $ P(x,y) $ involving two variables we may use \textit{double induction}, deducing $ P(x,y) $ from the truth of all propositions of the form $ P(x^L,y), P(x^R,y), P(x,y^L), P(x,y^R)$ (and, if necessary, $ P(x^L,y^L). P(x^L,y^R), P(x^R,y^L), P(x^R,y^R) $). Such multiple inductions can be justified in the usual way in terms of repeated single inductions.

We shall allow ourselves to use certain expression $ \{L|R\} $ that are not numbers, since they do not satisfy the condition that no member of $ L $ shall be $ \ge $ any member of $ R $. In general we may write down any expression $ \{L|R\} $ and even discuss inequalities between such expressions before establishing that they are numbers, but if we wish such an expression to represent a number we must establish the condition on $ L \text{ and } R $. In the more general theory developed in the next part of the book, we show that when the condition on $ L \text{ and } R $
 is omitted we obtain the most general notion of a \textit{game}.

Our next comments concern the motives for these particular definitions. Now it is our intention that each number $ x $ shall lie between the numbers $ x^L $ (to the left) and $ x^R $ (to the right), and that$ \ge, +, -,., \text{etc} $, shall have their usual properties. So that if(say) $ y\ge\text{some } x^R $ we would not have $ x\ge y $, for then $ x\ge x^R $. Similarly, we could not allow $ x\ge y $ if $ x \le \text{some }y^L $. So we define $ x\ge y $ in all other cases. (This conforms with ourmotto, and helps to ensure that numbers are totally ordered.)

The spirit of the definitions is to ask what we know already(i.e. by the answers to simpler questions) about the object being defined, and to make and if $ x $ is between $ x^L $ and $ x^R $, and $ y $ between $ y^L $ and $ y^R $, the we know "already" that $ x+y $ must lie between both $ x^L+y $ and $ x+y^L $ (on the left) and $ x^R+y $ and $ x+y^R $(on the right), which yields the difinition of $ x+y $. Similarly$ -x $ will lie between $ -x^R $(on the left) and $ -x^L $(on the right), which suffice to define $ -x $.

It is not nearly so easy to find exactly what we "already" know about $ xy $. It might seem, for instance, that we know that $ xy $ lies between $ x^Ly $ and $ xy^L $(on the left) and $ x^Ry $ and $ xy^R $ (on the right), which would yeild the definition
\[ xy=\{x^L,y, xy^L| x^Ry,xy^R\}. \]

But this fails in two ways. Firstly what we "knew" here is sometimes false(consider negative numbers), and secondly, even when it is true it need not be the strongest information we "already" know. In fact, of course, this defines the same function as $ x+y $.

It takes a great deal of thought to find the correct definitions, which comes from the observation that (for instance) from $ x-x^L > 0 $ and $ y - y^L >0 $ we can deduce $ (x-x^L)(y-y^L)>0 $, so that we must have $ xy>x^Ly+xy^L-x^Ly^L $. Since all the products here are simpler ones, and since we regard addition and subtraction as already defined, we can regard this inequality as already known when we come to define $ xy $, and the other inequalities in the definition are similar. [Note that for positive numbers $ x $ and $ y $ the inequaltiy $ xy>x^Ly+xy^L-x^Ly^L $is stronger than both inequalities $ xy>x^Ly, xy>xy^L $. ]

We can summarise our comments by saying that the definitions of the various operations and realtions are just the simplest possible definitions that are consistent with their intended properties. In the next chapter, we shall verity that these intended properties really hold of all numbers, but in the rest of this chapter we shall simplu explore the system in a more informal way. To simplify the notation, when $ L $ is the set $ \{a, b, c, \dots \} $ and $ R $ the set $ \{ \dots, x,y,z \} $, we shall simply write $ \{ a, b, c, \dots | \dots, x, y, z \} $ for $ \{ L|R \} $.

\section{Examples of numbers, and some of their properties}
\subsection{The number 0}
According to the construction, every number has the form $ \{ L|R \} $, where $ L $ and $ R $ are two sets of earlier constructed numbers. So how can the system possibly get "off the ground", since initially there will be no earlier constructed numbers?

The answer, of course, is that even before we have any numbers, we have a certain \textit{set} of numbers, namely \textit{the empty set} $ \emptyset $! So the earliest constructed number can only be $ \{L|R\} $ with both $ L=R=\emptyset $, or in the simplified notation, the number $ \{|\} $. This number we call 0.

Is 0 a number? Yes, since we cannot have inquality of the form $ 0^L\ge 0^R $, for these is neither a $ 0^L $ nor a $ 0^R $!

Is $ 0\ge 0 $? Yes, for we can have no inequality of the form $ 0^R \le 0 $ or $ 0\le 0^L $. So by the definition, and happily, we have $ 0=0 $. We also see from the definitions that $ -0=0+0=0 $, since there is no number of any of the forms $ -0^R, -0^L, 0^L+0, 0+0^L, 0^R+0, 0+0^R $. A slightly more complicated observation of the same type is that $ x0=0 $, since in every one of the terms defining $ xy $ there is a mention of $ y^L $ or $ y^R $, so that when $ y=0 $ no term is needed and the expression for the expression of $ xy $ reduces to $ \{|\} =0 $. So the number 0 has at least some of the properties we know and love. 



\end{document}