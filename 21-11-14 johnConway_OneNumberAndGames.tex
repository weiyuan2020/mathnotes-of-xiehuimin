\documentclass[10pt,a4paper]{book}
\usepackage[utf8]{inputenc}
\usepackage[T1]{fontenc}
\usepackage{amsmath}
\usepackage{amsthm}
\usepackage{ctex}
\usepackage{amsfonts}
\usepackage{amssymb}
\usepackage{graphicx}

\usepackage{listings}   % include the package before using it

\newtheorem{theorem}{Theorem}[section]
\newtheorem{lemma}{Lemma}[section]
\newtheorem{corollary}{Corollary}[section]

%//LaTeX 头部添加
%\newtheorem{theorem}{Theorem}[section]
%
%\begin{theorem}
%	***//定理内容
%	\label{thm-1}
%\end{theorem}
%
%\begin{proof}
%	***//证明过程
%\end{proof}
%
%//LaTeX 头部添加
%\newtheorem{lemma}{Lemma}[section]
%
%\begin{lemma} 
%	***//引理内容
%	\label{lem-1}
%\end{lemma}
%
%//LaTeX 头部添加
%\newtheorem{corollary}{Corollary}[section]
%
%\begin{corollary} 
%	***//推论内容
%	\label{cor-1}
%\end{corollary}
\newtheorem{myDef}{Definition}
%\begin{myDef}
%	\label{label}
%	...
%\end{myDef}
%


\usepackage{geometry}
%\geometry{right=2.0cm,left=2.0cm}% 
\geometry{right=2.0cm,left=2.0cm,top = 2.0cm, bottom = 2.0cm}

\title{one numbers and games reading notes}
\author{weiyuan}
\date{2021.11.14}
\begin{document}
	\maketitle
\chapter{All numbers gerat and small}
Construction

Two sets of numbers $  L,R  $\\
no member of $ L $ is $ \ge  $ any members of $ \mathbb{R} $\\
then there is a number$ \{L|R\} $\\
All numbers are constructed in this way.\\

Convension

If $ x\in\{L|R\} $ we write $ x^L $ for the typical member of L, and $ x^R $ for the typical member of $ \mathbb{R} $. For $ x $ itself we then write $ \{x^L|x^R\} $ 
\begin{gather}
	x = \{ a,b,c,\dots | d,e,f,\dots \}
\end{gather}
means that $ x\in \{L|R\} $, where $ a,b,c,\dots $ are the typical members of $ L $, and $ d,e,f,\dots $ the typical members of $ R $\\


\begin{myDef}
	$ x \ge y, x \le y $\\
	\begin{gather*}
		x \ge y \quad \iff \text{no} x^R \le y \text{and} x \le \text{no} y^L\\
	\end{gather*}
	and $ x\le y \iff y\ge x $
	We write $ x\ngeq y $ to mean that $ x \le y $ does not hold
	
\end{myDef}

\begin{myDef}
	$ x=y, x>y, x<y $\\
	\begin{gather*}
		x = y \iff x\ge y \text{and} y \ge x\\
		x > y \iff x\ge y \text{and} y \ngeq x\\
		x < y \iff y>x\\		
	\end{gather*}
	
\end{myDef}

\begin{myDef}
	$ x+y $\\
	\begin{gather*}
		x+y = \{x^L+y , x+y^L | x^R+y, x+y^R\}
	\end{gather*}
\end{myDef}
	
	\begin{myDef}
		$ -x $\\
		\begin{gather*}
			-x = \{ -x^R | -x^L \}
		\end{gather*}
	\end{myDef}
	
	\begin{myDef}
		$x\cdot y$\\
		\begin{gather*}
			x\cdot y = \{x^L+y+xy^L-x^Ly^L, x^Ry+xy^R-x^Ry^R|\\
			\qquad\quad x^Ly+xy^R-x^Ly^R, x^Ry+xy^L-x^Ry^L\}
		\end{gather*}
	\end{myDef}
	
	\begin{proof}
		\begin{gather}
			xy>x^Ly+xy^L-x^Ly^L\\
			\because	(x-x^L)(y-y^L)>0
		\end{gather}
	\end{proof}
\end{document}