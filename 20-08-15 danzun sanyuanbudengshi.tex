\documentclass[10pt,a4paper]{book}
\usepackage[utf8]{inputenc}
\usepackage[T1]{fontenc}
\usepackage{amsmath}
\usepackage{amsthm}
\usepackage{ctex}
\usepackage{amsfonts}
\usepackage{amssymb}
\usepackage{graphicx}

\usepackage{listings}   % include the package before using it

\newtheorem{theorem}{Theorem}[section]
\newtheorem{lemma}{Lemma}[section]
\newtheorem{corollary}{Corollary}[section]

%//LaTeX 头部添加
%\newtheorem{theorem}{Theorem}[section]
%
%\begin{theorem}
%	***//定理内容
%	\label{thm-1}
%\end{theorem}
%
%\begin{proof}
%	***//证明过程
%\end{proof}
%
%//LaTeX 头部添加
%\newtheorem{lemma}{Lemma}[section]
%
%\begin{lemma} 
%	***//引理内容
%	\label{lem-1}
%\end{lemma}
%
%//LaTeX 头部添加
%\newtheorem{corollary}{Corollary}[section]
%
%\begin{corollary} 
%	***//推论内容
%	\label{cor-1}
%\end{corollary}


\usepackage{geometry}
%\geometry{right=2.0cm,left=2.0cm}% 
\geometry{right=2.0cm,left=2.0cm,top = 2.0cm, bottom = 2.0cm}

\title{又一个三元不等式}
\author{单墫}
\date{2020.8.15}
\begin{document}
	\maketitle
$ a,b,c > 0 $
\begin{gather}
	a^2+b+c=11 \label{01}\\
	abc=3\label{02}
\end{gather}
	证明
	\begin{gather}
		\sqrt{a+b}+\sqrt{a+c}+\sqrt{b+c} \ge 4+\sqrt{2}
		\label{key-3}
	\end{gather}
解: $ a=3,b=c=1 $,此时 (\ref{key-3})中等号成立

仍用枚举法

已知$ a\le 3 $(否则(\ref{01}),(\ref{02})不能同时成立)--> why?

1. 若$ a\le 1 $ 则
\begin{gather}
	b+c = 11 - a^2 \ge 10\\
	(\sqrt{a+b}+\sqrt{a+c})^2 =2a+b+c+2\sqrt{a+b}\sqrt{a+c}>b+c \ge 10 \\
	\sqrt{a+b} + \sqrt{a+c}+\sqrt{b+c}\ge \sqrt{10}+\sqrt{10} = 2\sqrt{10}>4+\sqrt{2}
\end{gather}

2. 若$ 1<a\le 3 $, 则
\begin{gather}
	b+c = 11-a^2 \ge 2, \quad  bc = \frac{3}{a}\ge 1\\
	\text{let} x=a+b, y=a+c\\
	x+y = 2a+b+c = 2a+11-a^2\\
	xy = (a+b)(a+c)=a^2a(b+c)+bc
	\ge a^2+2a+1 = (a+1)^2\\
	\therefore \sqrt{x}+\sqrt{y} = \sqrt{(\sqrt{x}+\sqrt{y})^2}\ge \sqrt{2a+11-a^2+2(a+1)}\\
	\sqrt{2a+11-a^2+2(a+1)} = \sqrt{13-a^2+4a} = \sqrt{17-(a-2)^2}\\
	-1<a-2\le 1\quad (a-2)^2 \le 1 \quad 17-(a-2)^2 \ge 16 \therefore \sqrt{x}+\sqrt{y}\ge 4\\
	\therefore 	\sqrt{a+b}+\sqrt{a+c}+\sqrt{b+c} \ge 4+\sqrt{2}
\end{gather}


\end{document}