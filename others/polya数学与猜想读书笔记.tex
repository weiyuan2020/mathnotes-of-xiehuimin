%\documentclass[10pt,a4paper]{book}
%\usepackage[utf8]{inputenc}
%\usepackage[T1]{fontenc}
%\usepackage{amsmath}
%\usepackage{amsthm}
%\usepackage{ctex}
%\usepackage{amsfonts}
%\usepackage{amssymb}
%\usepackage{graphicx}
%
%\usepackage{listings}   % include the package before using it
%
%\newtheorem{theorem}{Theorem}[section]
%\newtheorem{lemma}{Lemma}[section]
%\newtheorem{corollary}{Corollary}[section]
%
%%//LaTeX 头部添加
%%\newtheorem{theorem}{Theorem}[section]
%%
%%\begin{theorem}
%%	***//定理内容
%%	\label{thm-1}
%%\end{theorem}
%%
%%\begin{proof}
%%	***//证明过程
%%\end{proof}
%%
%%//LaTeX 头部添加
%%\newtheorem{lemma}{Lemma}[section]
%%
%%\begin{lemma} 
%%	***//引理内容
%%	\label{lem-1}
%%\end{lemma}
%%
%%//LaTeX 头部添加
%%\newtheorem{corollary}{Corollary}[section]
%%
%%\begin{corollary} 
%%	***//推论内容
%%	\label{cor-1}
%%\end{corollary}
%\newtheorem{myDef}{Definition}
%%\begin{myDef}
%%	\label{label}
%%	...
%%\end{myDef}
%%
%
%
%\usepackage{geometry}
%%\geometry{right=2.0cm,left=2.0cm}% 
%\geometry{right=2.0cm,left=2.0cm,top = 2.0cm, bottom = 2.0cm}

\documentclass[mode=geye ]{ elegantnote}
\title{波利亚 数学与猜想 笔记}
\author{weiyuan}
\date{\today}
\begin{document}
	\maketitle
	\section{归纳方法}
	一个猜想性的一般命题若能在新特例中得到证实,则变得更加可信。\\
	\begin{enumerate}
		\item 随时准备修正我们的信念
		\item 有充分理由使我们改变信念,则应改变
		\item 无充分理由,不应轻易改变信念
	\end{enumerate}
	
	例题:
	
	1. 找规律
	\begin{gather*}
		11, 31, 41, 61, 71, 101, 131, \dots
	\end{gather*}
	以上数字都是素数
	\begin{gather*}
		21 = 3 \times 7\\
		51 = 3 \times 17\\
		81 = 3^4\\
		91 = 7 \times 13\\
		111 = 3 \times 37 \\
		121 = 11^2 \\
		141 = 3 \times 47\\
	\end{gather*}

	2. 
	\begin{gather*}
		\begin{aligned}
		&1			&=&0+1	\\
		&2+3+4		&=&1+8	\\
		&5+6+7+8+9	&=&8+27	\\
		&10+11+\dots+16	&=&27+64 \\
		\dots \\
		&n^2+1 + \dots + (n+1)^2 &=& n^3+(n+1)^3\\
		\end{aligned}
	\end{gather*}
	
	3.
	\begin{gather*}
		1,1+3,1+3+5,1+3+5+7,\dots
	\end{gather*}
	规律\\
	\begin{gather*}
		1+3+\dots +(2n-1) = \frac{(1+2n-1)\times n}{2} = n^2
	\end{gather*}
	
	4.
	\begin{gather*}
		1,1+8,1+8+27,1+8+27+64, \dots
	\end{gather*}
	\begin{gather*}
		1^3+2^3+\dots + n^3 = (1+2+\dots+n)^2 = \frac{n^2(n+1)^2}{4}
	\end{gather*}

	
\end{document}