\documentclass[10pt,a4paper]{article}
\usepackage[utf8]{inputenc}
\usepackage[T1]{fontenc}
\usepackage{amsmath}
\usepackage{ctex}
\usepackage{amsfonts}
\usepackage{amssymb}
\usepackage{graphicx}
\begin{document}
	20.08.07\\
	
	A-G 不等式\\ $a_1, a_2, \dots, a_n$ n个非负实数\\
	\begin{equation}
		\frac{a_1 + a_2 + \dots + a_n}{n} \geq \sqrt[n]{a_1\dots a_n}
	\end{equation}
	等号成立 $\iff a_1 = a_2 = \dots = a_n$	\\
	\\
	证明 1\\
	数学归纳法\\ $n=1$时结论平凡\\
	$n=2\qquad \frac{a_1+a_2}{2} \geq \sqrt{a_1a_2}$\\
	\[(a_1 - a_2)^2 = a_1^2 - 2 a_1 a_2 + a_2^2 \geq 0 \]
	\[a_1^2 + 2a_1a_2 + a_2^2 \geq 4a_1a_2\]
	\[(a_1+a_2)^2\geq 4a_1a_2\]
	\[\frac{a_1+a_2}{2}\geq \sqrt{a_1a_2}\]
	$n=k$时,假设 $\frac{a_1+\dots+a_k}{k}\geq \sqrt[k]{a_1\dots a_k}$成立\\
	$ n=k+1 $
	\begin{equation}
	\begin{aligned}
		&\frac{a_1+\dots + a_k + a_{k+1}}{k+1}-\frac{a_1+\dots +a_k}{k} \\
		=&\frac{k(a_1+\dots+a_{k+1})-(k+1)(a_1+\dots+a_k)}{k(k+1)}\\
		=&\frac{ka_{k+1}-(a_1+\dots+a_k)}{k(k+1)}\\		
	\end{aligned}
	\end{equation}
	we found $\frac{a_1+\dots + a_k + a_{k+1}}{k+1} =  \frac{a_1+\dots + a_k}{k} + \frac{ka_{k+1}-(a_1+\dots + a_k)}{k(k+1)} $\\
	note $ A := \frac{a_1+\dots + a_k}{k} , \qquad B:=\frac{ka_{k+1}-(a_1+\dots + a_k)}{k(k+1)} $
	
	\begin{equation}
		(\frac{a_1+\dots + a_k + a_{k+1}}{k+1})^{k+1}=(A+B)^{k+1}\geq A^{k+1}+(k+1)A^k B
	\end{equation}
	使用二项式展开需要对$ a_i $从小到大重排,而使用Bernoulli不等式则只需要$ A\geq 0, (A+B)\geq 0 $即可
	\begin{equation}
		A^{k+1}+(k+1)A^k B = A^k(A+(k+1)B)
	\end{equation}
	\begin{equation}
		\begin{aligned}
			A^k& =	(\frac{a_1+\dots + a_k + a_{k+1}}{k+1})^{k+1} \geq a_1\dots a_k \quad \text{assume at}(n=k)\\
			A+(k+1)B&= \frac{a_1+\dots + a_k}{k} + \frac{ka_{k+1}-(a_1+\dots + a_k)}{k} = a_{k+1}\\
			\because& (A+B)^{k+1}\geq A^k(A+(k+1)B)\geq a_1 \dots a_k  a_{k+1}\\
			\therefore & 	\frac{a_1+\dots + a_k + a_{k+1}}{k+1} \geq  \sqrt[k+1]{a_1 \dots a_k  a_{k+1}}\\
		\end{aligned}
	\end{equation}
	
	使用二项式展开定理的条件:\\
	在归纳法第二步对$a_1 \dots a_{k+1}  $重编号,使$ a_{k+1} $为其中最大的数(之一)\\
	这使得分解式右边第二项$ \frac{ka_{k+1}-(a_1+\dots+a_k)}{k(k+1)} $ 一定是非负数
	
	
	证明 2\\
	Forward and backward (Cauchy, 1897)\\
	Forward Part:\\
	$ n=2 $ 
	\begin{equation}
		\frac{a_1+a_2}{2}\geq \sqrt{a_1a_2}
	\end{equation}
	$ n=4 $ 
	\begin{equation}
		\begin{aligned}
			\frac{a_1+a_2+a_3+a_4}{4}
			&\geq \sqrt{\frac{a_1+a_2}{2}\frac{a_3+a_4}{2}}\\
			&\geq \sqrt{\sqrt{a_1a_2}\sqrt{a_3a_4}}\\
			&\geq \sqrt[4]{a_1a_2a_3a_4}\\
		\end{aligned}
	\end{equation}
	$ n=2^k $ 假设不等式$ \frac{a_1+\dots +a_{2^k}}{2^k}\geq \sqrt[2^k]{a_1\dots a_{2^k}} $成立\\
	$ n=2^{k+1} $
	\begin{equation}
		\begin{aligned}
			\frac{a_1+\dots+a_{2^k}+\dots+a_{2^{k+1}}}{2^{k+1}}
			&\geq \sqrt{\frac{a_1+\dots +a_{2^k}}{2^k}\frac{a_{2^k+1}+\dots +a_{2^{k+1}}}{2^k}}\\
			&\geq \sqrt{\sqrt[2^k]{a_1\dots a_{2^k}}\sqrt[2^k]{a_{2^k+1}\dots a_{2^{k+1}}}}\\
			&\geq \sqrt[2^{k+1}]{a_1\dots a_{2^{k+1}}}
		\end{aligned}
	\end{equation}
	
	Backward Part:
	A-G不等式对某个$ n\geq 2 $成立,则它对$ n-1 $也成立
	\begin{equation}
		\begin{aligned}
			\frac{1}{n-1}\sum_{i=1}^{n-1}a_i 
			&= \frac{1}{n}(\frac{n}{n-1})\sum_{i=1}^{n-1}a_i\\
			&=\frac{1}{n}(\sum_{i=1}^{n-1}a_i+\frac{1}{n-1}\sum_{i=1}^{n-1}a_i)
		\end{aligned}
	\end{equation}
	将$ \frac{1}{n-1}\sum_{i=1}^{n-1}a_i $看作$ a_n $
	\begin{equation}
		\begin{aligned}
			\frac{1}{n-1}\sum_{i=1}^{n-1}a_i
			&\geq \sqrt[n]{(\prod_{i=1}^{n-1}a_i) (\frac{1}{n-1}\sum_{i=1}^{n-1}a_i)}\\
			(\frac{1}{n-1}\sum_{i=1}^{n-1}a_i)^n
			&\geq \prod_{i=1}^{n-1}a_i(\frac{1}{n-1}\sum_{i=1}^{n-1}a_i)\\
			(\frac{1}{n-1}\sum_{i=1}^{n-1}a_i)^{n-1}
			&\geq \prod_{i=1}^{n-1}a_i\\
			\frac{1}{n-1}\sum_{i=1}^{n-1}a_i
			&\geq \sqrt[n-1]{\prod_{i=1}^{n-1}a_i}\\
		\end{aligned}
	\end{equation}
\end{document}