	1.3.1 Bernoulli inequality, Suppose that $ h>-1, n\in\mathbb{N} $, Then:
	\begin{equation}\label{BernoulliIneq}
		(1+h)^n\ge1+nh
 	\end{equation}
 	When $ n>1 $, the inequality became equation iff $ h=0 $.\\
 	
 	proof:\\ 	
 	When $ n=1 $, $ 1+h=1+h $\\
 	$ h=0 $, $ 1^n=1 $\\
 	Let's consider the condition $ n>1, h\ne 0 $.\\
 	i). $ h>0 $, $ (1+h)^n = \binom{n}{0}h^0+\binom{n}{1}h^1+\binom{n}{2}h^2+\cdots + \binom{n}{n}h^n$.\\
 	$ \because \binom{n}{2}h^2+\cdots + \binom{n}{n}h^n >0 $, $ \therefore (1+h)^n > 1+nh $\\
 	ii). $ -1<h<0, 0<1+h<1 $.
 	\begin{align*}
 		(1+h)^n-1 &=(1+h-1)\Big(1+(1+h)+(1+h)^2+\dots+(1+h)^{n-1}\Big)\\
 		&=h\Big(1+(1+h)+(1+h)^2+\dots+(1+h)^{n-1}\Big)
 	\end{align*}
 	$ \because 1+(1+h)+(1+h)^2+\dots+(1+h)^{n-1} < n $ when$ h<0 $\\
 	$ \therefore (1+h)^n > 1+nh $
 	
 	Two variable extension of the Bernoulli inequality, Suppose
 	$ h = \frac{B}{A}, A>0, A+B>0 $, Then$ 1+h>0 $ is established.
 	
 	1.3.2 Suppose $ A>0, A+B>0, n\in\mathbb{N} $, Then the inequality is true:
 	\begin{equation}\label{BernoulliIneqExts}
 		(A+B)^n\ge A^n+nA^{n-1}B
 	\end{equation}
 	The inequalty became equation iff $ B=0 $.
 	
 	proof:\\
 	divide $ A^n $ on both side of the inequality \ref{BernoulliIneqExts}. Set$ h=\frac{B}{A} (A>0) $, Then the inequality became Eq \ref{BernoulliIneq}.
 	So we can prove Eq \ref{BernoulliIneqExts} by prove Eq \ref{BernoulliIneq}.
 	Eq \ref{BernoulliIneq} is true when $ h>-1 $. $ \therefore 1+h>0, 1+\frac{B}{A}>0 $, $ \because A>0 $, $ \therefore A+B>0 $. And when $ n>1 $ the equation is true iff $h=0 $.$ \frac{B}{A}=0,\therefore B=0 $.
 	
 	\subsection{Ex 1.3.2}
 	exercise 8\\
 	$ a,c,t,g \ge 0, a+c+t+g =1 $. Prove that $ a^2+c^2+t^2+g^2 \ge \frac{1}{4} $.\\ The inequality became equatio iff $ a=c=t=g=\frac{1}{4} $.\\
 	
 	proof:\\
 	from A.G inequality,
 	\begin{equation}\label{AGofDNA}
 		\frac{a+c+t+g}{4} \ge \sqrt[4]{actg}, \quad a+c+t+g=1
 	\end{equation}
 	$ \therefore \sqrt[4]{actg} \le \frac{1}{4} $
% 	\begin{equation*}
% 		\frac{a^2+c^2+t^2+g^2}{4} \ge \sqrt[4]{a^2c^2t^2g^2} = |actg|
% 	\end{equation*}
 	\begin{equation*}
 		a+c+t+g=1, (a+c+t+g)^2=1
 	\end{equation*}
 	\begin{equation}\label{DNAeq2}
 		(a+c+t+g)^2=a^2+c^2+t^2+g^2+2ac+2at+2ag+2ct+2cg+2tg = 1
 	\end{equation}
 	\begin{align}
 		a^2+c^2 \ge 2ac &c^2+t^2 \ge 2ct \\
 		a^2+t^2 \ge 2at &c^2+g^2 \ge 2cg \\
 		a^2+g^2 \ge 2ag &t^2+g^2 \ge 2tg
 	\end{align}
 	substitude $ 2ac,2ag,\dots $ in equation \ref{DNAeq2}, we can get
 	\begin{equation*}
 		4(a^2+c^2+t^2+g^2)\ge a^2+c^2+t^2+g^2+2ac+2at+2ag+2ct+2cg+2tg
 	\end{equation*}
 	Then we get the inequality \ref{AGofDNA}.
 	
 	\section{1.4}
 	The law of duality: 
 	$ \forall(\exists) \rightarrow \exists(\forall) $
 	with negative statement
 	
 	Inverse proposition? 
 	
 	1. A have upper limit, $ \exists M>0. \forall x\in A, x\le M $.\\
 	It's negative statement is 'A don't have upper limit'. $ \forall M >0, \exists x\in A, x>M $.
 	
 	2. the minum item in A is b, $ b\in A,\forall x\in A, x\ge b $.\\
 	It's negative statement is 'b is not the minum item in A'. $ b\in A, \exists x\in A, x< b $.
 	
 	3. $ f\in (a,b) $ is a monotonic augmentation function, $ \forall x,y \in (a,b), x<y, f(x)\le f(y) $.(or $ f(x) < f(y) $, depends on monotonic function's definition)\\
 	It's negative statement is '$ f\in (a,b) $ isn't a monotonic augmentation function'. $ \exists x,y \in (a,b), x<y, f(x)>f(y) $ (or $ f(x)\ge f(y) $).
 	
 	4. $ f\in (a,b) $ is a monotonic function, $ \forall x,y,z \in (a,b), x<y<z, (f(x)-f(y))(f(y)-f(z))\ge 0 $.\\
 	It's negative statement is '$ f\in (a,b) $ isn't a monotonic function'. $ \exists x,y,z \in (a,b), x<y<z, (f(x)-f(y))(f(y)-f(z))< 0 $.\\
 	(Another way $ \forall x,y \in (a,b), x<y, f(x)-f(y)\ge 0 \text{ or } f(x)-f(y)\le 0 $.)
 	
 	5. $ A \subset B $, $ \forall x\in A, x\in B $.\\
 	It's negative statement is $ A \subsetneq B $, $ \exists x\in A, x\notin B $.
 	
 	6. $ A-B\neq \emptyset $, $ \exists x\in A, x\in B $.\\
 	It's negative statement is $ A-B = \emptyset $, $ \forall x\in A, x\notin B $.
 	
 	7. {$ x_n $} is an infinitesimal amounts, $ \forall \epsilon >0, \exists N\in \mathbb{N}^+, \forall n>N, |x_n|<\epsilon $.\\
 	It's negative statement is '{$ x_n $} is not an infinitesimal amounts', $ \exists \epsilon >0, \forall N\in \mathbb{N}^+, \exists n>N, |x_n|\ge\epsilon $.
 	
 	8. {$ x_n $} is infinitely large, $ \forall M >0, \exists N\in \mathbb{N}^+, \forall n>N, x_n>M $.\\
 	It's negative statement is '{$ x_n $} is not infinitely large', $ \exists M >0, \forall N\in \mathbb{N}^+, \exists n>N, x_n\le M $.
 	