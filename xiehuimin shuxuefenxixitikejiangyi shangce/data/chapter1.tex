\chapter{引论}
\section{关于习题课教案的组织}
\subsection{书中常用记号}
\begin{enumerate}
	\item $ \mathbf{N}_{+} $:所有正整数组成的集合.
	\item $ \mathbf{R} $:所有实数组成的集合(同时也用于表示无限区间$( -\infty,\infty )$).
	\item $ \mathbf{Q} $:所有有理数组成的集合.
	\item $ \mathbf{C} $:所有复数组成的集合.
	\item $ \iff $是等价关系的记号.$ A\iff B $ 表示$ A $和$ B $等价. 例如,$ A $代表$ x > 3 $,$ B $代表$ x-3>0 $,则$ x>3\iff x-3>0 $.
	\item $ [x] $是实数$ x $的整数部分,即不超过$ x $的最大整数. 例如,$ [\sqrt{2}] = 1, [-\sqrt{2}] = -2 $. 关于$ [x] $的基本不等式是:$ [x]\le x < [x] + 1 $,或$ x-1<[x]\le x $
%	\item $ \qedsymbol $表示一个证明或解的结束.
	\item $  $空心方块表示一个证明或解的结束.
	\item $ {n \choose k} = C_n^k = \frac{n(n-1)\cdots(n-k+1)}{k!}$.
	\item 记号$ \approx $表示近似值. 例如$ \sqrt{2}\approx 1.4 $.
	\item 复合函数$ f(g(x)) $也写成$ (f\circ g)(x) $或$ f \circ g $.
	\item 若$ A $和$ B $为两个集合,则用记号 $ A-B $ 或 $ A \backslash B $ 表示$ A $与$ B $的差集,也就是集合$ \{x|x\in A \text{且} x\notin B\} $.
	\item 用 $ O_\delta (a) $ 表示以$ a $为中心,以$ \delta >0 $ 为半径的邻域. 它就是开区间$ (a-\delta, a+\delta) $(也可以用$ U_\delta (a) $等记号). 如不必指出半径,则可简记为$ O(a)$ (或$ U(a) $).
\end{enumerate}


\subsection{几个常用的初等不等式}
\subsubsection{几个初等不等式的证明}
A.G 不等式 $ a_1,a_2,\cdots,a_n $ , n个非负实数

\begin{equation}
	\frac{a_1+a_2+\cdots+a_n}{n} \ge \sqrt[n]{a_1\cdots a_n}
\end{equation}
$\ge \text{in inequation became} = \iff a_1=a_2=\cdots=a_n$


\begin{proof}
	
	method 1. induction method

\begin{align*}
	&k=1 \qquad a_1 = a_1 \\
	&k=2 \qquad \frac{a_1+a_2}{2}\ge\sqrt{a_1a_2} \\	
	&k=n  \qquad \text{suppose }\quad \frac{a_1+a_2+\cdots+a_n}{n} \ge \sqrt[n]{a_1\cdots a_n}  \\
	&k = n+1\\ 	&\qquad\frac{a_1+a_2+\cdots+a_{n+1}}{n+1} - \frac{a_1+a_2+\cdots+a_{n}}{n} \\
	&\qquad=\frac{n(a_1+a_2+\cdots+a_{n+1})-(n+1)(a_1+a_2+\cdots+a_n)}{n(n+1)} \\
	&\qquad=\frac{na_{n+1}-(a_1+a_2+\cdots+a_n)}{n(n+1)}  \\
\end{align*}




Set $ A = \frac{a_1+a_2+\cdots+a_n}{n} $, $ B = \frac{na_{n+1} - (a_1+a_2\cdots+a_n)}{n(n+1)} $\\
\begin{equation*}
	(\frac{a_1+a_2+\cdots+a_{n+1}}{n+1})^{n+1} = (A+B)^{n+1}
\end{equation*}
$ A>0, B\ge 0$
\begin{align*}
	&(A+B)^{n+1} \ge A^{n+1}+(n+1)A^n B\\
	&A^{n+1} + (n+1)A^n B = A^n(A+(n+1)B)\\	
	&A^n = (\frac{a_1+a_2+\cdots+a_n}{n})^n \ge a_1a_2\cdots a_n\\
	&A + (n+1)B = \frac{a_1+a_2+\cdots+a_n}{n} + \frac{na_{n+1}-(a_1+a_2+\cdots+a_n)}{n} = a_{n+1}\\
	&\therefore (A+B)^{n+1} \ge A^n(A+(n+1)B)\ge a_1a_2\cdots a_n\cdot a_{n+1}\\
	&\therefore \frac{a_1+a_2+\cdots+a_{n+1}}{n+1} \ge \sqrt[n+1]{a_1a_2\cdots a_na_{n+1}}\\
\end{align*}

\textbf{使用二项式展开定理的条件}

在归纳法第二步,将$ a_1,a_2,\cdots,a_{n+1} $重编号,使得$ {n+1} $为其中最大的数(之一),这使得分解式右边第二项$ (na_{n+1}-(a_1+a_2+\cdots+a_n))/n(n+1) $一定是非负数。

method 2. Forward and Backward (Cauchy, 1897)

Forward part
\begin{align*}
	&k = 2. \frac{a_1+a_2}{2}\ge \sqrt{a_1a_2}.\\
	&k = 4. \frac{a_1+a_2+a_3+a_4}{4} \ge\sqrt{(\frac{a_1+a_2}{2})\cdot(\frac{a_3+a_4}{2})}.\\
	&\qquad  \ge \sqrt{\sqrt{a_1a_2}\sqrt{a_3a_4}} = \sqrt[4]{a_1a_2a_3a_4}.\\
	&k = 2^n. \text{Suppose}\quad \frac{a_1+a_2+\cdots a_{2^n}}{2^n} \ge \sqrt[2^n]{a_1a_2\cdots a_{2^n}}\\
	&k = 2^{n+1}.\\
	&	\frac{a_1+a_2+\cdots+a_{2^n}+\cdots+a_{2^{n+1}}}{2^{n+1}} \ge \sqrt{(\frac{a_1+a_2+\cdots+a_2^{n}}{2^n})\cdot(\frac{a_{2^{n} + 1}+a_{2^{n}+2}+\cdots+a_2^{n+1}}{2^n})}\\
	&I\ge \sqrt{\sqrt[2^n]{a_1a_2\cdots a_{2^n}}\sqrt[2^n]{a_{2^n+1} a_{2^n+2}\cdots a_{2^{n+1}}}} = \sqrt[2^{n+1}]{a_1a_2\cdots a_{2^{n+1}}}\\
\end{align*}

Backward part
suppose A.G inequality is valid when $ k = n $, Consider $ k = n-1 $.
\begin{align*}
	&\frac{1}{n-1}\sum_{i=1}^{n-1} a_i = \frac{1}{n}(\frac{n}{n-1})\sum_{i=1}^{n-1}a_i\\
	&\frac{1}{n-1}\sum_{i=1}^{n-1} a_i = \frac{1}{n}(\sum_{i=1}^{n-1} a_i + \frac{1}{n-1} \sum_{i=1}^{n-1} a_i)\\
	&\frac{1}{n-1} \sum_{i=1}^{n-1} a_i \ge \sqrt[n]{(\prod_{i=1}^{n-1} a_i)(\frac{1}{n-1}\sum_{i=1}^{n-1} a_i)}\\
	&(\frac{1}{n-1} \sum_{i=1}^{n-1} a_i)^n \ge (\prod_{i=1}^{n-1} a_i)(\frac{1}{n-1}\sum_{i=1}^{n-1} a_i)\\
	&(\frac{1}{n-1} \sum_{i=1}^{n-1} a_i)^{n-1} \ge (\prod_{i=1}^{n-1} a_i)\\
	&\frac{1}{n-1} \sum_{i=1}^{n-1} a_i \ge \sqrt[n-1]{\prod_{i=1}^{n-1} a_i}
\end{align*}
\end{proof}

\begin{proposition}
	 柯西-施瓦茨不等式
	对$ a_1,a_2,\cdots,a_n $ 和 $ b_1,b_2,\cdots,b_n $,成立
\end{proposition}


\begin{equation*}
	|\sum_{i=1}^n a_i b_i \le \sqrt{\sum_{i=1}^n a_i^2}\sqrt{\sum_{i=1}^n b_i^2}|
\end{equation*}

\begin{proof}
	\begin{equation*}
		 0 \le \sum_{i=1}^n (a_i - \lambda b_i)^2  = \sum_{i=1}^n a_i^2 - 2 \lambda \sum_{i=1}^n a_i b_i  + \lambda^2 \sum_{i=1}^n b_i^2
	\end{equation*}
	由韦达定理 (视$ \lambda $为未知数). 原方程无解或只有唯一解。
	\begin{align*}
		&\Delta = b^2-4ac \le 0\\
		&(-2\sum_{i=1}^n a_i b_i)^2 - 4\sum_{i=1}^n a_i^2 \sum_{i=1}^n b_i^2 \le 0 \\
		&(\sum_{i=1}^n a_i b_i)^2 \le \sum_{i=1}^n a_i^2 \sum_{i=1}^n b_i^2 \\
		&\sum_{i=1}^n a_i b_i \le \sqrt{\sum_{i=1}^n a_i^2} \sqrt{\sum_{i=1}^n b_i^2} 
	\end{align*}
\end{proof}

\subsubsection{练习题}
\begin{example}	关于 Bernoulli 不等式的推广:\\	
	(1) 证明: 当 $ -2 \le h \le -1 $ 时 Bernoulli 不等式 $ (1+h)^n \ge 1+nh $ 仍成立;\\
	(2) 证明: 当 $ h \ge 0 $ 时成立不等式 $ (1+h)^n \ge \frac{n(n-1)h^2}{2} $, 并推广之;\\
	(3) 证明: 若$ a_i > -1(i = 1,2,\dots,n) $ 且同号,则成立不等式
	\[\prod_{i=1}^{n}(1+a_i) \ge 1 + \sum_{i=1}^n a_i \]
	\begin{proof}
		(1)			
		\begin{align*}
			& -2 \le h \le -1  & \\
			& -1 \le 1+h\le 0  & -1 \le (1+h)^n \le 0 \\
			& -2n \le nh \le -n& 1-2n \le 1+nh\le 1-n
		\end{align*}	
		\begin{align*}
			n=0. && (1+h)^0 = 1 = 1 + 0\times h\\
			n=1. && 1+h = 1+h\\
			n \ge 2. && 1-n \le -2
		\end{align*}	
		\begin{align*}
			0 \ge (1+h)^n \ge -1 \ge -2 \ge 1-n \ge 1+nh \ge 1-2n
		\end{align*}	
		\begin{align*}		
			(1+h)^n \ge 1+nh
	 	\end{align*}
	
		(2)
		\begin{align*}
			h\ge 0&\\
			(1+h)^n = &1+nh+\frac{n(n-1)}{2}h^2 + \dots \ge \frac{n(n-1)}{2}h^2\\
		\end{align*}
		推广:
		\begin{equation*}
			(1+h)^n \ge \binom{n}{3}h^3, \binom{n}{4}h^4, \dots, \binom{n}{k}h^k, 0\le k \le n
		\end{equation*}
		
		(3)
%		\begin{align*}
%			\prod_{i=1}^n(1+a_i) = 1+\sum_{i=1}^n a_i + \sum_{i=1}^n\sum_{j=1}^n a_i a_j + 	\sum_{i=1}^n\sum_{j=1}^n\sum_{k=1}^n a_i a_j a_k + \dots
%		\end{align*}
%		分类讨论\\
%		1. $ a_i\ge 0, 1+a_i \ge 1 $. 
%		\[ \prod_{i=1}^n (1+a_i) \ge \frac{\prod_{i=1}^n (1+a_i)}{1+a_k} \qquad\forall k \in 1,2, \dots,n \]
		$ k = 1 $ 时显然成立. 使用归纳法证明. 假设$ k = n $ 时不等式
		$ \prod_{i=1}^{n}(1+a_i) \ge 1 + \sum_{i=1}^n a_i  $
		成立, 证明 $ k = n+1 $时 
		$  \prod_{i=1}^{n+1}(1+a_i) \ge 1 + \sum_{i=1}^{n+1} a_i  $
		成立.		
		\begin{align*}
			k = n+1 \qquad  
			& \prod_{i=1}^{n+1}(1+a_i) = \prod_{i=1}^{n}(1+a_i)  (1+a_{n+1})\\
			& \because \prod_{i=1}^{n}(1+a_i) \ge 1 + \sum_{i=1}^n a_i  \\
			& \prod_{i=1}^{n}(1+a_i)  (1+a_{n+1}) \ge (1 + \sum_{i=1}^n a_i)(1+a_{n+1})
		\end{align*}
		\begin{align*}
			(1 + \sum_{i=1}^n a_i)(1+a_{n+1}) & = 1 + \sum_{i=1}^n a_i + a_{n+1} + a_{n+1} \sum_{i=1}^n a_i\\
			& = 1 + \sum_{i=1}^{n+1} a_i + a_{n+1} \sum_{i=1}^n a_i\\
			& \ge 1 + \sum_{i=1}^{n+1} a_i
		\end{align*}
	\end{proof}
\end{example}

\begin{example}
	利用 A.G.不等式求解:\\
	(1). $ n! \le (\frac{n+1}{2})^n, \text{while } n >1 $\\
	(2). $ (n!)^2 = (n\cdot 1)[(n-1)\cdot 2]\dots (1\cdots n) $. 证明: 当$ n > 1 $时成立
	\[n!< (\frac{n+2}{6})^n\]
	(3). 比较上述两个不等式的优劣\\
	(4). 证明: 对任意实数r成立:
	\begin{equation}\label{FactorialNPowerR}
		(n!)^r\le  \frac{1}{n^n} (\sum_{k=1}^n k^r)^n 
	\end{equation}
	\begin{proof}
		(1).
		\begin{align*}
			n>1 \qquad & n! = 1 \times 2 \times \cdots \times n < (\frac{1+2+\cdots + n}{n})^n = (\frac{(1+n)n}{2n})^n = (\frac{n+1}{2})^n
		\end{align*}
		$ \because 1 \neq 2 \neq \cdots n $, 所以不会有等号出现的情况\\
		(2). $ n > 1 $
		\begin{align*}		
			 (n!)^2 
			 & = (n\cdot 1)[(n-1)\cdot 2]\dots (1\cdots n)\\
			 & < (\frac{n\times1 + (n-1)\times 2 + \dots + 1\times n}{n})^n\\
		\end{align*}
		Consider this equation 
		\begin{equation}\label{keyn1nminus12}
			 (\frac{n\times1 + (n-1)\times 2 + \dots + 1\times n}{n})^n 
		\end{equation}
		\begin{align*}
			\sum_{k=1}^n (n-k+1)k 
			& = (n+1)\sum_{k=1}^nk - \sum_{k=1} k^2\\
			& = (n+1)\frac{(n+1)n}{2} - \frac{n(n+1)(2n+1)}{6}\\	
			& = \frac{n(n+1)}{6}(3(n+1) - (2n+1))\\
			& = \frac{n(n+1)(n+2)}{6}
		\end{align*}
		\begin{align*}		
			(n!)^2 
			& <  (\frac{n\times1 + (n-1)\times 2 + \dots + 1\times n}{n})^n\\
			& = (\frac{(n+1)(n+2)}{6})^n
		\end{align*}
		$ \because n+1<n+2 $,
		$ \therefore n! < (\frac{n+2}{\sqrt{6}})^n $\\
		(3). $ n>3 $时,$ \frac{n+2}{\sqrt{6}} < \frac{n+1}{2} $ (2)的结果较好.\\
		(4).$ \forall r\in \mathbb{R} $, prove formula \ref{FactorialNPowerR}
		\begin{align*}
			&\frac{1}{n}\sum_{k=1}^n k^r \ge \sqrt[n]{\prod_{k=1}^n k^r}\\
			&(n!)^r =\prod_{k=1}^n k^r \le (\frac{1}{n}\sum_{k=1}^n k^r)^n = \frac{1}{n^n}(\sum_{k=1}^n k^r)^n 
		\end{align*}
	my answer
	\begin{align*}
		&\forall r\in \mathbb{R},\qquad(\sum_{k=1}^n k^r)^n \ge n^n(n!)^r\\
		&(n!)^r = \sum_{k=1}^n k^r \le (\frac{1^r+2^r+\dots+n^r}{n})^n = \frac{1}{n^n}(\sum_{k=1}^n k^r)^n\\
		&\therefore\quad (\sum_{k=1}^n k^r)^n \ge n^n(n!)^r
 	\end{align*}
	\end{proof}
\end{example}

\begin{example}
	$ a_k>0, k = 1,2,\dots,n$ 证明几何--调和平均值不等式
	\[(\prod_{k=1}^n a_k)^{\frac{1}{n}} \ge \frac{n}{\sum_{k=1}^n\frac{1}{a_k}}\]
	\begin{proof}
		from A.G inequality
		\[\frac{\sum_{k=1}^n\frac{1}{a_k}}{n} \ge \sqrt[n]{\prod_{k=1}^n \frac{1}{a_k}} = \frac{1}{\sqrt[n]{\prod_{k=1}^n a_k}}\]
		\[a_k>0, \quad \sqrt[n]{\prod_{k=1}^n {a_k}} \ge  \frac{n}{\sum_{k=1}^n\frac{1}{a_k}} \]
	\end{proof}
\end{example}

\begin{example}
	$ a,b,c\ge 0. $ prove $ \sqrt[3]{abc}\le\sqrt{\frac{ab+bc+ca}{3}} \le \frac{a+b+c}{3} $.
	并推广到n个非负数的情况
	\begin{proof}
		1. $ \sqrt[3]{abc} = \sqrt{\sqrt[3]{ab\cdot bc\cdot ca}} \le \sqrt{\frac{ab+bc+ca}{3}} $.\\
		2. 
		\begin{align*}
			\sqrt{\frac{ab+bc+ca}{3}} 
			\le &\sqrt{\frac{(\frac{a+b}{2})^2+(\frac{b+c}{2})^2+(\frac{c+a}{2})^2}{3}} \\
			&= \sqrt{\frac{2(a^2+b^2+c^2)+2(ab+bc+ca)}{12}}\\
			&= \sqrt{\frac{a^2+b^2+c^2+ab+bc+ca}{6}}
		\end{align*}
		$ a,b,c\ge 0 $, 希望证明\[\sqrt{\frac{ab+bc+ca}{3}}\le \frac{a+b+c}{3}\]
		\begin{align*}
			\frac{ab+bc+ca}{3} &\le \frac{a^2+b^2+c^2}{6} + \frac{ab+bc+ca}{6}\\
			\frac{ab+bc+ca}{2} &\le \frac{a^2+b^2+c^2}{6} + 2\frac{ab+bc+ca}{6}\qquad(\text{add} \frac{ab+bc+ca}{6})\\
			\frac{ab+bc+ca}{3} &\le \frac{ab+bc+ca}{2} \le (\frac{a+b+c}{3})^2\\
			&\sqrt{\frac{ab+bc+ca}{3}} \le \frac{a+b+c}{3}
		\end{align*}
		推广至n个
		\begin{align*}[l]
			n=2\qquad& \sqrt{ab} \le\frac{a+b}{2}\\
			n=3\qquad& \sqrt[3]{abc}  \le \sqrt{\frac{ab+bc+ca}{3}} \le \frac{a+b+c}{3}\\
			n=4\qquad& \sqrt[4]{abcd}  \le \sqrt[3]{\frac{abc+bcd+cda+dab}{4}} \le \sqrt{\frac{a+b+c}{3}}\le\frac{a+b+c+d}{4}
		\end{align*}
		\begin{align*}
			k = n \qquad& \sqrt[n]{a_1a_2\dots a_n} \le 		\sqrt{\frac{a_1+a_2+\dots+a_n}{n}}\le\frac{a_1+a_2+\dots+a_n}{n}
		\end{align*}
		This is
		\begin{align*}
			\sqrt[n]{\sum_{k=1}^n a_k} \le \sqrt{\frac{\sum_{k=1}^n a_k}{k}}\le\frac{\sum_{k=1}^n a_k}{k}
		\end{align*}
		1. $ \sqrt[n]{a_1 a_2 \dots a_n} = \sqrt{\sqrt[n]{a_1^2 a_2^2 \dots a_n^2}} \le \sqrt{\frac{a_1 a_2 + a_2 a_3 + \dots + a_n a_1}{n}}$\\
		2. $ \sqrt{\frac{a_1 a_2 + a_2 a_3 + \dots + a_n a_1}{n}} \le \sqrt{\frac{a_1+ a_2 + \dots + a_n}{n}} $?
	\end{proof}
\end{example}

\begin{example}
	(1) $ |\alpha + \beta | \le |\alpha| + |\beta|$
	\begin{proof}
		let $ \alpha = a-b, \beta = b $, the identity became $ |(a-b) + b| \le |a-b| + |b| $. This is $ |a - b| \ge |a| - |b| $.
		\begin{equation*}
			||a| - |b|| = \Big\lbrace %\left\lbrace  \right % empty \right for tex grammer check!
			\begin{aligned}
				|a| - |b|. \qquad a \ge b\\
				|b| - |a|. \qquad a < b
			\end{aligned}
		\end{equation*}
		When $ a\ge b $, $ ||a| - |b|| = |a| - |b| $. There is $ |a-b| \ge |a| - |b| = ||a|-|b|| $\\
		When $ a < b $, $ |a-b| = |b-a| \ge |b| - |a|  = ||a|-|b|| $.\\
		$ \therefore $, We have$ |a-b| \ge ||a|-|b|| $
	\end{proof}
		
	(2) $ \sum |a_k| \ge |\sum a_k|$ 
	\begin{proof}
		We can prove this statement by induction.
		\begin{align*}
			k=2,\qquad&|a_1| + |a_2| \ge |a_1+a_2|\\
			k=3,\qquad&|a_1| + |a_2| + |a_3| \ge |a_1+a_2+a_3|\\
			\text{Suppose } k = n, \qquad & \sum_{k=1}^n |a_k| \ge |\sum_{k=1}^n a_k| \\
			k = n+1, \qquad& \text{prove}\sum_{k=1}^{n+1} |a_k| \ge |\sum_{k=1}^{n+1} a_k|
		\end{align*}
		\begin{align*}
			\sum_{k=1}^{n+1} |a_k| 
			&= \sum_{k=1}^{n} |a_k| + |a_{n+1}| \\
			&\ge |\sum_{k=1}^n a_k| + |a_{n+1}| \\
			&\ge |\sum_{k=1}^{n+1} a_k|
		\end{align*}
	
		\begin{align*}
			k=2,\qquad&|a_1| - |a_2| \le |a_1-a_2|\\
			\text{Suppose } k = n, \qquad & |a_1| - \sum_{k=2}^n |a_k| \le |\sum_{k=1}^n a_k| \\
			k = n+1, \qquad& \text{prove}|a_1| - \sum_{k=2}^{n+1} |a_k| \le |\sum_{k=1}^{n+1} a_k|
		\end{align*}
		\begin{align*}
			|a_1| - \sum_{k=2}^{n+1} |a_k| 
			&= |a_1| - \sum_{k=2}^{n} |a_k| - |a_{n+1}| \\
			&\le |\sum_{k=1}^n a_k| - |a_{n+1}| \\
			&\le |\sum_{k=1}^{n+1} a_k|
		\end{align*}
		Can left side became$ \left| |a_1| - \sum_{k=2}^n|a_k|\right| $?
		\begin{equation}\label{a1ak_eq1}
			\left| |a_1| - \sum_{k=2}^n|a_k|\right| = |a_1| - \sum_{k=2}^n|a_k| \qquad |a_1| \ge\sum_{k=2}^n a_k
		\end{equation}
		\begin{equation}\label{a1ak_eq2}
			\left| |a_1| - \sum_{k=2}^n|a_k|\right| = \sum_{k=2}^n|a_k| - |a_1| \qquad |a_1| \ge\sum_{k=2}^n a_k
		\end{equation}
		in eq\ref{a1ak_eq1}, the inequality is still vaild. However in eq\ref{a1ak_eq2}, $ \sum_{k=2}^n |a_k| - |a_1| $ and $ |a_1| $
	\end{proof}
	(3). $\frac{|a+b|}{1+|a+b|}\le \frac{|a|}{1+|a|} + \frac{|b|}{1+|b|} $
	\begin{proof}
%		\begin{align*}
%			\frac{|a+b|}{1+|a+b|}\le &\frac{|a|}{1+|a|} + \frac{|b|}{1+|b|}\\
%			1-\frac{1}{1+|a+b|}\le&2-(\frac{1}{1+|a|} + \frac{1}{1+|b|})\\
%			& = 2-\frac{2+|a|+|b|}{(1+|a|)(1+|b|)}\\
%			& = 1-\frac{2+|a|+|b|-(1+|a|)(1+|b|)}{(1+|a|)(1+|b|)}\\
%			& = 1-\frac{1-|a||b|}{(1+|a|)(1+|b|)}
%		\end{align*}
	\begin{align*}
		\frac{|a+b|}{1+|a+b|}\le &\frac{|a|}{1+|a|} + \frac{|b|}{1+|b|}\\
		\frac{|a+b|}{1+|a+b|}\le &\frac{|a|+|b|+2|a||b|}{(1+|a|)(1+|b|)}\\
		1-\frac{|a+b|}{1+|a+b|}\ge &1-\frac{|a|+|b|+2|a||b|}{(1+|a|)(1+|b|)}\\
		\frac{1}{1+|a+b|}\ge &\frac{1-|a||b|}{(1+|a|)(1+|b|)}\\
		1+|a|+|b|+|a||b| \ge & 1+|a+b|-|a||b|-|a||b||a+b|\\
		|a|+|b|+2|a||b|+|a||b||a+b|>&0, \text{Since }+2|a||b|+|a||b||a+b|\ge|a+b|
	\end{align*}
	Therefore $\frac{|a+b|}{1+|a+b|}\le \frac{|a|}{1+|a|} + \frac{|b|}{1+|b|} $
	\end{proof}
%	\begin{align*}
%		1-\frac{1}{1+|a+b|}\le
%		&2-(\frac{1}{1+|a|} + \frac{1}{1+|b|})\\
%		& = 2-\frac{2+|a|+|b|}{(1+|a|)(1+|b|)}\\
%		& = 1-\frac{2+|a|+|b|-(1+|a|)(1+|b|)}{(1+|a|)(1+|b|)}\\
%		& = 1-\frac{1-|a||b|}{(1+|a|)(1+|b|)}
%	\end{align*}
%	We only need to prove that $ \frac{1}{1+|a+b|} \ge \frac{1-|a||b|}{(1+|a|)(1+|b|) $. We already know that $ |a||b| = |ab| $.
%		\begin{align*}
%			&(1+|a|)(1+|b|) &\ge (1+|a+b|)(|ab|-1)\\
%			&1+|a|+|b|+|ab| &\ge |ab|+|ab|\cdot|a+b|-1-|a+b|\\
%			&2+|a|+|b|+|a+b| &\ge |ab|\cdot|a+b|\
%		\end{align*}
%		\begin{align*}
%			|a+b|&\le|a|+|b|\\
%			1+|a+b|&\le 1+|a|+|b|\\
%			&\le 1+|a|+|b|+|a||b|\\
%			&=(1+|a|)(1+|b|)
%		\end{align*}
\end{example}

\begin{example}
	(4).$ |(a+b)^n-a^n| \le (|a|+|b|)^n-|a|^n $
	\begin{align*}
		(a+b)^n-a^n &= \binom{n}{1}a^{n-1}b^{1} + \binom{n}{2}a^{n-2}b^{2} + \cdots + \binom{n}{n}a^{0}b^{n}\\
		(|a|+|b|)^n-|a|^n &= \binom{n}{1}|a|^{n-1}|b|^{1} + \binom{n}{2}|a|^{n-2}|b|^{2} + \cdots + \binom{n}{n}|a|^{0}|b|^{n}
	\end{align*}
	\begin{align*}
		&\because |a|^j|b|^k \ge a^j b^k\\
		&\therefore \sum |a|^j|b|^k \ge |\sum a^j b^k|
	\end{align*}
\begin{align*}
	|(a+b)^n-a^n| = \Big\lbrace 
	\begin{aligned}
		(a+b)^n-a^n,\qquad a+b\ge a; b\ge0\\
		a^n-(a+b)^n,\qquad a+b< a; b<0
	\end{aligned}
\end{align*}
\begin{equation}\label{keyEx134answer}
	|(a+b)^n-a^n| \le (|a|+|b|)^n - |a|^n.
\end{equation}
\end{example}

\begin{proposition}
	1.3.5(Cauchy inequality)\\
	For $ a_1,a_2,\dots,a_n.$ and $b_1,b_2,\dots,b_n$. $ a_i,b_i\in\mathbb{R} $, There is
	\begin{equation}\label{CauchyInequality}
		\Big|\sum_{i=1}^n a_i b_i\Big| \le \sqrt{\sum_{i=1}^n a_i^2}\sqrt{\sum_{i=1}^n b_i^2}
	\end{equation}
\end{proposition}
\begin{proof}
	Let's prove eq\ref{CauchyInequality}\\
	First way on book:\\
	Use variable $ \lambda $, change the inequality into nonnegative binomial.
	\begin{align*}
		0&\le\sum_{i=1}^n (a_i -\lambda b_i)^2 
		&=\sum_{i=1}^n a_i^2 - 2\lambda\sum_{i=1}^na_ib_i + \lambda^2\sum_{i=1}^n\\
		\Delta &= B^2-4AC &=(-2\sum_{i=1}^n a_ib_i)^2 - 4(\sum_{i=1}^na_i^2)(\sum_{i=1}^nb_i^2)\le0\\	
	\end{align*}
	\begin{equation*}
		(\sum_{i=1}^n a_ib_i)^2 \le (\sum_{i=1}^na_i^2)(\sum_{i=1}^nb_i^2)
	\end{equation*}
	sqrt on both side of the inequality above, we can get
	\begin{equation*}
		\Big|\sum_{i=1}^n a_i b_i\Big| \le \sqrt{\sum_{i=1}^n a_i^2}\sqrt{\sum_{i=1}^n b_i^2}
	\end{equation*}
\end{proof}
6. Cauchy 不等式的不同证明

(1). 数学归纳法.\\
\begin{align*}
	k=1,\quad& |ab|=\sqrt{a^2}\sqrt{b^2}\\
	k=1,\quad& |a_1b_1+a_2b_2|=\sqrt{a_1^2+a_2^2}\sqrt{b_1^2+b_1^2}\\
	\text{Suppose }k=n,\quad&|\sum_{i=1}^n a_ib_i|=\sqrt{\sum_{i=1}^n a_i}\sqrt{\sum_{i=1}^n b_i}\\
	k=n+1,\quad&|\sum_{i=1}^{n+1} a_ib_i|=|\sum_{i=1}^{n} a_ib_i + a_{n+1}b_{n+1}|
\end{align*}
\begin{align*}
	|\sum_{i=1}^{n+1} a_ib_i|
	&=|\sum_{i=1}^{n} a_ib_i + a_{n+1}b_{n+1}|\\
%	&\le|\sum_{i=1}^{n} a_ib_i| + |a_{n+1}b_{n+1}|\\
	&\le|\sqrt{\sum_{i=1}^n a_i}\sqrt{\sum_{i=1}^n b_i} + a_{n+1}b_{n+1}|\\
\end{align*}
Note that $ A = \sqrt{\sum_{i=1}^n a_i} $, $ B = \sqrt{\sum_{i=1}^n b_i} $
\begin{align*}
	|\sum_{i=1}^{n+1} a_ib_i|
	&\le|AB+a_{n+1}b_{n+1}|\\
	&\le\sqrt{A^2+a_{n+1}^2}\sqrt{B^2+b_{n+1}^2}\\
	&=\sqrt{\sum_{i=1}^na_i^2}\sqrt{\sum_{i=1}^nb_i^2}
\end{align*}
(2) Lagrange 恒等式
\begin{equation}
	\sum_{i=1}^n a_k^2\sum_{i=1}^n b_k^2 - (\sum_{i=1}^n |a_kb_k|) = \frac{1}{2}\sum_{i=1}^n\sum_{k=1}^n(|a_k||b_i|-|a_i||b_k|)^2
\end{equation}
\begin{align*}
	(|a_k||b_i|-|a_i||b_k|)^2 
	&= |a_k|^2|b_i|^2-2|a_i||a_k||b_i||b_k|+|b_k|^2|a_i|^2\\
	&= a_k^2b_i^2+b_k^2a_i^2-2|a_ia_kb_ib_k|
\end{align*}
\begin{equation*}
	\sum_{i=1}^n\sum_{k=1}^n(|a_k||b_i|-|a_i||b_k|)^2 = 2\sum_{i=1}^n a_i^2 \sum_{k=1}^n b_k^2 - 2\sum_{i=1}^n\sum_{k=1}^n|a_ia_kb_ib_k|
\end{equation*}
\begin{equation*}
	\sum_{i=1}^n a_k^2\sum_{i=1}^n b_k^2 - (\sum_{i=1}^n |a_kb_k|) = \frac{1}{2}\sum_{i=1}^n\sum_{k=1}^n(|a_k||b_i|-|a_i||b_k|)^2 \ge 0
\end{equation*}
\begin{align*}
	&\therefore (\sum_{i=1}^n|a_ib_i|)^2 \le \sum_{i=1}^na_i^2 \sum_{i=1}^nb_i^2\\
	&\because |\sum_{i=1}^n a_i b_i| \le \sum_{i=1}^n|a_ib_i|\\
	&\therefore (|\sum_{i=1}^n a_i b_i|)^2 \le (\sum_{i=1}^n|a_ib_i|)^2\\
	&\therefore (|\sum_{i=1}^n a_i b_i|)^2 \le \sum_{i=1}^n a_i^2 \sum_{i=1}^n b_i^2
\end{align*}
不等式两边开平方,得到:
\begin{equation*}
	|\sum_{i=1}^n a_i b_i| \le \sqrt{\sum_{i=1}^n a_i^2} \sqrt{\sum_{i=1}^n b_i^2}
\end{equation*}

(3). 用不等式 $ |AB| \le \frac{A^2+B^2}{2} $
\begin{align*}
	|a_ib_i| &\le \frac{a_i^2+b_i^2}{2}&\\
	|\sum_{i=1}^na_ib_i| &\le \sum_{i=1}^n|a_ib_i|&\le \frac{\sum_{i=1}^n a_i^2+\sum_{i=1}^n b_i^2}{2}\\
	\frac{\sum_{i=1}^na_i^2+\sum_{i=1}^nb_i^2}{2} &\ge \sqrt{\sum_{i=1}^n a_i^2} \sqrt{\sum_{i=1}^n b_i^2} &\text{??}
\end{align*}
如何用均值不等式证明Cauchy不等式?\\
由切比雪夫不等式,有
\begin{equation}\label{CauchyInequality002}
	\frac{a_1b_1+a_2b_2+\cdots+a_nb_n}{n} \le (\frac{a_1+a_2+\cdots+a_n}{n}) (\frac{b_1+b_2+\cdots+b_n}{n})
\end{equation}
由均值不等式,有
\begin{align*}
	\frac{a_1+a_2+\cdots+a_n}{n} &\le \sqrt{\frac{a_1^2+a_2^2+\cdots+a_n^2}{n}}\\
	\frac{b_1+b_2+\cdots+b_n}{n} &\le \sqrt{\frac{b_1^2+b_2^2+\cdots+b_n^2}{n}}
\end{align*}
\begin{align*}
	\therefore \frac{a_1b_1+a_2b_2+\cdots+a_nb_n}{n} 
	&\le (\frac{a_1+a_2+\cdots+a_n}{n}) (\frac{b_1+b_2+\cdots+b_n}{n}) \\
	&\le \sqrt{\frac{a_1^2+a_2^2+\cdots+a_n^2}{n}} \sqrt{\frac{b_1^2+b_2^2+\cdots+b_n^2}{n}}\\
	& = \frac{1}{n}\sqrt{a_1^2+a_2^2+\cdots+a_n^2} \sqrt{b_1^2+b_2^2+\cdots+b_n^2}\\
\end{align*}
This is 
\begin{equation*}
	\sum_{i=1}^n a_i b_i \le \sqrt{\sum_{i=1}^n a_i^2} \sqrt{\sum_{i=1}^n b_i^2}
\end{equation*}
Square on both side of the inequality, The calculate square root. We can get eq\ref{CauchyInequality002}:

(4). 构造复的辅助数列
\begin{equation*}
	c_k = a_k^2 - b_k^2 + 2\i |a_k b_k|, \qquad k = 1,2,\dots, n
\end{equation*}
Then we use
\begin{equation*}
	\Big|\sum_{k=1}^n c_k\Big| \le \sum_{k=1}^n |c_k|
\end{equation*}

\begin{solve}
	\begin{align*}
		&c_k = (|a_k| + \i |b_k|)^2 = a_k^2 + b_k^2 + 2\i |a_k b_k|\\
		&\sum_{k=1}^n c_k = \sum_{k=1}^n a_k^2 + \sum_{k=1}^n b_k^2 + 2\i \sum_{k=1}^n |a_k b_k|\\
		&|c_k| = \sqrt{\Re^2 c_k + \Im^2 c_k} = \sqrt{(a_k^2-b_k^2)^2 + (2a_kb_k)^2} = a_k^2+b_k^2
	\end{align*}
	\begin{align*}
		&\therefore \Big| \sum_{k=1}^n a_k^2 + \sum_{k=1}^n b_k^2 + 2\i \sum_{k=1}^n |a_k b_k| \Big|
		= \sqrt{\Re^2 \sum_{k=1}^n c_k + \Im^2 \sum_{k=1}^n c_k}\\
		&= \sqrt{(\sum_{k=1}^n (a_k^2-b_k^2))^2 + \sum_{k=1}^n (2a_kb_k)^2}\\
		&= \sqrt{(\sum_{k=1}^n a_k^2)^2 + (\sum_{k=1}^n a_k^2)^2 - 2(\sum_{k=1}^n a_k^2)(\sum_{k=1}^n a_k^2) + 4\sum_{k=1}^n (a_kb_k)^2}\\	
 		&\because |\sum c_k| \le \sum|c_k| \\
 		&\therefore  (\sum_{k=1}^n a_k^2)^2 + (\sum_{k=1}^n a_k^2)^2 - 2(\sum_{k=1}^n a_k^2)(\sum_{k=1}^n a_k^2) + 4\sum_{k=1}^n (a_kb_k)^2 \le (\sum_{k=1}^n a_k^2 + \sum_{k=1}^n b_k^2)^2\\
 		&\therefore 4(\sum_{k=1}^n a_k b_k)^2 \le 4(\sum_{k=1}^n a_k^2) (\sum_{k=1}^n b_k^2)\\
 		& \text{extracting both side: } \Big|\sum_{k=1}^n a_k b_k\Big| \le \sqrt{\sum_{k=1}^n a_k^2} \sqrt{\sum_{k=1}^n b_k^2}
\end{align*}
\end{solve}

\begin{example}
	7. Suppose $ 0<x_i\le \frac{1}{2}, i=1,2,\dots,n $, then
	\begin{equation}\label{ex17FormulaByFanKy}
		\frac{\prod_{i=1}^n x_i}{(\sum_{i=1}^n x_i)^n} \le \frac{\prod_{i=1}^n(1-x_i)}{(\sum_{i=1}^n(1-x_i))^n}
	\end{equation}
\begin{proof}
	Let's prove eq\ref{ex17FormulaByFanKy} by induction method.
	\begin{equation*}
		n=2,\qquad \frac{x_1x_2}{(x_1+x_2)^2}\le \frac{(1-x_1)(1-x_2)}{((1-x_1)+(1-x_2))^2}
	\end{equation*}
	\begin{align*}
		&\frac{(x_1x_2)}{(x_1^2+2x_1x_2+x_2^2)} \le \frac{1-x_1-x_2+x_1x_2}{(1-x_1)^2+2(1-x_1)(1-x_2)+(1-x_2)^2}\\
		&\frac{(x_1+x_2)^2}{(x_1x_2)} \ge \frac{((1-x_1)(1-x_2))^2}{1-x_1-x_2+x_1x_2}\\
		&\frac{x_1}{x_2} + 2 + \frac{x_2}{x_1} \ge \frac{1-x_1}{1-x_2} + 2\frac{1-x_2}{1-x_1}\\
		&\frac{x_1}{x_2} - \frac{1-x_1}{1-x_2} \ge \frac{1-x_2}{1-x_1} - \frac{x_2}{x_1}\\
		&\frac{x_1(1-x_2) - x_2(1-x_1)}{x_2(1-x_2)} \ge \frac{x_1(1-x_2)-x_2(1-x_1)}{x_1(1-x_1)}\\
		&\frac{x_1-x_2}{x_2(1-x_2)} \ge \frac{x_1 - x_2}{x_1(1-x_1)}
	\end{align*}
$ f(x) = x-x^2, f'(x) = 1-2x>0,\text{ while }x\in(0,\frac{1}{2})  $\\
When $ x_1>x_2 , 0<x_2<x_1\le \frac{1}{2}, x_1-x_1^2 \ge x_2-x_2^2, x_1-x_2>0$\\
When $ x_1<x_2 , 0<x_1<x_2\le \frac{1}{2}, x_1-x_1^2 \le x_2-x_2^2, x_1-x_2<0$\\
\begin{equation*}
	 \frac{x_1-x_2}{x_2(1-x_2)} \ge \frac{x_1 - x_2}{x_1(1-x_1)} 
\end{equation*}
\begin{align*}
	&k=2,\qquad \frac{x_1x_2}{(x_1+x_2)^2}\le \frac{(1-x_1)(1-x_2)}{((1-x_1)+(1-x_2))^2}\\
	&k=4,\qquad \frac{x_1x_2x_3x_4}{(x_1+x_2+x_3+x_4)^2}\le 
	\frac{(1-x_1)(1-x_2)(1-x_3)(1-x_4)}{((1-x_1)+(1-x_2)+(1-x_3)+(1-x_4))^2}\\
	&\text{Use Cauchy's forward and backward method, We can prove this equation}\\
	&\text{Suppose }k=n,\frac{\prod_{i=1}^n x_i}{(\sum_{i=1}^n x_i)^2} \le 
	\frac{\prod_{i=1}^n (1-x_i)}{(\sum_{i=1}^n (1-x_i))^2}\\
	&k = n-1,\qquad \text{prove  } \frac{\prod_{i=1}^{n-1} x_i}{(\sum_{i=1}^{n-1} x_i)^2} \le 
	\frac{\prod_{i=1}^{n-1} (1-x_i)}{(\sum_{i=1}^{n-1} (1-x_i))^2}
\end{align*}
todo! need to complete!
\end{proof}
\end{example}
