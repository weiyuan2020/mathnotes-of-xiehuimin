% \documentclass[openany,twoside,scheme=chinese,fontset=none]{book}
\usepackage{ctex}
\usepackage{geometry}
\geometry{
    paperheight=260mm,
    paperwidth=185mm,
    top=25mm,
    bottom=15mm,
    left=25mm, % 左侧留 5mm 装订线距离
    right=15mm
}

\setmainfont{XITS}  % 英文字体, Times 风格

\setCJKmainfont{Source Han Serif SC}[         % 方正书宋_GBK
    BoldFont=Source Han Serif SC Bold,  % 思源宋体粗体
    ItalicFont=FZKai-Z03                % 方正楷体_GBK
    ]
\setCJKsansfont{Source Han Sans SC}[             % 方正黑体_GBK
    BoldFont=Source Han Sans SC Bold    % 思源黑体粗体
    ]
\setCJKmonofont{FZFangSong-Z02}         % 方正仿宋_GBK

\setCJKfamilyfont{zhsong}{FZShuSong-Z01}
\setCJKfamilyfont{zhxbs}{Source Han Serif SC Bold}
\setCJKfamilyfont{zhdbs}{Source Han Serif SC Heavy}
\setCJKfamilyfont{zhhei}{FZHei-B01}
\setCJKfamilyfont{zhdh}{Source Han Sans SC Bold}
\setCJKfamilyfont{zhfs}{FZFangSong-Z02}
\setCJKfamilyfont{zhkai}{FZKai-Z03}

\newcommand{\songti}{\CJKfamily{zhsong}}
\newcommand{\xbsong}{\CJKfamily{zhxbs}}
\newcommand{\dbsong}{\CJKfamily{zhdbs}}
\newcommand{\heiti}{\CJKfamily{zhhei}}
\newcommand{\dahei}{\CJKfamily{zhdh}}
\newcommand{\fangsong}{\CJKfamily{zhfs}}
\newcommand{\kaishu}{\CJKfamily{zhkai}}


\usepackage{amsmath}
\usepackage{amsthm} % amsthm 与 ntheorem 冲突
\usepackage{amssymb}
\usepackage{hyperref} % \url
\usepackage{graphicx} % \includegraphics
\usepackage{fancybox} % 使用盒子
\usepackage{xcolor} % 使用颜色 color
% \usepackage{amsfonts}  % 
\usepackage{mathrsfs}  % 花体字 mathscr

\usepackage{enumerate}  % 计数器
\usepackage{paralist}  % 计数器 允许行间公式

\usepackage{tikz,lipsum,lmodern}
\usepackage[most]{tcolorbox} % \DeclareTColorBox


\theoremstyle{plain} % default
\newtheorem{thm}{Theorem}[chapter] % 如果不采用章节号做前缀, 则不用[section]
\newtheorem{myLemma}[thm]{Lemma}
\newtheorem{mynewthm}{Theorem}[section] % 如果不采用章节号做前缀, 则不用[section]

\theoremstyle{definition} % definition

\newtheorem{myDef}[thm]{Definition}
\newtheorem{myExample}[thm]{Example}
\newtheorem{myRemark}[thm]{Remark}
\newtheorem{myProposition}[thm]{Proposition}
\newtheorem{myNotation}[thm]{Notation}

\newtheorem*{myCorollary}{Corollary}
\newtheorem{mySolve}{Solve}

\newtheorem{myExercise}{Exercise}[chapter]
% \theoremstyle{definition}
\newtheorem{definition}{Definition}[section]

% 我的笔记环境
\newcommand{\mybox}[1]{
    % \fbox{mynotes:\\#1}
    % \vskip 2.5mm
    % \fbox{\parbox{135mm}{mynotes:\\#1}}
    % \vskip 2.5mm
    \begin{tcolorbox}
        my notes:\\#1
    \end{tcolorbox}
}
\newcommand{\myproof}[1]{\begin{proof}#1\end{proof}}
\newcommand{\mycommand}[2]{\begin{#1}#2\end{#1}}
\newcommand{\mythm}[1]{\mycommand{thm}{#1}}
\newcommand{\mymyDef}[1]{\mycommand{myDef}{#1}}
\newcommand{\mymyExample}[1]{\mycommand{myExample}{#1}}
\newcommand{\mymyRemark}[1]{\mycommand{myRemark}{#1}}
\newcommand{\mymyProposition}[1]{\mycommand{myProposition}{#1}}
\newcommand{\mymyNotation}[1]{\mycommand{myNotation}{#1}}
\newcommand{\mymyCorollary}[1]{\mycommand{myCorollary}{#1}}
\newcommand{\mymySolve}[1]{\mycommand{mySolve}{#1}}
\newcommand{\mymyExercise}[1]{\mycommand{myExercise}{#1}}

%New Commands

%Special Symbols

\newcommand{\R}{{\protect\mathbb R}}
\newcommand{\Cc}{{\protect\mathbb C}}
\newcommand{\K}{{\protect\mathbb K}}
\newcommand{\N}{{\protect\mathbb N}}
\newcommand{\Q}{{\protect\mathbb Q}}
\newcommand{\Z}{{\protect\mathbb Z}}
\newcommand{\sequence}[1]{\protect\{#1\}}
\newcommand{\mybinom}[2]{\protect \binom{#1}{#2}}
% \documentclass[openany,twoside,scheme=chinese,fontset=none]{ctexbook}
\documentclass[openany,twoside,scheme=chinese,fontset=none]{book}
\usepackage{ctex}
\usepackage{geometry}
\geometry{
    paperheight=260mm,
    paperwidth=185mm,
    top=25mm,
    bottom=15mm,
    left=25mm, % 左侧留 5mm 装订线距离
    right=15mm
}

\setmainfont{XITS}  % 英文字体, Times 风格

\setCJKmainfont{Source Han Serif SC}[         % 方正书宋_GBK
    BoldFont=Source Han Serif SC Bold,  % 思源宋体粗体
    ItalicFont=FZKai-Z03                % 方正楷体_GBK
    ]
\setCJKsansfont{Source Han Sans SC}[             % 方正黑体_GBK
    BoldFont=Source Han Sans SC Bold    % 思源黑体粗体
    ]
\setCJKmonofont{FZFangSong-Z02}         % 方正仿宋_GBK

\setCJKfamilyfont{zhsong}{FZShuSong-Z01}
\setCJKfamilyfont{zhxbs}{Source Han Serif SC Bold}
\setCJKfamilyfont{zhdbs}{Source Han Serif SC Heavy}
\setCJKfamilyfont{zhhei}{FZHei-B01}
\setCJKfamilyfont{zhdh}{Source Han Sans SC Bold}
\setCJKfamilyfont{zhfs}{FZFangSong-Z02}
\setCJKfamilyfont{zhkai}{FZKai-Z03}

\newcommand{\songti}{\CJKfamily{zhsong}}
\newcommand{\xbsong}{\CJKfamily{zhxbs}}
\newcommand{\dbsong}{\CJKfamily{zhdbs}}
\newcommand{\heiti}{\CJKfamily{zhhei}}
\newcommand{\dahei}{\CJKfamily{zhdh}}
\newcommand{\fangsong}{\CJKfamily{zhfs}}
\newcommand{\kaishu}{\CJKfamily{zhkai}}


\usepackage{amsmath}
\usepackage{amsthm} % amsthm 与 ntheorem 冲突
\usepackage{amssymb}
\usepackage{hyperref} % \url
\usepackage{graphicx} % \includegraphics
\usepackage{enumerate} % 罗列专用宏包
\usepackage{fancybox} % 使用盒子
\usepackage{xcolor} % 使用颜色 color
% \usepackage{amsfonts}  % 
\usepackage{mathrsfs}  % 花体字 mathscr
\usepackage{enumerate}  % 计数器


\theoremstyle{plain} % default
\newtheorem{thm}{Theorem}[chapter] % 如果不采用章节号做前缀, 则不用[section]
\newtheorem{myLemma}[thm]{Lemma}
\newtheorem{mynewthm}{Theorem}[section] % 如果不采用章节号做前缀, 则不用[section]

\theoremstyle{definition} % definition

\newtheorem{myDef}[thm]{Definition}
\newtheorem{myExample}[thm]{Example}
\newtheorem{myRemark}[thm]{Remark}
\newtheorem{myProposition}[thm]{Proposition}
\newtheorem{myNotation}[thm]{Notation}

\newtheorem*{myCorollary}{Corollary}
\newtheorem{mySolve}{Solve}

\newtheorem{myExercise}{Exercise}[chapter]
% \theoremstyle{definition}
\newtheorem{definition}{Definition}[section]

% 我的笔记环境
\newcommand{\mybox}[1]{
    % \fbox{mynotes:\\#1}
    \vskip 2.5mm
    \fbox{\parbox{120mm}{mynotes:\\#1}}
    \vskip 2.5mm
}
\newcommand{\myproof}[1]{\begin{proof}#1\end{proof}}
\newcommand{\mycommand}[2]{\begin{#1}#2\end{#1}}
\newcommand{\mythm}[1]{\mycommand{thm}{#1}}
\newcommand{\mymyDef}[1]{\mycommand{myDef}{#1}}
\newcommand{\mymyExample}[1]{\mycommand{myExample}{#1}}
\newcommand{\mymyRemark}[1]{\mycommand{myRemark}{#1}}
\newcommand{\mymyProposition}[1]{\mycommand{myProposition}{#1}}
\newcommand{\mymyNotation}[1]{\mycommand{myNotation}{#1}}
\newcommand{\mymyCorollary}[1]{\mycommand{myCorollary}{#1}}
\newcommand{\mymySolve}[1]{\mycommand{mySolve}{#1}}
\newcommand{\mymyExercise}[1]{\mycommand{myExercise}{#1}}

%New Commands

%Special Symbols

\newcommand{\R}[1]{{\protect\mathbb R}^{#1}}
\newcommand{\Cc}{{\protect\mathbb C}}
\newcommand{\K}{{\protect\mathbb K}}
\newcommand{\N}{{\protect\mathbb N}}
\newcommand{\Q}{{\protect\mathbb Q}}
\newcommand{\Z}{{\protect\mathbb Z}}
\newcommand{\sequence}[1]{\protect\{#1\}}
\newcommand{\mybinom}[2]{\protect \binom{#1}{#2}}

\title{baby-rudin reading notes}
\author{weiyuan}
\date{\today}

\begin{document}
    \maketitle
    \tableofcontents
    \mainmatter
    % chap 1 the real and complex number system
\chapter{the real and complex number system}

\section{Introduction}
First we use $\sqrt{2}$ to construct real number system from integer and rational numbers.

% 1.1 Example
\begin{Example}
\begin{equation}\label{eq:1-001}
    p^2=2
\end{equation}
$p$ is not a rational number.
\end{Example}

\begin{proof}
    
% Proof: 
(反证法) 假设 $p$ 是有理数,  $\exists m,n \in \mathbf{N}$, s.t. $p=m/n$. $\gcd (m,n) = 1$.
Then \ref{eq:1-001}

\begin{equation}\label{eq:1-002}
    m^2 = 2n^2.
\end{equation}

$m$ is even, $m = 2k$.
那么有 $(2k)^2 = 2n^2$, $2k^2 = n^2$, $k$ is even, $\gcd (m,n)=2\neq 1$,
contrary to our choice of $m$ and $n$. Hence p can't be a rational number.
\end{proof}

After proving $\sqrt{2}$ isn't a rational number, rudin use $\sqrt{2}$ to divide the rationals
在证明 $\sqrt{2}$ 不是有理数后, 使用 $\sqrt{2}$ 将有理数集分成两部分.  引出了分划的概念? 

\begin{align*}
    A = \{p|p^2<2\}\\
    B = \{p|p^2>2\}
\end{align*}

$A$ \emph{contains no largest number},
$B$ \emph{contains no smallest number}.

$\forall p\in A$, $\exists q\in A$, s.t. $p<q$,
$\forall p\in B$, $\exists q\in B$, s.t. $p>q$,

$\forall p>0$

\begin{equation}\label{eq:1-003}
    q = p-\frac{p^2-2}{p+2} = \frac{2p+2}{p+2}
\end{equation}

Then 
\begin{equation}
    \label{eq:1-004}
    q^2 - 2 = \frac{2(p^2-2)}{(p+2)^2}
\end{equation}

If $p\in A$, $p^2<2$. \ref{eq:1-003} shows that $q>p$, \ref{eq:1-004} shows that $q^2<2$, $q\in A$.
If $p\in B$, $p^2>2$. \ref{eq:1-003} shows that $q<p$, \ref{eq:1-004} shows that $q^2>2$, $q\in B$.


\begin{Remark}
    
% 1.2 Remark

The purpose of the above discussion has been to show that the rational number system has certain gaps, 
in spite of the fact that between any two rationals there is another: If $r<s$ then $r<(r+s)/2<s$.
The real number system fills these gaps.
This is the principal reason for the fundamental role which it plays in analysis.
\end{Remark}

\mybox{
% mynotes:
有理数的稠密性与实数的连续性. 在分析中, 考察极限等需要的是数系的连续性, 因此需要先建立实数系. 
事实上, 我们是先有微积分, 后有实数理论的. 
三次数学危机:
无理数, 微积分基础, 集合论
实数理论是极限的基础. 
}

In order to elucidate its structure, as well as that of the complex numbers, 
we start with a brief discussion of the genral concepts of \emph{ordered set} and \emph{field}.

mynotes:
rudin引入复数的方法非常怪, 对初学者非常不友好, 过于抽象了. 
想起一个法国笑话, 问小学生$2+3$等于几, 回答 $2+3=3+2$ 加法是一个交换群(Abel 群). . . 

Here is some of the standard set-theoretic terminology taht will be used throughout this book.
接下来引入一些集合论的定义

1.3 Definitions
If $A$ is any set (whose elements(这里elements还没定义, 笑啦) may be numbers or any other objects(object指代什么? 我个人认为集合理解的难点在于集合的集合. 这一点可以引出罗素悖论)), we write $x\in A$ to indicate that $x$ is a member (or an element) of $A$.

If $x$ is not a member of $A$, we write: $x\notin A$.

\emph{empty set} $\varnothing$ contains no element, If a set has at least one element, it is called \emph{nonempty}.

$A,B$ are sets, $\forall x\in A$, $x\in B$, we say that $A$ is a \emph{subset} of $B$, $A\subset B$ or $B\supset A$. If $\exists x\in B$, $x\notin A$, A is a \emph{proper subset} of $B$, $A \subsetneqq B$.
Note that $A\subset A$ for every set $A$.

(Bernstein) If $A\subset B$ and $B\subset A$, we write $A = B$. Otherwise $A\neq B$.

mynotes:
这条性质在证明集合相等时很常用

1.4 Definitions
Throughout Chap. 1, the set of all rational numbers will be denoted by $\mathbb{Q}$.

有理数集$\mathbb{Q}$

\section{Ordered sets}
有序集

1.5 Definitions
Let $S$ be a set. An \emph{order} on $S$ is a relation, denoted by $<$, with the following two properties:

(i) If $x\in S$ and $y\in S$ then one and only one of the statements
\begin{equation*}
    x<y, \quad
    x=y, \quad
    y<x
\end{equation*}
The statement $x<y$ may be read as 
$x$ is less than $y$, or 
$x$ is smaller than $y$, or
$x$ precedes $y$.
(It's often convenient to write $y>x$ in place of $x<y$)
(less-great, smaller-bigger, precedes-succeeds)
% form wiki,
% The relationship x precedes y is written $x ≺ y$. The relation x precedes or is equal to y is written x ≼ y.
% The relationship x succeeds (or follows) y is written x ≻ y. The relation x succeeds or is equal to y is written x ≽ y.
% ≺ \prec 
% ≼ \preccurlyeq 
% ≻ \succ 
% ≽ \succcurlyeq  

(ii) If $x,y,z\in S$, if $x<y$ and $y<z$, then $x<z$.

偏序关系
1. 三歧性
2. 传递性

建立偏序关系后, 可以使用不等式进行分析. 在后续根据极限定义计算时, 需要大量使用不等式分析数列和函数的极限计算结果. 


$x\leq y$ indicates taht $x<y$ or $x=y$, without specofying which of these two is to hold.
In other words, $x\leq y$ is the negation of $x>y$.

1.6 Definitions
An \emph{ordered set} is a set $S$ in which an order is defined.

For example, $\mathbb{Q}$ is an ordered set if $r<s$ is defined to mean that $s-r$ is a positive rational number.

存在偏序关系的集合称为有序集
$\mathbb{Q}, \mathbb{R}$ 均是有序集, 但$\mathbb{C}$ 不是有序集. 

1.7 Definitions (bounded above)
Suppose $S$ is an ordered set, and $E \subset S$. If there exists a
$\beta \in S$ such that $x \leq \beta$ for every $x \in E$, we say that $E$ is \emph{bounded above}, and call
$\beta$ an \emph{upper bound} of $E$.

Lower bounds are defined in the same way (with $\geq$ in place of $\leq$).


1.8 Definitions (least upper bound)
Suppose $S$ is an ordered set, $E \subset S$, and $E$ is bounded above.
Suppose there exists an $a\alpha \in S$ with the following properties:

(i) $\alpha$ is an upper bound of $E$.
(ii) If $\gamma <\alpha$ then $\gamma$ is not an upper bound of $E$.

Then $\alpha$ is called the \emph{least upper bound} of $E$ [that there is at most one such
$\alpha$ is clear from (ii)] or the \emph{supremum} of $E$, and we write

\begin{equation*}
    \alpha = \sup E.
\end{equation*}

The \emph{greatest lower bound}, or \emph{infimum}, of a set $E$ which is bounded below
is defined in the same manner: The statement

\begin{equation*}
    \alpha = \inf E
\end{equation*}

means that $\alpha$ is a lower bound of $E$ and that no $\beta$ with $\beta > \alpha$ is a lower bound
of $E$.

从上界引出最小上界, 没有直接定义最大下界, 而是使用对称定义引出. 
从最小上界引出的最小上界性质更为常用. Dedekind分划

1.9 example
(a) Consider the set $A, B$
\begin{equation*}
    A = \{p|p^2 < 2\},\quad
    B = \{p|p^2 > 2\}.
\end{equation*}
$A$ has no least upper bound in $\mathbb{Q}$.
$B$ has no great lower bound in $\mathbb{Q}$.

(b) If $\alpha = \sup E$ exists, $\alpha\in E$ or $\alpha \notin E$.
\begin{align*}
    E_1 = \{r |r\in Q, r < 0\}\\
    E_2 = \{r |r\in Q, r \leq 0\}
\end{align*}
\begin{equation*}
    \sup E_1 = \sup E_2 = 0,
\end{equation*}
and $0\not\in E_1$, $0\in E_2$.

(c) $E = \{1/n | n = 1,2,3,...\}$. Then $\sup E = 1$, which is in $E$, and $\inf E = 0$, which is not in $E$.

1.10 Definitions (least-upper-bound property)(important!!)
An ordered set $S$ is said to have the \emph{least-upper-bound property}
if the following is true:

If $E \subset S$, $E$ is not empty, and $E$ is bounded above, then $\sup E$ exists in $S$.

Example 1.9(a) shows that $\mathbb{Q}$ does not have the least-upper-bound property.

We shall now show that there is a close relation between greatest lower
bounds and least upper bounds, and that every ordered set with the least-upper-bound property also has the greatest-lower-bound property.


1.11 Theorem 
Suppose $S$ is an ordered set with the least-upper-bound property,
$B \subset S$, $B$ is not empty, and $B$ is bounded below. Let $L$ be the set of all lower
bounds of $B$. Then

\begin{equation*}
    \alpha = \sup L
\end{equation*}

exists in $S$, and $\alpha = \inf B$.

In particular, $\inf B$ exists in $S$.

Proof 
Since $B$ is bounded below, $L$ is not empty. Since $L$ consists of
exactly those $y \in S$ which satisfy the inequality $y \leq x$ for every $x \in B$, we
see that \emph{every} $x \in B$ \emph{is an upper bound of} $L$. Thus $L$ is bounded above.
Our hypothesis about $S$ implies therefore that $L$ has a supremum in $S$;
call it $\alpha$.

If $\gamma <\alpha$ then (see Definition 1.8) $\gamma$ is not an upper bound of $L$,
hence $\gamma \not\in B$. It follows that $\alpha \leq x$ for every $x \in B$. Thus $\alpha \in L$.

If $\alpha < \beta$ then $\beta \not\in L$, since $\alpha$ is an upper bound of $L$.

We have shown that $\alpha \in L$ but $\beta \not\in  L$ if $\beta>\alpha$. In other words, $\alpha$
is a lower bound of $B$, but $\alpha$ is not if $\beta > \alpha$. This means that $\alpha = \inf B$.

mynotes
这个证明第一次看比较难理清
我试着用自己的话重写梳理一下:
已知条件
$S$, ordered set + least-upper-bound property.
$B\in S$, $B\neq \varnothing $, $B$ is bounded below.
$L$ is the set of all lower bounds of $B$.

$\exists \alpha\in S$, $\alpha = \sup L$, and $\alpha = \inf B$.

proof:
思路 由最小上界 $\rightarrow $ 最大下界

% \begin{align*}
%     \text{最小上界}  & \rightarrow  &\text{最大下界} \\
%     \downarrow      &               &\uparrow \\
%     L\text{最小上界}  & \rightarrow  &B\text{最大下界} \\
% \end{align*}

$L = \{y| y\in S; \forall x\in B, y\leq x\}$
    关于 $L$ 中有没有不在 $S$ 中的元素这一点我还没想明白. 定理中只是说 $L$ 是 $B$ 的下界组成的. $B$ 是 $S$ 的子集, 但 $B$ 的下界不一定全在 $S$ 中. 

$L$ 由 $B$ 在 $S$ 中的全部下界组成

$\forall x\in B$, $x$ 为 $L$ 的上界. $L\subset S$.
$S$ 有最小上界性质,
$\therefore \exists \alpha\in S$, $\alpha = \sup L$.

$\forall \gamma <\alpha$ 由 $\alpha = \sup L$ 的定义 (Definitions 1.8)
$\gamma$ 不是 $L$ 的上界.

$\forall x \in B$, $x$ 为 $L$ 的上界, $x \geq \alpha$. $\therefore \alpha \in L$.

$\alpha < \beta$, $\alpha = \sup L$. $\therefore \beta \not\in L$.
$L$ 由 $B$ 在 $S$ 中的全部下界组成, $\beta \not\in L$.
$\beta$ 不是 $B$ 的下界.

$\therefore \alpha = \inf B$, $\inf B\in S$.


\section{fields}
域, 交换除环 $<\mathbb{R},+,\times>$ 
$<\mathbb{R},+>$, $<\mathbb{R}\backslash\{0\},\times>$
都是交换群, 且满足分配律. 
则 $<\mathbb{R},+,\times>$ 是域. 

axiom 公理

(A) Axioms for addition

(Al) If $x\in F$  and $y \in F$, then their sum \(x + y\) is in F.

(A2) Addition is commutative: \(x + y=y+ x\) for all \(x, y \in F\).

(A3) Addition is associative: \((x+ y)+z = x + (y+ z)\) for all \(x, y, z \in F\).

(A4) $F$ contains an element $0$ such that $0 + x = x$ for every $x \in F$.

(A5) To every $x\in F$ corresponds an element $-x\in F$ such that

\begin{equation*}
    x+(-x)=0.
\end{equation*}

(M) Axioms for multiplication

(M1) If $x\in F$ and $x\in F$, then their product $xy$ is in $F$.

(M2) Multiplication is commutative: $xy = yx$ for all $x, y \in  F$.

(M3) Multiplication is associative: $(xy)z = x(yz)$ for all $x, y, z \in  F$.

(M4) $F$ contains an element $1 \neq 0$ such that $1x = x$ for every $x \in F$.

(M5) If $x \in F$ and $x \neq 0$ then there exists an element $1/x \in F$ such that

\begin{equation*}
    x\cdot(1/x)=1.
\end{equation*}
% 6 PRINCIPLES OF MATHEMATICAL ANALYSIS

(D) The distributive law

\begin{equation*}
    x(y+z)=xy+ xz
\end{equation*}

holds for all $x, y, z \in F$.

1.13 Remark

(a) Our usual writes (in any filed)

只定义了加法和乘法, 使用逆元分别表示减法和除法.
$x-y = x+(-y)$, $x/y=x\cdot (1/y)$.

(b) The field axioms clearly hold in $\mathbb{Q}$, the set of all rational numbers, if
addition and multiplication have their customary meaning. Thus $\mathbb{Q}$ is a
field.

全体有理数的集合是一个域.

(c) Although it is not our purpose to study fields (or any other algebraic
structures) in detail, it is worthwhile to prove that some familiar properties
of $\mathbb{Q}$ are consequences of the field axioms; once we do this, we will \underline{not
need to do it} again for the real numbers and for the complex numbers.

1.14 Proposition

The axioms for addition imply the following statements.

(a) If $x+y=x+z$ then $y=z$.
(b) If $x+y=x$ then $y=0$.
(c) If $x+y=0$ then $y= -x$.
(d) $-(-x)=x$.

Statement (a) is a cancellation law. Note that (b) asserts the uniqueness
of the element whose existence is assumed in (A4), and that (c) does the same
for (A5).

mynotes
what is the difference between axiom and proposition?
An axiom is a proposition regarded as self-evidently true without proof. The word "axiom" is a slightly archaic synonym for postulate. Compare conjecture or hypothesis, both of which connote apparently true but not self-evident statements.
A proposition is a mathematical statement such as "3 is greater than 4," "an infinite set exists," or "7 is prime." An axiom is a proposition that is assumed to be true. With sufficient information, mathematical logic can often categorize a proposition as true or false, although there are various exceptions (e.g., "This statement is false").
\url{https://www.nutritionmodels.com/terminology.html}


Proof(rudin)
If $x + y =x + z$, the axioms (A) give

\begin{align*}
    y =0+y&=(-x+x)+y=-x+(x+y)\\
    &=-x+(x+z)=(-x+x)+z=0+z=z
\end{align*}

This proves (a). Take $z = 0$ in (a) to obtain (b). Take $z= -x$ in (a) to
obtain (c).
Since $-x + x = 0$, (c) (with $-x$ in place of $x$) gives (d).

mynotes 我自己证明上述四条性质时都是从定义开始的, 而 rudin 这里在后一步的证明中都利用了刚推导出的结论, 这一点需要借鉴.
% THE REAL AND COMPLEX NUMBER SYSTEMS 7

1.15 Proposition 
The axioms for multiplication imply the following statements.

(a) If $x\neq0$ and $xy=xz$ then $y=z$.

(b) If $x\neq0$ and $xy=x$ then $y=1$.

(c) If $x\neq0$ and $xy=1$ then $y=1/x$.

(d) If $x\neq0$ then $1/(1/x) = x$.


The proof is so similar to that of Proposition 1.14 that we omit it.

mynotes
Proof
(a) 
\begin{align*}
    y&=1\cdot y=\left(\frac{1}{x}\cdot x\right)y =\frac{1}{x}\left( xy \right)\\
    &=\frac{1}{x}(xz) =\left(\frac{1}{x}x\right)z = z
\end{align*}

    % chap02
\chapter{Basic topology}
% chap02sec01
\section{Finite, countable, and uncountable sets}

We begin this section with a definition of the \myKeywordblue{function} concept.

\mybox{
    Function

    \url{https://mathworld.wolfram.com/Function.html}


A function is a relation that uniquely associates members of one set with members of another set. 
More formally, a function from $A$ to $B$ is an object $f$ such that every $a$ in $A$ is uniquely associated with an object $f(a)$ in $B$. 
A function is therefore a many-to-one (or sometimes one-to-one) relation. 
The set $A$ of values at which a function is defined is called its domain, 
while the set $f(A)$ subset $B$ of values that the function can produce is called its range. 
Here, the set $B$ is called the codomain of $f$.

In the context of univariate, real-valued functions $f:A \subset \R\rightarrow \R$, 
the fact that domain elements are mapped to unique range elements can be expressed graphically by way of the vertical line test.

In some literature, the term 
``map''
 is synonymous with function. 
Some caution must be exhibited, however, as it is not uncommon for the term map to denote a function with some sort of unspoken regularity assumption, 
e.g., in point-set topology, where 
``map''
 sometimes refers to a function which is continuous with respect to some topology.
}

\mybox{

\begin{center}
    \begin{tikzpicture}[x=0.7pt,y=0.7pt,yscale=-1,xscale=1]
    %uncomment if require: \path (0,300); %set diagram left start at 0, and has height of 300
    
    %Shape: Axis 2D [id:dp21476035781689284] 
    \draw  (165.52,146.23) -- (314.72,146.23)(202.6,85.23) -- (202.6,213.23) (307.72,141.23) -- (314.72,146.23) -- (307.72,151.23) (197.6,92.23) -- (202.6,85.23) -- (207.6,92.23)  ;
    %Shape: Wave [id:dp7563242846602689] 
    \draw   (128.6,145.93) .. controls (132.68,149.37) and (136.58,152.63) .. (141.1,152.63) .. controls (145.62,152.63) and (149.52,149.37) .. (153.6,145.93) .. controls (157.68,142.5) and (161.58,139.23) .. (166.1,139.23) .. controls (170.62,139.23) and (174.52,142.5) .. (178.6,145.93) .. controls (182.68,149.37) and (186.58,152.63) .. (191.1,152.63) .. controls (195.62,152.63) and (199.52,149.37) .. (203.6,145.93) .. controls (207.68,142.5) and (211.58,139.23) .. (216.1,139.23) .. controls (220.62,139.23) and (224.52,142.5) .. (228.6,145.93) .. controls (232.68,149.37) and (236.58,152.63) .. (241.1,152.63) .. controls (245.62,152.63) and (249.52,149.37) .. (253.6,145.93) .. controls (257.68,142.5) and (261.58,139.23) .. (266.1,139.23) .. controls (270.62,139.23) and (274.52,142.5) .. (278.6,145.93) .. controls (282.68,149.37) and (286.58,152.63) .. (291.1,152.63) .. controls (294.73,152.63) and (297.96,150.53) .. (301.2,147.92) ;
    %Shape: Parabola [id:dp3307772220116144] 
    \draw   (167.6,53.23) .. controls (190.93,177.23) and (214.27,177.23) .. (237.6,53.23) ;
    %Straight Lines [id:da017349586183422416] 
    \draw    (148.3,200.53) -- (256.9,91.93) ;
    
\end{tikzpicture}
\end{center}

Examples of functions over the reals $\R$ include $\sin x$ (many-to-one), $x$ (one-to-one), $x^2$ (two-to-one except for the single point $x=0$), etc.

Unfortunately, the term ``function'' is also used to refer to relations that map single points in the domain to possibly multiple points in the range. 
These ``functions'' are called multivalued functions (or multiple-valued functions), and arise prominently in the theory of complex functions, 
where the presence of multiple values engenders the use of so-called branch cuts.

Several notations are commonly used to represent (non-multivalued) functions. 
The most rigorous notation is $f:x\rightarrow f(x)$, which specifies that f is function acting upon a single number $x$ (i.e., f is a univariate, or one-variable, function) and returning a value $f(x)$. 
To be even more precise, a notation like ``$f:R\rightarrow R$, where $f(x)=x^2$''
 is sometimes used to explicitly specify the domain and codomain of the function. 
The slightly different 
``maps to''
 notation $f:x|\rightarrow f(x)$ is sometimes also used when the function is explicitly considered as a 
``map''.


Generally speaking, the symbol $f$ refers to the function itself, while $f(x)$ refers to the value taken by the function when evaluated at a point $x$. 
However, especially in more introductory texts, the notation $f(x)$ is commonly used to refer to the function $f$ itself (as opposed to the value of the function evaluated at $x$). 
In this context, the argument $x$ is considered to be a dummy variable whose presence indicates that the function $f$ takes a single argument (as opposed to $f(x,y)$, etc.). 
While this notation is deprecated by professional mathematicians, it is the more familiar one for most nonprofessionals. 
Therefore, unless indicated otherwise by context, the notation $f(x)$ is taken in this work to be a shorthand for the more rigorous $f:x\rightarrow f(x)$.
}

\begin{mydef}
    \label{mydef:2.1}
    Consider two sets $A$ and $B$, whose elements may be any objects whatsoever, 
    and suppose that with each element $x$ of $A$ there is associated, 
    in some manner, an element of $B$, which we denote by $f(x)$. 
    Then $f$ is said to be a \myKeywordblue{function} from $A$ to $B$ (or a \myKeywordblue{mapping} of $A$ into $B$). 
    The set $A$ is called the \myKeywordblue{domain} of $f$ (we also say $f$ is defined on $A$), 
    and the elements $f(x)$ are called the \myKeywordblue{values} of $f$.
    The set of all values of $f$ is called the \myKeywordblue{range} of $f$.
\end{mydef}
\mybox{\myKeywordblue{Codomain}:  A set within which the values of a function lie (as opposed to the range, which is the set of values that the function actually takes). 

\myKeywordblue{Range}:   If $f:D\rightarrow Y$ is a map (a.k.a. function, transformation, etc.) over a domain $D$, 
then the range of $f$, also called the image of $D$ under $f$, 
is defined as the set of all values that $f$ can take as its argument varies over $D$, i.e.,
\begin{equation*}
    \operatorname{Range}(f)=f(D)={f(\mathbf{X}):\mathbf{X} \in D}.
\end{equation*}

Note that among mathematicians, the word 
``image''
 is used more commonly than 
``range.''


The range is a subset of $Y$ and does not have to be all of $Y$.

Unfortunately, term 
``range''
 is often used to mean domain--its precise opposite--in probability theory, with Feller (1968, p.200) and Evans et al. (2000, p.5) calling the set of values that a variate $X$ can assume (i.e., the set of values $x$ that a probability density function $P(x)$ is defined over) the 
``range''
, denoted by $R_X$ (Evans et al. 2000, p.5).

Even worse, statistics most commonly uses 
``range''
 to refer to the completely different statistical quantity as the difference between the largest and smallest order statistics. In this work, this form of range is referred to as 
``statistical range.''
 
}

\begin{mydef}
    \label{mydef:2.2}
    Let $A$ and $B$ be two sets and let $f$ be a mapping of $A$ into $B$.
    If $E \subset A$, $f(E)$ is defined to be the set of all elements $f(x)$, for $x \in E$. We call $f(E)$ the image of $E$ under $f$. In this notation, $f(A)$ is the range of $f$. It is clear that $f(A) \subset B$. If $f(A) = B$, we say that $f$ maps $A$ \myKeywordblue{onto} $B$. (Note that, according
    to this usage, \myKeywordblue{onto} is more specific than \myKeywordblue{into}.)
    \mybox{onto 满射? into 映射?} 

    If $E \subset B$, $f^{-1}(E)$ denotes the set of all $x \in A$ such that $f(x)\in E$. We call $f^{-1}(E)$ the \myKeywordblue{inverse image} of $E$ under $f$. If $y \in B$, $f^{-1}(y)$ is the set of all $x \in A$ such that $f(x) =y$. If, for each $y\in B$, $f^{-1}(y)$ consists of at most one element of $A$, then $f$ is said to be a 1-1 (\myKeywordblue{one-to-one}) mapping of $A$ into $B$. This may also be expressed as follows: $f$ is a 1-1 mapping of $A$ into $B$ provided that $f(x_1) \neq f(x_2)$ whenever $x_1 \neq x_2$, $x_1 \in A$, $x_2 \in A$.

    (The notation $x_1 \neq x_2$, means that $x_1$ and $x_2$ are distinct elements; otherwise we write $x_1 = x_2$.)
\end{mydef}

\begin{mydef}
    \label{mydef:2.3}
    If there exists a 1-1 mapping of $A$ \myKeywordblue{onto} $B$, we say that $A$ and $B$ can be putin 1-1 correspondence, or that $A$ and $B$ have the same cardinal number, or, briefly, that $A$ and $B$ are equivalent, and we write $A\sim B$. This relation
    clearly has the following properties :

    It is reflexive: $A\sim A$.

    It is symmetric: If $A\sim B$, then $B\sim A$.

    It is transitive: If $A\sim B$ and $B\sim C$, then $A\sim C$.

    Any relation with these three properties is called an equivalence relation.    
\end{mydef}
\mybox{等价关系:
reflexive   自反性,
symmetric   对称性,
transitive  传递性.

集合等势是一种等价关系, 其满足自反性, 对称性, 传递性.}

\begin{mydef}
    \label{mydef:2.4}
    $\forall n\in \mathbb{N}^+$, $J_n = \{1,2,...,n\}$, $J = \{1,2,...,n,...\}$, (set consisting of all positive integers).

    $A$ is finite, $A\sim J_n$ for some n,

    $A = \varnothing$. empty set is also considered to be finite.

    $A$ is infinite, $A$ is not finite.

    $A$ is countable, $A \sim J$
    
    $A$ is uncountable. $A$ is neither finite nor countable.

    countable set and finite set are called at most countable.
\end{mydef}

\mybox{
    \begin{equation*}
        \left\{
        \begin{array}{lll}
        finite & A\sim J_n\\
        infinite &\left\{
            \begin{array}{ll}
                countable& A\sim J\\
                uncountable& \\
            \end{array}
        \right.
        \end{array}
        \right.
    \end{equation*}
    % todo 添加tikz注释
}

countable sets, enumerable, denumerable.

$A, B \in$ finite set\\
$A\sim B$ $\Longleftrightarrow$ $A, B$ contains same number of elements

$A, B \in$ infinite set\\
same number or elements? vague\\
1-1 correspondence. retains its clarity.

\begin{myExample}
    $f:J\rightarrow A$
    \begin{equation*}
        f(n) = \left\{
            \begin{array}{ll}
                \cfrac{n}{2} & (n \text{even})\\
                -\cfrac{n-1}{2} & (n \text{odd})
            \end{array}
        \right.
    \end{equation*}
\end{myExample}
\mybox{$f(n)=(-1)^n\left\lfloor \cfrac{n}{2} \right\rfloor $}

\begin{myRemark}
    a finite set cannot be equivalent to one of its proper subsets, but it's possible for infinite sets.
\end{myRemark}

$J = 1,2,3,4,...$, $A = 0,1,-1,2,-2,...$,$J, A$are infinite sets, $J \subset A$.\\
but there exist a function $f:J\rightarrow A$, $J \sim A$

\begin{mydef}
    \label{mydef:2.7}
    $f(x)$, $x\in J = \mathbb{N}^+$.\\
    $\{x_n\}$, $x_1,x_2,x_3,...$\\
    $x_n$, terms of the sequence.\\
    $\forall n\in J$, $x_n\in A$, $\{x_n\}$ is a sequence in $A$, or a sequence of elements of $A$.
\end{mydef}

every countable set is range of a sequence of distinct terms.
the elements of any countable set can be ``arranged in a sequence''.
replace $J(\mathbb{N}^+)$ by $\mathbb{N} = \{x|, x\in Z,x \geq 0\}$, start with $0$ rather than $1$.

\begin{thm}
    \label{thm:2.8}
    Every infinite subset of a countable set $A$ is countable
\end{thm}

$E\subset A$. $E$ is infinite. 
To prove $E$ is countable, we need a 1-1 correspondation of $J$ to $E$, $f:J\rightarrow E$.
\mybox{
    my first guess is $A$ is a countable set, $A\sim J$ (by def).
    $\exists$ 1-1 mapping $g:$ $J$ onto $A$.
    $x\in J$, $g(x)\in A$.
    $E\subset A$, $\exists  g(x)\in E$.
    $g(x_i)\in E$, $x_i\in J$, $g:J\rightarrow E$.\\
    再证 $x_i$ 不是有限的. $E$ is infinite, there exist infinite $g(x_i)\in E$. $\because g$ is a 1-1 mapping, $\{x_i\}$ is infinite. $\therefore J\sim E$.
}

\begin{proof}
    Suppose $E\subset A$, $E$ is infinite. 
    arrange the elements $x$ of $A$ in a sequence $\{x_n\}$ of a distinct elements. Construct a sequence $n_k$ as follows.\\
    Let $n_1$ be the smallest positive int, s.t. $x_{n_1}\in E$.
    Having chosen $n_1,...n_{k-1}$,$(k=2,3,...)$, let $n_k$ be the smallest integer greater than $n_{k-1}$, s.t. $x_{n_k} \in E$.\\
    Putting $f(k) = x_{n_k}$, $f:J\rightarrow E$ is a 1-1 mapping.
\end{proof}

Countable sets represent the ``smallest'' infinity.

No uncountable set ca be a subset of a countable set.

\mybox{
    rudin 这里尝试区分实无穷与浅无穷, 
    使用集合的势来说明更为具体, 全体整数组成的集合为 ``最小'' 的无穷大, 
    其势为$\aleph_0$, 康托尔使用一一对应关系作为无穷集合之间的等价关系.
    }

\begin{mydef}
    \label{mydef:2.9}
    $\forall \alpha\in A$, $E_\alpha \subset \Omega$, $\{E_\alpha\}$ debites elements of $E_\alpha$. 
    collection of sets (or family of sets)  
    \mybox{sets of sets sounds strange}  
    union
    \begin{equation}
        \label{eq:2.1}
        S = \bigcup_{\alpha\in A} E_\alpha
    \end{equation}
    if $A$ consists of the integers $1,2,...,n$.
    \begin{equation}
        \label{eq:2.2}
        S = \bigcup_{m=1}^n E_m
    \end{equation}
    \begin{equation}
        \label{eq:2.3}
        S = E_1 \bigcup E_2 \bigcup \cdots \bigcup E_n.
    \end{equation}
    if $A$ is the set of all positive integers.
    \begin{equation}
        \label{eq:2.4}
        S = \bigcup_{m=1}^{\infty} E_m.
    \end{equation}
    intersection
    \begin{equation}
        \label{eq:2.5}
        P = \bigcap_{\alpha\in A} E_\alpha
    \end{equation}
    \begin{equation}
        \label{eq:2.6}
        S = \bigcap_{m=1}^n E_m = E_1 \cap E_2 \cap \cdots \cap E_n.
    \end{equation}
    \begin{equation}
        \label{eq:2.7}
        S = \bigcap_{m=1}^{\infty} E_m.
    \end{equation}

    $A$ and $B$ intersect if $A\bigcap B$ is not empty, otherwise they are disjoint.
\end{mydef}

\begin{myExample}
    some example of set relation
\end{myExample}

\begin{myRemark}
    Many properties of unions and intersections are quite similar to those of sums and products; in fact, the words sum and product were sometimes used in this connection, and the symbols $\sum$ and $\prod$ were written in place of $\bigcup$ and $\bigcap$.
\end{myRemark}

The commutative and associative laws are trivial:
\begin{align}
        A \cup B &= B \cup A; &
        A \cap B &= B \cap A \label{eq:2.8} \\
        \left(A \cup B\right) \cup C &= A \cup \left(B \cup C\right); &
        \left(A \cap B\right) \cap C &= A \cap \left(B \cap C\right);\label{eq:2.9}
\end{align}

Thus the omission of parentheses in \ref{eq:2.3} and \ref{eq:2.6} is justified.

The distributive law also holds:
\begin{equation}
    \label{eq:2.10}
    A \cap \left( B \cup C\right) = 
    \left(A \cap B\right) \cup \left(A \cap C\right).
\end{equation}
To prove this, let the left and right members of \ref{eq:2.10} be denoted by $E$ and $F$, respectively.

Suppose $x \in E$. Then $x \in A$ and $x \in B \cup C$, that is, $x \in B$ or$ x \in C$ (possibly both). Hence $x \in A\cap B$ or $x \in A\cap C$, so that $x \in F$. Thus $E \subset F$.

Next, suppose $x \in F$. Then $x \in A\cap B$ or $x \in A\cap C$. That is, $x \in A$, and $x \in B\cup C$. Hence $x \in A\cap \left(B \cup C\right)$, so that $F \subset E$.

It follows that $E = F$.

We list a few more relations which are easily verified:

\begin{align}
    A \subset A \cup B, \label{eq:2.11}\\
    A \cap B \subset B, \label{eq:2.12}
\end{align}

If $0$ denotes the empty set, then
\begin{equation}
    \begin{array}{cc}
        A \cup 0 = A, & A \cap 0 = 0.
    \end{array}
\end{equation}
If $A \subset B$, then
\begin{equation}
    \begin{array}{cc}
        A \bigcup B = B, & A \bigcap B = A.
    \end{array}
\end{equation}

\mybox{现在一般使用 $\varnothing$ 指代空集}


\begin{thm}
    \label{thm:2.12}
    Let $\{E_n\}, n=1,2,3,...,$ be a sequence of countable sets, and put
    \begin{equation}
        \label{eq:2.15}
        S = \bigcup_{n=1}^{\infty} E_n.
    \end{equation}
    Then S is countable.
\end{thm}
\mybox{
    将 $E_n$ 按顺序排成一张表格, 按反对角线重新排列成新的序列, 
    得到 $T$, $S\sim T$.
    $S$ is at most countable.
    同时存在无限集合(infinite set) $E_1$, $E_1 \subset S$, 
    $S$ is countable.
}

\begin{proof}
    Let every set $E_n$ be arranged in a sequence $\sequence{x_{nk}}$, 
    $k = 1,2,3,\dots$,
    and consider the infinite array
    \begin{equation}
        \label{eq:2.16}
        \begin{array}{ccccc}
            x_{11} & x_{12} & x_{13} & x_{14} & \cdots \\  
            x_{21} & x_{22} & x_{23} & x_{24} & \cdots \\  
            x_{31} & x_{32} & x_{33} & x_{34} & \cdots \\  
            x_{41} & x_{42} & x_{43} & x_{44} & \cdots \\  
            \cdots & \cdots & \cdots & \cdots & \cdots \\
        \end{array}
    \end{equation}
    in which the elements of $E_n$ form the $n$th row. 
    The array contains all elements of $S$.
    As indicated by the arrows, 
    these elements can be arranged in a sequence
    \begin{equation}
        \label{eq:2.17}
        x_{11};
        x_{21}, x_{12};
        x_{31}, x_{22}, x_{13};
        x_{41}, x_{32}, x_{23}, x_{14};
        \cdots
    \end{equation}
    
    If any two of the sets En have elements in common, 
    these will appear more than once in (\ref{eq:2.17}). 
    Hence there is a subset $T$ of the set of all positive integers 
    such that $S \sim T$, 
    which shows that $S$ is at most countable (Theorem \ref{thm:2.8}). 
    Since $E_1 \subset S$, and $E_1$ is infinite, 
    $S$ is infinite, and thus countable.
\end{proof}

\begin{myCorollary*}
    Suppose $A$ is at most countable, and, for every $\alpha \in A, B$, is at most countable. Put
    \begin{equation*}
        T = \bigcup_{\alpha\in A} B_\alpha.
    \end{equation*}
    Then T is at most countable.
\end{myCorollary*}

For $T$ is equivalent to a subset of \ref{eq:2.15}.

\begin{thm}
    \label{thm:2.13}
    Theorem Let $A$ be a countable set, and let $B_n$ be the set of all $n$-tuples $(a_1, ...,a_n)$, where $a_k \in  A (k=1,...,n)$, and the elements $a_1, ...,a_n$ need not be distinct. Then $B_n$ is countable.
\end{thm}

\begin{proof}
    That $B_1$ is countable is evident, since $B_1 = A$. 
    Suppose $B_{n-1}$ is countable $(n = 2, 3, 4, ... )$. 
    The elements of $B_n$ are of the from
    \begin{equation}
        \label{eq:2.18}
        (b,a)
        \quad
        (b \in B_{n-1},a \in A).
    \end{equation}
    For every fixed $b$, the set of pairs $(b, a)$ is equivalent to $A$, and hence countable. 
    Thus $B_n$ is the union of a countable set of countable sets. 
    By Theorem \ref{thm:2.12}, Bn is countable.
The theorem follows by induction.
\end{proof}

\begin{myCorollary*}
    The set of all rational numbers is countable.
\end{myCorollary*}

\begin{proof}
    We apply Theorem \ref{thm:2.13}, 
    with $n = 2$, noting that every rational $r$ is of the form $b / a$, 
    where $a$ and $b$ are integers. 
    The set of pairs $(a, b)$, 
    and therefore the set of fractions $b / a$, is countable.
\end{proof}

In fact, even the set of all algebraic numbers is countable (see Exercise 2).

That not all infinite sets are, however, countable, is shown by the next
theorem.

\begin{thm}
    \label{thm:2.14}
    Theorem Let $A$ be the set of all sequences whose elements are the digits $0$ and $1$. This set $A$ is uncountable. 
\end{thm}

The elements of $A$ are sequences like $1, 0, 0, 1, 0, 1, 1, 1, ... .$

\begin{proof}
    Let $E$ be a countable subset of $A$, 
    and let $E$ consist of the sequences $s_1, s_2 , s_3 , ...$. 
    We construct a sequences as follows. 
    If the $n$th digit in $s_n$ is $1$, 
    we let the $n$th digit of $s$ be $0$, and vice versa. 
    Then the sequence $s$ differs from every member of $E$ in at least one place; hence $s \not\in E$. 
    But clearly $s \in A$, so that $E$ is a proper subset of $A$.

    We have shown that 
    every countable subset of $A$ is a proper subset of $A$. 
    It follows that $A$ is uncountable 
    (for otherwise $A$ would be a proper subset of $A$, which is absurd).
\end{proof}

The idea of the above proof was first used by Cantor, 
and is called Cantor's diagonal process; 
for, if the sequences $s_1, s_2 , s_3 ,\dots$ are placed in an array like (\ref{eq:2.16}), 
it is the elements on the diagonal which are involved in the construction of the new sequence.

Readers who are familiar with the binary representation of the real numbers (base 2 instead of 10) will notice that 
Theorem \ref{thm:2.14} implies that the set of all real numbers is uncountable. 
We shall give a second proof of this fact in Theorem \ref{thm:2.43}.
    % chap03.tex
\chapter{Numerical sequences and series}
% chap03sec01.tex
\section{Convergent sequences}

\begin{myDef}\label{myDef:3.1 converge}
    A sequences 
    $\{p_n\}$ 
    in metric space $X$ is said to converge if there is a point $p \in X$ with the following property:
    
    For every $\varepsilon >0$ there is an integer $N$ s.t. $n \geq N$ implies that $d(p_n, p) < \varepsilon$. (Here $d$ denotes the distance in $X$.)

    In this case we also say that $\{p_n\}$ converges to $p$, or that $p$ is the limit of $\{p_n\}$. [see Th 3.2(b)], and we write $p_n \rightarrow p$, or

    \begin{equation*}
        \lim_{n \to \infty} p_n = p.
    \end{equation*}

    if $\{p_n\}$ does not converge, it is said to diverge.
\end{myDef}

our definition of ``convergent sequence'' depends not only on $\{p_n\}$ but also on $X$. For instance, the sequence $\{1/n\}$ converges in $\mathbb{R}^1$(to $0$), but fails to converge in the set of all positive real numbers [with $d(x,y) = |x-y|$]. 
In cases of possible ambiguity, we can be more
precise and specify ``convergent in $X$'' rather than ``convergent''.

we recall that the set of all points $p_n (n=1,2, 3,...)$ is the range of 
$\{p_n\}$.
The range of a sequence may be a finite set, or it may be infinite. The sequence
$\{p_n\}$ is said to be bounded if its range is bounded.

As examples, consider the following sequences of complex numbers
(that is, $X = \mathbb{R}^2$):

\begin{enumerate}[(a)]
    \item If $s_n=1/n$, then $\lim_{n \to \infty} s_n = 0$; the range is infinite, and the sequence is bounded.
    \item If $s_n=n^2$ the sequence $\{s_n\}$ is unbounded, is divergent, and has infinite range.
    \item If $s_n = 1+[(- 1)^n/n]$, the sequence $\{s_n\}$ converges to $1$, is bounded, and has infinite range.
    \item If $s_n =i^n$ the sequence $\{s_n\}$ is divergent, is bounded, and has finite range.
    \item If $s_n = 1(n=1,2,3,...)$, then $\{s_n\}$ converges to $1$, is bounded, and has finite range.
\end{enumerate}

% We now summarize some important properties of convergent sequences in metric spaces.

\begin{thm}\label{thm:3.2 convergence sequence in metric space}
    Let$\{p_n\}$ be a sequence in a metric space $X$.
    \begin{enumerate}[(a)]
        \item $\{p_n\}$ converges to $p \in X$ if and only if every neighborhood of $p$ contains $p_n$ for all but finitely many $n$.
        \item If $p\in X$, $p^\prime \in X$, and if $\{p_n\}$ converges to $p$ and to $p'$, then $p^\prime =p$.
        \item If $\{p_n\}$ converges, then $\{p_n\}$ is bounded.
        \item If $E \subset X$ and if $p$ is a limit point of $E$, then there is a sequence$\{p_n\}$ in $E$ such that $p = \lim_{n \to \infty} p_n$.
    \end{enumerate}
\end{thm}

\begin{proof}
    (d) For each positive integer $n$, there is a point $p_n \in E$ such that $d(p_n,p) <1/n$. Given $\varepsilon > 0$, choose $N$ so that $N \varepsilon >1$. If $n>N$, it follows that $d(p_n, p) <\varepsilon$. Hence $p_n \rightarrow p$.
\end{proof}

\begin{thm}\label{thm:3.3}
    Suppose $\{s_n\}, \{t_n\}$ are complex sequences, and 
    $\lim_{n \to \infty} s_n = s$,
    $\lim_{n \to \infty} t_n = t$.
    Then
    \begin{enumerate}[(a)]
        \item $\lim_{n \to \infty} (s_n + t_n) = s + t$;
        \item $\lim_{n \to \infty} c s_n = cs$, $\lim_{n \to \infty} (c + s_n) = c + s$, for any number $c$;
        \item $\lim_{n \to \infty} s_n t_n = st$;
        \item $\lim_{n \to \infty} \frac{1}{s_n} = \frac{1}{s}$, provided $s_n \neq 0(n = 1,2,3,\dots)$, and $s \neq 0$.
    \end{enumerate}
\end{thm}

\myproof{
    \begin{equation}
        \label{eq:3.1}
        s_n t_n - st = (s_n - s)(t_n - t) + s(t_n - t) + t(s_n - s).
    \end{equation}
}

\begin{thm} As:\\
    \label{thm:3.4}
    (a) Suppose $\mathbf{x}_n \in R^k (n = 1,2,3,\dots)$ and
    \begin{equation*}
        \mathbf{x_n} = (
            \alpha_{1,n},\dots
            \alpha_{k,n}
        ).
    \end{equation*}
    Then $\{\mathbf{x}_n\}$ converges to $\mathbf{x} = (\alpha_1, \dots, \alpha_k)$ if and only if
    \begin{equation}
        \lim_{n \to \infty} \alpha_{j,n} = \alpha_j \qquad (1\leq j\leq k).
    \end{equation}

    (b) Suppose $\{\mathbf{x}_n\}$, $\{\mathbf{y}_n\}$ are sequences in $\mathbb{R}^k$, $\{\beta_n\}$ is a sequence of real numbers, and 
    $\mathbf{x}_n \rightarrow \mathbf{x}$,
    $\mathbf{y}_n \rightarrow \mathbf{y}$,
    $\beta_n \rightarrow \beta$. Then
    \begin{equation*}
        \lim_{n \to \infty} (\mathbf{x_n} + \mathbf{y_n}) = \mathbf{x} + \mathbf{y}, \quad
        \lim_{n \to \infty} \mathbf{x_n} \cdot \mathbf{y_n} = \mathbf{x} \cdot \mathbf{y}, \quad
        \lim_{n \to \infty} \beta_n \mathbf{x_n} = \beta \mathbf{x}.
    \end{equation*}
\end{thm}

% chap03sec02

\section{Subsequences}
\begin{mydef}\label{def:3.5}
    Given a sequence $\{p_n\}$, consider a sequence $\{n_k\}$ of positive integers, such that $n_1 <n_2 <n_3 <....$ Then the sequence $\{p_{n_i}\}$ is called a \emph{subsequence} of $\{p_n\}$. If $\{p_{n_i}\}$ converges, its limit is called a subsequential limit of $\{p_n\}$.
\end{mydef}
It is clear that $\{p_n\}$ converges to $p$ if and only if every subsequence of $\{p_{n}\}$ converges to $p$. We leave the details of the proof to the reader.

\begin{thm}\label{thm:3.6}
    (a) If $\{p_{n}\}$ is a sequence in a compact metric space $X$, then some subsequence of $\{p_{n}\}$ converges to a point of $X$.

    (b) Every bounded sequence in $\R^k$ contains a convergent subsequence.
\end{thm}

\begin{thm}\label{thm:3.7}
    The subsequential limits of a sequence $\{p_{n}\}$  in a metric space $X$ form a closed subset of $X$.
\end{thm}


% chap03sec03
\section{Cauchy sequences}
\begin{myDef}\label{myDef:3.8}
    A sequence $\{p_n\}$ in a metric space $X$ is said to be a \emph{Cauchy sequence} if for every $\varepsilon > 0$ there is an integer $N$ such that $d(p_n, p_m) <e$ if $n \geq N$ and $m \geq N$. 
\end{myDef}

% $\R{1}$
In our discussion of Cauchy sequences, as well as in other situations
which will arise later, the following geometric concept will be useful.

\begin{myDef}\label{myDef:3.9}
    % 3.9 Definition 
    Let $E$ be a nonempty subset of a metric space $X$, and let $S$ be the set of all real numbers of the form $d(p, q)$, with $p \in E$ and $q \in E$. The sup of $S$ is called the diameter of $E$.    
\end{myDef}

If $\sequence{p_n}$ is a sequence in $X$ and if $E_N$ consists of the points $p_N, p_{N+1}, p_{N+2},\dots$, it is clear from the two preceding definitions that $\sequence{p_n}$ is a \emph{Cauchy sequence} \emph{if and only if}

\begin{equation*}
    \lim_{N \to \infty} \text{diam } E_N = 0.
\end{equation*}

\begin{thm}\label{thm:3.10}
    (a) If $\bar{E}$ is the closure of a set $E$ in a metric space $X$, then 
    \begin{equation*}
        \rm{diam }\; E = \rm{diam }\; E.
    \end{equation*}
    
    (b) If $K_n$ is a sequence of compact sets in $X$ such that $K_n \supset K_{n+1} $ $(n=1,2,3,...) $and if
    \begin{equation*}
        \lim_{n \to \infty} \rm {diam }\; K_n = 0,
    \end{equation*}
    then $\cap_1^\infty K_n$ consists of exactly one point.
\end{thm}

% https://blog.sciencenet.cn/blog-783377-668028.html
% latex斜体变正体需在代码前加 rm
% y=exp(log(x+1))
% 代码:y=exp(log(x+1))
% y=exp(log(x+1))
% 代码:\rm y=exp(log(x+1))
% y=exp(log(x+1))
% 代码:y={rm exp}({rm log}(x+1))
% {}可以控制作用域的范围

\begin{thm}\label{thm:3.11}
    (a) Inany metric space $X$, every convergent sequence is a Cauchy sequence.

    (b) If $X$ is a compact metric space and if $\sequence{p_n}$ is a Cauchy sequence in $X$, then $\sequence{p_n}$ converges to some point of $X$.

    (c) In $\R{k}$, every Cauchy sequence converges.
\end{thm}

Note: The difference between the definition of convergence and
the definition of a Cauchy sequence is that the limit is explicitly involved
in the former, but not in the latter. Thus Theorem 3.11(b) may enable us
to decide whether or not a given sequence converges without knowledge
of the limit to which it may converge.

The fact (contained in Theorem 3.11) that a sequence converges in
$\R{k}$ if and only if it is a Cauchy sequence is usually called the 
\emph{Cauchy criterion} for convergence.

\begin{myDef}\label{myDef:3.12}
    A metric space in which every Cauchy sequence converges is
    said to be \emph{complete}.
\end{myDef}


Thus Theorem 3.11 says that \emph{all compact metric spaces and all Euclidean spaces are complete}. Theorem 3.11 implies also that every closed subset $E$ of a complete metric space $X$ is complete. (Every Cauchy sequence in $E$ is a Cauchy sequence in $X$. hence it converges to some $p \in X$, and actually $p \in E$ since $E$ is closed.) An example of a metric space which is not complete is the space of all
rational numbers, with $d(x, y) = |x - y|$.

Theorem 3.2(c) and example (d) of Definition 3.1 show that convergent sequences are bounded, but that bounded sequences in $\R{k}$ need not converge. However, there is one important case in which convergence is equivalent to boundedness; this happens for monotonic sequences in $\R{1}$.


\begin{myDef}\label{myDef:3.13}
    A sequence $\sequence{s_n}$ of real numbers is said to be

    (a) monotonically increasing if $s_n \leq s_{n+1}$ $(n=1,2,3,...)$;

    (b) monotonically decreasing if $s_n \geq s_{n+1}$ $(n=1,2,3,...)$.
\end{myDef}

\begin{thm}\label{thm:3.14}
    Theorem Suppose $\sequence{s_n}$ is monotonic. Then $\sequence{s_n}$ converges if and only if it is bounded.
\end{thm}

% chap03sec04
\section{Upper and lower limits}

\begin{myDef}
    \label{myDef:3.15}
    Let $\sequence{s_n}$ be a sequence of real numbers with the following property: For every real $M$ there is an integer $N$ such that $n > N$ implies $s_n \geq M$. We then write
    \begin{equation*}
        s_n \rightarrow +\infty.
    \end{equation*}
    Similarly, if for every real $M$ there is an integer $N$ such that $n > N$ implies $s_n \leq M$, we write
    \begin{equation*}
        s_n \rightarrow -\infty.
    \end{equation*}
\end{myDef}

It should be noted that we now use the symbol $rightarrow$ (introduced in Definition 3.1) for certain types of divergent sequences, as well as for convergent sequences, but that the definitions of convergence and of limit, given in Definition 3.1, are in no way changed.

\begin{myDef}
    \label{myDef:3.16}
    Let $\sequence{s_n}$ be a sequence of real numbers. Let $E$ be the set of numbers $x$ (in the extended real number system) such that $s_n rightarrow x$ for some subsequence $\sequence{s_n}$. This set $E$ contains all subsequential limits as defined in Definition 3.5, plus possibly the numbers $+\infty, -\infty$.

    We now recall Definitions 1.8 and 1.23 and put
    \begin{align*}
        s^{*} = \sup E,
        s_{*} = \inf E,
    \end{align*}
    The numbers $s^{*}, s_{*}$, are called the upper and lower limits of $\sequence{s_n}$; we use the notation 
    \begin{equation*}
        \limsup_{n \rightarrow \infty} s_n = s^{*},\quad \liminf_{n \rightarrow \infty} s_n = s_{*}.        
    \end{equation*}
\end{myDef}
% chap03sec05
\section{Some special sequences}

some sequences occur frequently.
remark: If $0\leq x_n \leq s_n$ for $n \geq N$, where $N$ is some fixed number, and if $s_n \rightarrow 0$, then $x_n \rightarrow 0$.

\begin{thm}
    \label{thm:3.20}
    (a) If $p > 0$, then $\lim_{n \to \infty} \frac{1}{n^p} = 0$. 

    (b) If $p > 0$, then $\lim_{n \to \infty} \sqrt[n]{p} = 1$.
    
    (c) $\lim_{n \to \infty} \sqrt[n]{n} = 1$.
    
    (d) If $p > 0$ and $\alpha$ is real, then $\lim_{n \to \infty} \frac{n^\alpha}{(1+p)^n} = 0$.

    (e) If $|x|<1$, then $\lim_{n \to \infty} x^n = 0$.
\end{thm}
% chap03sec06
\section{Series}
Consider complex-valued sequences and series

\begin{mydef}
    \label{def:3.21}
    Given a sequence $\sequence{a_n}$, we use the notation
    \begin{equation*}
        \sum_{n=p}^{q} a_n \quad (p \leq q)
    \end{equation*}
    to denote the sum $a_p+a_{p+1}+\dots+a_q$. With $\sequence{a_n}$ we associate a sequence $\sequence{s_n}$, where
    \begin{equation*}
        s_n = \sum_{k=1}^{n} a_k.
    \end{equation*}
    For $\sequence{s_n}$ we also use the symbolic expression
    \begin{equation*}
        a_1 + a_2 + a_3 + \dots
    \end{equation*}
    or, more concisely
    \begin{equation}
        \sum_{n=1}^{\infty} a_n.
    \end{equation}
    we call this \emph{infinite series}, or just a \emph{series}. The numbers $\sequence{s_n}$ are called the \emph{partial sums} of the series. If $\sequence{s_n}$ converges to $s$, we say that that the series \emph{converges}, and write
    \begin{equation*}
        \sum_{n=1}^{\infty} a_n = s.
    \end{equation*}
    The number $s$ is called the sum of the series; but it should be clearly understood that $s$ is the \emph{limit of a sequence of sums}, and is not obtained simply by addition.
    
    If $\sequence{s_n}$ diverges, the series is said to diverge.

    Sometimes, for convenience of notation, we shall consider series of the form
    \begin{equation}
        \sum_{n=0}^{\infty} a_n.
    \end{equation}
    And frequently, when there is no possible ambiguity, or when the distinction is immaterial, we shall simply write $\sum a_n$ , in place of (4) or (5).

    It is clear that every theorem about sequences can be stated in terms of series (putting $a_1 = s_1$, and $a_{n} = s_{n} - s_{n-1}$ for $n > 1$), and vice versa. But it is nevertheless useful to consider both concepts.
\end{mydef}
The Cauchy criterion (Theorem 3.11) can be restated in the following form:
    
\begin{thm}
    \label{thm:3.22}
    $\sum a_n$  converges if and only if for every $\varepsilon \in > 0$ there is an integer $N$ such that
    \begin{equation}
        \left|
            \sum_{k=n}^{m} a_k 
        \right| \leq \varepsilon
    \end{equation}
    if $m \geq n \geq N$. 
\end{thm}

In particular, by taking $m = n$, (6) becomes
\begin{equation*}
    |a_n| \leq \varepsilon \quad (n \geq N).
\end{equation*}

\begin{thm}
    \label{thm:3.23}
    If $\sum a_n$ converges, then $\lim_{n \rightarrow \infty} a_n = 0$. 
\end{thm}

The condition $a_n \rightarrow 0$ is not sufficient to ensure convergence of $\sum a_n$. For instance, the series
\begin{equation*}
    \sum_{n=1}^{\infty}\frac{1}{n}
\end{equation*}
diverges; for the proof we refer to Theorem 3.28.

Theorem 3.14, concerning monotonic sequences, also has an immediate
counterpart for series.
\begin{thm}
    \label{thm:3.24}
    A series of nonnegative terms converges if and only if its partial sums form a bounded sequence.
\end{thm}

\begin{thm}
    \label{thm:3.25}
    (a) If $|a_n| \leq c_n$, for $n \geq N_0$, where $N_0$ is some fixed integer, and if $\sum c_n$ converges, then $\sum a_n$ converges.

    (b) If $a_n \geq d_n \geq 0$ for $n \geq N_0$, and if $\sum d_n$, diverges, then $\sum a_n$ diverges.
\end{thm}

\mybox{比较审敛法}
% chap03sec07
\section{Series of nonnegative terms}

\begin{thm}
    \label{thm:3.26}
    If $0 \leq x < 1$, then
    \begin{equation*}
        \sum_{n=0}^{\infty} x^n = \frac{1}{1-x}.
    \end{equation*}
    If $x \geq 1$, the series diverges. 
\end{thm}

\mybox{几何级数收敛条件}


\mybox{Cauchy use ``thin'' subsequence of $\sequence{a_n}$ determines the convergence or divergence of $\sum a_n$}

\begin{thm}
    \label{thm:3.27 Cauchy}
    Suppose $a_1 \geq a_2 \geq a_3 \geq \dots \geq 0$.
    Then the series $\sum_{n=1}^{\infty}a_n$ converges if and only if the series
    \begin{equation}
        \sum_{k=0}^{\infty} 2^k a_{2^k}
        = a_1 + 2 a_2 + 4 a_4 + 8 a_8 + \dots
    \end{equation}
    converges.
\end{thm}

\begin{proof}
    By Theorem \ref{thm:3.24}, it suffices to consider boundedness of the partial sums.
    Let
    \begin{align*}
        s_n &= a_1 + a_2 + \dots + a_n, \\
        t_n &= a_1 + 2 a_2 + \dots + 2^{k} a_{2^k}.
    \end{align*}
    For $n < 2^k$,
    \begin{align*}
        s_n 
        &\leq a_1 + (a_2 + a_3) + \dots + (a_{2^k}+\dots+a_{2^{k+1}-1}) \\
        &\leq a_1 + 2a_2 + \dots + 2^k a_{2^k} \\
        &= t_k,
    \end{align*}
    so that
    \begin{equation}
        \label{3.8}
        s_n \leq t_k.
    \end{equation}
    On the other hand, if $n > 2^k$,
    \begin{align*}
        s_n 
        &\geq a_1 + a_2 + (a_3 + a_4) + \dots + (a_{2^{k-1}+1}+\dots+a_{2^{k}}) \\
        &\geq \frac{1}{2}a_1 + a_2 + 2 a_4 + \dots + 2^{k+1} a_{2^k} \\
        &= \frac{1}{2}t_k,
    \end{align*}
    so that
    \begin{equation}
        \label{eq:3.9}
        2 s_n \geq t_k.
    \end{equation}
    $\sequence{s_n}$,
    $\sequence{t_n}$ are both bounded or both unbounded.
\end{proof}

\begin{thm}
    \label{thm:3.28}
    $\sum \frac{1}{n^p}$ converges if $p>1$ and diverges if $p\leq 1$. 
\end{thm}

\begin{thm}
    \label{thm:3.29}
    If $p > 1$ ,
    \begin{equation}
        \label{eq:3.10}
        \sum_{n=2}^{\infty} \frac{1}{n (\log n)^p}
    \end{equation}
    converges; if $p \leq 1$ , the series diverges.
\end{thm}

``$\log n$'' the logarithm of $n$ to the base $e$ (compare Exercise 7, Chap. 1);
the number $e$ will be defined in a moment (see Def 3.30). We let the series start with $n=2$ , since $\log 1 = 0$ .

\begin{proof}
    The monotonicity of the logarithmic function (which will be discussed in more detail in Chap. 8) implies that ($\log n$) increase. Hence ($1/n \log n$) decreases, and we can apply Theorem \ref{thm:3.27 Cauchy} to (\ref{eq:3.10}); this leads us the series
    \begin{equation}
        \label{eq:3.11}
        \sum_{k=1}^{\infty}2^k\cdot\frac{1}{2^k (\log 2^k)^p} = 
        \sum_{k=1}^{\infty}\cdot\frac{1}{(k\log 2)^p} =
        \frac{1}{(\log 2)^p}\sum_{k=1}^{\infty}\cdot\frac{1}{k^p}
    \end{equation}
    and Theorem \ref{thm:3.29} follows from Theorem \ref{thm:3.28}.
\end{proof}


This procedure may evidently be continued. For instance,
\begin{equation}
    \label{eq:3.12}
    \sum_{n=3}^{\infty}\frac{1}{n \log n \log \log n}
\end{equation}
diverges, whereas
\begin{equation}
    \label{eq:3.13}
    \sum_{n=3}^{\infty}\frac{1}{n \log n (\log \log n)^2}
\end{equation}
converges.

Series (\ref{eq:3.12}) differ very littel from (\ref{eq:3.13}). Still, one diverges, the other converges.
If we continue the process which led us from Theorem \ref{thm:3.28} to Theorem \ref{thm:3.29}, we get pairs of convergent and divergent series whose terms differ even less than those of (\ref{eq:3.12}) and (\ref{eq:3.13}).
One might thus be led to the conjecture that there is a limiting situation of some sort, a ``boundary'' with all convergent series on one side, all divergent series on the other side --- at least as far as series with monotonic coefficients are converned. 
This notion of ``boundary'' is of course quite vague.
The point we wish to make is this: No matter how we make this notion precise, the conjecture is false. Exercises 11(b) and 12(b) may serve as illustrations.

More deeper aspect of convergence theory can refer to Knopp's \emph{``Theory and Application of Infinite Series''}, Chap IX, particularly Sec. 41.

% chap03sec08
\section{The number $e$}

\begin{myDef}
    \label{myDef:3.30 e}
    $e = \sum_{n=0}^{\infty}\frac{1}{n!}.$

    Here $n! = 1 \cdot 2 \cdot 3 \cdots n$ if $n \geq 1$ , and $0! = 1$.  
\end{myDef}

Since
\begin{align*}
    s_n
    &= 1 + 1 
    + \frac{1}{1 \cdot 2} 
    + \frac{1}{1 \cdot 2 \cdot 3} 
    + \cdots 
    + \frac{1}{1 \cdot 2 \cdots n} \\
    &< 1 + 1
    + \frac{1}{2}
    + \frac{1}{2^2}
    + \cdots
    + \frac{1}{2^{n-1}} 
    < 3
\end{align*}
The series converges, and the definition makes sense. In fact, the series converges very rapidly and allows us to compute $e$ with great accuracy.

It is of interest to note that $e$ can also be difined by means of another limit process; the proof provides a good illustration of operations with limits:

\begin{thm}
    \label{thm:3.31 another def of e}
    $\lim_{n \to \infty} (1+1/n)^n = e.$ 
\end{thm}
\mybox{Is this equation found by Bernoulli?}
\begin{proof}
    Let
    \begin{equation*}
        s_n = \sum_{k=0}^{n} \frac{1}{k!}, \quad
        t_n = \sum_{k=0}^{n} (1 + \frac{1}{n})^n. 
    \end{equation*}
    by the binomial theorem,
    \begin{align*}
        t_n &= 1 + 1
        + \frac{1}{2!}\left(1 - \frac{1}{n}\right)
        + \frac{1}{3!}\left(1 - \frac{1}{n}\right)\left(1 - \frac{2}{n}\right)
        + \cdots \\
        &+ \frac{1}{n!}\left(1 - \frac{1}{n}\right)\left(1 - \frac{2}{n}\right)\cdots\left(1-\frac{n-1}{n}\right).
    \end{align*}
    Hence $t_n \leq s_n$ , so that
    \begin{equation}
        \label{eq:3.14}
        \limsup_{n \rightarrow \infty}  t_n \leq e,
    \end{equation}

    by Theorem \ref{thm:3.19}. Next, if $n \geq m$ ,
    \begin{equation*}
        t_n \geq 1 + 1
        + \frac{1}{2!}\left(1 - \frac{1}{n}\right)
        + \cdots
        + \frac{1}{m!}\left(1-\frac{1}{n}\right)
        + \cdots
        + \frac{1}{m!}\left(1-\frac{1}{n}\right)\cdots\left(1-\frac{m-1}{n}\right).
    \end{equation*}

    Let $n \rightarrow \infty$ , kepping $m$ fixed. We get 
    \begin{equation*}
        \liminf_{n \to \infty} t_n \geq 1 + 1 
        + \frac{1}{2!}
        + \cdots
        + \frac{1}{m!},
    \end{equation*}
    so that
    \begin{equation*}
        s_m \leq \liminf_{n \rightarrow \infty} t_n,
    \end{equation*}
    Letting $m \rightarrow \infty$, we finally get
    \begin{equation}
        \label{eq:3.15}
        e \leq \liminf_{n \rightarrow \infty} t_n.
    \end{equation} 
    
    The Theorem follows from (\ref{eq:3.14}) and (\ref{eq:3.15}).
\end{proof}

The rapidly with which the series $\sum 1/n!$ converges can be estimated as follows: If $s_n$ has the same meaning as above, we have
\begin{align*}
    e - s_n
    &= \frac{1}{(n+1)!}
    + \frac{1}{(n+2)!}
    + \frac{1}{(n+3)!}
    + \cdots \\
    &< \frac{1}{(n+1)!}\left\{
        1
        + \frac{1}{n+1}
        + \frac{1}{(n+1)^2}
        + \cdots
    \right\} = \frac{1}{n!n}
\end{align*}
so that
\begin{equation}
    \label{eq:3.16}
    0 < e - s_n < \frac{1}{n!n}.
\end{equation}
Thus $s_{10}$, for instance, approximates $e$ with an error less than $10^{-7}$.
The inequality (\ref{eq:3.16}) is of theoretical interest as well, since it enables us to prove the irrationality of $e$ very easily.

\begin{thm}
    \label{thm:3.32}
    $e$ is irrational.
\end{thm}

\begin{proof}
    Suppose $e$ is rational. Then $e = p/q$, where $p$ and $q$ are positive integers. 
    By (\ref{eq:3.16}),
    \begin{equation}
        \label{eq:3.17}
        0<q!(e-s_q)<\frac{1}{q}.
    \end{equation}
    By our assumption, $q!e$ is an integer. Since
    \begin{equation*}
        q!s_q = 
        q!\left(
            1 + 1 + \frac{1}{2!} + \cdots + \frac{1}{q!}
        \right)
    \end{equation*}
    is an integer, we see that $q!(e-s_q)$ is an integer.

    Since $q \geq 1$, (\ref{eq:3.17}) implies the existence of an integer between $0$ and $1$. We have thus reached a contradiction.
\end{proof}

Actually, $e$ is not even an algebraic number.
For a simple proof of this, see page 25 of Niven's book, or page 176 of Herstein's, cited in the Bibliography.

% \input{chap/chap03mynotes.tex}
    % chap04

\chapter{Continuity}

The function concept and some of the related terminology were introduced in
Definitions \ref{myDef:2.1} and \ref{myDef:2.2}. Although we shall (in later chapters) be mainly interested in real and complex functions (i.e., in functions whose values are real or complex numbers) we shall also discuss vector-valued functions (i.e., functions with values in $\R{k}$) and functions with values in an arbitrary metric space. The theorems we shall discuss in this general setting would not become any easier if we restricted ourselves to real functions, for instance, and it actually simplifies and clarifies the picture to discard unnecessary hypotheses and to state and prove theorems in an appropriately general context. 

The domains of definition of our functions will also be metric spaces, suitably specialized in various instances.

% chap04sec01
\section{Limits of functions}

\begin{myDef}
    \label{myDef:4.1}
    Let $X$ and $Y$ be metric spaces; suppose $E \subset X$, $f$ maps $E$ into $Y$, and $p$ is a limit point of $E$. We write $f(x) \rightarrow q$ as $x \rightarrow p$, or
    \begin{equation}
        \label{eq:4.1}
        \lim_{x \to p} f(x) = q
    \end{equation}
    if there is a point $q \in Y$ with the following property: For every $\varepsilon > 0$ there exists a $\delta > 0$ such that
    \begin{equation}
        \label{eq:4.2}
        d_Y (f(x), q) < \varepsilon
    \end{equation}
    for all points $x \in E$ for which
    \begin{equation}
        \label{eq:4.3}
        0 < d_X (x, p) < \delta.
    \end{equation}
    The symbols $d_X$ and $d_Y$ refer to the distances in $X$ and $Y$,  respectively.
\end{myDef}
\mybox{
    逐点连续 \\
    逐点连续 是 局部性质.\\
    一致连续 是 整体性质.
}
If $X$ and/or $Y$ are replaced by the real line, the complex plane, or by some euclidean space $\R^{k}$, the distances $d_X$, $d_Y$ are of course replaced by absolute values, or by norms of differences (see Sec. 2.16).

It should be noted that $p \in X$, but that $p$ need not be a point of $E$ in the above definition. Moreover, even if $p \in E$, we may very well have $f(p) \neq \lim_{x \to p} f(x)$ ➔ .

We can recast this definition in terms of limits of sequences:

\begin{thm}
    \label{thm:4.2}
    Let $X,Y,E,f$ , and $p$ be as in Definition 4.1. Then
    \begin{equation}
        \label{eq:4.4}
        \lim_{x \to p} f(x) = q
    \end{equation}
    if and only if 
    \begin{equation}
        \label{eq:4.5}
        \lim_{n \to \infty} f(p_n) = q
    \end{equation}
    for every sequence $\sequence{p_n}$ in $E$ such that
    \begin{equation}
        \label{eq:4.6}
        p_n \neq p, \quad
        \lim_{n \to \infty} p_n = p.
    \end{equation}
\end{thm}

\begin{proof}
    Suppose (\ref{eq:4.4}) holds. Choose $\sequence{p_n}$ in $E$ satisfying (\ref{eq:4.6}). Let $\varepsilon > 0$ be given. Then there exists $\delta > 0$ such that $d_Y(f(x), q) < \varepsilon$ if $x \in E$ and $0 < d_X (x, p) < \delta$. Also, there exists $N$ such that $n > N$ implies $0 < d_X(p_n ,p) < \delta$. Thus, for $n > N$, we have $d_Y(f(p_n), q) < \delta$, which shows that (\ref{eq:4.5}) holds. 
    
    Conversely, suppose (\ref{eq:4.4}) is false. Then there exists some $\varepsilon > 0$ such that for every $\delta > 0$ there exists a point $x \in E$ (depending on $\delta$), for which $d_Y(f(x), q) \geq \varepsilon$ but $0 < d_X(x, p) < \delta$. Taking $\delta_n = 1/n (n = 1, 2, 3, ... )$, we thus find a sequence in $E$ satisfying (\ref{eq:4.6}) for which (\ref{eq:4.5}) is false.
\end{proof}

\begin{myCorollary}
    If $f$ has a limit at $p$, this limit is unique.
\end{myCorollary}

\begin{myDef}
    \label{myDef:4.3}
    Suppose we have two complex functions, $f$ and $g$, both defined on $E$. By $f + g$ we mean the function which assigns to each point $x$ of $E$ the number $f(x) + g(x)$. 
    Similarly we define the difference $f - g$, 
    the product $fg$, 
    and the quotient $f/g$ of the two functions, 
    with the understanding that the quotient is defined only at those points $x$ of $E$ at which $g(x) \neq 0$. 
    If $f$ assigns to each point $x$ of $E$ the same number $c$, 
    then $f$ is said to be a constant function, or simply a constant, 
    and we write $f = c$. 
    If $f$ and $g$ are real functions, and if $f(x) \geq g(x)$ for every $x \in E$, we shall sometimes write $f \geq g$, for brevity.

    Similarly, if $\mathbf{f}$ and $\mathbf{g}$ map $E$ into $\R^{k}$, we define $\mathbf{f} + \mathbf{g}$ and $\mathbf{f} \cdot \mathbf{g}$ by
    \begin{equation*}
        (\mathbf{f} + \mathbf{g})(x) 
        = \mathbf{f}(x)  
        + \mathbf{g}(x) , \quad
        (\mathbf{f} \cdot \mathbf{g})(x) 
        = \mathbf{f}(x)  
        \cdot \mathbf{g}(x) ;        
    \end{equation*}
    and if $\lambda$ is a real number, $(\lambda \mathbf{f})(x) = \lambda \mathbf{f}(x)$.
\end{myDef}

\begin{thm}
    \label{thm:4.4}
    Suppose $E \subset X$, a metric space, $p$ is a limit point of $E$, $f$ and $g$ are complex functions on $E$, and
    \begin{equation*}
        \lim_{x \to p} f(x) = A, \quad
        \lim_{x \to p} g(x) = B.
    \end{equation*}
    Then \\
    (a) $\lim_{x \to p} (f + g)(x) = A + B$; \\
    (b) $\lim_{x \to p} (f   g)(x) = A   B$; \\
    (b) $\lim_{x \to p} (\frac{f}{g})(x) = \frac{A}{B}$, if $B \neq 0$. \\
\end{thm}

\begin{proof}
    In view of Theorem \ref{thm:4.2}, these assertions follow immediately from the analogous properties of sequences (Theorem \ref{thm:3.3}).
\end{proof}

Remark:
    If $f$ and $g$ map $E$ into $\R^{k}$, 
    then (a) remains true, 
    and (b) becomes (b') 
    $\lim_{x \to p} (\mathbf{f} \cdot \mathbf{g})(x) = \mathbf{A \cdot B}$;

(Compare Theorem \ref{thm:3.4}.)
% chap04sec02

\section{Continuous functions}

\begin{mydef}
    \label{mydef:4.5}
    Suppose $X$ and $Y$ are metric spaces, $E \subset X$, $p \in E$, and $f$ maps $E$ into $Y$. Then $f$ is said to be \emph{continuous at $p$} if for every $\varepsilon > 0$ there exists a $\varepsilon > 0$ such that
    \begin{equation*}
        d_Y (f(x), f(p)) < \varepsilon
    \end{equation*}
    for all points $x \in  E$ for which $d_X(x, p) < \delta$.
    
    If $f$ is continuous at every point of $E$, then $f$ is said to be \emph{continuous on $E$}.
    
    It should be noted that $f$ has to be defined at the point $p$ in order to be continuous at $p$. (Compare this with the remark following Definition \ref{mydef:4.1}.)

    If $p$ is an isolated point of $E$, then our definition implies that every function $f$ which has $E$ as its domain of definition is continuous at $p$. For, no matter which $\varepsilon > 0$ we choose, we can pick $\delta > 0$ so that the only point $x \in  E$ for which $d_X(x,p) <\delta$ is $x = p$; then
    \begin{equation*}
        d_Y(f(x),f(p)) = 0 < \varepsilon.
    \end{equation*}
\end{mydef}

\begin{thm}
    \label{thm:4.6}
    In the situation given in Definition \ref{mydef:4.5}, assume also that $p$ is a limit point of $E$. Then $f$ is continuous at $p$ if and only if $\lim_{x \to p}  f(x) = f(p)$.
\end{thm}
\begin{proof}
    This is clear if we compare Definitions \ref{mydef:4.1} and \ref{mydef:4.5}.
\end{proof}


We now turn to compositions of functions. 
A brief statement of the following theorem is that a continuous function of a continuous function is continuous.
\begin{thm}
    \label{thm:4.7}
    Suppose $X, Y, Z$ are metric spaces, $E \subset X$, $f$ maps $E$ into $Y$, $g$ maps the range of $f$, $f(E)$, into $Z$, and $h$ is the mapping of $E$ into $Z$ defined by 
    \begin{equation*}
        h(x) = g(f(x)) \quad
    (x \in  E).
    \end{equation*}
    If $f$ is continuous at a point $p \in E$ and if $g$ is continuous at the point $f(p)$, then $h$ is continuous at $p$.

    This function his called the composition or the composite of $f$ and $g$. The notation
    \begin{equation*}
        h = g \circ f
    \end{equation*}
    is frequently used in this context.
\end{thm}

\begin{proof}
    Let $\varepsilon > 0$ be given. Since $g$ is continuous at $f(p)$, there exists $\eta > 0$ such that 
    \begin{equation*}
        d_Z(g(y), g(f(p))) < \varepsilon \text{ if } d_Y(y,f(p)) < \eta, \text{ and } y \in f(E).
    \end{equation*}
Since $f$ is continuous at $p$, there exists $\delta > 0$ such that
\begin{equation*}
    d_Y(f(x),f(p)) < \eta, \text{ if } d_X(x, p) <\delta \text{ and } x \in E.
\end{equation*}
It follows that
\begin{equation*}
    d_Z(h(x), h(p)) = d_Z(g(f(x)), g(f(p))) < \varepsilon
\end{equation*}
if $d_X(x, p) < \delta$ and $x \in E$. Thus $h$ is continuous at $p$.
\end{proof}

\begin{thm}
    \label{thm:4.8}
    A mapping $f$ of a metric space $X$ into a metric space $Y$ is continuous on $X$ if and only if $f^{-1}(V)$ is open in $X$ for every open set $V$ in $Y$.
\end{thm}

(Inverse images are defined in Definition \ref{mydef:2.2}.) 
This is a very useful characterization of continuity.

\begin{myCorollary*}
    A mapping $f$ of a metric space $X$ into a metric space $Y$ is continuous if and only if $f^{-1} (C)$ is closed in $X$ for every closed set $C$ in $Y$.
\end{myCorollary*}

This follows from the theorem, since a set is closed if and only if its complement is open, and since $f^{-1}(E^c) = [f^{-1}(E)]^c$ for every $E \subset Y$.

We now turn to complex-valued and vector-valued functions, and to
functions defined on subsets of $\R^{k}$.

\begin{thm}
    \label{thm:4.9}
    Let $f$ and $g$ be complex continuous functions on a metric space $X$.
    Then $f + g,fg$, and $f/g$ are continuous on $X$.
    
    In the last case, we must of course assume that $g(x) \neq 0$, for all $x \in  X$.
\end{thm}
\begin{proof}
    At isolated points of $X$ there is nothing to prove. At limit points,
    the statement follows from Theorems \ref{thm:4.4} and \ref{thm:4.6}.
\end{proof}

\begin{thm}
    \label{thm:4.10}
    (a) Let $f_1, \dots , f_k$ be real functions on a metric space $X$, and let $\mathbf{f}$ be the mapping of $X$ into $\R^{k}$ defined by
    \begin{equation}
        \label{eq:4.7}
        \mathbf{f}(x) = (f_1(x), ... ,f_k(x)) \quad    (x \in  X);
    \end{equation}
    then $\mathbf{f}$ is continuous if and only if each of the functions $f_1, ... , f_k$ is continuous.

    (b) If $\mathbf{f}$ and $\mathbf{g}$ are continuous mappings of $X$ into $\R^{k}$, 
    then $\mathbf{f + g}$ and $\mathbf{f \cdot g}$ are continuous on $X$.
\end{thm}
The functions $f_1, ... , f_k$ are called the \emph{components} of $\mathbf{f}$. 
Note that $\mathbf{f + g}$ is a mapping into $\R^{k}$, whereas $\mathbf{f \cdot g}$ is a \emph{real} function on $X$.
\begin{proof}
    Part (a) follows from the inequalities
    \begin{equation*}
        \left| f_j(x) -f_j(x) \right| \leq
        \left| \mathbf{f}(x) - \mathbf{f}(y) \right| =
        \left\{
            \sum_{i=1}^{k} \left| f_i(x) - f_i(y) \right|^2
        \right\}^{\frac{1}{2}},
    \end{equation*}
    for $j=1,2,...,k$ . Part (b) follows form (a) and Theorem \ref{thm:4.9}.
\end{proof}

\begin{myExample}
    \label{myExample:4.11}
    \begin{equation}
        \phi_i (\mathbf{x}) = x_i 
        \quad (\mathbf{x} \in \R^{k})
    \end{equation}

    \begin{equation}
        x_{1}^{n_1}
        x_{2}^{n_2}
        \dots
        x_{k}^{n_k}
    \end{equation}

    \begin{equation}
        P(\mathbf{x}) = \sum c_{n_1 \dots n_k} 
        x_{1}^{n_1}
        x_{2}^{n_2}
        \dots
        x_{k}^{n_k}
        \quad (\mathbf{x} \in \R^{k})
    \end{equation}

    \begin{equation}
        \left| 
            \left| \mathbf{x} \right| -
            \left| \mathbf{y} \right|  
        \right| \leq
        \left| \mathbf{x-y} \right| 
        \quad (\mathbf{x,y} \in \R^{k})
    \end{equation}
\end{myExample}

\begin{myRemark}
    \label{myRemark:4.12}
We defined the notion of continuity for functions defined on a subset $E$ of a metric space $X$. 
However, the complement of $E$ in $X$ plays no role whatever in this definition 
(note that the situation was somewhat different for limits of functions). 
Accordingly, we lose nothing of interest by discarding the complement of the domain of $f$.
This means that we may just as well talk only about continuous mappings of one metric space into another, rather than
of mappings of subsets. This simplifies statements and proofs of some theorems.
We have already made use of this principle in Theorems \ref{thm:4.8} to \ref{thm:4.10}, and will continue to do so in the following section on compactness.
\end{myRemark}
% chap04sec03

\section{Continuity and compactness}
\mybox{连续性与紧致性}
\begin{myDef}
    \label{myDef:4.13}
    A mapping $\mathbf{f}$ of a set $E$ into $\R^{k}$ is said to be \emph{bounded} 
    if there is a real number $M$ such that $\left| f(x) \right| \leq M$ for all $x \in E$.
\end{myDef}
\mybox{映射有界 代表其定义域上所有点的映射组成的集合有界.}
\begin{thm}
    \label{thm:4.14}
    Suppose $f$ is a continuous mapping of a compact metric space $X$ into a metric space $Y$. Then $f(X)$ is compact.
\end{thm}
\mybox{$f$为紧度量空间 $X$ 到度量空间 $Y$ 的连续映射, $f(X)$是紧的}
\myproof{
    Let $\sequence{V_\alpha}$ be an open cover of $f(X)$. Since $f$ is continuous, 
    Theorem \ref{thm:4.8} shows that each of the sets $f^{-1}(V_{\alpha})$ is open. 
    Since $X$ is compact, there are finitely many indices, say $\alpha_1,  , \alpha_n$, such that
    \begin{equation}
        \label{eq:4.12}
        X \subset 
        f^{-1} (V_{\alpha_1})
        \cup \cdots \cup
        f^{-1} (V_{\alpha_n}).    
    \end{equation}
    Since $f(f^{-1}(E)) \subset E$ for every $E \subset Y$, 
    (\ref{eq:4.12}) implies that 
    \begin{equation}
        \label{eq:4.13}
        f(X) \subset 
        (V_{\alpha_1})
        \cup \cdots \cup
        (V_{\alpha_n}).
    \end{equation}
    
    This completes the proof
}

Note: We have used the relation $f(f^{- 1}(E)) \subset E$, valid for $E \subset Y$. 
If $E \subset X$, then $f^{- 1}(f(E)) \supset E$; equality need not hold in either case.

We shall now deduce some consequences of Theorem \ref{thm:4.14}

\mythm{
    \label{thm:4.15}
    If $\mathbf{f}$ is a continuous mapping of a compact metric space $X$ into $\R^{k}$, 
    then $\mathbf{f}(X)$ is closed and bounded. 
    Thus, $\mathbf{f}$ is bounded.
}

This follows from Theorem \ref{thm:2.41}. 
The result is particularly important when $f$ is real:

\mythm{
    \label{thm:4.16}
    Suppose $f$ is a continuous real function on a compact metric space $X$, and
    \begin{equation}
        \label{eq:4.14}
        M = \sup_{p\in X} f(p), \quad
        m = \inf_{p\in X} f(p).
    \end{equation}
    Then there exist points $p, q \in X$ 
    such that $f(p) = M$ and $f(q) = m$.
}
The notation in (\ref{eq:4.14}) means that 
$M$ is the least upper bound of the set of all numbers $f(p)$, 
where $p$ ranges over $X$, 
and that $m$ is the greatest lower bound of this set of numbers.

The conclusion may also be stated as follows: 
\emph{There exist points $p$ and $q$
in $X$ such that $f(q) \leq f(x) \leq f(p)$ for all $x \in X$;} 
that is, $f$ attains its maximum (at $p$) and its minimum (at $q$).

\myproof{
    By Theorem \ref{thm:4.15}, 
    $f(X)$ is a closed and bounded set of real numbers; 
    hence $f(X)$ contains
    \begin{equation*}
        M = \sup f(X), \quad
        m = \inf f(X).        
    \end{equation*}
    By Theorem \ref{thm:2.28}
}

\mythm{
    \label{thm:4.17}
    Suppose $f$ is a continuous 1-1 mapping of a compact metric space $X$ onto a metric space $Y$. 
    Then the inverse mapping 1-1 defined on $Y$ by 
    \begin{equation*}
        f^{-1}(f(x)) = x \quad
        (x \in X)
    \end{equation*}
    is a continuous mapping of $Y$ onto $X$.
}
\mybox{1-1映射---逆映射}

\mymyDef{
    \label{myDef:4.18}
    Let $f$ be a mapping of a metric space $X$ into a metric space $Y$.
    We say that $f$ is \emph{uniformly continuous} on $X$ 
    if for every $\varepsilon > 0$ there exists $\delta > 0$
    such that
    \begin{equation}
        \label{eq:4.15}
        d_Y(f(p),f(q)) < \varepsilon
    \end{equation}
    for all $p$ and $q$ in $X$ for which $d_X(p, q) < \delta$.
}
\mybox{一致连续}
Let us consider the differences between the concepts of continuity and of
uniform continuity. 
First, uniform continuity is a property of a function on a set, 
whereas continuity can be defined at a single point. 
To ask whether a given function is uniformly continuous at a certain point is meaningless. 
Second, if $f$ is continuous on $X$, 
then it is possible to find, 
for each $\varepsilon > 0$ and for each point $p$ of $X$, 
a number $\delta > 0$ having the property specified in Definition \ref{myDef:4.5}. 
This $\delta$ depends one $\varepsilon$ \emph{and} on $p$. 
If $f$ is, however, uniformly continuous on $X$, 
then it is possible, for each $\varepsilon > 0$, 
to find \emph{one} number $\delta > 0$ which will do for \emph{all} points $p$ of $X$.

Evidently, every uniformly continuous function is continuous. 
That the two concepts are equivalent on compact sets follows from the next theorem. 

\mythm{
    \label{thm:4.19}
    Let $f$ be a continuous mapping of a compact metric space $X$ into a metric space $Y$. 
    Then $f$ is uniformly continuous on $X$.
}

\myproof{
    Let $\varepsilon > 0$ be given.
    Since $f$ is continuous, we can associate to each point $p \in X$ a positive number $\phi(p)$ such that 
    \begin{equation}
        \label{eq:4.16}
        q\in X, d_X(p, q) < \phi(p)
        \text{ implies }
        d_Y (f(p), f(q)) < \frac{\varepsilon}{2}.
    \end{equation}

    Let $J(p)$ be the set of all $q \in X$ for which
    \begin{equation}
        \label{eq:4.17}
        d_X(p, q) < \frac{1}{2}\phi(p).
    \end{equation}

    Since $p \in J(p)$, the collection of all sets $J(p)$ is an open cover of $X$;
    and since $X$ is compact, there is a finite set of points $p_1,...,p_n$ in $X$, such that 
    \begin{equation}
        \label{eq:4.18}
        X \subset J(p_1) \cup \cdots \cup J(p_n).
    \end{equation}
    We put 
    \begin{equation}
        \delta = \frac{1}{2} \min [\phi(p_1), ..., \phi(p_n)].
    \end{equation}
    Then $\delta > 0$ .
    (This is one point where the finiteness of the covering,
    inherent in the definition of compactness, is essential.
    The minimum of a finite set of positive numbers is positive,
    whereas the inf of an infinite set of positive numbers may very well be 0.)

    Now let $q$ and $p$ be points of $X$, such that $d_X(p, q) < \delta$,
    By (\ref{eq:4.18}), there is an integer $m$, $1 \leq m \leq n$, 
    such that $p \in J(p_m)$; hence 
    \begin{equation}
        \label{eq:4.20}
        d_X(p, p_m) < \frac{1}{2}\phi(p_m),
    \end{equation}
    and we also have 
    \begin{equation*}
        d_X(q, p_m) \leq
        d_X(p, q) +
        d_X(p, p_m) <
        \delta + \frac{1}{2}\phi(p_m) \leq
        \phi(p_m).
    \end{equation*}
    Finally, (\ref{eq:4.16}) shows that therefore 
    \begin{equation*}
        d_Y(f(p), f(q)) \leq
        d_Y(f(p), f(p_m)) +
        d_Y(f(q), f(p_m)) <
        \varepsilon .
    \end{equation*}
    This complete the proof.
}

An alternative proof is sketched in Exercise 10.

We now proceed to show that compactness is essential in the hypotheses
of Theorems \ref{thm:4.14}, \ref{thm:4.15}, \ref{thm:4.16}, and \ref{thm:4.19}.

\mythm{
    \label{thm:4.20}
    Let $E$ be a noncompact set in $\R^{1}$ Then

    (a) there exists a continuous function on $E$ which is not bounded, 
    
    (b) there exists a continuous and bounded function on $E$ which has no maximum.

    If, in addition, $E$ is bounded, then 
    
    (c) there exists a continuous function on $E$ which is not uniformly continuous.
}

\myproof{
    Suppose first that $E$ is bounded, 
    so that there exists a limit point $x_0$ of $E$ 
    which is not a point of $E$. 
    Consider
    \begin{equation}
        \label{eq:4.21}
        f(x) = \frac{1}{x - x_0}
        \quad
        (x \in E).
    \end{equation}
    This is continuous on $E$ (Theorem 4.9), but evidently unbounded. 
    To see that (\ref{eq:4.21}) is not uniformly continuous, 
    let $\varepsilon > 0$ and $\delta > 0$ be arbitrary, 
    and choose a point $x \in E$ such that $\left| x - x_0 \right| < \delta$.
    Taking $t$ close enough to $x_0$ , 
    we can then make the difference $\left| f(t) - f(x) \right|$ greater than $\varepsilon$, although $\left| t-x \right| < \delta$.
    Since this is true for every $\delta > 0$, 
    $f$ is not uniformly continuous on $E$.

    The function $g$ given by
    \begin{equation}
        \label{eq:4.22}
        g(x) = \frac{1}{1+(x-x_0)^2}
        \quad
        (x \in E)
    \end{equation}
    is continuous on $E$, and is bounded, since $0 < g(x) < 1$. 
    It is clear that 
    \begin{equation*}
        \sup_{x \in E} g(x) = 1,
    \end{equation*}
    whereas $g(x) < l$ for all $x \in E$. Thus $g$ has no maximum on $E$.

    Having proved the theorem for bounded sets $E$, 
    let us now suppose that $E$ is unbounded. 
    Then $f(x) = x$ establishes (a), whereas
    \begin{equation}
        \label{eq:4.23}
        h(x) = \frac{x^2}{1 + x^2}
        \quad 
        (x \in E)
    \end{equation}
    establishes (b), since
    \begin{equation*}
        \sup_{x \in E} h(x) = 1
    \end{equation*}
    and $h(x) < 1$ for all $x \in E$.

    Assertion (c) would be false if boundedness were omitted from the
    hypotheses. 
    For, let $E$ be the set of all integers. 
    Then every function defined on $E$ is uniformly continuous on $E$. 
    To see this, we need merely take $\delta < 1$ in Definition \ref{myDef:4.18}.
}

We conclude this section by showing that compactness is also essential in
Theorem \ref{thm:4.17}.

\mymyExample{
    Let $X$ be the half-open interval $[0, 2\pi)$ on the real line, 
    and let $f$ be the mapping of $X$ onto the circle $Y$ consisting of all points whose distance from the origin is $1$, given by
    \begin{equation}
        \label{eq:4.24}
        f(t) = (\cos t, \sin t)
        \quad
        (0 \leq t < 2\pi).
    \end{equation}
    The continuity of the trigonometric functions cosine and sine, 
    as well as their periodicity properties, will be established in Chap. 8. 
    These results show that $f$ is a continuous 1-1 mapping of $X$ onto $Y$.
    
    However, the inverse mapping (which exists, since $f$ is one-to-one and onto) fails to be continuous at the point $(1, 0) = \mathbf{f}(0)$. 
    Of course, $X$ is not compact in this example. 
    (It may be of interest to observe that $\mathbf{f}^{-1}$ fails to be continuous in spite of the fact that $Y$ \emph{is} compact!)
}
% chap04sec04

\section{Continuity and connectedness}
\mybox{连续性与连通性}
\begin{thm}
    \label{thm:4.22}
    If $f$ is a continuous mapping of a metric space $X$ into a metric space $Y$, 
    and if $E$ is a connected subset of $X$, then $f(E)$ is connected.
\end{thm}

\begin{thm}
    \label{thm:4.23}
    Let $f$ be a continuous real function on the interval $[a, b]$. 
    If $f(a) <f(b)$ and if $c$ is a number such that $f(a) < c < f(b)$, 
    then there exists a point $x \in (a, b)$ such that $f(x) = c$.
\end{thm}

\begin{myremark}
    \label{myremark:4.24}
    At first glance, it might seem that Theorem \ref{thm:4.23} has a converse.
    That is, one might think that if for any two points $x_1 < x_2$ 
    and for any number $c$ between $f(x_1)$ and $f(x_2)$ 
    there is a point $x$ in $(x_1 , x_2)$ such that $f(x) = c$, 
    then $f$ must be continuous.
\end{myremark}
% chap04sec05
\section{Discontinuities}

If $x$ is a point in the domain of definition of the function $f$ at which $f$ is not continuous, 
we say that $f$ is \emph{discontinuous} at $x$, 
or that $f$ has a \emph{discontinuity} at $x$.
If $f$ is defined on an interval or on a segment, 
it is customary to divide discontinuities into two types. 
Before giving this classification, 
we have to define the right-hand and the left-hand limits of $f$ at $x$, 
which we denote by $f(x+)$ and $f(x-)$, respectively.

\begin{myDef}
    \label{myDef:4.25}
    Let $f$ be defined on $(a, b)$. 
    Consider any point $x$ such that $a \leq x < b$. 
    We write
    \begin{equation*}
        f(x+) = q        
    \end{equation*}
    if $f(t_n) \rightarrow q$ as $n \rightarrow \infty$, 
    for all sequences $\sequence{t_n}$ in $(x, b)$ such that $t_n \rightarrow x$. 
    To obtain the definition of $f(x-)$, 
    for $a < x \leq b$, 
    we restrict ourselves to sequences $\sequence{t_n}$ in $(a, x)$.
    It is clear that any point $x$ of $(a, b)$, 
    $\lim_{t \to x} f(t)$ exists if and only if
    \begin{equation*}
        f(x+) = f(x-) = \lim_{t \to x} f(t).
    \end{equation*}
\end{myDef}

\begin{myDef}
    \label{myDef:4.26}
    Let $f$ be defined on $(a, b)$. 
    If $f$ is discontinuous at a point $x$,
    and if $f(x +)$ and $f (x-)$ exist, 
    then $f$ is said to have a discontinuity of the \emph{first kind}, 
    or a \emph{simple discontinuity}, at $x$. 
    Otherwise the discontinuity is said to be of the \emph{second kind}.

    There are two ways in which a function can have a simple discontinuity:
    either $f(x+) \neq f(x-)$ [in which case the value $f(x)$ is immaterial], 
    or $f(x+) = f(x-) \neq f(x)$.
\end{myDef}
\mybox{
    第一类间断点和第二类间断点, \\
    第一类间断点也称为可去间断点, 跳跃间断点.\\ 
    第二类间断点也称为无穷间断点, 震荡间断点.
}

\begin{myExample}
    \begin{asparaenum}[(a)]
        \item Define 
    \begin{equation*}
        f(x) = \left\{
            \begin{array}{lc}
                1 & (x \text{ rational}),\\
                0 & (x \text{ irrational}).
            \end{array}
        \right.
    \end{equation*}
    Then $f$ has a discontinuity of the second kind at every point $x$. 
    since neither $f(x+)$ nor $f(x-)$ exists.
    \item Define
    \begin{equation*}
        f(x) = \left\{
            \begin{array}{lc}
                x & (x \text{ rational}),\\
                0 & (x \text{ irrational}).
            \end{array}
        \right.
    \end{equation*}
    Then $f$ is continuous at $x = 0$ and has a discontinuity of the second kind at every other point.

    \item Define 
    \begin{equation*}
        f(x) = \left\{
            \begin{array}{lc}
                 x + 2  & (-3 <    x < -2),\\
                -x - 2  & (-2 \leq x <  0),\\
                 x + 2  & ( 0 \leq x <  1).
            \end{array}
        \right.
    \end{equation*}
    Then $f$ has a simple discontinuity at $x = 0$ and is continuous at every other point of $(-3, 1)$.
    \item Define
    \begin{equation*}
        f(x) = \left\{
            \begin{array}{lc}
                \sin \frac{1}{x} & (x \neq 0),\\
                0 & (x = 0).
            \end{array}
        \right.
    \end{equation*}
    Since neither $f(0+)$ nor $f(0-)$ exists,
    $f$ has a discontinuity of the second kind at $x = 0$. 
    We have not yet shown that $\sin x$ is a continuous function. 
    If we assume this result for the moment, 
    Theorem \ref{thm:4.7} implies that $f$ is continuous at every point $x \neq 0$.
    \end{asparaenum}
\end{myExample}
% chap04sec06
\section{Monotonic functions}
\mybox{单调函数}
We shall now study those functions which never decrease 
(or never increase) on a given segment.

\begin{mydef}
    \label{def:4.28}
    Let $f$ be real on $(a, b)$. 
    Then $f$ is said to be \emph{monotonically increasing} on $(a, b)$ 
    if $a< x < y < b$ implies $f(x) \leq f(y)$. 
    If the last inequality is reversed, 
    we obtain the definition of a \emph{monotonically decreasing} function. 
    The class of monotonic functions consists of both the increasing and the decreasing functions.
\end{mydef}

\begin{thm}
    \label{thm:4.29}
    Let f be monotonically increasing on $(a, b)$. 
    Then $f(x+)$ and $f(x-)$ exist at every point of $x$ of $(a, b)$. 
    More precisely,
    \begin{equation}
        \label{eq:4.25}
        \sup_{a < t < x} f(t) = f(x-) 
        \leq f(x) \leq 
        f(x+) = \inf_{x < t < b} f(t).
    \end{equation}
    Furthermore, if $a < x < y < b$, then
    \begin{equation}
        \label{eq:4.26}
        f(x+) \leq f(y-).
    \end{equation}
\end{thm}

Analogous results evidently hold for monotonically decreasing functions.

\begin{proof}
    By hypothesis, the set of numbers $f(t)$, where $a< t < x$, 
    is bounded above by the number $f(x)$, 
    and therefore has a least upper bound which we shall denote by $A$. 
    Evidently $A \leq f(x)$. We have to show that $A =f(x-)$. 
    
    Let $\varepsilon > 0$ be given. 
    It follows from the definition of $A$ 
    as a least upper bound that there exists $\delta > 0$ 
    such that $a < x - \delta < x$ and
    \begin{equation}
        \label{eq:4.27}
        A - \varepsilon < f(x - \delta) \leq A.
    \end{equation}

    Since $f$ is monotonic, we have
    \begin{equation}
        \label{eq:4.28}
        f(x-\delta) \leq f(t) \leq A
        \quad 
        (x-\delta < t < x).
    \end{equation}

    Combining (\ref{eq:4.27}) and (\ref{eq:4.28}), we see that
    \begin{equation*}
        \left| f(t) - A \right| < \varepsilon
        \quad
        (x - \delta < t < x).
    \end{equation*}
    Hence $f(x-) = A$.

    The second half of (\ref{eq:4.25}) is proved in precisely the same way.

    Next, if $a < x < y < b$, we see from (\ref{eq:4.25}) that
    \begin{equation}
        \label{eq:4.29}
        f(x+) 
        = \inf_{x < t < b} f(t)
        = \inf_{x < t < y} f(t)
    \end{equation}
    The last equality is obtained by applying (\ref{eq:4.25}) to $(a, y)$ in place of $(a, b)$. 
    Similarly,
    \begin{equation}
        \label{eq:4.30}
        f(y-) 
        = \sup_{a < t < y} f(t)
        = \sup_{x < t < y} f(t)
    \end{equation}
    Comparison of (\ref{eq:4.29}) and (\ref{eq:4.30}) gives (\ref{eq:4.26}).
\end{proof}

\begin{myCorollary*}
    Monotonic functions have no discontinuities of the second kind.
\end{myCorollary*}

This corollary implies that every monotonic function is discontinuous at a countable set of points at most. 
Instead of appealing to the general theorem whose proof is sketched in Exercise 17, 
we give here a simple proof which is applicable to monotonic functions.

\begin{thm}
    \label{thm:4.30}
    Let $f$ be monotonic on $(a, b)$. 
    Then the set of points of $(a, b)$ 
    at which $f$ is discontinuous 
    is at most countable.
\end{thm}

\begin{myRemark}
    It should be noted that the discontinuities of a monotonic function need not be isolated. 
    In fact, given any countable subset $E$ of $(a, b)$, 
    which may even be dense, 
    we can construct a function $f$, monotonic on $(a, b)$, 
    discontinuous at every point of $E$, and at no other point of $(a, b)$.

    To show this, let the points of $E$ be arranged in a sequence $\sequence{x_n}$, $n = 1, 2, 3,...$. 
    Let $\sequence{c_n}$ be a sequence of positive numbers such that $\sum c_n$ converges. 
    Define
    \begin{equation}
        \label{eq:4.31}
        f(x) = \sum_{x_n < x} c_n
        \quad 
        (a < x < b).
    \end{equation}
    The summation is to be understood as follows: 
    Sum over those indices $n$ for which $x_n < x$. 
    If there are no points $x_n$ to the left of $x$, the sum is empty; 
    following the usual convention, we define it to be zero. 
    Since (\ref{eq:4.31}) converges absolutely, 
    the order in which the terms are arranged is immaterial.
\end{myRemark}

\begin{asparaenum}[(a)]
    \item $f$ is monotonically increasing on $(a, b)$;
    \item $f$ is discontinuous at every point of $E$; in fact,
    \begin{equation*}
        f(x_n+) - f(x_n-) = c_n
    \end{equation*}
    \item $f$ is continuous at every other point of $(a, b)$.
\end{asparaenum}

Moreover, it is not hard to see that $f(x-) =f(x)$ at all points of $(a, b)$. 
If a function satisfies this condition, 
we say that $f$ is \emph{continuous from the left}. 
If the summation in (\ref{eq:4.31}) were taken over all indices $n$ for which $x_n \leq x$, 
we would have $f(x+) = f(x)$ at every point of $(a, b)$; 
that is, $f$ would be \emph{continuous from the right}.

Functions of this sort can also be defined by another method; 
for an example we refer to Theorem 6.16.
% chap04sec07
\section{Infinite limits and limits at infinity}
To enable us to operate in the extended real number system, 
we shall now enlarge the scope of Definition \ref{myDef:4.1}, 
by reformulating it in terms of neighborhoods.

For any real number $x$, we have already defined a neighborhood of $x$ to
be any segment $(x - \delta, x + \delta)$.

\begin{myDef}
    \label{myDef:4.32}
    For any real $c$, the set of real numbers $x$ such that $x > c$ 
    is called a neighborhood of $+\infty$ and is written $(c, +\infty)$. 
    Similarly, the set $(-\infty , c)$ is a neighborhood of $-\infty$ .
\end{myDef}

\begin{myDef}
    \label{myDef:4.33}
    Let $f$ be a real function defined on $E \subset R$. 
    We say that 
    \begin{equation*}
        f(t) \rightarrow A \text{ as } t \rightarrow x,
    \end{equation*}
    where $A$ and $x$ are in the extended real number system, 
    if for every neighborhood $U$ of $A$ 
    there is a neighborhood $V$ of $x$ 
    such that $V \cap E$ is not empty, 
    and such that $f(t) \in U$ for all $t \in V \cap E$, $t \neq x$.
    
    A moment's consideration will show that 
    this coincides with Definition \ref{myDef:4.1} when $A$ and $x$ are real.
    The analogue of Theorem \ref{thm:4.4} is still true, 
    and the proof offers nothing new. 
    We state it, for the sake of completeness.
\end{myDef}
\mybox{sake 目的、理由、缘故...等意思, 经常以for the sake of 的形式出现}

\begin{thm}
    \label{thm:4.34}
    Let $f$ and $g$ be defined on $E \in \R$. Suppose
    \begin{equation*}
        f(t) \rightarrow A, \quad
        g(t) \rightarrow B, \quad
        \text{ as } t \rightarrow x.
    \end{equation*}
    Then
\begin{enumerate}[(a)]
    \item $f(t) \rightarrow A'$ implies $A' = A$.
    \item $(f + g)(t) \rightarrow A + B$,
    \item $(fg)(t) \rightarrow AB$,
    \item $(f /g)(t) \rightarrow A/B$,
\end{enumerate}
provided the right members of (b), (c), and (d) are defined.
\end{thm}

Note that $\infty  - \infty$ , $0 \cdot \infty$ , $\infty /\infty$ , 
$A/0$ are not defined (see Definition 1.23).
\end{document}