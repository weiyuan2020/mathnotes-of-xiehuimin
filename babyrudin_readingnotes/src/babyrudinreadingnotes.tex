% \documentclass[openany,twoside,scheme=chinese,fontset=none]{book}
\usepackage{ctex}
\usepackage{geometry}
\geometry{
    paperheight=260mm,
    paperwidth=185mm,
    top=25mm,
    bottom=15mm,
    left=25mm, % 左侧留 5mm 装订线距离
    right=15mm
}

\setmainfont{XITS}  % 英文字体, Times 风格

\setCJKmainfont{Source Han Serif SC}[         % 方正书宋_GBK
    BoldFont=Source Han Serif SC Bold,  % 思源宋体粗体
    ItalicFont=FZKai-Z03                % 方正楷体_GBK
    ]
\setCJKsansfont{Source Han Sans SC}[             % 方正黑体_GBK
    BoldFont=Source Han Sans SC Bold    % 思源黑体粗体
    ]
\setCJKmonofont{FZFangSong-Z02}         % 方正仿宋_GBK

\setCJKfamilyfont{zhsong}{FZShuSong-Z01}
\setCJKfamilyfont{zhxbs}{Source Han Serif SC Bold}
\setCJKfamilyfont{zhdbs}{Source Han Serif SC Heavy}
\setCJKfamilyfont{zhhei}{FZHei-B01}
\setCJKfamilyfont{zhdh}{Source Han Sans SC Bold}
\setCJKfamilyfont{zhfs}{FZFangSong-Z02}
\setCJKfamilyfont{zhkai}{FZKai-Z03}

\newcommand{\songti}{\CJKfamily{zhsong}}
\newcommand{\xbsong}{\CJKfamily{zhxbs}}
\newcommand{\dbsong}{\CJKfamily{zhdbs}}
\newcommand{\heiti}{\CJKfamily{zhhei}}
\newcommand{\dahei}{\CJKfamily{zhdh}}
\newcommand{\fangsong}{\CJKfamily{zhfs}}
\newcommand{\kaishu}{\CJKfamily{zhkai}}


\usepackage{amsmath}
\usepackage{amsthm} % amsthm 与 ntheorem 冲突
\usepackage{amssymb}
\usepackage{hyperref} % \url
\usepackage{graphicx} % \includegraphics
\usepackage{fancybox} % 使用盒子
\usepackage{xcolor} % 使用颜色 color
% \usepackage{amsfonts}  % 
\usepackage{mathrsfs}  % 花体字 mathscr

\usepackage{enumerate}  % 计数器
\usepackage{paralist}  % 计数器 允许行间公式

\usepackage{tikz,lipsum,lmodern}
\usepackage[most]{tcolorbox} % \DeclareTColorBox


\theoremstyle{plain} % default
\newtheorem{thm}{Theorem}[chapter] % 如果不采用章节号做前缀, 则不用[section]
\newtheorem{myLemma}[thm]{Lemma}
\newtheorem{mynewthm}{Theorem}[section] % 如果不采用章节号做前缀, 则不用[section]

\theoremstyle{definition} % definition

\newtheorem{myDef}[thm]{Definition}
\newtheorem{myExample}[thm]{Example}
\newtheorem{myRemark}[thm]{Remark}
\newtheorem{myProposition}[thm]{Proposition}
\newtheorem{myNotation}[thm]{Notation}

\newtheorem*{myCorollary}{Corollary}
\newtheorem{mySolve}{Solve}

\newtheorem{myExercise}{Exercise}[chapter]
% \theoremstyle{definition}
\newtheorem{definition}{Definition}[section]

% 我的笔记环境
\newcommand{\mybox}[1]{
    % \fbox{mynotes:\\#1}
    % \vskip 2.5mm
    % \fbox{\parbox{135mm}{mynotes:\\#1}}
    % \vskip 2.5mm
    \begin{tcolorbox}
        my notes:\\#1
    \end{tcolorbox}
}
\newcommand{\myproof}[1]{\begin{proof}#1\end{proof}}
\newcommand{\mycommand}[2]{\begin{#1}#2\end{#1}}
\newcommand{\mythm}[1]{\mycommand{thm}{#1}}
\newcommand{\mymyDef}[1]{\mycommand{myDef}{#1}}
\newcommand{\mymyExample}[1]{\mycommand{myExample}{#1}}
\newcommand{\mymyRemark}[1]{\mycommand{myRemark}{#1}}
\newcommand{\mymyProposition}[1]{\mycommand{myProposition}{#1}}
\newcommand{\mymyNotation}[1]{\mycommand{myNotation}{#1}}
\newcommand{\mymyCorollary}[1]{\mycommand{myCorollary}{#1}}
\newcommand{\mymySolve}[1]{\mycommand{mySolve}{#1}}
\newcommand{\mymyExercise}[1]{\mycommand{myExercise}{#1}}

%New Commands

%Special Symbols

\newcommand{\R}{{\protect\mathbb R}}
\newcommand{\Cc}{{\protect\mathbb C}}
\newcommand{\K}{{\protect\mathbb K}}
\newcommand{\N}{{\protect\mathbb N}}
\newcommand{\Q}{{\protect\mathbb Q}}
\newcommand{\Z}{{\protect\mathbb Z}}
\newcommand{\sequence}[1]{\protect\{#1\}}
\newcommand{\mybinom}[2]{\protect \binom{#1}{#2}}
\documentclass[openany,twoside,scheme=chinese,fontset=none]{ctexbook}
\usepackage{geometry}
\geometry{
    paperheight=260mm,
    paperwidth=185mm,
    top=25mm,
    bottom=15mm,
    left=25mm, % 左侧留 5mm 装订线距离
    right=15mm
}

\setmainfont{XITS}  % 英文字体, Times 风格

\setCJKmainfont{Source Han Serif SC}[         % 方正书宋_GBK
    BoldFont=Source Han Serif SC Bold,  % 思源宋体粗体
    ItalicFont=FZKai-Z03                % 方正楷体_GBK
    ]
\setCJKsansfont{Source Han Sans SC}[             % 方正黑体_GBK
    BoldFont=Source Han Sans SC Bold    % 思源黑体粗体
    ]
\setCJKmonofont{FZFangSong-Z02}         % 方正仿宋_GBK

\setCJKfamilyfont{zhsong}{FZShuSong-Z01}
\setCJKfamilyfont{zhxbs}{Source Han Serif SC Bold}
\setCJKfamilyfont{zhdbs}{Source Han Serif SC Heavy}
\setCJKfamilyfont{zhhei}{FZHei-B01}
\setCJKfamilyfont{zhdh}{Source Han Sans SC Bold}
\setCJKfamilyfont{zhfs}{FZFangSong-Z02}
\setCJKfamilyfont{zhkai}{FZKai-Z03}

\newcommand{\songti}{\CJKfamily{zhsong}}
\newcommand{\xbsong}{\CJKfamily{zhxbs}}
\newcommand{\dbsong}{\CJKfamily{zhdbs}}
\newcommand{\heiti}{\CJKfamily{zhhei}}
\newcommand{\dahei}{\CJKfamily{zhdh}}
\newcommand{\fangsong}{\CJKfamily{zhfs}}
\newcommand{\kaishu}{\CJKfamily{zhkai}}


\usepackage{amsmath}
\usepackage{amsthm} % amsthm 与 ntheorem 冲突
\usepackage{amssymb}
\usepackage{hyperref} % \url
\usepackage{graphicx} % \includegraphics
\usepackage{enumerate} % 罗列专用宏包
\usepackage{fancybox} % 使用盒子
\usepackage{xcolor} % 使用颜色 color
% \usepackage{amsfonts}  % 
\usepackage{mathrsfs}  % 花体字 mathscr

\theoremstyle{plain} % default
\newtheorem{thm}{Theorem}[chapter] % 如果不采用章节号做前缀, 则不用[section]
\newtheorem{myLemma}[thm]{Lemma}
\theoremstyle{definition} % definition
\newtheorem{myDefinition}[thm]{Definition}
\newtheorem{myExample}[thm]{Example}
\newtheorem{myRemark}[thm]{Remark}
\newtheorem{myProposition}[thm]{Proposition}
\newtheorem{myNotation}[thm]{Notation}
\newtheorem*{myCorollary}{Corollary}
% 我的笔记环境
\newcommand{\mybox}[1]{
    % \fbox{mynotes:\\#1}
    \vskip 2.5mm
    \fbox{\parbox{120mm}{mynotes:\\#1}}
    \vskip 2.5mm
}


\title{baby-rudin reading notes}
\author{weiyuan}
\date{\today}

\begin{document}
    \maketitle
    \mainmatter
    % chap 1 the real and complex number system
\chapter{the real and complex number system}

\section{Introduction}
First we use $\sqrt{2}$ to construct real number system from integer and rational numbers.

% 1.1 Example
\begin{Example}
\begin{equation}\label{eq:1-001}
    p^2=2
\end{equation}
$p$ is not a rational number.
\end{Example}

\begin{proof}
    
% Proof: 
(反证法) 假设 $p$ 是有理数,  $\exists m,n \in \mathbf{N}$, s.t. $p=m/n$. $\gcd (m,n) = 1$.
Then \ref{eq:1-001}

\begin{equation}\label{eq:1-002}
    m^2 = 2n^2.
\end{equation}

$m$ is even, $m = 2k$.
那么有 $(2k)^2 = 2n^2$, $2k^2 = n^2$, $k$ is even, $\gcd (m,n)=2\neq 1$,
contrary to our choice of $m$ and $n$. Hence p can't be a rational number.
\end{proof}

After proving $\sqrt{2}$ isn't a rational number, rudin use $\sqrt{2}$ to divide the rationals
在证明 $\sqrt{2}$ 不是有理数后, 使用 $\sqrt{2}$ 将有理数集分成两部分.  引出了分划的概念? 

\begin{align*}
    A = \{p|p^2<2\}\\
    B = \{p|p^2>2\}
\end{align*}

$A$ \emph{contains no largest number},
$B$ \emph{contains no smallest number}.

$\forall p\in A$, $\exists q\in A$, s.t. $p<q$,
$\forall p\in B$, $\exists q\in B$, s.t. $p>q$,

$\forall p>0$

\begin{equation}\label{eq:1-003}
    q = p-\frac{p^2-2}{p+2} = \frac{2p+2}{p+2}
\end{equation}

Then 
\begin{equation}
    \label{eq:1-004}
    q^2 - 2 = \frac{2(p^2-2)}{(p+2)^2}
\end{equation}

If $p\in A$, $p^2<2$. \ref{eq:1-003} shows that $q>p$, \ref{eq:1-004} shows that $q^2<2$, $q\in A$.
If $p\in B$, $p^2>2$. \ref{eq:1-003} shows that $q<p$, \ref{eq:1-004} shows that $q^2>2$, $q\in B$.


\begin{Remark}
    
% 1.2 Remark

The purpose of the above discussion has been to show that the rational number system has certain gaps, 
in spite of the fact that between any two rationals there is another: If $r<s$ then $r<(r+s)/2<s$.
The real number system fills these gaps.
This is the principal reason for the fundamental role which it plays in analysis.
\end{Remark}

\mybox{
% mynotes:
有理数的稠密性与实数的连续性. 在分析中, 考察极限等需要的是数系的连续性, 因此需要先建立实数系. 
事实上, 我们是先有微积分, 后有实数理论的. 
三次数学危机:
无理数, 微积分基础, 集合论
实数理论是极限的基础. 
}

In order to elucidate its structure, as well as that of the complex numbers, 
we start with a brief discussion of the genral concepts of \emph{ordered set} and \emph{field}.

mynotes:
rudin引入复数的方法非常怪, 对初学者非常不友好, 过于抽象了. 
想起一个法国笑话, 问小学生$2+3$等于几, 回答 $2+3=3+2$ 加法是一个交换群(Abel 群). . . 

Here is some of the standard set-theoretic terminology taht will be used throughout this book.
接下来引入一些集合论的定义

1.3 Definitions
If $A$ is any set (whose elements(这里elements还没定义, 笑啦) may be numbers or any other objects(object指代什么? 我个人认为集合理解的难点在于集合的集合. 这一点可以引出罗素悖论)), we write $x\in A$ to indicate that $x$ is a member (or an element) of $A$.

If $x$ is not a member of $A$, we write: $x\notin A$.

\emph{empty set} $\varnothing$ contains no element, If a set has at least one element, it is called \emph{nonempty}.

$A,B$ are sets, $\forall x\in A$, $x\in B$, we say that $A$ is a \emph{subset} of $B$, $A\subset B$ or $B\supset A$. If $\exists x\in B$, $x\notin A$, A is a \emph{proper subset} of $B$, $A \subsetneqq B$.
Note that $A\subset A$ for every set $A$.

(Bernstein) If $A\subset B$ and $B\subset A$, we write $A = B$. Otherwise $A\neq B$.

mynotes:
这条性质在证明集合相等时很常用

1.4 Definitions
Throughout Chap. 1, the set of all rational numbers will be denoted by $\mathbb{Q}$.

有理数集$\mathbb{Q}$

\section{Ordered sets}
有序集

1.5 Definitions
Let $S$ be a set. An \emph{order} on $S$ is a relation, denoted by $<$, with the following two properties:

(i) If $x\in S$ and $y\in S$ then one and only one of the statements
\begin{equation*}
    x<y, \quad
    x=y, \quad
    y<x
\end{equation*}
The statement $x<y$ may be read as 
$x$ is less than $y$, or 
$x$ is smaller than $y$, or
$x$ precedes $y$.
(It's often convenient to write $y>x$ in place of $x<y$)
(less-great, smaller-bigger, precedes-succeeds)
% form wiki,
% The relationship x precedes y is written $x ≺ y$. The relation x precedes or is equal to y is written x ≼ y.
% The relationship x succeeds (or follows) y is written x ≻ y. The relation x succeeds or is equal to y is written x ≽ y.
% ≺ \prec 
% ≼ \preccurlyeq 
% ≻ \succ 
% ≽ \succcurlyeq  

(ii) If $x,y,z\in S$, if $x<y$ and $y<z$, then $x<z$.

偏序关系
1. 三歧性
2. 传递性

建立偏序关系后, 可以使用不等式进行分析. 在后续根据极限定义计算时, 需要大量使用不等式分析数列和函数的极限计算结果. 


$x\leq y$ indicates taht $x<y$ or $x=y$, without specofying which of these two is to hold.
In other words, $x\leq y$ is the negation of $x>y$.

1.6 Definitions
An \emph{ordered set} is a set $S$ in which an order is defined.

For example, $\mathbb{Q}$ is an ordered set if $r<s$ is defined to mean that $s-r$ is a positive rational number.

存在偏序关系的集合称为有序集
$\mathbb{Q}, \mathbb{R}$ 均是有序集, 但$\mathbb{C}$ 不是有序集. 

1.7 Definitions (bounded above)
Suppose $S$ is an ordered set, and $E \subset S$. If there exists a
$\beta \in S$ such that $x \leq \beta$ for every $x \in E$, we say that $E$ is \emph{bounded above}, and call
$\beta$ an \emph{upper bound} of $E$.

Lower bounds are defined in the same way (with $\geq$ in place of $\leq$).


1.8 Definitions (least upper bound)
Suppose $S$ is an ordered set, $E \subset S$, and $E$ is bounded above.
Suppose there exists an $a\alpha \in S$ with the following properties:

(i) $\alpha$ is an upper bound of $E$.
(ii) If $\gamma <\alpha$ then $\gamma$ is not an upper bound of $E$.

Then $\alpha$ is called the \emph{least upper bound} of $E$ [that there is at most one such
$\alpha$ is clear from (ii)] or the \emph{supremum} of $E$, and we write

\begin{equation*}
    \alpha = \sup E.
\end{equation*}

The \emph{greatest lower bound}, or \emph{infimum}, of a set $E$ which is bounded below
is defined in the same manner: The statement

\begin{equation*}
    \alpha = \inf E
\end{equation*}

means that $\alpha$ is a lower bound of $E$ and that no $\beta$ with $\beta > \alpha$ is a lower bound
of $E$.

从上界引出最小上界, 没有直接定义最大下界, 而是使用对称定义引出. 
从最小上界引出的最小上界性质更为常用. Dedekind分划

1.9 example
(a) Consider the set $A, B$
\begin{equation*}
    A = \{p|p^2 < 2\},\quad
    B = \{p|p^2 > 2\}.
\end{equation*}
$A$ has no least upper bound in $\mathbb{Q}$.
$B$ has no great lower bound in $\mathbb{Q}$.

(b) If $\alpha = \sup E$ exists, $\alpha\in E$ or $\alpha \notin E$.
\begin{align*}
    E_1 = \{r |r\in Q, r < 0\}\\
    E_2 = \{r |r\in Q, r \leq 0\}
\end{align*}
\begin{equation*}
    \sup E_1 = \sup E_2 = 0,
\end{equation*}
and $0\not\in E_1$, $0\in E_2$.

(c) $E = \{1/n | n = 1,2,3,...\}$. Then $\sup E = 1$, which is in $E$, and $\inf E = 0$, which is not in $E$.

1.10 Definitions (least-upper-bound property)(important!!)
An ordered set $S$ is said to have the \emph{least-upper-bound property}
if the following is true:

If $E \subset S$, $E$ is not empty, and $E$ is bounded above, then $\sup E$ exists in $S$.

Example 1.9(a) shows that $\mathbb{Q}$ does not have the least-upper-bound property.

We shall now show that there is a close relation between greatest lower
bounds and least upper bounds, and that every ordered set with the least-upper-bound property also has the greatest-lower-bound property.


1.11 Theorem 
Suppose $S$ is an ordered set with the least-upper-bound property,
$B \subset S$, $B$ is not empty, and $B$ is bounded below. Let $L$ be the set of all lower
bounds of $B$. Then

\begin{equation*}
    \alpha = \sup L
\end{equation*}

exists in $S$, and $\alpha = \inf B$.

In particular, $\inf B$ exists in $S$.

Proof 
Since $B$ is bounded below, $L$ is not empty. Since $L$ consists of
exactly those $y \in S$ which satisfy the inequality $y \leq x$ for every $x \in B$, we
see that \emph{every} $x \in B$ \emph{is an upper bound of} $L$. Thus $L$ is bounded above.
Our hypothesis about $S$ implies therefore that $L$ has a supremum in $S$;
call it $\alpha$.

If $\gamma <\alpha$ then (see Definition 1.8) $\gamma$ is not an upper bound of $L$,
hence $\gamma \not\in B$. It follows that $\alpha \leq x$ for every $x \in B$. Thus $\alpha \in L$.

If $\alpha < \beta$ then $\beta \not\in L$, since $\alpha$ is an upper bound of $L$.

We have shown that $\alpha \in L$ but $\beta \not\in  L$ if $\beta>\alpha$. In other words, $\alpha$
is a lower bound of $B$, but $\alpha$ is not if $\beta > \alpha$. This means that $\alpha = \inf B$.

mynotes
这个证明第一次看比较难理清
我试着用自己的话重写梳理一下:
已知条件
$S$, ordered set + least-upper-bound property.
$B\in S$, $B\neq \varnothing $, $B$ is bounded below.
$L$ is the set of all lower bounds of $B$.

$\exists \alpha\in S$, $\alpha = \sup L$, and $\alpha = \inf B$.

proof:
思路 由最小上界 $\rightarrow $ 最大下界

% \begin{align*}
%     \text{最小上界}  & \rightarrow  &\text{最大下界} \\
%     \downarrow      &               &\uparrow \\
%     L\text{最小上界}  & \rightarrow  &B\text{最大下界} \\
% \end{align*}

$L = \{y| y\in S; \forall x\in B, y\leq x\}$
    关于 $L$ 中有没有不在 $S$ 中的元素这一点我还没想明白. 定理中只是说 $L$ 是 $B$ 的下界组成的. $B$ 是 $S$ 的子集, 但 $B$ 的下界不一定全在 $S$ 中. 

$L$ 由 $B$ 在 $S$ 中的全部下界组成

$\forall x\in B$, $x$ 为 $L$ 的上界. $L\subset S$.
$S$ 有最小上界性质,
$\therefore \exists \alpha\in S$, $\alpha = \sup L$.

$\forall \gamma <\alpha$ 由 $\alpha = \sup L$ 的定义 (Definitions 1.8)
$\gamma$ 不是 $L$ 的上界.

$\forall x \in B$, $x$ 为 $L$ 的上界, $x \geq \alpha$. $\therefore \alpha \in L$.

$\alpha < \beta$, $\alpha = \sup L$. $\therefore \beta \not\in L$.
$L$ 由 $B$ 在 $S$ 中的全部下界组成, $\beta \not\in L$.
$\beta$ 不是 $B$ 的下界.

$\therefore \alpha = \inf B$, $\inf B\in S$.


\section{fields}
域, 交换除环 $<\mathbb{R},+,\times>$ 
$<\mathbb{R},+>$, $<\mathbb{R}\backslash\{0\},\times>$
都是交换群, 且满足分配律. 
则 $<\mathbb{R},+,\times>$ 是域. 

axiom 公理

(A) Axioms for addition

(Al) If $x\in F$  and $y \in F$, then their sum \(x + y\) is in F.

(A2) Addition is commutative: \(x + y=y+ x\) for all \(x, y \in F\).

(A3) Addition is associative: \((x+ y)+z = x + (y+ z)\) for all \(x, y, z \in F\).

(A4) $F$ contains an element $0$ such that $0 + x = x$ for every $x \in F$.

(A5) To every $x\in F$ corresponds an element $-x\in F$ such that

\begin{equation*}
    x+(-x)=0.
\end{equation*}

(M) Axioms for multiplication

(M1) If $x\in F$ and $x\in F$, then their product $xy$ is in $F$.

(M2) Multiplication is commutative: $xy = yx$ for all $x, y \in  F$.

(M3) Multiplication is associative: $(xy)z = x(yz)$ for all $x, y, z \in  F$.

(M4) $F$ contains an element $1 \neq 0$ such that $1x = x$ for every $x \in F$.

(M5) If $x \in F$ and $x \neq 0$ then there exists an element $1/x \in F$ such that

\begin{equation*}
    x\cdot(1/x)=1.
\end{equation*}
% 6 PRINCIPLES OF MATHEMATICAL ANALYSIS

(D) The distributive law

\begin{equation*}
    x(y+z)=xy+ xz
\end{equation*}

holds for all $x, y, z \in F$.

1.13 Remark

(a) Our usual writes (in any filed)

只定义了加法和乘法, 使用逆元分别表示减法和除法.
$x-y = x+(-y)$, $x/y=x\cdot (1/y)$.

(b) The field axioms clearly hold in $\mathbb{Q}$, the set of all rational numbers, if
addition and multiplication have their customary meaning. Thus $\mathbb{Q}$ is a
field.

全体有理数的集合是一个域.

(c) Although it is not our purpose to study fields (or any other algebraic
structures) in detail, it is worthwhile to prove that some familiar properties
of $\mathbb{Q}$ are consequences of the field axioms; once we do this, we will \underline{not
need to do it} again for the real numbers and for the complex numbers.

1.14 Proposition

The axioms for addition imply the following statements.

(a) If $x+y=x+z$ then $y=z$.
(b) If $x+y=x$ then $y=0$.
(c) If $x+y=0$ then $y= -x$.
(d) $-(-x)=x$.

Statement (a) is a cancellation law. Note that (b) asserts the uniqueness
of the element whose existence is assumed in (A4), and that (c) does the same
for (A5).

mynotes
what is the difference between axiom and proposition?
An axiom is a proposition regarded as self-evidently true without proof. The word "axiom" is a slightly archaic synonym for postulate. Compare conjecture or hypothesis, both of which connote apparently true but not self-evident statements.
A proposition is a mathematical statement such as "3 is greater than 4," "an infinite set exists," or "7 is prime." An axiom is a proposition that is assumed to be true. With sufficient information, mathematical logic can often categorize a proposition as true or false, although there are various exceptions (e.g., "This statement is false").
\url{https://www.nutritionmodels.com/terminology.html}


Proof(rudin)
If $x + y =x + z$, the axioms (A) give

\begin{align*}
    y =0+y&=(-x+x)+y=-x+(x+y)\\
    &=-x+(x+z)=(-x+x)+z=0+z=z
\end{align*}

This proves (a). Take $z = 0$ in (a) to obtain (b). Take $z= -x$ in (a) to
obtain (c).
Since $-x + x = 0$, (c) (with $-x$ in place of $x$) gives (d).

mynotes 我自己证明上述四条性质时都是从定义开始的, 而 rudin 这里在后一步的证明中都利用了刚推导出的结论, 这一点需要借鉴.
% THE REAL AND COMPLEX NUMBER SYSTEMS 7

1.15 Proposition 
The axioms for multiplication imply the following statements.

(a) If $x\neq0$ and $xy=xz$ then $y=z$.

(b) If $x\neq0$ and $xy=x$ then $y=1$.

(c) If $x\neq0$ and $xy=1$ then $y=1/x$.

(d) If $x\neq0$ then $1/(1/x) = x$.


The proof is so similar to that of Proposition 1.14 that we omit it.

mynotes
Proof
(a) 
\begin{align*}
    y&=1\cdot y=\left(\frac{1}{x}\cdot x\right)y =\frac{1}{x}\left( xy \right)\\
    &=\frac{1}{x}(xz) =\left(\frac{1}{x}x\right)z = z
\end{align*}

    % chap02
\chapter{Basic topology}
% chap02sec01
\section{Finite, countable, and uncountable sets}

We begin this section with a definition of the \myKeywordblue{function} concept.

\mybox{
    Function

    \url{https://mathworld.wolfram.com/Function.html}


A function is a relation that uniquely associates members of one set with members of another set. 
More formally, a function from $A$ to $B$ is an object $f$ such that every $a$ in $A$ is uniquely associated with an object $f(a)$ in $B$. 
A function is therefore a many-to-one (or sometimes one-to-one) relation. 
The set $A$ of values at which a function is defined is called its domain, 
while the set $f(A)$ subset $B$ of values that the function can produce is called its range. 
Here, the set $B$ is called the codomain of $f$.

In the context of univariate, real-valued functions $f:A \subset \R\rightarrow \R$, 
the fact that domain elements are mapped to unique range elements can be expressed graphically by way of the vertical line test.

In some literature, the term 
``map''
 is synonymous with function. 
Some caution must be exhibited, however, as it is not uncommon for the term map to denote a function with some sort of unspoken regularity assumption, 
e.g., in point-set topology, where 
``map''
 sometimes refers to a function which is continuous with respect to some topology.
}

\mybox{

\begin{center}
    \begin{tikzpicture}[x=0.7pt,y=0.7pt,yscale=-1,xscale=1]
    %uncomment if require: \path (0,300); %set diagram left start at 0, and has height of 300
    
    %Shape: Axis 2D [id:dp21476035781689284] 
    \draw  (165.52,146.23) -- (314.72,146.23)(202.6,85.23) -- (202.6,213.23) (307.72,141.23) -- (314.72,146.23) -- (307.72,151.23) (197.6,92.23) -- (202.6,85.23) -- (207.6,92.23)  ;
    %Shape: Wave [id:dp7563242846602689] 
    \draw   (128.6,145.93) .. controls (132.68,149.37) and (136.58,152.63) .. (141.1,152.63) .. controls (145.62,152.63) and (149.52,149.37) .. (153.6,145.93) .. controls (157.68,142.5) and (161.58,139.23) .. (166.1,139.23) .. controls (170.62,139.23) and (174.52,142.5) .. (178.6,145.93) .. controls (182.68,149.37) and (186.58,152.63) .. (191.1,152.63) .. controls (195.62,152.63) and (199.52,149.37) .. (203.6,145.93) .. controls (207.68,142.5) and (211.58,139.23) .. (216.1,139.23) .. controls (220.62,139.23) and (224.52,142.5) .. (228.6,145.93) .. controls (232.68,149.37) and (236.58,152.63) .. (241.1,152.63) .. controls (245.62,152.63) and (249.52,149.37) .. (253.6,145.93) .. controls (257.68,142.5) and (261.58,139.23) .. (266.1,139.23) .. controls (270.62,139.23) and (274.52,142.5) .. (278.6,145.93) .. controls (282.68,149.37) and (286.58,152.63) .. (291.1,152.63) .. controls (294.73,152.63) and (297.96,150.53) .. (301.2,147.92) ;
    %Shape: Parabola [id:dp3307772220116144] 
    \draw   (167.6,53.23) .. controls (190.93,177.23) and (214.27,177.23) .. (237.6,53.23) ;
    %Straight Lines [id:da017349586183422416] 
    \draw    (148.3,200.53) -- (256.9,91.93) ;
    
\end{tikzpicture}
\end{center}

Examples of functions over the reals $\R$ include $\sin x$ (many-to-one), $x$ (one-to-one), $x^2$ (two-to-one except for the single point $x=0$), etc.

Unfortunately, the term ``function'' is also used to refer to relations that map single points in the domain to possibly multiple points in the range. 
These ``functions'' are called multivalued functions (or multiple-valued functions), and arise prominently in the theory of complex functions, 
where the presence of multiple values engenders the use of so-called branch cuts.

Several notations are commonly used to represent (non-multivalued) functions. 
The most rigorous notation is $f:x\rightarrow f(x)$, which specifies that f is function acting upon a single number $x$ (i.e., f is a univariate, or one-variable, function) and returning a value $f(x)$. 
To be even more precise, a notation like ``$f:R\rightarrow R$, where $f(x)=x^2$''
 is sometimes used to explicitly specify the domain and codomain of the function. 
The slightly different 
``maps to''
 notation $f:x|\rightarrow f(x)$ is sometimes also used when the function is explicitly considered as a 
``map''.


Generally speaking, the symbol $f$ refers to the function itself, while $f(x)$ refers to the value taken by the function when evaluated at a point $x$. 
However, especially in more introductory texts, the notation $f(x)$ is commonly used to refer to the function $f$ itself (as opposed to the value of the function evaluated at $x$). 
In this context, the argument $x$ is considered to be a dummy variable whose presence indicates that the function $f$ takes a single argument (as opposed to $f(x,y)$, etc.). 
While this notation is deprecated by professional mathematicians, it is the more familiar one for most nonprofessionals. 
Therefore, unless indicated otherwise by context, the notation $f(x)$ is taken in this work to be a shorthand for the more rigorous $f:x\rightarrow f(x)$.
}

\begin{mydef}
    \label{mydef:2.1}
    Consider two sets $A$ and $B$, whose elements may be any objects whatsoever, 
    and suppose that with each element $x$ of $A$ there is associated, 
    in some manner, an element of $B$, which we denote by $f(x)$. 
    Then $f$ is said to be a \myKeywordblue{function} from $A$ to $B$ (or a \myKeywordblue{mapping} of $A$ into $B$). 
    The set $A$ is called the \myKeywordblue{domain} of $f$ (we also say $f$ is defined on $A$), 
    and the elements $f(x)$ are called the \myKeywordblue{values} of $f$.
    The set of all values of $f$ is called the \myKeywordblue{range} of $f$.
\end{mydef}
\mybox{\myKeywordblue{Codomain}:  A set within which the values of a function lie (as opposed to the range, which is the set of values that the function actually takes). 

\myKeywordblue{Range}:   If $f:D\rightarrow Y$ is a map (a.k.a. function, transformation, etc.) over a domain $D$, 
then the range of $f$, also called the image of $D$ under $f$, 
is defined as the set of all values that $f$ can take as its argument varies over $D$, i.e.,
\begin{equation*}
    \operatorname{Range}(f)=f(D)={f(\mathbf{X}):\mathbf{X} \in D}.
\end{equation*}

Note that among mathematicians, the word 
``image''
 is used more commonly than 
``range.''


The range is a subset of $Y$ and does not have to be all of $Y$.

Unfortunately, term 
``range''
 is often used to mean domain--its precise opposite--in probability theory, with Feller (1968, p.200) and Evans et al. (2000, p.5) calling the set of values that a variate $X$ can assume (i.e., the set of values $x$ that a probability density function $P(x)$ is defined over) the 
``range''
, denoted by $R_X$ (Evans et al. 2000, p.5).

Even worse, statistics most commonly uses 
``range''
 to refer to the completely different statistical quantity as the difference between the largest and smallest order statistics. In this work, this form of range is referred to as 
``statistical range.''
 
}

\begin{mydef}
    \label{mydef:2.2}
    Let $A$ and $B$ be two sets and let $f$ be a mapping of $A$ into $B$.
    If $E \subset A$, $f(E)$ is defined to be the set of all elements $f(x)$, for $x \in E$. We call $f(E)$ the image of $E$ under $f$. In this notation, $f(A)$ is the range of $f$. It is clear that $f(A) \subset B$. If $f(A) = B$, we say that $f$ maps $A$ \myKeywordblue{onto} $B$. (Note that, according
    to this usage, \myKeywordblue{onto} is more specific than \myKeywordblue{into}.)
    \mybox{onto 满射? into 映射?} 

    If $E \subset B$, $f^{-1}(E)$ denotes the set of all $x \in A$ such that $f(x)\in E$. We call $f^{-1}(E)$ the \myKeywordblue{inverse image} of $E$ under $f$. If $y \in B$, $f^{-1}(y)$ is the set of all $x \in A$ such that $f(x) =y$. If, for each $y\in B$, $f^{-1}(y)$ consists of at most one element of $A$, then $f$ is said to be a 1-1 (\myKeywordblue{one-to-one}) mapping of $A$ into $B$. This may also be expressed as follows: $f$ is a 1-1 mapping of $A$ into $B$ provided that $f(x_1) \neq f(x_2)$ whenever $x_1 \neq x_2$, $x_1 \in A$, $x_2 \in A$.

    (The notation $x_1 \neq x_2$, means that $x_1$ and $x_2$ are distinct elements; otherwise we write $x_1 = x_2$.)
\end{mydef}

\begin{mydef}
    \label{mydef:2.3}
    If there exists a 1-1 mapping of $A$ \myKeywordblue{onto} $B$, we say that $A$ and $B$ can be putin 1-1 correspondence, or that $A$ and $B$ have the same cardinal number, or, briefly, that $A$ and $B$ are equivalent, and we write $A\sim B$. This relation
    clearly has the following properties :

    It is reflexive: $A\sim A$.

    It is symmetric: If $A\sim B$, then $B\sim A$.

    It is transitive: If $A\sim B$ and $B\sim C$, then $A\sim C$.

    Any relation with these three properties is called an equivalence relation.    
\end{mydef}
\mybox{等价关系:
reflexive   自反性,
symmetric   对称性,
transitive  传递性.

集合等势是一种等价关系, 其满足自反性, 对称性, 传递性.}

\begin{mydef}
    \label{mydef:2.4}
    $\forall n\in \mathbb{N}^+$, $J_n = \{1,2,...,n\}$, $J = \{1,2,...,n,...\}$, (set consisting of all positive integers).

    $A$ is finite, $A\sim J_n$ for some n,

    $A = \varnothing$. empty set is also considered to be finite.

    $A$ is infinite, $A$ is not finite.

    $A$ is countable, $A \sim J$
    
    $A$ is uncountable. $A$ is neither finite nor countable.

    countable set and finite set are called at most countable.
\end{mydef}

\mybox{
    \begin{equation*}
        \left\{
        \begin{array}{lll}
        finite & A\sim J_n\\
        infinite &\left\{
            \begin{array}{ll}
                countable& A\sim J\\
                uncountable& \\
            \end{array}
        \right.
        \end{array}
        \right.
    \end{equation*}
    % todo 添加tikz注释
}

countable sets, enumerable, denumerable.

$A, B \in$ finite set\\
$A\sim B$ $\Longleftrightarrow$ $A, B$ contains same number of elements

$A, B \in$ infinite set\\
same number or elements? vague\\
1-1 correspondence. retains its clarity.

\begin{myExample}
    $f:J\rightarrow A$
    \begin{equation*}
        f(n) = \left\{
            \begin{array}{ll}
                \cfrac{n}{2} & (n \text{even})\\
                -\cfrac{n-1}{2} & (n \text{odd})
            \end{array}
        \right.
    \end{equation*}
\end{myExample}
\mybox{$f(n)=(-1)^n\left\lfloor \cfrac{n}{2} \right\rfloor $}

\begin{myRemark}
    a finite set cannot be equivalent to one of its proper subsets, but it's possible for infinite sets.
\end{myRemark}

$J = 1,2,3,4,...$, $A = 0,1,-1,2,-2,...$,$J, A$are infinite sets, $J \subset A$.\\
but there exist a function $f:J\rightarrow A$, $J \sim A$

\begin{mydef}
    \label{mydef:2.7}
    $f(x)$, $x\in J = \mathbb{N}^+$.\\
    $\{x_n\}$, $x_1,x_2,x_3,...$\\
    $x_n$, terms of the sequence.\\
    $\forall n\in J$, $x_n\in A$, $\{x_n\}$ is a sequence in $A$, or a sequence of elements of $A$.
\end{mydef}

every countable set is range of a sequence of distinct terms.
the elements of any countable set can be ``arranged in a sequence''.
replace $J(\mathbb{N}^+)$ by $\mathbb{N} = \{x|, x\in Z,x \geq 0\}$, start with $0$ rather than $1$.

\begin{thm}
    \label{thm:2.8}
    Every infinite subset of a countable set $A$ is countable
\end{thm}

$E\subset A$. $E$ is infinite. 
To prove $E$ is countable, we need a 1-1 correspondation of $J$ to $E$, $f:J\rightarrow E$.
\mybox{
    my first guess is $A$ is a countable set, $A\sim J$ (by def).
    $\exists$ 1-1 mapping $g:$ $J$ onto $A$.
    $x\in J$, $g(x)\in A$.
    $E\subset A$, $\exists  g(x)\in E$.
    $g(x_i)\in E$, $x_i\in J$, $g:J\rightarrow E$.\\
    再证 $x_i$ 不是有限的. $E$ is infinite, there exist infinite $g(x_i)\in E$. $\because g$ is a 1-1 mapping, $\{x_i\}$ is infinite. $\therefore J\sim E$.
}

\begin{proof}
    Suppose $E\subset A$, $E$ is infinite. 
    arrange the elements $x$ of $A$ in a sequence $\{x_n\}$ of a distinct elements. Construct a sequence $n_k$ as follows.\\
    Let $n_1$ be the smallest positive int, s.t. $x_{n_1}\in E$.
    Having chosen $n_1,...n_{k-1}$,$(k=2,3,...)$, let $n_k$ be the smallest integer greater than $n_{k-1}$, s.t. $x_{n_k} \in E$.\\
    Putting $f(k) = x_{n_k}$, $f:J\rightarrow E$ is a 1-1 mapping.
\end{proof}

Countable sets represent the ``smallest'' infinity.

No uncountable set ca be a subset of a countable set.

\mybox{
    rudin 这里尝试区分实无穷与浅无穷, 
    使用集合的势来说明更为具体, 全体整数组成的集合为 ``最小'' 的无穷大, 
    其势为$\aleph_0$, 康托尔使用一一对应关系作为无穷集合之间的等价关系.
    }

\begin{mydef}
    \label{mydef:2.9}
    $\forall \alpha\in A$, $E_\alpha \subset \Omega$, $\{E_\alpha\}$ debites elements of $E_\alpha$. 
    collection of sets (or family of sets)  
    \mybox{sets of sets sounds strange}  
    union
    \begin{equation}
        \label{eq:2.1}
        S = \bigcup_{\alpha\in A} E_\alpha
    \end{equation}
    if $A$ consists of the integers $1,2,...,n$.
    \begin{equation}
        \label{eq:2.2}
        S = \bigcup_{m=1}^n E_m
    \end{equation}
    \begin{equation}
        \label{eq:2.3}
        S = E_1 \bigcup E_2 \bigcup \cdots \bigcup E_n.
    \end{equation}
    if $A$ is the set of all positive integers.
    \begin{equation}
        \label{eq:2.4}
        S = \bigcup_{m=1}^{\infty} E_m.
    \end{equation}
    intersection
    \begin{equation}
        \label{eq:2.5}
        P = \bigcap_{\alpha\in A} E_\alpha
    \end{equation}
    \begin{equation}
        \label{eq:2.6}
        S = \bigcap_{m=1}^n E_m = E_1 \cap E_2 \cap \cdots \cap E_n.
    \end{equation}
    \begin{equation}
        \label{eq:2.7}
        S = \bigcap_{m=1}^{\infty} E_m.
    \end{equation}

    $A$ and $B$ intersect if $A\bigcap B$ is not empty, otherwise they are disjoint.
\end{mydef}

\begin{myExample}
    some example of set relation
\end{myExample}

\begin{myRemark}
    Many properties of unions and intersections are quite similar to those of sums and products; in fact, the words sum and product were sometimes used in this connection, and the symbols $\sum$ and $\prod$ were written in place of $\bigcup$ and $\bigcap$.
\end{myRemark}

The commutative and associative laws are trivial:
\begin{align}
        A \cup B &= B \cup A; &
        A \cap B &= B \cap A \label{eq:2.8} \\
        \left(A \cup B\right) \cup C &= A \cup \left(B \cup C\right); &
        \left(A \cap B\right) \cap C &= A \cap \left(B \cap C\right);\label{eq:2.9}
\end{align}

Thus the omission of parentheses in \ref{eq:2.3} and \ref{eq:2.6} is justified.

The distributive law also holds:
\begin{equation}
    \label{eq:2.10}
    A \cap \left( B \cup C\right) = 
    \left(A \cap B\right) \cup \left(A \cap C\right).
\end{equation}
To prove this, let the left and right members of \ref{eq:2.10} be denoted by $E$ and $F$, respectively.

Suppose $x \in E$. Then $x \in A$ and $x \in B \cup C$, that is, $x \in B$ or$ x \in C$ (possibly both). Hence $x \in A\cap B$ or $x \in A\cap C$, so that $x \in F$. Thus $E \subset F$.

Next, suppose $x \in F$. Then $x \in A\cap B$ or $x \in A\cap C$. That is, $x \in A$, and $x \in B\cup C$. Hence $x \in A\cap \left(B \cup C\right)$, so that $F \subset E$.

It follows that $E = F$.

We list a few more relations which are easily verified:

\begin{align}
    A \subset A \cup B, \label{eq:2.11}\\
    A \cap B \subset B, \label{eq:2.12}
\end{align}

If $0$ denotes the empty set, then
\begin{equation}
    \begin{array}{cc}
        A \cup 0 = A, & A \cap 0 = 0.
    \end{array}
\end{equation}
If $A \subset B$, then
\begin{equation}
    \begin{array}{cc}
        A \bigcup B = B, & A \bigcap B = A.
    \end{array}
\end{equation}

\mybox{现在一般使用 $\varnothing$ 指代空集}


\begin{thm}
    \label{thm:2.12}
    Let $\{E_n\}, n=1,2,3,...,$ be a sequence of countable sets, and put
    \begin{equation}
        \label{eq:2.15}
        S = \bigcup_{n=1}^{\infty} E_n.
    \end{equation}
    Then S is countable.
\end{thm}
\mybox{
    将 $E_n$ 按顺序排成一张表格, 按反对角线重新排列成新的序列, 
    得到 $T$, $S\sim T$.
    $S$ is at most countable.
    同时存在无限集合(infinite set) $E_1$, $E_1 \subset S$, 
    $S$ is countable.
}

\begin{proof}
    Let every set $E_n$ be arranged in a sequence $\sequence{x_{nk}}$, 
    $k = 1,2,3,\dots$,
    and consider the infinite array
    \begin{equation}
        \label{eq:2.16}
        \begin{array}{ccccc}
            x_{11} & x_{12} & x_{13} & x_{14} & \cdots \\  
            x_{21} & x_{22} & x_{23} & x_{24} & \cdots \\  
            x_{31} & x_{32} & x_{33} & x_{34} & \cdots \\  
            x_{41} & x_{42} & x_{43} & x_{44} & \cdots \\  
            \cdots & \cdots & \cdots & \cdots & \cdots \\
        \end{array}
    \end{equation}
    in which the elements of $E_n$ form the $n$th row. 
    The array contains all elements of $S$.
    As indicated by the arrows, 
    these elements can be arranged in a sequence
    \begin{equation}
        \label{eq:2.17}
        x_{11};
        x_{21}, x_{12};
        x_{31}, x_{22}, x_{13};
        x_{41}, x_{32}, x_{23}, x_{14};
        \cdots
    \end{equation}
    
    If any two of the sets En have elements in common, 
    these will appear more than once in (\ref{eq:2.17}). 
    Hence there is a subset $T$ of the set of all positive integers 
    such that $S \sim T$, 
    which shows that $S$ is at most countable (Theorem \ref{thm:2.8}). 
    Since $E_1 \subset S$, and $E_1$ is infinite, 
    $S$ is infinite, and thus countable.
\end{proof}

\begin{myCorollary*}
    Suppose $A$ is at most countable, and, for every $\alpha \in A, B$, is at most countable. Put
    \begin{equation*}
        T = \bigcup_{\alpha\in A} B_\alpha.
    \end{equation*}
    Then T is at most countable.
\end{myCorollary*}

For $T$ is equivalent to a subset of \ref{eq:2.15}.

\begin{thm}
    \label{thm:2.13}
    Theorem Let $A$ be a countable set, and let $B_n$ be the set of all $n$-tuples $(a_1, ...,a_n)$, where $a_k \in  A (k=1,...,n)$, and the elements $a_1, ...,a_n$ need not be distinct. Then $B_n$ is countable.
\end{thm}

\begin{proof}
    That $B_1$ is countable is evident, since $B_1 = A$. 
    Suppose $B_{n-1}$ is countable $(n = 2, 3, 4, ... )$. 
    The elements of $B_n$ are of the from
    \begin{equation}
        \label{eq:2.18}
        (b,a)
        \quad
        (b \in B_{n-1},a \in A).
    \end{equation}
    For every fixed $b$, the set of pairs $(b, a)$ is equivalent to $A$, and hence countable. 
    Thus $B_n$ is the union of a countable set of countable sets. 
    By Theorem \ref{thm:2.12}, Bn is countable.
The theorem follows by induction.
\end{proof}

\begin{myCorollary*}
    The set of all rational numbers is countable.
\end{myCorollary*}

\begin{proof}
    We apply Theorem \ref{thm:2.13}, 
    with $n = 2$, noting that every rational $r$ is of the form $b / a$, 
    where $a$ and $b$ are integers. 
    The set of pairs $(a, b)$, 
    and therefore the set of fractions $b / a$, is countable.
\end{proof}

In fact, even the set of all algebraic numbers is countable (see Exercise 2).

That not all infinite sets are, however, countable, is shown by the next
theorem.

\begin{thm}
    \label{thm:2.14}
    Theorem Let $A$ be the set of all sequences whose elements are the digits $0$ and $1$. This set $A$ is uncountable. 
\end{thm}

The elements of $A$ are sequences like $1, 0, 0, 1, 0, 1, 1, 1, ... .$

\begin{proof}
    Let $E$ be a countable subset of $A$, 
    and let $E$ consist of the sequences $s_1, s_2 , s_3 , ...$. 
    We construct a sequences as follows. 
    If the $n$th digit in $s_n$ is $1$, 
    we let the $n$th digit of $s$ be $0$, and vice versa. 
    Then the sequence $s$ differs from every member of $E$ in at least one place; hence $s \not\in E$. 
    But clearly $s \in A$, so that $E$ is a proper subset of $A$.

    We have shown that 
    every countable subset of $A$ is a proper subset of $A$. 
    It follows that $A$ is uncountable 
    (for otherwise $A$ would be a proper subset of $A$, which is absurd).
\end{proof}

The idea of the above proof was first used by Cantor, 
and is called Cantor's diagonal process; 
for, if the sequences $s_1, s_2 , s_3 ,\dots$ are placed in an array like (\ref{eq:2.16}), 
it is the elements on the diagonal which are involved in the construction of the new sequence.

Readers who are familiar with the binary representation of the real numbers (base 2 instead of 10) will notice that 
Theorem \ref{thm:2.14} implies that the set of all real numbers is uncountable. 
We shall give a second proof of this fact in Theorem \ref{thm:2.43}.
\end{document}