\documentclass[openany,twoside,scheme=chinese,fontset=none]{ctexbook}
\usepackage{geometry}
\geometry{
    paperheight=260mm,
    paperwidth=185mm,
    top=25mm,
    bottom=15mm,
    left=25mm, % 左侧留 5mm 装订线距离
    right=15mm
}

\setmainfont{XITS}  % 英文字体, Times 风格

\setCJKmainfont{Source Han Serif SC}[         % 方正书宋_GBK
    BoldFont=Source Han Serif SC Bold,  % 思源宋体粗体
    ItalicFont=FZKai-Z03                % 方正楷体_GBK
    ]
\setCJKsansfont{Source Han Sans SC}[             % 方正黑体_GBK
    BoldFont=Source Han Sans SC Bold    % 思源黑体粗体
    ]
\setCJKmonofont{FZFangSong-Z02}         % 方正仿宋_GBK

\setCJKfamilyfont{zhsong}{FZShuSong-Z01}
\setCJKfamilyfont{zhxbs}{Source Han Serif SC Bold}
\setCJKfamilyfont{zhdbs}{Source Han Serif SC Heavy}
\setCJKfamilyfont{zhhei}{FZHei-B01}
\setCJKfamilyfont{zhdh}{Source Han Sans SC Bold}
\setCJKfamilyfont{zhfs}{FZFangSong-Z02}
\setCJKfamilyfont{zhkai}{FZKai-Z03}

\newcommand{\songti}{\CJKfamily{zhsong}}
\newcommand{\xbsong}{\CJKfamily{zhxbs}}
\newcommand{\dbsong}{\CJKfamily{zhdbs}}
\newcommand{\heiti}{\CJKfamily{zhhei}}
\newcommand{\dahei}{\CJKfamily{zhdh}}
\newcommand{\fangsong}{\CJKfamily{zhfs}}
\newcommand{\kaishu}{\CJKfamily{zhkai}}


\usepackage{amsmath}
\usepackage{amsthm}
\usepackage{amssymb}
\usepackage{hyperref} % \url
\usepackage{graphicx} % \includegraphics


\usepackage{enumerate}%罗列专用宏包
\usepackage{fancybox} %使用盒子
 

% \newtheorem{example}{例}

% https://jingyan.baidu.com/article/09ea3ede21248cc0afde3944.html
% \newtheorem{thm}{Theorem}[section] % 如果不采用章节号做前缀, 则不用[section]
\newtheorem{thm}{Theorem}[chapter] % 如果不采用章节号做前缀, 则不用[section]
\newtheorem{Definition}[thm]{Definition} % 这句定义使得 defn 环境和 thm 共享编号
\newtheorem{Lemma}[thm]{Lemma}% 这句定义使得 lem 环境和 thm 共享编号
\newtheorem{Example}[thm]{Example}% 这句定义使得 example 环境和thm 共享编号
\newtheorem{Remark}[thm]{Remark}
\newtheorem{Axiom}[thm]{Axiom}
\newtheorem{Proposition}[thm]{Proposition}

\newcommand{\mybox}[1]{
    % \fbox{mynotes:\\#1}
    \fbox{\parbox{120mm}{mynotes:\\#1}}
    \vskip 5mm
}

\title{babyrudin reading notes}
\author{weiyuan}
\date{\today}

\begin{document}
    \maketitle
    \mainmatter
    % chap 1 the real and complex number system
\chapter{the real and complex number system}
\section{Introduction}
First we use $\sqrt{2}$ to construct real number system from integer and rational numbers.

\begin{myExample}
    \label{Example:1.1}
    \begin{equation}
        \label{eq:1.1}
        p^2=2
    \end{equation}
    $p$ is not a rational number.
\end{myExample}

\begin{proof}
    If $p$ is rational,  $\exists m,n \in \mathbf{N}$, 
    s.t. $p=m/n$. $\gcd (m,n) = 1$.
    Then \ref{eq:1.1}
    \begin{equation}
        \label{eq:1.2}
        m^2 = 2n^2.
    \end{equation}

    $m$ is even, $m = 2k$.
    Then $(2k)^2 = 2n^2$, $2k^2 = n^2$, $k$ is even, $\gcd (m,n)=2\neq 1$,
    contrary to our choice of $m$ and $n$. Hence p can't be a rational number.
\end{proof}

After proving $\sqrt{2}$ isn't a rational number, rudin use $\sqrt{2}$ to divide the rationals

\mybox{在证明 $\sqrt{2}$ 不是有理数后, 使用 $\sqrt{2}$ 将有理数集分成两部分.  引出了分划的概念? }

\begin{align*}
    A = \{p|p^2<2\}\\
    B = \{p|p^2>2\}
\end{align*}
$A$ \emph{contains no largest number},\\
$B$ \emph{contains no smallest number}.\\
$\forall p\in A$, $\exists q\in A$, s.t. $p<q$,\\
$\forall p\in B$, $\exists q\in B$, s.t. $p>q$,\\
$\forall p>0$
\begin{equation}
    \label{eq:1.3}
    q = p-\frac{p^2-2}{p+2} = \frac{2p+2}{p+2}
\end{equation}

Then 
\begin{equation}
    \label{eq:1.4}
    q^2 - 2 = \frac{2(p^2-2)}{(p+2)^2}
\end{equation}

If $p\in A$, $p^2<2$. \ref{eq:1.3} shows that $q>p$, \ref{eq:1.4} shows that $q^2<2$, $q\in A$.

If $p\in B$, $p^2>2$. \ref{eq:1.3} shows that $q<p$, \ref{eq:1.4} shows that $q^2>2$, $q\in B$.


\begin{myRemark}
    \label{Remark:1.2}
    The purpose of the above discussion has been to show that 
    the rational number system has {\color{red}certain gaps}, 
    in spite of the fact that between any two rationals there is another: 
    If $r<s$ then $r<(r+s)/2<s$.
    The real number system fills these gaps.
    This is the principal reason for the fundamental role which it plays in analysis.
\end{myRemark}

\mybox{
    有理数的稠密性与实数的连续性. 
    在分析中, 考察极限等需要的是数系的连续性, 因此需要先建立实数系. 
    事实上, 我们是先有微积分, 后有实数理论的. \\
    三次数学危机:无理数, 微积分基础, 集合论
    实数理论是极限的基础. 
}

In order to elucidate its structure, as well as that of the complex numbers, 
we start with a brief discussion of the general concepts of \emph{ordered set} and \emph{field}.

\mybox{
    rudin引入复数的方法非常怪, 对初学者非常不友好, 过于抽象了. 
    想起一个法国笑话, 问小学生$2+3$等于几, 回答 $2+3=3+2$ 加法是一个交换群(Abel 群)...
} 

Here is some of the standard set-theoretic terminology that will be used throughout this book.

\mybox{接下来引入一些集合论的定义}

\begin{myDef}
    \label{myDef:1.3}
    If $A$ is any set (whose elements may be numbers or any other objects), 
    we write $x\in A$ to indicate that $x$ is a member (or an element) of $A$.
    \mybox{
        element 还没定义\\    
        object指代什么? 我个人认为集合理解的难点在于集合的集合. 
        % 这一点可以引出罗素悖论
        }
    If $x$ is not a member of $A$, we write: $x\notin A$.

    \emph{empty set} $\varnothing$ contains no element, If a set has at least one element, it is called \emph{nonempty}.

    $A,B$ are sets, 
    $\forall x\in A$, $x\in B$, we say that $A$ is a \emph{subset} of $B$, $A \subset B$ or $B \supset A$. 
    If $\exists x\in B$, $x\notin A$, A is a \emph{proper subset} of $B$, $A \subsetneqq B$.
    Note that $A\subset A$ for every set $A$.

    (Bernstein) If $A\subset B$ and $B\subset A$, we write $A = B$. Otherwise $A \neq B$.
\end{myDef}
\mybox{这条性质在证明集合相等时很常用}

\begin{myDef}
    \label{myDef:1.4}
    Throughout Chap. 1, the set of all rational numbers will be denoted by $\Q $.
\end{myDef}
\section{Ordered sets}
\mybox{有序集}

\begin{myDef}\label{myDef:1.5}
    Let $S$ be a set. 
    An \emph{order} on $S$ is a relation, denoted by $<$,
    with the following two properties:
    \begin{asparaenum}[(i)]
        \item If $x\in S$ and $y\in S$ then one and only one of the statements
        \begin{equation*}
            x<y, \quad
            x=y, \quad
            y<x
        \end{equation*}
        is true.    
        \item If $x,y,z\in S$, if $x<y$ and $y<z$, then $x<z$.
    \end{asparaenum}

    The statement $x < y$ may be read as 
    $x$ is less than $y$, or 
    $x$ is smaller than $y$, or
    $x$ precedes $y$.
    (It's often convenient to write $y>x$ in place of $x<y$)
    (less-great, smaller-bigger, precedes-succeeds)

    \mybox{form wiki,
    The relationship x precedes y is written $x ≺ y$. 
    The relation x precedes or is equal to y is written x ≼ y.
    The relationship x succeeds (or follows) y is written x ≻ y. 
    The relation x succeeds or is equal to y is written x ≽ y.
    ≺ $\backslash\text{prec}$ 
    ≼ $\backslash\text{preccurlyeq}$ 
    ≻ $\backslash\text{succ}$ 
    ≽ $\backslash\text{succcurlyeq}$  }

    $x\leq y$ indicates that $x<y$ or $x=y$, 
    without specifying which of these two is to hold.
    In other words, $x\leq y$ is the negation of $x>y$.
\end{myDef}


\mybox{
  偏序关系:
1. 三歧性,
2. 传递性.

建立偏序关系后, 可以使用不等式进行分析. 
在后续根据极限定义计算时, 
需要大量使用不等式分析数列和函数的极限计算结果. }

\begin{myDef}\label{myDef:1.6}
% 1.6 myDef
An \emph{ordered set} is a set $S$ in which an order is defined.
\end{myDef}

For Example, $\Q $ is an ordered set 
if $r<s$ is defined to mean that $s-r$ is a positive rational number.

\mybox{
    存在偏序关系的集合称为有序集
$\Q , \R$ 均是有序集, 但$\mathbb{C}$ 不是有序集. }

\begin{myDef}
    \label{myDef:1.7}
    (bounded above)\\
    Suppose $S$ is an ordered set, and $E \subset S$. 
    If there exists a $\beta \in S$ 
    such that $x \leq \beta$ for every $x \in E$, 
    we say that $E$ is \emph{bounded above}, and call
    $\beta$ an \emph{upper bound} of $E$.

    Lower bounds are defined in the same way (with $\geq$ in place of $\leq$).
\end{myDef}

\begin{myDef}
    \label{myDef:1.8}
    (least upper bound)\\
    Suppose $S$ is an ordered set, $E \subset S$, and $E$ is bounded above.
    Suppose there exists an $a\alpha \in S$ with the following properties:
    \begin{enumerate}[(i)]
        \item $\alpha$ is an upper bound of $E$.
        \item If $\gamma <\alpha$ then $\gamma$ is not an upper bound of $E$.
    \end{enumerate}

    Then $\alpha$ is called the \emph{least upper bound} of $E$ 
    [that there is at most one such $\alpha$ is clear from (ii)] 
    or the \emph{supremum} of $E$, and we write
    \begin{equation*}
        \alpha = \sup E.
    \end{equation*}

    The \emph{greatest lower bound}, or \emph{infimum}, 
    of a set $E$ which is bounded below is defined in the same manner: The statement
    \begin{equation*}
        \alpha = \inf E
    \end{equation*}
    means that $\alpha$ is a lower bound of $E$ and 
    that no $\beta$ with $\beta > \alpha$ is a lower bound
    of $E$.
\end{myDef}
\mybox{
    从上界引出最小上界, 没有直接定义最大下界, 而是使用对称定义引出. 
    从最小上界引出的最小上界性质更为常用. Dedekind分划
}

\begin{myExample}
    \label{myExample:1.9}
    \begin{asparaenum}[(a)]
        \item Consider the set $A, B$
        \begin{equation*}
            A = \{p|p^2 < 2\},\quad
            B = \{p|p^2 > 2\}.
        \end{equation*}
        $A$ has no least upper bound in $\Q $.
        $B$ has no great lower bound in $\Q $.    
        \item If $\alpha = \sup E$ exists, $\alpha$ may be or may not be a member of $E$.
        \begin{align*}
            E_1 = \{r |r\in Q, r < 0\}\\
            E_2 = \{r |r\in Q, r \leq 0\}
        \end{align*}
        \begin{equation*}
            \sup E_1 = \sup E_2 = 0,
        \end{equation*}
        and $0\not\in E_1$, $0\in E_2$.
        \item $E = \{1/n | n = 1,2,3,...\}$. 
        Then $\sup E = 1$, which is in $E$, 
        and $\inf E = 0$, which is not in $E$.
    \end{asparaenum}
\end{myExample}

\begin{myDef}
    \label{myDef:1.10}
    {\color{red}{least-upper-bound property}}\\
    An ordered set $S$ is said to have the \emph{least-upper-bound property} 
    if the following is true:\\
    If $E \subset S$, $E$ is not empty, and $E$ is bounded above, then $\sup E$ exists in $S$.
\end{myDef}

Example \ref{myExample:1.9}(a) shows that $\Q $ does not have the least-upper-bound property.

We shall now show that 
there is a close relation between greatest lower bounds and least upper bounds, 
and that every ordered set with the least-upper-bound property 
also has the greatest-lower-bound property.

\mybox{最小上界性质与最大下界性质是等价的.}

\begin{thm}
    \label{thm:1.11}
    Suppose $S$ is an ordered set with the least-upper-bound property,
    $B \subset S$, $B$ is not empty, and $B$ is bounded below. 
    Let $L$ be the set of all lower bounds of $B$. 
    Then
    \begin{equation*}
        \alpha = \sup L
    \end{equation*}
    exists in $S$, and $\alpha = \inf B$.

    In particular, $\inf B$ exists in $S$.
\end{thm}

\begin{proof}
    Since $B$ is bounded below, $L$ is not empty. 
    Since $L$ consists of exactly those $y \in S$ 
    which satisfy the inequality $y \leq x$ for every $x \in B$, 
    we see that \emph{every} $x \in B$ \emph{is an upper bound of} $L$. 
    Thus $L$ is bounded above.
    Our hypothesis about $S$ implies therefore that 
    $L$ has a supremum in $S$; 
    call it $\alpha$.

    If $\gamma < \alpha$ then (see Definition \ref{myDef:1.8}) 
    $\gamma$ is not an upper bound of $L$, 
    hence $\gamma \not\in B$. 
    It follows that $\alpha \leq x$ for every $x \in B$. 
    Thus $\alpha \in L$.

    If $\alpha < \beta$ then $\beta \not\in L$, 
    since $\alpha$ is an upper bound of $L$.

    We have shown that $\alpha \in L$ 
    but $\beta \not\in  L$ if $\beta > \alpha$. 
    In other words, $\alpha$ is a lower bound of $B$, 
    but $\alpha$ is not if $\beta > \alpha$. 
    This means that $\alpha = \inf B$.
\end{proof}

\mybox{% mynotes
这个证明第一次看比较难理清
我试着用自己的话重写梳理一下:
已知条件
$S$, ordered set + least-upper-bound property.
$B\in S$, $B\neq \varnothing $, $B$ is bounded below.
$L$ is the set of all lower bounds of $B$.
$\exists \alpha\in S$, $\alpha = \sup L$, and $\alpha = \inf B$.}

\begin{proof}
    % proof:
    思路 由最小上界 $\rightarrow $ 最大下界
    \begin{equation*}
        \begin{array}{ccc}
            \text{最小上界}  & \rightarrow  &\text{最大下界} \\
            \downarrow      &               &\uparrow \\
            L\text{最小上界}  & \rightarrow  &B\text{最大下界} \\
        \end{array}
    \end{equation*}
    $L = \{y| y\in S; \forall x\in B, y\leq x\}$
    关于 $L$ 中有没有不在 $S$ 中的元素这一点我还没想明白.
    定理中只是说 $L$ 是 $B$ 的下界组成的. 
    $B$ 是 $S$ 的子集, 但 $B$ 的下界不一定全在 $S$ 中. 

    $L$ 由 $B$ 在 $S$ 中的全部下界组成

    $\forall x\in B$, $x$ 为 $L$ 的上界. $L\subset S$.
    $S$ 有最小上界性质,
    $\therefore \exists \alpha\in S$, $\alpha = \sup L$.

    $\forall \gamma <\alpha$ 由 $\alpha = \sup L$ 的定义 (\ref{myDef:1.8})
    $\gamma$ 不是 $L$ 的上界.

    $\forall x \in B$, $x$ 为 $L$ 的上界, $x \geq \alpha$. $\therefore \alpha \in L$.

    $\alpha < \beta$, $\alpha = \sup L$. $\therefore \beta \not\in L$.
    $L$ 由 $B$ 在 $S$ 中的全部下界组成, $\beta \not\in L$.
    $\beta$ 不是 $B$ 的下界.

    $\therefore \alpha = \inf B$, $\inf B\in S$.
\end{proof}



\section{fields}
\mybox{域, 交换除环 <$\mathbb{R},+,\times$> 
<$\mathbb{R},+$>, <$\mathbb{R}\backslash\{0\},\times$>
都是交换群, 且满足分配律. 
则 <$\mathbb{R},+,\times$> 是域. }

\begin{myAxiom}\label{myAxiom:1.12}
% Axiom % 公理

(A) Axioms for addition

(Al) If $x\in F$  and $y \in F$, then their sum \(x + y\) is in F.

(A2) Addition is commutative: \(x + y=y+ x\) for all \(x, y \in F\).

(A3) Addition is associative: \((x+ y)+z = x + (y+ z)\) for all \(x, y, z \in F\).

(A4) $F$ contains an element $0$ such that $0 + x = x$ for every $x \in F$.

(A5) To every $x\in F$ corresponds an element $-x\in F$ such that
\begin{equation*}
    x+(-x)=0.
\end{equation*}

(M) Axioms for multiplication

(M1) If $x\in F$ and $x\in F$, then their product $xy$ is in $F$.

(M2) Multiplication is commutative: $xy = yx$ for all $x, y \in  F$.

(M3) Multiplication is associative: $(xy)z = x(yz)$ for all $x, y, z \in  F$.

(M4) $F$ contains an element $1 \neq 0$ such that $1x = x$ for every $x \in F$.

(M5) If $x \in F$ and $x \neq 0$ then there exists an element $1/x \in F$ such that
\begin{equation*}
    x\cdot(1/x)=1.
\end{equation*}
% 6 PRINCIPLES OF MATHEMATICAL ANALYSIS

(D) The distributive law
\begin{equation*}
    x(y+z)=xy+ xz
\end{equation*}

holds for all $x, y, z \in F$.
\end{myAxiom}

\begin{myRemark}\label{myRemark:1.13}
    % 1.13 Remark
(a) Our usual writes (in any filed)

只定义了加法和乘法, 使用逆元分别表示减法和除法.
$x-y = x+(-y)$, $x/y=x\cdot (1/y)$.

(b) The field axioms clearly hold in $\mathbb{Q}$, the set of all rational numbers, if
addition and multiplication have their customary meaning. Thus $\mathbb{Q}$ is a
field.

全体有理数的集合是一个域.

(c) Although it is not our purpose to study fields (or any other algebraic
structures) in detail, it is worthwhile to prove that some familiar properties
of $\mathbb{Q}$ are consequences of the field axioms; once we do this, we will \underline{not
need to do it} again for the real numbers and for the complex numbers.
\end{myRemark}

\begin{myProposition}\label{myProposition:1.14}
% 1.14 Proposition
The axioms for addition imply the following statements.

(a) If $x+y=x+z$ then $y=z$.

(b) If $x+y=x$ then $y=0$.

(c) If $x+y=0$ then $y= -x$.

(d) $-(-x)=x$.
\end{myProposition}

Statement (a) is a cancellation law. Note that (b) asserts the uniqueness
of the element whose existence is assumed in (A4), and that (c) does the same
for (A5).

\mybox{
    % mynotes
what is the difference between axiom and proposition?

An axiom is a proposition regarded as self-evidently true without proof. The word "axiom" is a slightly archaic synonym for postulate. Compare conjecture or hypothesis, both of which connote apparently true but not self-evident statements.
A proposition is a mathematical statement such as "3 is greater than 4," "an infinite set exists," or "7 is prime." An axiom is a proposition that is assumed to be true. With sufficient information, mathematical logic can often categorize a proposition as true or false, although there are various exceptions (e.g., "This statement is false").
\url{https://www.nutritionmodels.com/terminology.html}}


\begin{proof}
    Proof(rudin)

If $x + y =x + z$, the axioms (A) give
\begin{align*}
    y =0+y&=(-x+x)+y=-x+(x+y)\\
    &=-x+(x+z)=(-x+x)+z=0+z=z
\end{align*}

This proves (a). Take $z = 0$ in (a) to obtain (b). Take $z= -x$ in (a) to
obtain (c).
Since $-x + x = 0$, (c) (with $-x$ in place of $x$) gives (d).
\end{proof}

\mybox{mynotes 我自己证明上述四条性质时都是从定义开始的, 而 rudin 这里在后一步的证明中都利用了刚推导出的结论, 这一点需要借鉴.}
% THE REAL AND COMPLEX NUMBER SYSTEMS 7

\begin{myProposition}\label{Proposition:1.15}
% 1.15 Proposition 
The axioms for multiplication imply the following statements.

(a) If $x\neq0$ and $xy=xz$ then $y=z$.

(b) If $x\neq0$ and $xy=x$ then $y=1$.

(c) If $x\neq0$ and $xy=1$ then $y=1/x$.

(d) If $x\neq0$ then $1/(1/x) = x$.
\end{myProposition}

The proof is so similar to that of Proposition 1.14 that we omit it.


\begin{proof}
mynotes
% Proof
(a),  
\begin{align*}
    y&=1\cdot y=\left(\frac{1}{x}\cdot x\right)y =\frac{1}{x}\left( xy \right)\\
    &=\frac{1}{x}(xz) =\left(\frac{1}{x}x\right)z = z
\end{align*}

(b), (a)取  $z=1$. $y=z=1$.

(c), (a)取  $z=\frac{1}{x}$. $y=z=\frac{1}{x}$.

(d), (c)取  $x=\frac{1}{x'}$. $y=1/(1/x')$.
\end{proof}

\begin{myProposition}\label{Proposition:1.16}
    The field axioms imply the following statements, for any $x, y, z \in F$.

    (a) $0x=0$.

    (b) If $x\neq 0$ and $y\neq 0$ then $xy\neq 0$.

    (c) $(-x)y=-(xy)=x(-y)$.

    (d) $(-x)(-y)=xy$.
\end{myProposition}

\begin{proof}
    $0x+0x=(0+0)x=0x$. Hence \ref{myProposition:1.14}(b) implies that $0x=0$, and (a) holds.

    Next, assume $x \neq 0$, $y \neq 0$, but $xy =0$. Then (a) gives
    \begin{equation*}
        1=
        \left(\frac{1}{y}\right)\left(\frac{1}{x}\right)xy=
        \left(\frac{1}{y}\right)\left(\frac{1}{x}\right)0=0.
    \end{equation*}

a contradiction. Thus (b) holds.

The first equality in (c) comes from
\begin{equation*}
    (-x)y +xy=(-x+x)y=0y=0,
\end{equation*}

combined with \ref{myProposition:1.14}(c); the other half of (c) is proved in the same way.\\
Finally,
\begin{equation*}
    (-x)(-y)=-[x(-y)]=-[-(xy)]=xy
\end{equation*}
by (c) and \ref{myProposition:1.14}(d).
\end{proof}

\begin{myDefinition}\label{myDefinition:1.17}
    An ordered field is a field $F$ which is also an ordered set, such
    that
    
    (i) $x+y<x+z$ if $x,y,z\in F$ and $y<z$,
    
    (ii) $xy>0$ if $x\in F$, $y\in F$, $x>0$, and $y>0$.
\end{myDefinition}
If $x > 0$, we call $x$ positive; 
if $x < 0$, $x$ is negative.

For example, $\mathbb{Q}$ is an ordered field.

All the familiar rules for working with inequalities apply in every ordered
field: Multiplication by positive [negative] quantities preserves [reverses] inequalities, no square is negative, etc. The following proposition lists some of
these.
% 8 PRINCIPLES OF MATHEMATICAL ANALYSIS

\mybox{有序域$F$也是有序集, 由于有理数域 $\mathbb{Q}$, 实数域 $\mathbb{R}$ 都是有序域, 这里使用有理数域 $\mathbb{Q}$ 证明的有序集的性质也可以直接用于实数域. $\mathbb{R}$}

\begin{myProposition}\label{myProposition:1.18}
    The following statements are true in every ordered field.

(a) If $x>0$ then $-x <0$, and vice versa.

(b) If $x>0$ and $y<z$ then $xy <xz$.

(c) If $x<0$ and $y<z$ then $xy> xz$.

(d) If $x \neq 0$ then $x^2 > 0$. In particular, $1 > 0$.

(e) If $0<x<y$ then $0<l/y<l/x$.
\end{myProposition}

\begin{proof}
    (a) $x>0$, $-x<0$. 
    \begin{align*}
        x   &> 0=(x+-x)\\
        x+0 &> x+(-x)\\
        (-x)&<0
    \end{align*}

    (b) $x>0$, $y<z$, $xy<xz$.
    \begin{align*}
        y<z, z-y&>y-y=0\\
        x(z-y)&>0\\
        x(z-y)+xy&>0+xy\\
        xz&>xy
    \end{align*}

    (c)
    \begin{align*}
        (z-y) &>y-y=0\\
        x<0,(-x)>0.\quad (-x)(z-y)&>0 \\
        x(z-y) &<0\\
        xz<xy    
    \end{align*}

    (d)
    \begin{align*}
        x>0  && x^2    >0  \\
        x<0  &&(-x)^2 >0, (-x)^2 = -[x(-x)] = -(-(x\cdot x)) =x^2, x^2>0
    \end{align*}
    $\because 1^2=1$, $1>0$.

    (e)
    If $y>0$ and $v \leq 0$, then $yv \leq 0$. But $y \cdot (1/y)=1>0$. Hence $1/y > 0$.
    Likewise, $1/x > 0$. If we multiply both sides of the inequality $x <y$ by
    the positive quantity $(1/x)(1/y)$, we obtain $1/y <1/x$.
\end{proof}



\section{The real field}

We now state the \emph{existence theorem} which is the core of this chapter.

\begin{thm}\label{thm:1.19}
    There exists an \emph{ordered field} $\R$ 
    which has the least-upper-bound property.

    Moreover, $\R$ contains $\Q $ as a \emph{subfield}.
\end{thm}

The second statement means that $\Q \subset \R$ 
and that the operations of addition and multiplication in $\R$, 
when applied to members of $\Q $, 
coincide with the usual operations on rational numbers; 
also, the positive rational numbers are positive elements of $\R$.

The members of $\R$ are called real numbers.

The proof of Theorem \ref{thm:1.19} is rather long and a bit tedious 
and is therefore presented in an Appendix to Chap. 1. The proof actually constructs $\R$ from $\Q$.

The next theorem could be extracted from this construction with very
little extra effort. 
However, we prefer to derive it from Theorem \ref{thm:1.19} since this
provides a good illustration of what one can do with the least-upper-bound property.

\mybox{$\R$ 具有最小上界性质的有序域
least-upper-bound $\rightarrow$ upper bound in the sets.
ordered field (ordered set, field).

$\Q \in \R$ subfield

$x\in \R$, $x$ is a real number
}
\mybox{
    proof of theorem \ref{thm:1.19} is tedious.
    construct $\R$ from $\Q $

    tedious 乏味的, 冗长的\\
    derive 取得, 得到
    }

\begin{thm}\label{thm:1.20}(archimedean property of $\R$)
    (a) If $x \in \R$, $y \in \R$, and $x > 0$, then there is a positive integer $n$ such that
    \begin{equation*}
        nx > y
    \end{equation*}

    (b) If $x \in \R$, $y \in \R$, and $x < y$, then there exists a $p \in Q$ such that $x < p < y$.
\end{thm}
\begin{proof}
    \begin{asparaenum}[(a)]
        \item Let $A$ be the set of all $nx$, 
        where $n$ runs through the positive integers.
        If (a) were false, then $y$ would be an upper bound of $A$. 
        But then $A$ has a least upper bound in $R$. 
        Put $\alpha = \sup A$. 
        Since $x > 0$, $\alpha - x < \alpha$, 
        and $\alpha - x$ is not an upper bound of $A$. 
        Hence $\alpha - x < mx$ for some positive integer $m$. 
        But then $\alpha < (m + l)x \in A$, 
        which is impossible, 
        since $\alpha$ is an upper bound of $A$.
        \item Since $x < y$, we have $y - x > $0, 
        and (a) furnishes a positive integer $n$ 
        such that
        \begin{equation*}
            n(y - x) > 1.
        \end{equation*}
        Apply (a) again, to obtain positive integers $m_1$ and $m_2$ 
        such that $m_1 > nx$,
        $m_2 > -nx$. Then
        $-m_2 < nx < m_1$
        Hence there is an integer $m$ (with $-m_2 \leq m \leq m_1$) 
        such that
        \begin{equation*}
            m - 1\leq 11x < m.
        \end{equation*}
        If we combine these inequalities, we obtain
        \begin{equation*}
            nx < m \leq 1 + nx < ny.
        \end{equation*}
        Since $n > 0$, it follows that
        \begin{equation*}
            x < \frac{m}{n} < y.
        \end{equation*}
        This proves (b), with $p = m/n$.
    \end{asparaenum}
\end{proof}

We shall now prove the existence of nth roots of positive reals. 
This proof will show how the difficulty pointed out in the Introduction 
(irrationality of $\sqrt{2}$) can be handled in $\R$.
\begin{thm}
    \label{thm:1.21}
    For every real $x > 0$ and every integer $n> 0$ 
    there is one and only one positive real $y$ 
    such that $y^n = x$.
\end{thm}

This number $y$ is written $\sqrt[n]{x}$ or $x^{1/n}$.

\begin{proof}
    That there is at most one such $y$ is clear, 
    since $0 < y_1 < y_2$ implies $y_1^n < y_2^n$.
    
    Let $E$ be the set consisting of all positive real numbers $t$ 
    such that $t^n < x$.
    
    If $t = x/(1 + x)$ then $0 \leq t < 1$. 
    Hence $t^{n} \leq t < x$.
    Thus $t \in E$, and $E$ is not empty.
    
    If $t > 1 + x$ then $t^{n} \geq t > x$, 
    so that $t \not\in E$. 
    Thus $1 + x$ is an upper bound of $E$.
    
    Hence Theorem \ref{thm:1.19} implies the existence of 
    \begin{equation*}
        y = \sup E.
    \end{equation*}
    To prove that $y^{n} = x$ we will show that 
    each of the inequalities $y^{n} < x$ and $y^{n} > x$ leads to a contradiction.
    
    The identity $b^{n} - a^{n}= (b - a)(b^{n-1} + b^{n}- 2a + \cdots + a^{n-1})$ yields the inequality
    \begin{equation*}
        b^{n} - a^{n} < (b - a)nb^{n-1}
    \end{equation*}
    when $0 < a < b$.
    
    Assume $y^{n} < x$. Choose $h$ so that $0 < h < 1$ and
    \begin{equation*}
        h < \frac{x - y^n}{n(y + 1)^{n-1}}.
    \end{equation*} 
    Put $a = y$, $b = y + h$. Then
    \begin{equation*}
        (y + h)^{n} - y^{n} 
        < hn(y + h)^{n-l} 
        < hn(y + l)^{n-1} 
        < x - y^{n}.
    \end{equation*}
    Thus $(y + h)^{n} < x$, and $y +h \in E$. 
    Since $y + h > y$, 
    this contradicts the fact that $y$ is an upper bound of $E$.

    Assume $y^{n} > x$. Put 
    \begin{equation*}
        k = \frac{y^n - x}{n y^{n-1}}
    \end{equation*}
    Then $0 < k < y$. 
    If $t \leq y - k$, we conclude that
    \begin{equation*}
        y^{n} - t^{n} 
        \leq y^{n} - (y - k)^{n} 
        < kny^{n-1} 
        = y^{n} - x.
    \end{equation*}
    Thus $t^{n} > x$, and $t \not\in E$. 
    It follows that $y - k$ is an upper bound of $E$.
    But $y - k < y$, 
    which contradicts the fact that $y$ is the least upper bound of $E$.
    Hence $y^{n} = x$, and the proof is complete.
\end{proof}
\mybox{
    很经典的等式证明, 
    从两边的不等式不成立出发证明有且仅有等式成立.
}

\begin{myCorollary*}
    If $a$ and $b$ are positive real numbers 
    and $n$ is a positive integer, 
    then
    \begin{equation*}
        (ab)^{1/n}= a^{1/n}b^{1/n}.
    \end{equation*}
\end{myCorollary*}

\begin{mydef}
    \label{mydef:1.22}
    (Decimals)\\
    We conclude this section by pointing out the relation between real numbers and decimals.
\end{mydef}



\section{The extended real number system}
\mybox{使用正负无穷大扩充实数系}
\begin{myDef}
    \label{thm:1.23}
    The extended real number system consists of the real field $\R$
    and two symbols, $+\infty$ and $-\infty$. 
    We preserve the original order in $\R$, 
    and define
    \begin{equation*}
        -\infty <x<+\infty
    \end{equation*}
    for every $x\in \R$
\end{myDef}

\begin{asparaenum}[(a)]
\item If $x$ is real then
\begin{equation*}
    x+\infty = +\infty,\quad
    x-\infty = -\infty,\quad
    \frac{x}{+\infty}=\frac{x}{-\infty}=0.
\end{equation*}

\item If $x>0$ 
then $x\cdot (+\infty) = +\infty$, 
$x\cdot(-\infty)=-\infty$.

\item If $x<\infty$ 
then $x\cdot(+\infty)=-\infty$, 
$x\cdot(-\infty)= +\infty$.
\end{asparaenum}



\section{THE COMPLEX FIELD}

\mybox{rudin 引入复数定义的方法很奇怪, 代数角度是一致的, 但理解起来比较困难, 我觉得使用几何方法引入复数更为合理且直观, rudin 这里对初学者不太友好}

\begin{myDefinition}\label{myDefinition:complexnumber1.24}
    % 1.24 Definition 
    A complex number is an ordered pair $(a, b)$ of real numbers.
"Ordered" means that $(a, b)$ and $(b, a)$ are regarded as distinct if $a \neq b$.

Let$x = (a, b)$, $y = (c,d)$ be two complex numbers. We write $x =y$ if and
only if $a =c$ and $b=d$. (Note that this definition is not entirely superfluous;
think of equality of rational numbers, represented as quotients of integers.) We
define
\begin{align*}
    x+y &= (a+c, b+d),\\
    xy  &= (ac - bd, ad + bc).
\end{align*}
\end{myDefinition}

\begin{thm}\label{thm:complexnumberisfield}
    % 1.25 'Theorem 
    These definitions of addition and multiplication turn the \emph{set} of all complex numbers into a field, with $(0, 0)$ and $(1, 0)$ in the role $0$ and $1$.
\end{thm}

proof (A1)~(A5), (M1)~(M5) and (D), then we can prove that $\mathbb{C}$ is a field.

\begin{thm}\label{thm:realincomplex}
    For any real numbers $a$ and $b$ we have
    \begin{equation*}
        (a,0)+ (b,0) = (a+ b,0),\quad
        (a,0)(b,0) = (ab,0).
    \end{equation*}
\end{thm}
The proof is trivial.

show that the notation $(a, b)$ is equivalent to the more customary $a + bi$.

\begin{myDefinition}\label{myDefinition:image_number}
    $i=(0,1)$    
\end{myDefinition}

\begin{thm}\label{thm:sqartrootofminus1}
    % 1.28 Theorem
    $i^2=-1$
\end{thm}

\begin{proof}
    \begin{equation*}
        i^2=(0,1)(0,1)=(-1,0)=-1.
    \end{equation*}
\end{proof}

\begin{thm}\label{thm:complexnumberTransfer}
    If $a$ and $b$ are real, then $(a,b) =a + bi$.
\end{thm}

\begin{proof}
    \begin{align*}
        a+bi
        &=(a,0)+(b,0)(0,1)\\
        &=(a,0)+(0,b)=(a,b)\\
    \end{align*}
\end{proof}

\begin{myDefinition}\label{myDefinition:conjugate}
    $a,b\in \mathbb{R}$, $z=a+bi$,the complex number $\bar{z}=a-bi$ is called the conjugate of $z$. the numbers $a$ and $b$ are the real part and imaginary part of $z$. respectively.
    \begin{equation*}
        a=\Re(z), \quad
        b=\Im(z)
    \end{equation*}
\end{myDefinition}

\begin{thm}\label{thm:complexProperty}
    % 1.31 Theorem
    If $z$ and $w$ are complex, then

    (a) $\bar{z+w}=\bar{z}+\bar{w}$,

    (b) $\bar{zw}=\bar{z}\cdot\bar{w}$,

    (c) $z+\bar{z}=2\Re(z)$, $z-\bar{z}=2\Im(z)$,

    (d) $z\bar{z}$ is real and positive (except when $z=0$).
\end{thm}
Proof (a), (b),and (c)are quite trivial. To prove (d), write $z = a + bi$,
and note that $z\bar{z} = a^2 + b^2$.

\begin{myDefinition}\label{myDefinition:complex_absolutevalue}
    If $z$ is a complex number, its absolute value $|z|$ is the nonnegative square root of $z\bar{z}$; that is, $|z| = (z\bar{z})^{1/2}$.
\end{myDefinition}
The existence (and uniqueness) of $|z|$ follows from Theorem \ref{thm:1.21} and
part (d) of Theorem \ref{thm:complexProperty}.

Note that when $x$ is real, then $\bar{x} = x$, hence $|x| = \sqrt{x^2}$. Thus $|x| = x$
if $x>0$, $|x| = -x$ if $x <0$.

\begin{thm}\label{thm:1.33}
    Let $z$ and $w$ be complex numbers. Then
    
    (a) $|z|>0$ unless $z=0$, $|0|=0$,

    (b) $\bar{z}=z$,

    (c) $|zw| = |z||w|$,

    (d) $| \Re(z)| \leq |z|$,

    (e) $|z+w| \leq|z|+|w|$.

\end{thm}


\begin{myNotation}\label{myNotation:sum}
% 1.34 notation 
(sum)
$x_1,x_2,\dots,x_n \in \mathbb{C}$,
\begin{equation*}
    x_1+x_2+\dots+x_n = \sum_{j=1}^{n} x_j.
\end{equation*}
\end{myNotation}

\begin{thm}\label{thm:schwarz_inequality}
    (Schwarz Inequality)

    If 
    $a_1,\dots,a_n$, 
    $b_1,\dots,b_n$, are complex numbers, then
    \begin{equation*}
        \left| \sum_{j=1}^{n}a_j \bar{b_j}\right|^2 \leq 
        \sum_{j=1}^{n}\left|a_j\right|^2
        \sum_{j=1}^{n}\left|b_j\right|^2.
    \end{equation*}    
\end{thm}

在正式证明之前, 先回忆 $\mathbb{R}$ 中的施瓦茨不等式是怎么证明的.
let $A = \sum a_j^2$, $B = \sum b_j^2$, $C = \sum a_j b_j$.
\begin{equation*}
    \sum (a_j+\lambda b_j)^2 = 
    \sum a_j^2 
    + 2\sum a_j b_j \lambda
    + \sum b_j^2 \lambda^2
\end{equation*}
由韦达定理, $\Delta \leq 0$, $\Delta= (2\sum a_j b_j )^2 - 4 \sum a_j^2\sum b_j^2$. 
因此 $(\sum a_j b_j )^2 \leq \sum a_j^2\sum b_j^2$
\begin{proof}
    Put $A = \sum |a_j|^2$, $B = \sum |b_j|^2$, $C = \sum a_j \bar{b_j}$, $j = 1,2,\dots,n$. 
    
    If $B = 0$, $b_1 = \dots = b_n = 0$, this conclusion is trivial.

    If $B > 0$, 
    \begin{align*}
        \sum \left|B a_j - C b_j\right|^2
        &= \sum (B a_j-C b_j)(B \bar{a_j} - \bar{C b_j})\\
        &= B^2\sum \left|a_j\right|^2 - B\bar{C}\sum a_j \bar{b_j} - BC \sum \bar{a_j} b_j + \left|C\right|^2\sum |b_j|^2\\
        &= B^2A-B|C|^2\\
        &= B(AB-|C|^2).
    \end{align*}
    Since each term in the first sum is nonnegative, we see that
    \begin{equation*}
        B(AB-|C|^2) \geq 0.
    \end{equation*}
    Since $B>0$, it follows that $AB-|C|^2 \geq 0$. This is the desired inequality.
\end{proof}

我的想法
\begin{align*}
    \sum (a_j + \lambda \bar{b_j})(\bar{a_j} + \lambda b_j)
    &=\sum(a_j\bar{a_j}+\lambda(\bar{a_j}b_j+a_j\bar{b_j})+\lambda^2 b_j\bar{b_j})\\
    &=\sum(a_j\bar{a_j}+\lambda 2\Re(a_j\bar{b_j}) +\lambda^2 b_j\bar{b_j})
\end{align*}
由韦达定理, $\Delta \leq 0$, $\Delta = (2\sum\Re(a_j\bar{b_j}))^2-4\sum a_j\bar{a_j}\sum b_j\bar{b_j}$, 
(这里推出的结论比原始结论弱?为什么?)
% 由于 $\left( \sum\Re(a_j\bar{b_j}) \right)^2 = $
% \begin{align*}
%     \left( \sum\Re(a_j\bar{b_j}) \right)^2 &\leq \sum a_j\bar{a_j}\sum b_j\bar{b_j}\\
%     \left| \left( \sum a_j \bar{b_j} \right)\right|^2
    
% \end{align*}
\section{Euclidean space}
欧式空间

\begin{myDefinition}\label{myDefinition:coordinates}
    % 1.36 Definitions
    For each positive integer $k$, let $\mathbb{R}^k$ be the set of all ordered k-tuples
    \begin{equation*}
        \mathbf{k} = \left(x_1,x_2,\dots,x_k\right),
    \end{equation*}
    where $x_1,x_2,\dots,x_k$ are real numbers, called the \textbf{coordinates} of $\mathbf{x}$. The elements of
    $\mathbb{R}^k$ are called points, or vectors, especially when $k > 1$. We shall denote vectors
    by boldfaced letters. 
    If $\mathbf{y} = \left(y_1,y_2,\dots,y_k\right)$ and if $\alpha$ is a real number, put
    \begin{align*}
        \mathbf{x} + \mathbf{y} &= \left(x_1+y_1,x_2+y_2,\dots,x_k+y_k\right),\\
        \alpha\mathbf{x}  = \left(\alpha x_1,\alpha x_2,\dots,\alpha x_k\right)
    \end{align*}
    so that $\mathbf{x} +\mathbf{y} \in \mathbb{R}^k$ and $\alpha\mathbf{x} \in \mathbb{R}^k$. This defines addition of vectors, as well as multiplication of a vector by a real number (a scalar). These two operations satisfy the commutative, associative, and distributive laws (the proof is trivial, in view of the analogous laws for the real numbers) and make $\mathbb{R}^k$ into a vector space over the \emph{real field}. The zero element of $\mathbb{R}^k$ (sometimes called the origin or the null vector) is the point $\mathbf{0}$, all of whose coordinates are $0$.

    We also define the so-called ``inner product'' (or scalar product) of $\mathbf{x}$ and $\mathbf{y}$ by
    \begin{equation*}
        \mathbf{x}\cdot\mathbf{y} = \sum_{j=1}^{k}x_j y_j
    \end{equation*}
    and the norm of $\mathbf{x}$ by
    \begin{equation*}
        |x| = (x\cdot x)^{1/2} = \left( \sum_{1}^{k} x_j^2 \right)^{1/2}.
    \end{equation*}

    The structure now defined (the vector space $\mathbb{R}^k$ with the above inner product and norm) is called euclidean $k$-space.
\end{myDefinition}

\begin{thm}\label{thm:1.37vector}
    Suppose $\mathbf{x}, \mathbf{y}, \mathbf{z}\in\mathbb{R}^k$, and $\alpha$ is real. Then

(a) $| \mathbf{x}| \geq 0$;

(b) $| \mathbf{x}| = 0$ if and only if $\mathbf{x} =0$;

(c) $| \alpha \mathbf{x}| = | \alpha||x|$

a |x-y| <|x[lyl;
a x+y <x +1yl;
Of) |x-z|<|x-y|+]|y-z|.
\end{thm}



\end{document}