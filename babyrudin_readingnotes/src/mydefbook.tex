\documentclass[openany,twoside,scheme=chinese,fontset=none]{book}
\usepackage{ctex}
\usepackage{geometry}
\geometry{
    paperheight=260mm,
    paperwidth=185mm,
    top=25mm,
    bottom=15mm,
    left=25mm, % 左侧留 5mm 装订线距离
    right=15mm
}

\setmainfont{XITS}  % 英文字体,  Times 风格

\setCJKmainfont{Source Han Serif SC}[         % 方正书宋_GBK
    BoldFont=Source Han Serif SC Bold,  % 思源宋体粗体
    ItalicFont=FZKai-Z03                % 方正楷体_GBK
    ]
\setCJKsansfont{Source Han Sans SC}[             % 方正黑体_GBK
    BoldFont=Source Han Sans SC Bold    % 思源黑体粗体
    ]
\setCJKmonofont{FZFangSong-Z02}         % 方正仿宋_GBK

\setCJKfamilyfont{zhsong}{FZShuSong-Z01}
\setCJKfamilyfont{zhxbs}{Source Han Serif SC Bold}
\setCJKfamilyfont{zhdbs}{Source Han Serif SC Heavy}
\setCJKfamilyfont{zhhei}{FZHei-B01}
\setCJKfamilyfont{zhdh}{Source Han Sans SC Bold}
\setCJKfamilyfont{zhfs}{FZFangSong-Z02}
\setCJKfamilyfont{zhkai}{FZKai-Z03}

\newcommand{\songti}{\CJKfamily{zhsong}}
\newcommand{\xbsong}{\CJKfamily{zhxbs}}
\newcommand{\dbsong}{\CJKfamily{zhdbs}}
\newcommand{\heiti}{\CJKfamily{zhhei}}
\newcommand{\dahei}{\CJKfamily{zhdh}}
\newcommand{\fangsong}{\CJKfamily{zhfs}}
\newcommand{\kaishu}{\CJKfamily{zhkai}}


\usepackage{amsmath}
\usepackage{amsthm} % amsthm 与 ntheorem 冲突
\usepackage{amssymb}
\usepackage{hyperref} % \url
\usepackage{graphicx} % \includegraphics
\usepackage{fancybox} % 使用盒子
\usepackage{xcolor} % 使用颜色 color
% \usepackage{amsfonts}  % 
\usepackage{mathrsfs}  % 花体字 mathscr

\usepackage{enumerate}  % 计数器
\usepackage{paralist}  % 计数器 允许行间公式

\usepackage{tikz,lipsum,lmodern}
\usepackage[most]{tcolorbox} % \DeclareTColorBox


\theoremstyle{plain} % default
\newtheorem{thm}{Theorem}[chapter] % 如果不采用章节号做前缀,  则不用[section]
\newtheorem{myLemma}[thm]{Lemma}
\newtheorem{mynewthm}{Theorem}[section] % 如果不采用章节号做前缀,  则不用[section]
\newtheorem{myCorollary}[thm]{Corollary}
\newtheorem*{myCorollary*}{Corollary}

\theoremstyle{definition} % definition

\newtheorem{myDef}[thm]{Definition}
\newtheorem{myExample}[thm]{Example}
\newtheorem{myRemark}[thm]{Remark}
\newtheorem{myProposition}[thm]{Proposition}
\newtheorem{myNotation}[thm]{Notation}

\newtheorem*{myRemark*}{Remark}
% \newtheorem{mySolve}{Solve}

\newtheorem{myExercise}{Exercise}[chapter]
% \theoremstyle{definition}
% \newtheorem{definition}{Definition}[section]

% 我的笔记环境
\newcommand{\mybox}[1]{
    % \fbox{mynotes:\\#1}
    % \vskip 2.5mm
    % \fbox{\parbox{135mm}{mynotes:\\#1}}
    % \vskip 2.5mm
    \begin{tcolorbox}
        my notes:\\#1
    \end{tcolorbox}
}
% 使用这种方式输入命令, 反向查找错误困难, 因此放弃
% \newcommand{\myproof}[1]{\begin{proof}#1\end{proof}}
% \newcommand{\mycommand}[2]{\begin{#1}#2\end{#1}}
% \newcommand{\mythm}[1]{\mycommand{thm}{#1}}
% \newcommand{\mymyDef}[1]{\mycommand{myDef}{#1}}
% \newcommand{\mymyExample}[1]{\mycommand{myExample}{#1}}
% \newcommand{\mymyRemark}[1]{\mycommand{myRemark}{#1}}
% \newcommand{\mymyProposition}[1]{\mycommand{myProposition}{#1}}
% \newcommand{\mymyNotation}[1]{\mycommand{myNotation}{#1}}
% \newcommand{\mymyCorollary*}[1]{\mycommand{myCorollary*}{#1}}
% \newcommand{\mymyExercise}[1]{\mycommand{myExercise}{#1}}

% \newcommand{\mySolve}[1]{\textbf{\color{blue} Solve: }{#1}}
% \newcommand{\mySolve}{\textbf{\color{blue} Solve: }}
\newcommand{\myKeyword}[1]{\textbf{\color{blue} {#1} }}
\newcommand{\mySolve}{\myKeyword{Solve}}
%New Commands

%Special Symbols

\newcommand{\R}{{\protect\mathbb R}}
\newcommand{\Cc}{{\protect\mathbb C}}
\newcommand{\K}{{\protect\mathbb K}}
\newcommand{\N}{{\protect\mathbb N}}
\newcommand{\Q}{{\protect\mathbb Q}}
\newcommand{\Z}{{\protect\mathbb Z}}
\newcommand{\sequence}[1]{\protect\{#1\}}
\newcommand{\mybinom}[2]{\protect \binom{#1}{#2}}

% \def\dif{\mathop{}\!\mathrm{d}} % 微分符号
\def\dif{\mathop{}\hphantom{\mskip-\thinmuskip}\mathrm{d}}%
\let\daccent\d
\let\d\relax
\newcommand\d{\ifmmode\dif\else\expandafter\daccent\fi}
