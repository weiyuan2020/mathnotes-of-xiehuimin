% chap03exercise
\section*{Exercises}

\begin{myExercise}
    \label{ex:3.1}
    Prove that convergence of $\{s_n\}$ implies convergence of $\{|s_n|\}$. 
    Is the converse true?
\end{myExercise}

\begin{myExercise}
    \label{ex:3.2}
    Calculate $\lim_{n \to \infty} (\sqrt{n^2+n}-n)$ 
\end{myExercise}

\mySolve
\begin{align*}
    \lim_{n \to \infty} n\left( \sqrt{1+\frac{1}{n}}-1 \right) 
    &= \lim_{n \to \infty} \frac{ \sqrt{1+\frac{1}{n}}-1 }{\frac{1}{n}} \\
    &= \lim_{n \to \infty} \frac{1+\frac{1}{2}\frac{1}{n}-1}{\frac{1}{n}} \\
    &= \frac{1}{2} .
\end{align*}

\begin{myExercise}
    \label{ex:3.3}
    If $s_1 = \sqrt{2}$ , and 
    \begin{equation*}
        s_{n+1} = \sqrt{2+\sqrt{s_n}} 
        \quad 
        (n = 1,2,3,\dots),
    \end{equation*}
    prove that $\{s_n\}$ converges, and that $s_n < 2$ for $n=1,2,3,...$ .
\end{myExercise}

\begin{myExercise}
    \label{ex:3.4}
    Find the upper and lower limits of the sequences $\{s_n\}$ defined by 
    \begin{equation*}
        s_1 = 0; \quad 
        s_{2m} = \frac{s_{2m-1}}{2}; \quad 
        s_{2m+1} = \frac{1}{2} + s_{2m} .
    \end{equation*}
\end{myExercise}


\begin{myExercise}
    \label{ex:3.5}
    For any two real sequences $\{a_n\}$, $\{b_n\}$, prove that
    \begin{equation*}
        \limsup_{n \to \infty} (a_n + b_n) \leq
        \limsup_{n \to \infty} a_n +
        \limsup_{n \to \infty} b_n ,
    \end{equation*}
    provided the sum on the right is not of the form $\infty - \infty$.
\end{myExercise}

\begin{myExercise}
    \label{ex:3.6}
    Investigate the behavior (convergence or divergence) of $\sum a_n$ if
    \begin{enumerate}[(a)]
        \item $a_n = \sqrt{n+1} - \sqrt{n}$ ;
        \item $a_n = \frac{\sqrt{n+1} - \sqrt{n}}{n}$ ;
        \item $a_n = (\sqrt[n]{n} - 1)^n$ ;
        \item $a_n = \frac{1}{1+z^n}$ , for complex values of $z$.
    \end{enumerate}
\end{myExercise}

\begin{myExercise}
    \label{ex:3.7}
    Prove that the convergence of $\sum a_n$ implies the convergence of
    \begin{equation*}
        \sum \frac{\sqrt{a_n}}{n},
    \end{equation*}
    if $a_n \geq 0$ .
\end{myExercise}


\begin{myExercise}
    \label{ex:3.8}
    If $\sum a_n$ converges, 
    and if $\{b_n\}$ is monotonic and bounded, 
    prove that $\sum a_n b_n$ converges.
\end{myExercise}


\begin{myExercise}
    \label{ex:3.9}
    Find the radius of convergence of each of the following power series:
    \begin{enumerate}[(a)]
        \item $\sum n^3z^n$ ,
        \item $\sum \frac{2^n}{n!}z^n$ ,
        \item $\sum \frac{2^n}{n^2}z^n$ ,
        \item $\sum \frac{n^3}{3^n}z^n$ ,
    \end{enumerate}
\end{myExercise}

\mySolve
\begin{enumerate}
    \item $\lim_{n \to \infty} \frac{1}{(1+1/n)^3} = 1$, $R = 1$.
    \item $\lim_{n \to \infty} \frac{n + 1}{2} = \infty$, $R = \infty$.
    \item $\lim_{n \to \infty} \frac{\left( 1+\frac{1}{n} \right)^2}{2} = \frac{1}{2}$, $R = \frac{1}{2}$.
    \item $\lim_{n \to \infty} \frac{3}{\left( 1+\frac{1}{n} \right)^3} = 3$, $R = 3$.
\end{enumerate}

\begin{myExercise}
    \label{ex:3.10}
    Suppose that the coefficients of the power series $\sum a_n z^n$ are integers, 
    infinitely many of which are distinct from zero. 
    Prove that the radius of convergence is at most 1.
\end{myExercise}

\begin{myExercise}
    \label{ex:3.11}
    Suppose $a_n > 0$, $s_n = a_1 + ... + a_n$ and $\sum a_n$ diverges.
    \begin{asparaenum}[(a)]
        \item Prove that $\sum \frac{a_n}{1+a_n}$ diverges.
        \item Prove that 
        \begin{equation*}
            \frac{a_{N+1}}{s_{N+1}} + \cdots +
            \frac{a_{N+k}}{s_{N+k}} \geq 
            1- \frac{s_N}{s_{N+k}}
        \end{equation*}
        and deduce that $\sum \frac{a_n}{s_n}$ diverges.
        \item Prove that 
        \begin{equation*}
            \frac{a_n}{s_n^2} \leq \frac{1}{s_{n-1}} - \frac{1}{s_n}
        \end{equation*}
        and deduce that $\sum \frac{a_n}{s_n^2}$ converges.
        \item What can be said about 
        \begin{equation*}
            \sum \frac{a_n}{1+n a_n} 
            \text{ and }
            \sum \frac{a_n}{1+n^2a_n}
        \end{equation*}
    \end{asparaenum}
\end{myExercise}

\begin{myExercise}
    \label{ex:3.12}
    Suppose $a_n > 0$ and $\sum a_n$ converges. Put
    \begin{equation*}
        r_n = \sum_{m=n}^{\infty} a_m .
    \end{equation*}
    \begin{asparaenum}[(a)]
        \item Prove that 
        \begin{equation*}
            \frac{a_m}{r_m} + \cdots +
            \frac{a_n}{r_n} > 
            1 - \frac{r_n}{r_m}
        \end{equation*}
        if $m<n$, and deduce that $\sum \frac{a_n}{r_n}$ diverges.
        \item Prove that 
        \begin{equation*}
            \frac{a_n}{\sqrt{r_n}} < 
            2\left( \sqrt{r_n} - \sqrt{r_{n+1}} \right)
        \end{equation*}
        and deduce that $\sum \frac{a_n}{\sqrt{r_n}}$ converges.
    \end{asparaenum}
\end{myExercise}


\begin{myExercise}
    \label{ex:3.13}
    Prove that the Cauchy product of two absolutely convergent series converges absolutely.
\end{myExercise}


\begin{myExercise}
    \label{ex:3.14}
    If $\{s_n\}$ is a complex sequence, define its arithmetic means $\sigma_n$ by 
    \begin{equation*}
        \sigma_n = \frac{s_0+s_1+\cdots+s_n}{n+1}
        \quad 
        (n=0,1,2,\dots).
    \end{equation*}
    \begin{asparaenum}[(a)]
        \item If $\lim s_n = s$, prove that $\lim \sigma_n = s$.
        \item Construct a sequence $\{s_n\}$ which does not converge, although $\lim \sigma_n= 0$.
        \item Can it happen that $s_n> 0$ for all $n$ and that $\limsup s_n = \infty$, although $\lim \sigma_n= 0$?
        \item Put $a_n = s_n - s_{n-1}$, for $n \geq 1$. 
        Show that
        \begin{equation*}
            s_n-\sigma_n = \frac{1}{n+1}\sum_{k=1}^{n}k a_k .
        \end{equation*}
        Assume that $\lim (n a_n)= 0$ and that $\{\sigma_n\}$ converges. 
        Prove that $\{s_n\}$ converges.
        [This gives a converse of (a), but under the additional assumption that $n a_n \rightarrow 0$.]
        \item Derive the last conclusion from a weaker hypothesis: 
        Assume $M < \infty$, $| n a_n | \leq M$ for all $n$, 
        and $\lim \sigma_n= \sigma$. 
        Prove that $\lim s_n = \sigma$, by completing the following outline:
        
        If $m < n$, then
        \begin{equation*}
            s_n - \sigma_n 
            = \frac{m+1}{n-m}(\sigma_n - \sigma_m)
            = \frac{  1}{n-m}\sum_{i=m+1}^{n}(s_n - s_i) .
        \end{equation*}
        For these $i$,
        \begin{equation*}
            \left| s_n - s_i \right| 
            \leq \frac{(n-i)M}{i+1} 
            \leq \frac{(n-m-1)M} {m+2} .
        \end{equation*}

        Fix $\varepsilon > 0$ and associate with each $n$ the integer $m$ that satisfies
        \begin{equation*}
            m \leq \frac{n - \varepsilon}{1 + \varepsilon}
            < m+1 .
        \end{equation*}
        Then $(m + 1)/(n - m) < 1/\varepsilon$ and $\left| s_n-s_i \right| < M\varepsilon$. Hence
        \begin{equation*}
            \limsup_{n \to \infty} \left| s_n - \sigma \right| \leq M \varepsilon .
        \end{equation*}
        Since $\varepsilon$ was arbitrary, $\lim s_n = \sigma$.
    \end{asparaenum}
\end{myExercise}

\begin{myExercise}
    \label{ex:3.15}
    Definition \ref{mydef:3.21} can be extended to the case
    in which the $a_n$ lie in some fixed $\R^k$.
    Absolute convergence is defined as convergence of $\sum \left| \mathbf{a_n} \right|$ , 
    Show that Theorems \ref{thm:3.22}, \ref{thm:3.23}, \ref{thm:3.25}(a), \ref{thm:3.33}, \ref{thm:3.34 ratio test}, \ref{thm:3.42}, \ref{thm:3.45}, \ref{thm:3.47}, and \ref{thm:3.55} are true in this more general setting. 
    (Only slight modifications are required in any of the proofs.)
\end{myExercise}

