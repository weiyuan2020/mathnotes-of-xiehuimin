% chap11sec01

\section{Set functions}

If $A$ and $B$ are any two sets, 
we write $A - B$ for the set of all elements $x$ such that 
$x \in A, x \not\in B$. 
The notation $A - B$ does not imply that $B \subset A$. 
We denote
the empty set by 0, 
and say that $A$ and $B$ are disjoint if $A \cap B = 0$.


\begin{myDef}
    A family $\mathscr{R}$ of sets is called a ring if A e $\mathscr{R}$ and Be $\mathscr{R}$ implies
    \begin{equation}
        \label{eq:11.1}
        A \cup B \in \mathscr{R}, \quad 
        A - B \in \mathscr{R}.
    \end{equation}
    Since $A n B = A - (A - B)$, we also have $A \cap B \in \mathscr{R}$ if $\mathscr{R}$ is a ring.

    A ring $\mathscr{R}$ is called a $\sigma$-\emph{ring} if 
    \begin{equation}
        \label{eq:11.2}
        \bigcup_{n=1}^{\infty} A_n \in \mathscr{R}
    \end{equation}
    whenever $A_n \in \mathscr{R} (n = 1,2,3,\dots)$. 
    Since 
    \begin{equation*}
        \bigcap_{n=1}^{\infty} A_n 
        = A_1 - \bigcup_{n=1}^{\infty} (A_1 - A_n),
    \end{equation*}
    we also have 
    \begin{equation*}
        \bigcap_{n=1}^{\infty} A_n \in \mathscr{R}
    \end{equation*}
    if $\mathscr{R}$ is a $\sigma$-ring.
\end{myDef}