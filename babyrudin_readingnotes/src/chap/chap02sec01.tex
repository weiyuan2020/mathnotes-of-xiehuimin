% chap02sec01
\section{FINITE, COUNTABLE, AND UNCOUNTABLE SETS}

We begin this section with a definition of the \textbf{function} concept.
\begin{myDefinition}\label{myDefinition:2.1function}
% 2.1 Definition
Consider two sets $A$ and $B$, whose elements may be any objects whatsoever, and suppose that with each element $x$ of $A$ there is associated, in some manner, an element of $B$, which we denote by $f(x)$. Then $f$ is said to be a \emph{function} from $A$ to $B$ (or a \emph{mapping} of $A$ into $B$). The set $A$ is called the \emph{domain} of $f$ (we also say fis defined on $A$), and the elements $f(x)$ are called the \emph{values} of $f$. The set of all values of $f$ is called the \emph{range} of $f$.
\end{myDefinition}

\begin{myDefinition}\label{myDefinition:2.2onto_onetoone}
% 22 Definition 
Let $A$ and $B$ be two sets and let $f$ be a mapping of $A$ into $B$.
If $E \subset A$, $f(E)$ is defined to be the set of all elements $f(x)$, for $x \in E$. We call $f(E)$ the image of $E$ under $f$. In this notation, $f(A)$ is the range of $f$. It is clear that $f(A) \subset B$. If $f(A) = B$, we say that $f$ maps $A$ \emph{onto} $B$. (Note that, according
to this usage, \emph{onto} is more specific than \emph{into}.)\footnote{onto 满射? into 映射?} 

If $E \subset B$, $f^{-1}(E)$ denotes the set of all $x \in A$ such that $f(x)\in E$. We call $f^{-1}(E)$ the \emph{inverse image} of $E$ under $f$. If $y \in B$, $f^{-1}(y)$ is the set of all $x \in A$ such that $f(x) =y$. If, for each $y\in B$, $f^{-1}(y)$ consists of at most one element of $A$, then $f$ is said to be a 1-1 (\emph{one-to-one}) mapping of $A$ into $B$. This may also be expressed as follows: $f$ is a 1-1 mapping of $A$ into $B$ provided that $f(x_1) \neq f(x_2)$ whenever $x_1 \neq x_2$, $x_1 \in A$, $x_2 \in A$.

(The notation $x_1 \neq x_2$, means that $x_1$ and $x_2$ are distinct elements; otherwise we write $x_1 = x_2$.)
\end{myDefinition}

\begin{myDefinition}\label{myDefinition:2.3equivalent}
If there exists a 1-1 mapping of $A$ \emph{onto} $B$, we say that $A$ and $B$ can be putin 1-1 correspondence, or that $A$ and $B$ have the same cardinal number, or, briefly, that $A$ and $B$ are equivalent, and we write $A\sim B$. This relation
clearly has the following properties :

It is reflexive: $A\sim A$.

It is symmetric: If $A\sim B$, then $B\sim A$.

It is transitive: If $A\sim B$ and $B\sim C$, then $A\sim C$.

Any relation with these three properties is called an equivalence relation.\footnote{等价关系:自反性,对称性,传递性}
\end{myDefinition}