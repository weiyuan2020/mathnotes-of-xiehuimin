
\section{THE COMPLEX FIELD}

\mybox{rudin 引入复数定义的方法很奇怪, 使用复数的代数定义直接引入(天上掉下来的定义)。理解起来比较困难, 我觉得使用几何方法引入复数更为合理且直观, rudin 这里对初学者不太友好}

\begin{myDef}\label{myDef:complexnumber1.24}
    % 1.24 Definition 
    A complex number is an ordered pair $(a, b)$ of real numbers.
"Ordered" means that $(a, b)$ and $(b, a)$ are regarded as distinct if $a \neq b$.

Let$x = (a, b)$, $y = (c,d)$ be two complex numbers. We write $x =y$ if and
only if $a =c$ and $b=d$. (Note that this definition is not entirely superfluous;
think of equality of rational numbers, represented as quotients of integers.) We
define
\begin{align*}
    x+y &= (a+c, b+d),\\
    xy  &= (ac - bd, ad + bc).
\end{align*}
\end{myDef}

\begin{thm}\label{thm:complexnumberisfield}
    % 1.25 'Theorem 
    These definitions of addition and multiplication turn the \emph{set} of all complex numbers into a field, with $(0, 0)$ and $(1, 0)$ in the role $0$ and $1$.
\end{thm}

proof (A1)--(A5), (M1)--(M5) and (D), then we can prove that $\mathbb{C}$ is a field.

\begin{thm}\label{thm:realincomplex}
    For any real numbers $a$ and $b$ we have
    \begin{equation*}
        (a,0)+ (b,0) = (a+ b,0),\quad
        (a,0)(b,0) = (ab,0).
    \end{equation*}
\end{thm}
The proof is trivial.

show that the notation $(a, b)$ is equivalent to the more customary $a + bi$.

\begin{myDef}\label{myDef:image_number}
    $i=(0,1)$    
\end{myDef}

\begin{thm}\label{thm:sqartrootofminus1}
    % 1.28 Theorem
    $i^2=-1$
\end{thm}

\begin{proof}
    \begin{equation*}
        i^2=(0,1)(0,1)=(-1,0)=-1.
    \end{equation*}
\end{proof}

\begin{thm}\label{thm:complexnumberTransfer}
    If $a$ and $b$ are real, then $(a,b) =a + bi$.
\end{thm}

\begin{proof}
    \begin{align*}
        a+bi
        &=(a,0)+(b,0)(0,1)\\
        &=(a,0)+(0,b)=(a,b)\\
    \end{align*}
\end{proof}

\begin{myDef}\label{myDef:conjugate}
    $a,b\in \mathbb{R}$, $z=a+bi$,the complex number $\bar{z}=a-bi$ is called the conjugate of $z$. the numbers $a$ and $b$ are the real part and imaginary part of $z$. respectively.
    \begin{equation*}
        a=\Re(z), \quad
        b=\Im(z)
    \end{equation*}
\end{myDef}

\begin{thm}\label{thm:complexProperty}
    % 1.31 Theorem
    If $z$ and $w$ are complex, then

    (a) $\bar{z+w}=\bar{z}+\bar{w}$,

    (b) $\bar{zw}=\bar{z}\cdot\bar{w}$,

    (c) $z+\bar{z}=2\Re(z)$, $z-\bar{z}=2\Im(z)$,

    (d) $z\bar{z}$ is real and positive (except when $z=0$).
\end{thm}
Proof (a), (b),and (c)are quite trivial. To prove (d), write $z = a + bi$,
and note that $z\bar{z} = a^2 + b^2$.

\begin{myDef}\label{myDef:complex_absolutevalue}
    If $z$ is a complex number, its absolute value $|z|$ is the nonnegative square root of $z\bar{z}$; that is, $|z| = (z\bar{z})^{1/2}$.
\end{myDef}
The existence (and uniqueness) of $|z|$ follows from Theorem \ref{thm:1.21} and
part (d) of Theorem \ref{thm:complexProperty}.

Note that when $x$ is real, then $\bar{x} = x$, hence $|x| = \sqrt{x^2}$. Thus $|x| = x$
if $x>0$, $|x| = -x$ if $x <0$.

\begin{thm}\label{thm:1.33}
    Let $z$ and $w$ be complex numbers. Then
    
    (a) $|z|>0$ unless $z=0$, $|0|=0$,

    (b) $\bar{z}=z$,

    (c) $|zw| = |z||w|$,

    (d) $| \Re(z)| \leq |z|$,

    (e) $|z+w| \leq|z|+|w|$.

\end{thm}


\begin{myNotation}\label{myNotation:sum}
% 1.34 notation 
(sum)
$x_1,x_2,\dots,x_n \in \mathbb{C}$,
\begin{equation*}
    x_1+x_2+\dots+x_n = \sum_{j=1}^{n} x_j.
\end{equation*}
\end{myNotation}

\begin{thm}\label{thm:schwarz_inequality}
    (Schwarz Inequality)

    If 
    $a_1,\dots,a_n$, 
    $b_1,\dots,b_n$, are complex numbers, then
    \begin{equation*}
        \left| \sum_{j=1}^{n}a_j \bar{b_j}\right|^2 \leq 
        \sum_{j=1}^{n}\left|a_j\right|^2
        \sum_{j=1}^{n}\left|b_j\right|^2.
    \end{equation*}    
\end{thm}

在正式证明之前, 先回忆 $\mathbb{R}$ 中的施瓦茨不等式是怎么证明的.
let $A = \sum a_j^2$, $B = \sum b_j^2$, $C = \sum a_j b_j$.
\begin{equation*}
    \sum (a_j+\lambda b_j)^2 = 
    \sum a_j^2 
    + 2\sum a_j b_j \lambda
    + \sum b_j^2 \lambda^2
\end{equation*}
由韦达定理, $\Delta \leq 0$, $\Delta= (2\sum a_j b_j )^2 - 4 \sum a_j^2\sum b_j^2$. 
因此 $(\sum a_j b_j )^2 \leq \sum a_j^2\sum b_j^2$
\begin{proof}
    Put $A = \sum |a_j|^2$, $B = \sum |b_j|^2$, $C = \sum a_j \bar{b_j}$, $j = 1,2,\dots,n$. 
    
    If $B = 0$, $b_1 = \dots = b_n = 0$, this conclusion is trivial.

    If $B > 0$, 
    \begin{align*}
        \sum \left|B a_j - C b_j\right|^2
        &= \sum (B a_j-C b_j)(B \bar{a_j} - \bar{C b_j})\\
        &= B^2\sum \left|a_j\right|^2 - B\bar{C}\sum a_j \bar{b_j} - BC \sum \bar{a_j} b_j + \left|C\right|^2\sum |b_j|^2\\
        &= B^2A-B|C|^2\\
        &= B(AB-|C|^2).
    \end{align*}
    Since each term in the first sum is nonnegative, we see that
    \begin{equation*}
        B(AB-|C|^2) \geq 0.
    \end{equation*}
    Since $B>0$, it follows that $AB-|C|^2 \geq 0$. This is the desired inequality.
\end{proof}

我的想法
\begin{align*}
    \sum (a_j + \lambda \bar{b_j})(\bar{a_j} + \lambda b_j)
    &=\sum(a_j\bar{a_j}+\lambda(\bar{a_j}b_j+a_j\bar{b_j})+\lambda^2 b_j\bar{b_j})\\
    &=\sum(a_j\bar{a_j}+\lambda 2\Re(a_j\bar{b_j}) +\lambda^2 b_j\bar{b_j})
\end{align*}
由韦达定理, $\Delta \leq 0$, $\Delta = (2\sum\Re(a_j\bar{b_j}))^2-4\sum a_j\bar{a_j}\sum b_j\bar{b_j}$, 
(这里推出的结论比原始结论弱?为什么?)
% 由于 $\left( \sum\Re(a_j\bar{b_j}) \right)^2 = $
% \begin{align*}
%     \left( \sum\Re(a_j\bar{b_j}) \right)^2 &\leq \sum a_j\bar{a_j}\sum b_j\bar{b_j}\\
%     \left| \left( \sum a_j \bar{b_j} \right)\right|^2
    
% \end{align*}