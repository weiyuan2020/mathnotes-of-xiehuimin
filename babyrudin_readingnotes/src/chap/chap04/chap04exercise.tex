% chap04exercise
\section*{Exercises}

\begin{myExercise}
    \label{ex:4.1}
    Suppose $f$ is a real function defined on $\R^1$ which satisfies
    \begin{equation*}
        \lim_{n \to \infty} \left[ f(x+h) - f(x-h) \right] = 0
    \end{equation*}
    for every $x \in \R^1$.
    Does this implies that $f$ is continuous?
\end{myExercise}


\begin{myExercise}
    \label{ex:4.2}
    If $f$ is a continuous mapping of a metric space $X$ into a metric space $Y$, prove that 
    \begin{equation*}
        f(\overline{E}) \subset \overline{f(E)}
    \end{equation*}
    for every set $E \subset X$.
    ($\overline{E}$ denotes the closure of $E$.)
    Show, by an example, that $f(\overline{E})$ can be a proper subset of $\overline{f(E)}$.
\end{myExercise}


\begin{myExercise}
    \label{ex:4.3}
    Let $f$ be a continuous real function on a metric space $X$. 
    Let $Z(f)$ (the \emph{zero set} of $f$) 
    be the set of all $p \in X$ at which $f(p) = 0$. 
    Prove that $Z(f)$ is closed.
\end{myExercise}


\begin{myExercise}
    \label{ex:4.4}
    Let $f$ and $g$ be continuous mappings of a metric space $X$ into a metric space $Y$,
    and let $E$ be a dense subset of $X$. 
    Prove that $f(E)$ is dense in $f(X)$. 
    If $g(p) = f(p)$ for all $p \in E$, 
    prove that $g(p) = f(p)$ for all $p \in X$. 
    (In other words, a continuous mapping is determined by its values on a dense subset of its domain.)
\end{myExercise}


\begin{myExercise}
    \label{ex:4.5}
    If $f$ is a real continuous function defined on a closed set $E \subset \R^1$, prove that there exist continuous real functions $g$ on $\R^1$ such that $g(x) = f(x)$ for all $x \in E$. 

    (Such functions $g$ are called \emph{continuous extensions} of $f$ from $E$ to $\R^1$ .) 
    
    Show that the result becomes false if the word ``closed'' is omitted. Extend the result to vector-valued functions. 
    
    \emph{Hint}: Let the graph of $g$ be a straight line on each of the segments which constitute the complement of $E$ (compare Exercise \ref{ex:2.29}). 
    The result remains true if $\R^1$ is replaced by any metric space, but the proof is not so simple.
\end{myExercise}