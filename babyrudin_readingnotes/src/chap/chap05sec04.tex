% chap05sec04
\section{L'hospital's rule}
\mybox{洛必达法则}
The following theorem is frequently useful in the evaluation of limits.

\begin{thm}
    \label{thm:5.13}
    Suppose $f$ and $g$ are real and differentiable in $(a, b)$, 
    and $g'(x) \neq 0$ for all $x \in (a, b)$, 
    where $-\infty \leq a < b \leq + \infty$. 
    Suppose
    \begin{equation}
        \label{eq:5.13}
        \frac{f'(x)}{g'(x)}\rightarrow A 
        \text{ as } x \rightarrow a.
    \end{equation}
    If 
    \begin{equation}
        \label{eq:5.14}
        f(x) \rightarrow 0
        \text{ and }
        g(x) \rightarrow 0
        \text{ as } x \rightarrow a.
    \end{equation}
    or if 
    \begin{equation}
        \label{eq:5.15}
        g(x) \rightarrow +\infty 
        \text{ as } x \rightarrow a.
    \end{equation}
    then
    \begin{equation}
        \label{eq:5.16}
        \frac{f(x)}{g(x)} \rightarrow A
        \text{ as } x \rightarrow a.
    \end{equation}
\end{thm}
The analogous statement is of course also true if $x \rightarrow b$, 
or if $g(x) \rightarrow -\infty $ in (\ref{eq:5.15}). 
Let us note that we now use the limit concept in the extended sense of
Definition \ref{myDef:4.33}.
\mybox{对洛必达法则的证明 
分两种情况讨论
1. 分子分母同时趋于0, 
2. 分母趋于无穷大.
}
\begin{proof}
    We first consider the case in which $- \infty \leq A < + \infty$.
    Choose a real number $q$ such that $A < q$, 
    and then choose $r$ such that $A < r < q$.
    By (\ref{eq:5.13}) there is a point $c \in (a, b)$ 
    such that $a < x < c$ implies
    \begin{equation}
        \label{eq:5.17}
        \frac{f'(x)}{g'(x)} < r.
    \end{equation}
    If $a< x < y < c$, 
    then Theorem \ref{thm:5.9} shows that 
    there is a point $t \in (x, y)$
    such that
    \begin{equation}
        \label{eq:5.18}
        \frac{f(x)-f(y)}{g(x)-g(y)} = 
        \frac{f'(t)}{g'(t)} < r.
    \end{equation}
    Suppose (\ref{eq:5.14}) holds. 
    Letting $x \rightarrow a$ in (\ref{eq:5.18}), 
    we see that
    \begin{equation}
        \label{eq:5.19}
        \frac{f(y)}{g(y)} \leq r < q
        \quad
        (a< y < c).
    \end{equation}
    Next, suppose (\ref{eq:5.15}) holds.
    Keeping $y$ fixed in (\ref{eq:5.18}), 
    we can choose a point $c_1 \in (a, y)$ 
    such that $g(x) > g(y)$ and $g(x) > 0$ if $a< x < c_1$. 
    Multiplying (\ref{eq:5.18}) by $[g(x) - g(y)]/g(x)$, 
    we obtain
    \begin{equation}
        \label{eq:5.20}
        \frac{f(x)}{g(x) < r - r\frac{g(y)}{g(x)} + \frac{f(y)}{g(x)}}
        \quad
        (a < x < c_1).
    \end{equation}
    If we let $x > a$ in (\ref{eq:5.20}), 
    (\ref{eq:5.15}) shows that there is a point $c_2 \in (a, c_1)$
    such that
    \begin{equation}
        \label{eq:5.21}
        \frac{f(x)}{g(x)} < q
        \quad 
        (a < x < c_2).
    \end{equation}
    Summing up, (\ref{eq:5.19}) and (\ref{eq:5.21}) show that 
    for any $q$, subject only to the condition $A < q$, 
    there is a point $c_2$ 
    such that $f(x)/g(x) < q$ if $a< x < c_2$.
    
    In the same manner, if $- \infty  < A \leq + \infty$ , 
    and $p$ is chosen so that $p < A$, we can find a point $c_3$ 
    such that
    \begin{equation}
        \label{eq:5.22}
        p < \frac{f(x)}{g(x)}
        \quad 
        (a < x < c_3).
    \end{equation}
    and (\ref{eq:5.16}) follows from these two statements.
\end{proof}
\mybox{没看明白 需要复习}
