% chap07sec06
\section{The stone-weierstrass theorem}

\begin{thm}
    \label{thm:7.26}
    If $f$ is a continuous complex function on $[a, b]$, 
    there exists a sequence of polynomials $P_n$ such that
    \begin{equation*}
        \lim_{n \to \infty} P_n(x) = f(x)        
    \end{equation*}
    uniformly on $[a, b]$. 
    If $f$ is real, the $P_n$ may be taken real.
\end{thm}

This is the form in which the theorem was originally discovered by
Weierstrass.

% todo add proof

\begin{myCorollary}
    \label{myCorollary:7.27}
    For every interval $[-a, a]$ 
    there is a sequence of real polynomials $P_n$ such that $P_n(0) = 0$
    and such that
    \begin{equation*}
        \lim_{n \to \infty} P_n(x) = \left| x \right| 
    \end{equation*}
    uniformly on $[-a, a]$.
\end{myCorollary}

% todo add proof

\begin{mydef}
    \label{mydef:7.28}
    A family $\mathscr{A}$ of complex functions defined on a set $E$
    is said to be an \emph{algebra} if 
    (i) $f + g \in \mathscr{A}$, 
    (ii) $fg \in \mathscr{A}$, and 
    (iii) $cf \in \mathscr{A}$ 
    for all $f \in \mathscr{A}$, $g \in \mathscr{A}$
    and for all complex constants $c$, 
    that is, if $\mathscr{A}$ is closed under addition, multiplication, and scalar multiplication. 
    We shall also have to consider algebras of real functions; 
    in this case, (iii) is of course only required to hold for all real $c$.
    
    If $\mathscr{A}$ has the property that $f \in \mathscr{A}$ whenever $f_n \in \mathscr{A}$ $(n = 1, 2, 3, ... )$ and
    $f_n \rightarrow f$ uniformly on $E$, 
    then $\mathscr{A}$ is said to be uniformly closed.
    
    Let $\mathscr{B}$ be the set of all functions which are limits of uniformly convergent sequences of members of $\mathscr{A}$. 
    Then $\mathscr{B}$ is called the uniform closure of $\mathscr{A}$. 
    (See Definition \ref{mydef:7.14}.)
    
    For example, the set of all polynomials is an algebra, and the Weierstrass theorem may be stated by saying that the set of continuous functions on $[a, b]$ is the uniform closure of the set of polynomials on $[a, b]$.
\end{mydef}

\begin{thm}
    \label{thm:7.29}
    Let $\mathscr{B}$ be the uniform closure of an algebra $\mathscr{A}$ of bounded functions. 
    Then $\mathscr{B}$ is a uniformly closed algebra.
\end{thm}

% todo add proof


\begin{mydef}
    \label{mydef:7.30}
    Let $\mathscr{A}$ be a family of functions on a set $E$. 
    Then $\mathscr{A}$ is said to separate points on $E$ 
    if to every pair of distinct points $x_1, x_2 \in E$ 
    there corresponds a function $f \in \mathscr{A}$ such that
    $f(x_1) \neq -f(x_2)$.

    If to each $x \in E$ there corresponds a function $g \in \mathscr{A}$ such that $g(x) \neq 0$,    
    we say that $\mathscr{A}$ \emph{vanishes at no point of} $E$.

    The algebra of all polynomials in one variable clearly has these properties on $\R^1$. 
    An example of an algebra which does not separate points is the set of all even polynomials, say on $[-1, 1]$, 
    since $f (-x) = f (x)$ for every even function $f$
    
    The following theorem will illustrate these concepts further.
\end{mydef}

\begin{thm}
    \label{thm:7.31}
    Suppose $\mathscr{A}$ is an algebra of functions on a set $E$, 
     $\mathscr{A}$ separates points on $E$, 
     and $\mathscr{A}$ vanishes at no point of $E$. 
     Suppose $x_1, x_2$ are distinct points of $E$, 
     and $c_1, c_2$ are constants 
     (real if $\mathscr{A}$ is a real algebra). 
     Then $\mathscr{A}$ contains a function $f$ such that
    \begin{equation*}
        f(x_1) = c_1, \quad 
        f(x_2) = c_2.
    \end{equation*}
\end{thm}

% todo add proof


We now have all the material needed for Stone's generalization of the
Weierstrass theorem.

\begin{thm}
    \label{thm:7.32}
    Let $\mathscr{A}$ be an algebra of real continuous functions on a compact set $K$. 
    If $\mathscr{A}$ separates points on $K$ and if $\mathscr{A}$ vanishes at no point of $K$, 
    then the uniform closure $\mathscr{B}$ of $\mathscr{A}$ consists of all real continuous functions on $K$.
\end{thm}

% todo add proof
We shall divide the proof into four steps.



\begin{thm}
    \label{thm:7.33}
    Suppose $\mathscr{A}$ is a self-adjoint algebra of complex continuous functions on a compact set $K$, 
    $\mathscr{A}$ separates points on $K$, 
    and $\mathscr{A}$ vanishes at no point of $K$. 
    Then the uniform closure $\mathscr{B}$ of $\mathscr{A}$ consists of all complex continuous functions on $K$. 
    In other words, $\mathscr{A}$ is dense $\mathscr{C}(K)$.
\end{thm}

% todo add proof


