% chap10sec01

\section{Integration}
\begin{mydef}
    \label{mydef:10.1}
    Suppose $I^k$ is a $k$-cell in $\R^k$, consisting of all
    \begin{equation*}
        \mathbf{x} = (x_1,\dots,x_k)
    \end{equation*}
    such that 
    \begin{equation}
        \label{eq:10.1}
        a_i \leq x_i \leq b_i 
        \quad 
        (i = 1,\dots, k) ,
    \end{equation}
    $I^j$ is the $j$-cell in $\R^j$ defined by the first $j$ inequalities (\ref{eq:10.1}), and f is a real continuous function on $I^k$.

    Put $f = f_k$, and define $f_{k-1}$ on $I^{k-1}$ by
    \begin{equation*}
        f_{k-1}(x_1,\dots,x_{k-1}) = 
        \int_{a_k}^{b_k} f_k (x_1,\dots,x_{k-1},x_k) \d x_k .
    \end{equation*}
    The uniform continuity of $f_k$ on $I^k$ shows that $f_{k-1}$ is continuous on $I^{k-1}$.
    Hence we can repeat this process and obtain functions $f_j$, continuous on $I^j$, 
    such that $f_{j-1}$ is the integral of $f_j$, with respect to $x_j$, over $[a_j, b_j]$. 
    After $k$ steps we arrive at a \emph{number} $f_0$, 
    which we call the \emph{integral of $f$ over $I^k$}; 
    we write it in the form
    \begin{equation}
        \label{eq:10.2}
        \int_{I^k} f(\mathbf{x}) \d \mathbf{x}
        \text{  or  }
        \int_{I^k} f.
    \end{equation}
    
    A priori, this definition of the integral depends on the order in which the $k$ integrations are carried out. 
    However, this dependence is only apparent. 
    To prove this, let us introduce the temporary notation $L(f)$ for the integral (\ref{eq:10.2}) and $L'(f)$ for the result obtained by carrying out the $k$ integrations in some other order.
\end{mydef}
\mybox{
    $L(f)$ 积分 (\ref{eq:10.2}) 暂时的记号
    
    $L'(f)$ 用另外的次序求这 $k$ 个积分的结果
}

\begin{thm}
    \label{thm:10.2}
    For every $f \in \mathscr{C}(I^k)$, $L(f) = L'(f)$.
\end{thm}
\mybox{对区间上的连续函数, 积分结果与积分顺序无关}
% todo add proof
\mybox{Stone-Weierstrass 定理能够用到这些函数上}

\begin{mydef}
    \label{mydef:10.3}
    The \emph{support} of a (real or complex) function $f$ on $\R^k$ is the
    closure of the set of all points $\mathbf{x} \in R^k$ 
    at which $f(\mathbf{x}) \neq 0$. 
    If $f$ is a continuous function with compact support, 
    let $I^k$ be any $k$-cell which contains the support of $f$, 
    and define
    \begin{equation}
        \label{eq:10.3}
        \int_{R^k} f =
        \int_{I^k} f .
    \end{equation}
    The integral so defined is evidently independent of the choice of $I^k$, provided only that $I^k$ contains the support of $f$.

\end{mydef}

\mybox{support 支集}

It is now tempting to extend the definition of the integral over $R^k$ to
functions which are limits (in some sense) of continuous functions with compact support. 
We do not want to discuss the conditions under which this can be done; 
the proper setting for this question is the Lebesgue integral. 
We shall merely describe one very simple example which will be used in the proof of Stokes' theorem.

\begin{myExample}
    % todo add example
    \begin{equation}
        \label{eq:10.4}
        \int_{Q^k} f =
        \int_{I^k} f .
    \end{equation}

\end{myExample}

Our next goal is the change of variables formula stated in Theorem 10.9.
To facilitate its proof, we first discuss so-called primitive mappings, and partitions of unity. 
Primitive mappings will enable us to get a clearer picture of the
local action of $\mathscr{C}'$-mapping with invertible derivative, and partitions of unity are a very useful device 
that makes it possible to use local information in a global setting.