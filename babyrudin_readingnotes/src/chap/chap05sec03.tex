% chap05sec03
\section{The continuity of derivatives}

We have already seen [Example 5.6(b)] that 
a function $f$ may have a derivative $f'$ which exists at every point, 
but is discontinuous at some point. 
However, not every function is a derivative. 
In particular, derivatives which exist at every point of an interval 
have one important property in common with functions
which are continuous on an interval:
Intermediate values are assumed (compare Theorem \ref{thm:4.23}). 
The precise statement follows

\mythm{
    Suppose $f$ is a real differentiable function on $[a, b]$ 
    and suppose $f'(a) < \lambda <f'(b)$. 
    Then there is a point $x \in (a, b)$ 
    such that $f'(x) = \lambda$.
}
A similar result holds of course if $f'(a) > f'(b)$.
\myproof{
    Put $g(t) = f(t) - \lambda t$. 
    Then $g'(a) < 0$, so that $g(t_1) < g(a)$ for some $t_1 \in (a, b)$, 
    and  $g'(b) > 0$, so that $g(t_2) < g(b)$ for some $t_2 \in (a, b)$. 
    Hence $g$ attains its minimum on $[a, b]$ (Theorem \ref{thm:4.16}) at some point $x$ such that $a < x < b$. 
    By Theorem \ref{thm:5.8}, $g'(x) = 0$. 
    Hence $f'(x) = \lambda$.
}

\textbf{Corollary}

If $f$ is differentiable on $[a, b]$, 
then $f'$ cannot have any simple discontinuities on $[a, b]$.

But $f'$ may very well have discontinuities of the second kind.