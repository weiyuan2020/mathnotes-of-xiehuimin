% chap03sec02

\section{Subsequences}
\begin{myDef}\label{myDef:3.5 Subsequence}
    Given a sequence $\{p_n\}$, consider a sequence $\{n_k\}$ of positive integers, such that $n_1 <n_2 <n_3 <....$ Then the sequence $\{p_{n_i}\}$ is called a \emph{subsequence} of $\{p_n\}$. If $\{p_{n_i}\}$ converges, its limit is called a subsequential limit of $\{p_n\}$.
\end{myDef}
It is clear that $\{p_n\}$ converges to $p$ if and only if every subsequence of $\{p_{n}\}$ converges to $p$. We leave the details of the proof to the reader.

\begin{thm}\label{thm:3.6}
    (a) If $\{p_{n}\}$ is a sequence in a compact metric space $X$, then some subsequence of $\{p_{n}\}$ converges to a point of $X$.

    (b) Every bounded sequence in $\mathbb{R}^k$ contains a convergent subsequence.
\end{thm}

\begin{thm}\label{thm:3.7}
    The subsequential limits of a sequence $\{p_{n}\}$  in a metric space $X$ form a closed subset of $X$.
\end{thm}

