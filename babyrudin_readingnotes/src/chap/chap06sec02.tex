% chap06sec02
\section{Properties of the integral}
\mybox{积分的性质}
\begin{thm}
    \label{thm:6.12}
    \begin{asparaenum}[(a)]
        \item If $f_1 \in \mathscr{R}(\alpha)$ and $f_2 \in \mathscr{R}(\alpha)$, then 
        \begin{equation*}
            f_1 + f_2 \in \mathscr{R}(\alpha),
        \end{equation*}
        $cf \in \mathscr{R}(\alpha)$ for every constant $c$, and 
        \begin{align*}
            \int_{a}^{b} (f_1+f_2) \d \alpha 
            &= \int_{a}^{b} f_1 \d \alpha 
            +  \int_{a}^{b} f_2 \d \alpha , \\ 
            \int_{a}^{b} cf \d \alpha 
            &= c \int_{a}^{b} f \d \alpha .
        \end{align*}
        
        \item If $f_1(x) \leq f_2(x)$ on $[a, b]$ , then
        \begin{equation*}
            \int_{a}^{b} f_1 \d \alpha \leq
            \int_{a}^{b} f_2 \d \alpha .
        \end{equation*}

        \item If $f \in \mathscr{R}(\alpha)$ on $[a, b]$ and if $a<c<b$ , 
        then $f \in \mathscr{R}(\alpha)$ on $[a, c]$ and on $[c, b]$ , and 
        \begin{equation*}
            \int_{a}^{c} f \d \alpha +
            \int_{c}^{b} f \d \alpha =
            \int_{a}^{b} f \d \alpha .
        \end{equation*}
        
        \item If $f \in \mathscr{R}(\alpha)$ on $[a, b]$ and 
        if $\left| f(x) \right| \leq M$ on $[a, b]$ , then 
        \begin{equation*}
            \left| \int_{a}^{b} f \d \alpha \right| \leq
            M\left[ \alpha(b) - \alpha(a) \right].
        \end{equation*}

        \item If $f \in \mathscr{R}(\alpha_1)$ and $f \in \mathscr{R}(\alpha_2)$, 
        then $f \in \mathscr{R}(\alpha_1 + \alpha_2)$ and 
        \begin{equation*}
            \int_{a}^{b} f \d (\alpha_1 + \alpha_2) = 
            \int_{a}^{b} f \d \alpha_1 + 
            \int_{a}^{b} f \d \alpha_2 .
        \end{equation*}
        If $f \in \mathscr{R}(\alpha)$ and $c$ is a positive constant,
        then $f \in \mathscr{R}(c\alpha)$ and 
        \begin{equation*}
            \int_{a}^{b} f \d c\alpha = 
            c\int_{a}^{b} f \d \alpha .
        \end{equation*}
    \end{asparaenum}
\end{thm}
\mybox{黎曼积分的性质
    \begin{enumerate}[(a)]
        \item 线性 (求和和数乘)
        \item 偏序性
        \item 积分域可加
        \item 积分结果范围估计(较粗略)
        \item 积分变量的线性?
    \end{enumerate}
}
\begin{proof}
    If $f = f_1 + f_2$ and $P$ is any partition of $[a, b]$, we have
    \begin{equation}
        \label{eq:6.20}
        L(P, f_1, \alpha) + 
        L(P, f_2, \alpha) \leq
        L(P, f, \alpha) \leq
        U(P, f, \alpha) \leq
        U(P, f_1, \alpha) +
        U(P, f_2, \alpha) .
    \end{equation}
    If $f_1 \in \mathscr{R}(\alpha)$ and $f_2 \in \mathscr{R}(\alpha)$, 
    let $\varepsilon > 0$ be given.
    There are partitions $P_j$ $(j = 1, 2)$ such that 
    \begin{equation*}
        U(P_j, f_j, \alpha) -
        L(P_j, f_j, \alpha) < \varepsilon .
    \end{equation*}
    These inequalities persist if $P_1$ and $P_2$ are replaced by their common refinement $P$. Then (\ref{eq:6.20}) implies 
    \begin{equation*}
        U(P, f, \alpha) -
        L(P, f, \alpha) < 2 \varepsilon .
    \end{equation*}
    which proves that $f \in \mathscr{R}(\alpha)$.

    With this same $P$ we have 
    \begin{equation*}
        U(P, f_j, \alpha) <
        \int f_j \d \alpha + \varepsilon
        \quad (j = 1, 2);
    \end{equation*}
    hence (\ref{eq:6.20}) implies
    \begin{equation*}
        \int f \d \alpha \leq
        U(P, f, \alpha) <
        \int f_1 \d \alpha +
        \int f_2 \d \alpha +
        2 \varepsilon.
    \end{equation*}
    Since $\varepsilon$ was arbitrary, we conclude that 
    \begin{equation}
        \label{eq:6.21}
        \int f \d \alpha \leq
        \int f_1 \d \alpha +
        \int f_2 \d \alpha .
    \end{equation}

    If we replace $f_1$ and $f_2$ in (\ref{eq:6.21}) by $-f_1$ and $-f_2$, the inequality is reversed, and the equality is proved.
    
    The proofs of the other assertions of Theorem \ref{thm:6.12} are so similar that we omit the details. 
    In part (c) the point is that (by passing to refinements) 
    we may restrict ourselves to partitions which contain the point $c$,
    in approximating $\int f \d \alpha$.
\end{proof}

\begin{thm}
    \label{thm:6.13}
    If $f \in \mathscr{R}(\alpha)$ and $g \in \mathscr{R}(\alpha)$ on $[a, b]$ , then 
    \begin{enumerate}[(a)]
        \item $fg \in \mathscr{R}(\alpha)$
        \item $\left| f \right| \in \mathscr{R}(\alpha)$ and $\left| \int_{a}^{b} f \d \alpha \right| \leq \int_{a}^{b} \left| f \right| \d \alpha$.
    \end{enumerate}
\end{thm}
% todo add proof

\begin{myDef}
    \label{myDef:6.14}
    The unit step function I is defined by
    \begin{equation*}
        I(x) = \left\{ 
            \begin{array}{ll}
                0 & (x \leq 0), \\
                1 & (x >    0).
            \end{array}
         \right.
    \end{equation*}
\end{myDef}

\begin{thm}
    \label{thm:6.15}
    If $a < s < b$, $f$ is bounded on $[a, b ]$, 
    $f$ is continuous at $s$, and
    $\alpha(x) = I(x - s)$, then
    \begin{equation*}
        \int_{a}^{b} f \d \alpha = f(s).
    \end{equation*}
\end{thm}
% todo add proof

\begin{thm}
    \label{thm:6.16}
    Suppose $c_n \geq 0$ for $1, 2, 3, \dots$,
    $\sum c_n$ converges, 
    $\sequence{s_n}$ is a sequence of distinct points in $(a, b)$, 
    and 
    \begin{equation}
        \label{eq:6.22}
        \alpha (x) = 
        \sum_{n=1}^{\infty} c_n I(x - s_n).
    \end{equation}
    Let $f$ be continuous on $[a,b]$. Then 
    \begin{equation}
        \label{eq:6.23}
        \int_{a}^{b} f \d \alpha =
        \sum_{n=1}^{\infty} c_n f(s_n).
    \end{equation}
\end{thm}
% todo add proof

\begin{thm}
    \label{thm:6.17}
    Assume $\alpha$ increases monotonically and 
    $\alpha' \in \mathscr{R}$ on $[a,b]$ .
    Let $f$ be a bounded real function on $[a,b]$ .

    Then $f \in \mathscr{R}(\alpha)$ if and only if $f \alpha' \in \mathscr{R}$.
    In that case 
    \begin{equation}
        \label{eq:6.27}
        \int_{a}^{b} f \d \alpha = 
        \int_{a}^{b} f(x) \alpha'(x) \d x .
    \end{equation}
\end{thm}
\mybox{积分变量求导}
% todo add proof

\begin{myRemark}
    \label{myRemark:6.18}
    The two preceding theorems illustrate the generality and flexibility which are inherent in the Stieltjes process of integration. 
    If $\alpha$ is a pure step function 
    [this is the name often given to functions of the form (22)], 
    the integral reduces to a finite or infinite series. 
    If $\alpha$ has an integrable derivative, 
    the integral reduces to an ordinary Riemann integral. 
    This makes it possible in many cases to study series and integrals simultaneously, rather than separately.
    
    To illustrate this point, consider a physical example. 
    The moment of inertia of a straight wire of unit length, 
    about an axis through an endpoint, at right angles to the wire, is
    \begin{equation}
        \label{eq:6.33}
        \int_{0}^{1} x^2 \d m
    \end{equation}
    where $m(x)$ is the mass contained in the interval $[0, x]$. 
    If the wire is regarded as having a continuous density $\rho$, 
    that is, if $m'(x) = \rho(x)$, then (\ref{eq:6.33}) turns into
    \begin{equation}
        \label{eq:6.34}
        \int_{0}^{1} x^2 \rho(x) dx.
    \end{equation}
    
    On the other hand, 
    if the wire is composed of masses $m_i$ concentrated at points $x_i$, 
    (\ref{eq:6.33}) becomes
    \begin{equation}
        \label{eq:6.35}
        \sum_{i} x_i^2 m_i.
    \end{equation}
    
    Thus (\ref{eq:6.33}) contains (\ref{eq:6.34}) and (\ref{eq:6.35}) as special cases, but it contains much more; 
    for instance, the case in which $m$ is continuous but not everywhere differentiable.
\end{myRemark}

\begin{thm}
    \label{thm:6.19}
    \textbf{change of variable}
    Suppose $\phi$ is a strictly increasing continuous function that
    maps an interval $[A, B]$ onto $[a, b]$. 
    Suppose $\alpha$ is monotonically increasing on $[a, b]$ and
    $f \in \mathscr{R}(\alpha)$ on $[a, b]$. 
    Define $\beta$ and $g$ on $[A, B]$ by
    \begin{equation}
        \label{eq:6.36}
        \beta(y) = \alpha(\phi(y)),
        \quad 
        g(y) = f(\phi(y)).
    \end{equation}
    Then $g \in \mathscr{R}(\beta)$ and 
    \begin{equation}
        \label{eq:6.37}
        \int_{A}^{B} g \d \beta =
        \int_{a}^{b} f \d \alpha.
    \end{equation}
\end{thm}
\mybox{更改积分变量}
% todo add proof
