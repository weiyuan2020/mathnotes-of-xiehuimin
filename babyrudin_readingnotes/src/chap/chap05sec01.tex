% chap05sec01
\section{The derivative of a real function}
\mybox{实函数的导数}
\begin{myDef}
    \label{myDef:5.1}
    Let $f$ be defined (and real-valued) on $[a, b]$. 
    For any $x \in [a, b]$ form the quotient
    \begin{equation}
        \label{eq:5.1}
        \phi(t) = \frac{f(t) - f(x)}{t - x}
        \quad
        (a < t < b, t \neq x),
    \end{equation}
    and define
    \begin{equation}
        \label{eq:5.2}
        f'(x) = \lim_{t \to x} \phi(t),
    \end{equation}
    provided this limit exists in accordance with Definition 4.1.

    We thus associate with the function $f$ a function $f'$ whose domain is the set of points $x$ at which the limit (\ref{eq:5.2}) exists; 
    $f'$ is called the \emph{derivative} of $f$.
    \mybox{函数的导数}

    If $f'$ is defined at a point $x$, we say that $f$ is \emph{differentiable} at $x$. 
    If $f'$ is defined at every point of a set $E \subset [a, b]$, we say that $f$ is differentiable on $E$.
    \mybox{在某点可导, 在整个区间可导}
    
    It is possible to consider right-hand and left-hand limits in (\ref{eq:5.2}); 
    this leads to the definition of right-hand and left-hand derivatives. 
    In particular, at the endpoints $a$ and $b$, the derivative, 
    if it exists, is a right-hand or left-hand derivative, respectively. 
    We shall not, however, discuss one-sided derivatives in any detail.
    \mybox{类似左连续和右连续, 左导和右导}
    
    If $f$ is defined on a segment $(a, b)$ and if $a < x < b$, 
    then $f'(x)$ is defined by (\ref{eq:5.1}) and (\ref{eq:5.2}), as above. 
    But $f'(a)$ and $f'(b)$ are not defined in this case.
\end{myDef}

\begin{thm}
    \label{thm:5.2}
    Let $f$ be defined on $[a, b]$. 
    If $f$ is differentiable at a point $x \in [a, b]$,
    then $f$ is continuous at $x$.
\end{thm}
\mybox{可导必连续,连续不一定可导}
\begin{proof}
    As $t \rightarrow x$, we have, by Theorem \ref{thm:4.4}
    \begin{equation*}
        f(t) - f(x) = \frac{f(t)-f(x)}{t-x}\cdot (t-x)
        \rightarrow f'(x)\cdot 0 = 0.
    \end{equation*}
\end{proof}
The converse of this theorem is not true. 
It is easy to construct continuous functions 
which fail to be differentiable at isolated points. 
In Chap. 7 we shall even become acquainted with a function 
which is continuous on the whole line 
without being differentiable at any point!

\begin{thm}
    \label{thm:5.3}
    Suppose $f$ and $g$ are defined on $[a, b]$ and are differentiable at a point $x \in [a, b]$. 
    Then $f + g$, $fg$, and $f/g$ are differentiable at $x$, and
    \begin{enumerate}[(a)]
        \item $(f + g)'(x) = f'(x) + g'(x)$;
        \item $(fg)'(x) = f'(x)g(x) + f(x)g'(x)$;
        \item $\left( \cfrac{f}{g} \right)'(x) = \cfrac{g(x)f'(x) - g'(x)f(x)}{g^{2}(x)}$.
    \end{enumerate}
    In (c), we assume of course that $g(x) \neq 0$.
\end{thm}
\begin{proof}
    (a) is clear, by Theorem \ref{thm:4.4}. Let $h = fg$. Then 
    \begin{equation*}
        h(t) - h(x) = 
        f(t)\left[g(t) - g(x)\right] + 
        g(x)\left[f(t) - f(x)\right].
    \end{equation*}
    If we divide this by $t - x$ and note that $f(t) \rightarrow f(x)$
    as $t \rightarrow x$ (Theorem 5.2), 
    (b) follows. 
    Next, let $h = f/g$. Then 
    \begin{equation*}
        \frac{h(t) - h(x)}{t - x} = 
        \frac{1}{g(t)g(x)}\left[
            g(x) \frac{f(t) - f(x)}{t - x} -
            f(x) \frac{g(t) - g(x)}{t - x}
        \right].
    \end{equation*}
    Letting $t \rightarrow x$, and applying Theorem \ref{thm:4.4} and \ref{thm:5.2}, we obtain (c).
\end{proof}

\begin{myExample}
    The derivative of any constant is clearly zero. 
    If $f$ is defined by $f(x) = x$, then $f'(x) = 1$. 
    Repeated application of (b) and (c) then shows that $x^n$ is differentiable, 
    and that its derivative is $nx^{n-1}$ , 
    for any integer $n$ (if $n < 0$, we have to restrict ourselves to $x \neq 0$). 
    Thus every polynomial is differentiable, 
    and so is every rational function, 
    except at the points where the denominator is zero.
\end{myExample}

The following theorem is known as the ``chain rule'' for differentiation. 
It deals with differentiation of composite functions 
and is probably the most important theorem about derivatives. 
We shall meet more general versions of it in Chap. 9.
\mybox{链导法则 by Leibniz}
\begin{thm}
    \label{thm:5.5}
    Suppose $f$ is continuous on $[a, b]$,
    $f'(x)$ exists at some point $x \in [a, b]$, 
    $g$ is defined on an interval $I$ which contains the range of $f$, 
    and $g$ is differentiable at the point $f(x)$. If
    \begin{equation*}
        h(t) = g(f(t))
        \quad
        (a \leq t \leq b),
    \end{equation*}
    then $h$ is differentiable at $x$, and 
    \begin{equation}
        \label{eq:5.3}
        h'(x) = g'(f(x))f'(x).
    \end{equation}
\end{thm}

\begin{proof}
    Let $y = f(x)$. By the definition of the derivative, we have
    \begin{equation}
        \label{eq:5.4}
        f(t) - f(x) = (t - x) \left[ f'(x) + u(t) \right],
    \end{equation}
    \begin{equation}
        \label{eq:5.5}
        g(s) - g(y) = (s - y) \left[ g'(y) + v(s) \right],
    \end{equation}
    where $t \in [a, b]$, $s \in I$, and $u(t) \rightarrow 0$ as $t \rightarrow x$, $v(s) \rightarrow 0$ as $s \rightarrow y$.
    Let $s = f(t)$.
    
    Using first (\ref{eq:5.5}) and then (\ref{eq:5.4}), we obtain 
    \begin{align*}
        h(t) - h(x)
        &= g(f(t)) - g(f(x)) \\
        &= \left[ f(t) - f(x) \right] \cdot \left[ g'(y) + v(s) \right] \\
        &= (t - x)\cdot \left[ f'(x) + u(t) \right]\cdot \left[ g'(y) + v(s) \right],
    \end{align*}
    or, if $t \neq x$,
    \begin{equation}
        \label{eq:5.6}
        \frac{h(t) - h(x)}{t - x} =
        \left[ g'(y) + v(s) \right]\cdot
        \left[ f'(x) + u(t) \right],
    \end{equation}
    Letting $t \rightarrow x$, we see that $s \rightarrow y$, 
    by the continuity of $f$, so that the right side of (\ref{eq:5.6}) ends to $g'(y)f'(x)$, which gives (\ref{eq:5.3}).
\end{proof}

\begin{myExample}
    \begin{asparaenum}[(a)]
        \item 
        \begin{equation}
            \label{eq:5.7}
            f(x) = 
            \begin{array}{lc}
                x\sin \frac{1}{x} & (x \neq 0),\\
                0 & (x = 0).
            \end{array}
        \end{equation}
        
        apply Theorem \ref{thm:5.3} and \ref{thm:5.5} whenever $x \neq 0$,
        \begin{equation}
            \label{eq:5.8}
            f'(x) = \sin \frac{1}{x} - \frac{1}{x} \cos \frac{1}{x}
            \quad
            (x \neq 0).
        \end{equation}
        At $x = 0$, these theorems do not apply any longer, since $1/x$ is not defined there, and we appeal directly to the definition: for $t \neq 0$,
        \begin{equation*}
            \frac{f(t) - f(0)}{t - 0} = \sin \frac{1}{t}.
        \end{equation*}
        As $t \rightarrow 0$, this does not tend to any limit, so that $f'(0)$ does not exist.
    
        \item 
        \begin{equation}
            \label{eq:5.9}
            f(x) = 
            \begin{array}{lc}
                x^2\sin \frac{1}{x} & (x \neq 0),\\
                0 & (x = 0).
            \end{array}
        \end{equation}
        As above, we obtain 
        \begin{equation}
            \label{eq:5.10}
            f'(x) = 2x \sin \frac{1}{x} - \cos \frac{1}{x}
            \quad
            (x \neq 0).
        \end{equation}
        At $x = 0$, we appeal to the definition, and obtain 
        \begin{equation*}
            \left| \frac{f(t) - f(0)}{t - 0} \right| = 
            \left| t \sin \frac{1}{t} \right| \leq \left| t \right| 
            \quad
            (t \neq 0);
        \end{equation*}
        letting $t \rightarrow 0$, we see that 
        \begin{equation}
            \label{eq:5.11}
            f'(0) = 0.
        \end{equation}
        Thus $f$ is differentiable at all points $x$,
        but $f'$ is not a continuous function,
        since $\cos (1/x)$ in (\ref{eq:5.10}) does not tend to a limit as $x \rightarrow 0$.
        \end{asparaenum}
\end{myExample}