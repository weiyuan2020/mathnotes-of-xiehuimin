% chap11
\chapter{The lebesgue theory}
\label{chap:11}

It is the purpose of this chapter to present the fundamental concepts of the
Lebesgue theory of measure and integration and to prove some of the crucial
theorems in a rather general setting, without obscuring the main lines of the
development by a mass of comparatively trivial detail. Therefore proofs are
only sketched in some cases, and some of the easier propositions are stated
without proof. However, the reader who has become familiar with the techniques 
used in the preceding chapters will certainly find no difficulty in 
supplying the missing steps.

The theory of the Lebesgue integral can be developed in several distinct
ways. Only one of these methods will be discussed here. For alternative
procedures we refer to the more specialized treatises on integration listed in
the Bibliography.

% chap11sec01

\section{Set functions}

If $A$ and $B$ are any two sets,
we write $A - B$ for the set of all elements $x$ such that
$x \in A, x \not\in B$.
The notation $A - B$ does not imply that $B \subset A$.
We denote
the empty set by 0,
and say that $A$ and $B$ are disjoint if $A \cap B = 0$.


\begin{mydef}
    A family $\mathscr{R}$ of sets is called a ring if A e $\mathscr{R}$ and Be $\mathscr{R}$ implies
    \begin{equation}
        \label{eq:11.1}
        A \cup B \in \mathscr{R}, \quad
        A - B \in \mathscr{R}.
    \end{equation}
    Since $A n B = A - (A - B)$, we also have $A \cap B \in \mathscr{R}$ if $\mathscr{R}$ is a ring.

    A ring $\mathscr{R}$ is called a $\sigma$-\emph{ring} if
    \begin{equation}
        \label{eq:11.2}
        \bigcup_{n=1}^{\infty} A_n \in \mathscr{R}
    \end{equation}
    whenever $A_n \in \mathscr{R} (n = 1,2,3,\dots)$.
    Since
    \begin{equation*}
        \bigcap_{n=1}^{\infty} A_n
        = A_1 - \bigcup_{n=1}^{\infty} (A_1 - A_n),
    \end{equation*}
    we also have
    \begin{equation*}
        \bigcap_{n=1}^{\infty} A_n \in \mathscr{R}
    \end{equation*}
    if $\mathscr{R}$ is a $\sigma$-ring.
\end{mydef}

\begin{mydef}
    \label{mydef:11.2}
    We say that $\phi$ is a set function defined on $\mathscr{R}$ if $\phi$ assigns to every $A \in \mathscr{R}$ a number $@f(A)$ of the extended real number system.
    $\phi$ is \emph{additive} if $A \cap B = 0$ implies
    \begin{equation}
        \label{eq:11.3}
        \phi \left( A \cup B \right) =
        \phi (A) + \phi (B),
    \end{equation}
    and $\phi$ is \emph{countably additive} if $A_i \cap A_j = 0 (i \neq j)$ implies
    \begin{equation}
        \label{eq:11.4}
        \phi\left( \bigcup_{n=1}^{\infty} A_n \right) =
        \sum_{n=1}^{\infty} \phi\left( A_n \right) .
    \end{equation}
    We shall always assume that the range of $\phi$ does not contain both $+ \infty$ and $- \infty$;
    for if it did, the right side of (\ref{eq:11.3}) could become meaningless.
    Also, we exclude set functions whose only value is $+ \infty$ or $- \infty$.

    It is interesting to note that the left side of (\ref{eq:11.4}) is independent of the order in which the $A_n$'s are arranged.
    Hence the rearrangement theorem shows that the right side of (\ref{eq:11.4}) converges absolutely if it converges at all;
    if it does not converge, the partial sums tend to $+ \infty$, or to $- \infty$.

    If $\phi$ is additive, the following properties are easily verified:
    \begin{align}
        \phi(0) & = 0 \label{eq:11.5}                              \\
        \phi \left( A_1 \cup \cdots \cup A_n \right)
                & = \phi(A_1) + \cdots + \phi(A_n) \label{eq:11.6}
    \end{align}
    if $A_i \cap A_j = 0$ whenever $i \neq j$.
    \begin{equation}
        \label{eq:11.7}
        \phi \left( A_1 \cup A_2 \right) +
        \phi \left( A_1 \cap A_2 \right) =
        \phi (A_1) + \phi (A_2).
    \end{equation}

    If $\phi(A) \geq 0$ for all $A$, and $A_1 \subset A_2$, then
    \begin{equation}
        \label{eq:11.8}
        \phi(A_1) \leq \phi(A_2) .
    \end{equation}

    Because of (\ref{eq:11.8}), nonnegative additive set functions are often called monotonic.
    \begin{equation}
        \label{eq:11.9}
        \phi\left( A - B \right) =
        \phi\left( A \right) -
        \phi\left( B \right)
    \end{equation}
    if $B \subset A$, and $\left| \left( \phi B \right) \right| < +\infty $.
\end{mydef}

\begin{thm}
    \label{thm:11.3}
    Suppose $\phi$ is countably additive on a ring $\mathscr{R}$.
    Suppose $A_n \in \mathscr{R} (n = 1,2,3,\dots)$,
    $A_1 \subset A_2 \subset A_3 \subset \cdots$, $A \in \mathscr{R}$, and
    \begin{equation*}
        A = \bigcup_{n=1}^{\infty} A_n .
    \end{equation*}
    Then, as $n \rightarrow \infty$,
    \begin{equation*}
        \phi(A_n) \rightarrow \phi(A) .
    \end{equation*}
\end{thm}


% chap11sec02

\section{Constriction of the lebesgue measure}

\begin{mydef}
    \label{mydef:11.4}
    Let $\R^p$ denote $p$-dimensional euclidean space.
    By an \emph{interval} in $\R^p$ we mean the set of points $\mathbf{x} = (x_1 , ... , x_p)$ such that
    \begin{equation}
        \label{eq:11.10}
        a_i \leq x_i \leq b_i
        \quad
        (i = 1,\dots,p),
    \end{equation}
    or the set of points which is characterized by (\ref{eq:11.10}) with any or all of the $\leq$ signs replaced by $<$.
    The possibility that $a_i = b_i$ for any value of $i$ is not ruled out;
    in particular, the empty set is included among the intervals.

    If $A$ is the union of a finite number of intervals,
    $A$ is said to be an \emph{elementary set}.

    If $I$ is an interval, we define
    \begin{equation*}
        m(I) = \prod_{i=1}^{p} (b_i - a_i) ,
    \end{equation*}
    no matter whether equality is included or excluded in any of the inequalities (\ref{eq:11.10}).

    If $A = I_1 \cup \cdots \cup I_n$, and if these intervals are pairwise disjoint, we set
    \begin{equation}
        \label{eq:11.11}
        m(A) =
        m(I_1) + \cdots +
        m(I_n) .
    \end{equation}

    We let $\mathscr{E}$ denote the family of all elementary subsets of $\R^p$.

    At this point, the following properties should be verified:
    \begin{enumerate}[(a)]
        \item $\mathscr{E}$ is a ring, but not a $\sigma$-ring.
        \item If $A \in \mathscr{E}$, then $A$ is the union of a finite number of \emph{disjoint} intervals.
        \item If $A \in \mathscr{E}$, $m(A)$ is well defined by (\ref{eq:11.11}); that is. if two different decompositions of $A$ into disjoint intervals are used, each gives rise to the same value of $m(A)$.
        \item $m$ is additive on $\mathscr{E}$
    \end{enumerate}

    Note that if $p = 1,2,3$, then $m$ is length, area, and volume, respectively.
\end{mydef}

\mybox{原书这里的列表项使用公式编号记录...}

\begin{mydef}
    \label{mydef:11.5}
    A nonnegative additive set function $\phi$ defined on $\mathscr{E}$ is said to be regular if the following is true:
    To every $A \in \mathscr{E}$ and to every $\varepsilon > 0$
    there exist sets $F \in \mathscr{E}$, $G \in \mathscr{E}$
    such that $F$ is closed, $G$ is open, $F \subset A \subset G$,
    and
    \begin{equation}
        \label{eq:11.16}
        \phi(G) - \varepsilon \leq
        \phi(A) \leq
        \phi(F) + \varepsilon .
    \end{equation}
\end{mydef}

\begin{newexample}
    \label{neqexample:11.6}
    \begin{asparaenum}[(a)]
        \item \emph{The set function $m$ is regular.}
        If $A$ is an interval, it is trivial that the requirements of Definition \ref{mydef:11.5} are satisfied. The general case follows from \ref{mydef:11.4} property (b).
        \item Take $\R^p = \R^1$, and let $\alpha$ be a monotonically increasing function, defined for all real $x$. Put
        \begin{align*}
            \mu([a,b]) & = \alpha(b-)-\alpha(a-), \\
            \mu([a,b]) & = \alpha(b+)-\alpha(a+), \\
            \mu([a,b]) & = \alpha(b+)-\alpha(a+), \\
            \mu([a,b]) & = \alpha(b-)-\alpha(a-).
        \end{align*}
        Here $[a,b)$ is the set $a \leq x < b$, etc.
        Because of the possible discontinuities of $\alpha$, these cases have to be distinguished.
        If $\mu$ is defined for elementary sets as in (\ref{eq:11.11}), $\mu$ is regular on $\mathscr{E}$.
        The proof is just like that of (a)
    \end{asparaenum}
\end{newexample}

Our next objective is to show that every regular set function on $\mathscr{E}$ can be
extended to a countably additive set function on a $\sigma$-ring which contains $\mathscr{E}$.

\begin{mydef}
    \label{mydef:11.7}
    outer measure $\mu^*(E)$
    % Let µ be additive, regular, nonnegative, and finite on 8.
    % Consider countable coverings of any set E c RP by open elementary sets An:
\end{mydef}

\begin{thm}
    \label{thm:11.8}
    \begin{asparaenum}[(a)]
        \item For every $A \in \mathscr{E}$, $\mu^* (A) = \mu (A)$.
        \item If $E = \cup_1^{\infty} E_n$, then
        \begin{equation}
            \label{eq:11.19}
            \mu^* (E) \leq
            \sum_{n=1}^{\infty} \mu^* (E_n) .
        \end{equation}
    \end{asparaenum}
\end{thm}

Note that (a) asserts that $\mu^*$ is an extension of $\mu$ from $\mathscr{E}$ to the family of all subsets of $\R^P$.
The property (\ref{eq:11.19}) is called \emph{subadditivity}.

% todo add proof 

\begin{mydef}
    \label{mydef:11.9}
    finitely $\mu$-measurable
    $A \in \mathfrak{M}_F(\mu)$
\end{mydef}

\begin{thm}
    \label{thm:11.10}
    $\mathfrak{M}(\mu)$ is a $\sigma$-ring, and $\mu^*$ is countably additive on $\mathfrak{M}(\mu)$.
\end{thm}

properties of $S(A, B)$ and $d(A, B)$.

\begin{myremark}
    \label{myremark:11.11}
\end{myremark}
% chap11sec03

\section{Measure space}


\begin{mydef}
    measurable space

    notation 
    \begin{equation}
        \label{eq:11.41}
        \{x |P\} 
    \end{equation}
    the set of all elements $x$ which have the property $P$.
\end{mydef}
% chap11sec04

\section{Measurable functions}

\begin{mydef}
    \label{mydef:11.13}
    measurable function
    \begin{equation}
        \label{eq:11.42}
        \{x|f(x) > a\}
    \end{equation}
    is measurable for every real $a$.
\end{mydef}

\begin{newexample}
    If $X = \R^P$ and $\mathfrak{M} = \mathfrak{M}(\mu)$ 
    % (µ) 
    as defined in Definition \ref{mydef:11.9}, every continuous $f$ is measurable, since then (\ref{eq:11.42}) is an open set.
\end{newexample}

\begin{thm}
    \label{thm:11.15}
    Each of the following four conditions implies the other three:
    \begin{enumerate}[(a)]
        \item $\{x|f(x) >    a\}$ is measurable for every real $a$.
        \item $\{x|f(x) \geq a\}$ is measurable for every real $a$.
        \item $\{x|f(x) <    a\}$ is measurable for every real $a$.
        \item $\{x|f(x) \leq a\}$ is measurable for every real $a$.
    \end{enumerate}
\end{thm}

\begin{thm}
    \label{thm:11.16}
    If $f$ is measurable, then $\left| f \right|$ is measurable. 
\end{thm}

\begin{proof}
    \begin{equation*}
        \{x | \left| f(x) \right| < a\} = 
        \{x | f(x) <  a\} \cap 
        \{x | f(x) > -a\} .
    \end{equation*}
\end{proof}

\begin{thm}
    \label{thm:11.17}
    Let $\{f_n\}$ be a sequence of measurable functions. 
    For $x \in X$, put
    \begin{align*}
        g(x) &= \sup f_n(x) \quad (n=1,2,3,\dots), \\
        f(x) &= \limsup_{n \rightarrow \infty} f_n (x) .
    \end{align*}
    Then $g$ and $h$ are measurable.
\end{thm}

The same is of course true of the $\inf$ and $\liminf$.

\begin{proof}
    \begin{align*}
        \{x | g(x) > a\} &= \bigcup_{n=1}^{\infty} \{x | f_n(x) > a\} , \\
        h(x) &= \inf g_m (x),
    \end{align*}
    where $g_m (x) = \sup f_n (x) (n \geq m)$.
\end{proof}

\begin{myCorollary*}
    \begin{asparaenum} [(a)]
        \item If $f$ and $g$ are measurable, then $\max(f, g)$ and $\min(f, g)$ are measurable. 
        If
        \begin{equation}
            \label{eq:11.47}
            f^+ =  \max (f, 0), \quad
            f^- = -\min (f, 0),
        \end{equation}
        it follows, in particular, that $f^+$ and $f^-$ are measurable.
        \item The limit of a convergent sequence of measurable functions is measurable.
    \end{asparaenum}
\end{myCorollary*}

\begin{thm}
    \label{thm:11.18}
    Let $f$ and $g$ be measurable real-valued functions defined on $X$,
    let $F$ be real and continuous on $\R^2$ , and put
    \begin{equation*}
        h(x) = F(f (x), g(x))
        \quad (x \in X)
    \end{equation*}
    Then $h$ is measurable.

    In particular, $f + g$ and $fg$ are measurable.
\end{thm}


chap11sec05

\section{Simple functions}

\begin{mydef}
    \label{mydef:11.19}
    simple function

    characteristic function
\end{mydef}

\begin{thm}
    \label{thm:11.20}
    Let $f$ be a real function on $X$. 
    There exists a sequence $\sequence{s_n}$ of simple functions such that $s_n(x) \rightarrow J(x)$ as $n \rightarrow  \infty$,for every $x \in X$. 
    If $f$ is measurable, $\sequence{s_n}$ may be chosen to be a sequence of measurable functions. 
    If $f \geq 0$, $\sequence{s_n}$ may be chosen to be a monotonically increasing sequence.
\end{thm}
% chap11sec06

\section{Integration}

\begin{mydef}
    \label{mydef:11.21}
    Suppose 
    \begin{equation}
        \label{eq:11.51}
        s(x) = \sum_{i=1}^{n} c_i K_{E_i} (x) 
        \quad (x \in X, x_i > 0)
    \end{equation}
    is measurable, 
    and suppose $E \in \mathfrak{M}$.
    We define
    \begin{equation}
        \label{eq:11.52}
        I_E(s) = 
        \sum_{i=1}^{n} c_i \mu \left( E \cap E_1 \right) .
    \end{equation}

    If $f$ is measurable and nonnegative, 
    we define 
    \begin{equation}
        \label{eq:11.53}
        \int_E f \d \mu = 
        \sup I_E (s).
    \end{equation}
    where the sup is taken over all measurable simple functions $s$ such that $0 \leq s \leq f$
    
    The left member of (\ref{eq:11.53}) is called the Lebesgue integral of $f$, with respect to the measure $\mu$, over the set $E$. 
    It should be noted that the integral may have the value $+ \infty$. 
    
    It is easily verified that
    \begin{equation}
        \label{eq:11.54}
        \int_E s \d \mu =
        I_E (s)
    \end{equation}
    for every nonnegative simple measurable function $s$.
\end{mydef}

\begin{mydef}
    \label{mydef:11.22}
    Let $f$ be measurable, and consider the two integrals
    \begin{equation}
        \label{eq:11.55}
        \int_E f^+ \d \mu , \quad 
        \int_E f^- \d \mu ,
    \end{equation}
    where $f^+$ and $f^-$ are defined as in (\ref{eq:11.47}).

    If at least one of the integrals in (\ref{eq:11.55}) is finite, 
    we define 
    \begin{equation}
        \label{eq:11.56}
        \int_E f \d \mu = 
        \int_E f^+ \d \mu -
        \int_E f^- \d \mu 
    \end{equation}
    
    If both integrals in (\ref{eq:11.55}) are finite, then (\ref{eq:11.56}) is finite,
    and we say that $f$ is \emph{integrable} (or \emph{summable}) on $E$ in the Lebesgue sense, with respect to $\mu$;
    we write $f \in \mathscr{L}(\mu)$ on $E$.
    If $\mu = m$, the usual notation is: 
    $f \in \mathscr{L}$ on $E$.

    This terminology may be a little confusing:
    If (\ref{eq:11.56}) is $+\infty$ or $-\infty$,
    then the integral of $f$ over $E$ is defined, 
    although $f$ is not integrable in the above sense of the word;
    $f$ is integrable on $E$ only if its integral over $E$ is finite.

    We shall be mainly interested in integrable functions,
    although in some cases it is desirable to deal with the more general situation.
\end{mydef}

\begin{myRemark}
    \label{myRemark:11.23}
    The following properties are evident:
    \begin{asparaenum}[(a)]
        \item If $f$ is measurable and bounded on $E$, and if $\mu(E) < + \infty$, then $f \in \mathscr{L}(\mu)$ on $E$.
        \item If $a \leq f(x) \leq b$ for $x \in E$, and $\mu(E) < + \infty$, 
        then 
        \begin{equation*}
            a\mu(E) \leq \int_E f \d \mu \leq b\mu(E) .
        \end{equation*}
        \item If $f$ and $g \in \mathsf{L}(\mu)$ on $E$, and if $f(x) \leq g(x)$ for $x \in E$, then
        \begin{equation*}
            \int_E f \d \mu \leq
            \int_E g \d \mu .
        \end{equation*}
        \item  If $f \in \mathscr{L}(\mu)$ on $E$, then $cf \in \mathscr{L}(\mu)$ on $E$, for every finite constant $c$, and
        \begin{equation*}
            \int_E cf \d \mu \leq
            c \int_E f \d \mu .            
        \end{equation*}
        \item If $\mu(E) = 0$, and $f$ is measurable, then
        \begin{equation*}
            \int_E f \d \mu = 0.            
        \end{equation*}
        \item If $f \in \mathscr{L}(\mu)$ on $E$, $A \in \mathfrak{M}$, and $A \subset E$, then $f \in \mathscr{L}(\mu)$ on $A$.
    \end{asparaenum}
\end{myRemark}

\begin{thm}
    \label{thm:11.24}
    \begin{asparaenum}[(a)]
        \item Suppose $f$ is measurable and nonnegative on $X$. For $A \in \mathfrak{M}$, define 
        \begin{equation}
            \label{eq:11.57}
            \phi(A) = \int_A f \d \mu .
        \end{equation}
        Then $\phi$ is countably additive on $\mathfrak{M}$.
        \item The same conclusion holds if $f \in \mathscr{L}(\mu)$ on $X$.
    \end{asparaenum}
\end{thm}

\begin{myCorollary*}
    If $A \in \mathfrak{M}$, $B \in \mathfrak{M}$, $B \subset A$, and $\mu(A-B)=0$, then
    \begin{equation*}
        \int_A f \d \mu = 
        \int_B f \d \mu .
    \end{equation*}
    Since $A =B\cup (A - B)$, this follows from Remark \ref{myRemark:11.23}(e).
\end{myCorollary*}

\begin{myRemark}
    \label{myRemark:11.25}
    The preceding corollary shows that sets of measure zero are negligible in integration. 

    Let us write $f \sim g$ on $E$ if the set
    \begin{equation*}
        \int_A f \d \mu =
        \int_B f \d \mu .
    \end{equation*}
    has measure zero.

    Then $f \sim f$; $f \sim g$ implies $g \sim f$; 
    and  $f \sim g$, $g \sim h$ implies $f \sim h$.
    That is, the relation $\sim$ is an equivalence relation.

    If $f \sim g$ on $E$, we clearly have 
    \begin{equation*}
        \int_A f \d \mu =
        \int_A g \d \mu ,
    \end{equation*}
    provided the integrals exists, for every measurable subset $A$ of $E$.
    % todo
\end{myRemark}

\begin{thm}
    \label{thm:11.26}
    If $f \in \mathscr{L}(\mu)$ on $E$, then $\left| f \right| \in \mathscr{L}(\mu)$ on $E$, and 
    \begin{equation}
        \label{eq:11.63}
        \left| \int_E f \d \mu \right| \leq
        \int_E \left| f \right| \d \mu .
    \end{equation}
\end{thm}

\begin{thm}
    \label{thm:11.27}
    Suppose $f$ is measurable on $E$, $\left| f \right| \leq g$, and $g \in \mathscr{L}(\mu)$ on $E$. 
    Then $f \in \mathscr{L}(\mu)$ on $E$.
\end{thm}

\begin{proof}
    We have $f^+ \leq g$ and $f^- \leq g$.
\end{proof}

\begin{thm}
    \label{thm:11.28}
    \myKeyword{Lebesgue's monotone convergence theorem} 
    Suppose $E \in \mathfrak{M}$. Let $\sequence{f_n}$ be
    a sequence of measurable functions such that
    \begin{equation}
        \label{eq:11.64}
        0 \leq f_1(x) \leq f_2(x) \leq \cdots 
        \quad (x \in E).
    \end{equation}

    Let $f$ be defined by 
    \begin{equation}
        \label{eq:11.65}
        f_n(x) \rightarrow f(x)
        \quad (x \in E)
    \end{equation}
    as $n \rightarrow \infty$. 
    Then 
    \begin{equation}
        \label{eq:11.66}
        \int_E f_n \d \mu \rightarrow
        \int_E f \d \mu 
        \quad (n \rightarrow \infty).
    \end{equation}
\end{thm}

\begin{thm}
    \label{thm:11.29}
    Suppose $f = f_1 + f_2$, where $f_i \in \mathsf{L}(\mu)$ on $E$ $(i = 1,2)$. 
    Then $f \in \mathsf{L}(\mu)$ on $E$, and 
    \begin{equation}
        \label{eq:11.73}
        \int_E f \d \mu = 
        \int_E f_1 \d \mu +
        \int_E f_2 \d \mu .
    \end{equation}
\end{thm}

We are now in a position to reformulate Theorem \ref{thm:11.28} for series.

\begin{thm}
    \label{thm:11.30}
    Suppose $E \in \mathfrak{M}$. 
    If $\sequence{f_n}$ is a sequence of nonnegative measurable functions and 
    \begin{equation}
        \label{eq:11.76}
        f(x) = \sum_{n=1}^{\infty} f_n (x)
        \quad (x \in E),
    \end{equation}
    then 
    \begin{equation*}
        \int_E f \d \mu = 
        \sum_{n=1}^{\infty} \int_E f_n \d \mu .
    \end{equation*}
\end{thm}

\begin{proof}
    The partial sums of (\ref{eq:11.76}) form a monotonically increasing sequence.
\end{proof}

\begin{thm}
    \label{thm:11.31}
    \myKeyword{Fatou's theorem}
    Suppose $E \in \mathfrak{M}$. 
    If $\sequence{f_n}$ is a sequence of nonnegative measurable functions and 
    \begin{equation*}
        f(x) = \liminf_{n \rightarrow \infty} f_n (x)
        \quad (x \in E),
    \end{equation*}
    then 
    \begin{equation}
        \label{eq:11.77}
        \int_E f \d \mu \leq
        \liminf _{n \rightarrow \infty} f_n \d \mu .
    \end{equation}
\end{thm}

Strict inequality may hold in (\ref{eq:11.77}). 
An example is given in Exercise 5.

\begin{thm}
    \label{thm:11.32}
    \myKeyword{Lebesgue's dominated convergence theorem}
    Suppose $E \in \mathfrak{M}$.
    Let $\sequence{f_n}$ be a sequence of measurable functions such that 
    \begin{equation}
        \label{eq:11.82}
        f_n(x) \rightarrow f(x)
        \quad (x \in E).
    \end{equation}
    as $n \rightarrow \infty$.
    If there exists a functions such that 
    \begin{equation}
        \label{eq:11.83}
        \left| f_n(x) \right| \leq g(x)
        \quad (n = 1,2,3,\dots,x \in E),
    \end{equation}
    then 
    \begin{equation}
        \label{eq:11.84}
        \lim_{n \to \infty} \int_E f_n \d \mu =
        \int_E f \d \mu .
    \end{equation}
\end{thm}

\begin{myCorollary*}
    If $\mu(E) < +\infty$, $\sequence{f_n}$ is uniformly bounded on $E$, and $f_n(x) \rightarrow f(x)$ on $E$, then (\ref{eq:11.84}) holds.
\end{myCorollary*}

A uniformly bounded convergent sequence is often said to be boundedly
convergent.
% chap11sec07

\section{Comparison with the Riemann integral}

Our next theorem will show that every function which is Riemann-integrable
on an interval is also Lebesgue-integrable, 
and that Riemann-integrable functions are subject to rather stringent continuity conditions. 
Quite apart from the fact that the Lebesgue theory therefore enables us to integrate a much larger class of functions, 
its greatest advantage lies perhaps in the ease with which many limit operations can be handled; 
from this point of view, 
Lebesgue's convergence theorems may well be regarded as the core of the Lebesgue theory.


One of the difficulties which is encountered in the Riemann theory is
that limits of Riemann-integrable functions 
(or even continuous functions)
may fail to be Riemann-integrable. 
This difficulty is now almost eliminated,
since limits of measurable functions are always measurable.

Let the measure space $X$ be the interval $[a, b]$ of the real line, with $\mu = m$
(the Lebesgue measure), and $\mathfrak{M}$ the family of Lebesgue-measurable subsets
of $[a, b]$. Instead of
\begin{equation*}
    \int_X f \d m
\end{equation*}
it is customary to use the familiar notation
\begin{equation*}
    \int_{a}^{b} f \d x
\end{equation*}
for the Lebesgue integral of $f$ over $[a, b]$. 
To distinguish Riemann integrals from Lebesgue integrals, 
we shall now denote the former by
\begin{equation*}
    \mathfrak{R} \int_{a}^{b} f \d x .
\end{equation*}

\begin{thm}
    \label{thm:11.33}
    \begin{asparaenum}[(a)]
        \item If $f \in \mathscr{R}$ on $[a,b]$, then $f \in \mathscr{L}$ on $[a,b]$, 
        and 
        \begin{equation}
            \label{eq:11.87}
            \int_{a}^{b} f \d x = 
            \mathscr{R} \int_{a}^{b} f \d x .
        \end{equation}
        \item Suppose $f$ is bounded on $[a,b]$. 
        Then $f \in \mathscr{R}$ on $[a,b]$ 
        if and only if $f$ is continuous almost everywhere on $[a,b]$.
    \end{asparaenum}
\end{thm}
% chap11sec08

\section{Integration of complex functions}

Suppose $f$ is a complex-valued function defined on a measure space $X$,
and $f = u + iv$,
where $u$ and $v$ are real.
We say that $f$ is measurable if and only if
both $u$ and $v$ are measurable.

It is easy to verify that sums and products of complex measurable functions
are again measurable. Since
\begin{equation*}
    \left| f \right| = (u^2 + v^2)^{1/2},
\end{equation*}
Theorem \ref{thm:11.18} shows that $|f|$ is measurable for every complex measurable $f$.

Suppose $\mu$ is a measure on $X$,
$E$ is a measurable subset of $X$,
and $f$ is a complex function on $X$.
We say that $f \in \mathscr{L}(\mu)$ on $E$ provided that $f$ is measurable
and
\begin{equation}
    \label{eq:11.97}
    \int_E \left| f \right| \d \mu < +\infty ,
\end{equation}
and we define
\begin{equation*}
    \int_E f \d \mu =
    \int_E u \d \mu + i
    \int_E v \d \mu
\end{equation*}
if (\ref{eq:11.97}) holds.
Since $|u| \leq |f|$, $|v| \leq |f|$, and $|f | \leq | u | + | v |$,
it is clear that
(\ref{eq:11.97}) holds if and only if $u \in \mathscr{L}(\mu)$ and $v \in \mathscr{L}(\mu)$ on $E$.

Theorems \ref{myremark:11.23}(a), (d), (e), (f), \ref{thm:11.24}(b), \ref{thm:11.26}, \ref{thm:11.27}, \ref{thm:11.29}, and \ref{thm:11.32}
can now be extended to Lebesgue integrals of complex functions.
The proofs are quite straightforward.
That of Theorem \ref{thm:11.26} is the only one that offers
anything of interest:

If $f \in \mathscr{L}(\mu)$ on $E$, there is a complex number $c$, $|c| = 1$, such that
\begin{equation*}
    c \int_E f \d \mu \geq 0 .
\end{equation*}
Put $g = cf = u + iv$, $u$ and $v$ real.
Then
\begin{equation*}
    \left| \int_E f \d \mu \right| =
    c \int_E f \d \mu =
    \int_E g \d \mu =
    \int_E u \d \mu \leq
    \int_E | f | \d \mu .
\end{equation*}
The third of the above equalities holds since the preceding ones show that
$\int f \d \mu$ is real.

% chap11sec09

\section{Functions of Class $\mathscr{L}^2$}

As an application of the Lebesgue theory, 
we shall now extend the Parseval theorem 
(which we proved only for Riemann-integrable functions in Chap. 8)
and prove the Riesz-Fischer theorem for orthonormal sets of functions.

\begin{mydef}
    \label{mydef:11.34}
    Let $X$ be a measurable space. 
    We say that a complex
    function $f \in \mathscr{L}^2(\mu)$ on $X$ if $f$ is measurable and if
    \begin{equation*}
        \int_X |f|^2 \d \mu < +\infty .
    \end{equation*}
    If $\mu$ Lebesgue measure, 
    we say $f \in \mathscr{L}^2$. 
    For $f \in \mathscr{L}^2(\mu)$
    (we shall omit the phrase ``on $X$'' from now on) we define
    \begin{equation*}
        \left\| f \right\| =
        \left\{ \int_X \left| f \right|^2 \d \mu \right\}^{1/2}
    \end{equation*}
    and call $\|f\|$ the $\mathscr{L}^2(\mu)$ norm of $f$.
\end{mydef}
\mybox{omit 省略}

\begin{thm}
    \label{thm:11.35}
    Suppose $ f \in \mathscr{L}^2(\mu)$ 
    and     $ g \in \mathscr{L}^2(\mu)$. 
    Then    $fg \in \mathscr{L}  (\mu)$, 
    and
    \begin{equation}
        \label{eq:11.98}
        \int_X \left| fg \right| \d \mu \leq
        \left\| f \right\| \left\| g \right\| .
    \end{equation}
\end{thm}

This is the Schwarz inequality, 
which we have already encountered for series and for Riemann integrals. 
It follows from the inequality 
\begin{equation*}
    0 \leq 
    \int_X \left( | f | + \lambda | g |  \right)^2 \d \mu =
    \left\| f \right\|^2 + 
    2 \lambda \int_X | fg | \d \mu + \lambda^2 \left\| g \right\|^2 ,
\end{equation*}
which holds for every real $\lambda$.

\begin{thm}
    \label{thm:11.36}
    If   $f \in \mathscr{L}^2(\mu)$
    and  $f \in \mathscr{L}^2(\mu)$,
    then $f + g \in \mathscr{L}^2(\mu)$,
    and 
    \begin{equation*}
        \left\| f + g \right\| \leq
        \left\| f \right\| + \left\| g \right\| .
    \end{equation*}
\end{thm}

\begin{proof}
    The Schwarz inequality shows that 
    \begin{align*}
        \left\| f + g \right\|^2
        &= \int |f|^2 + \int f\bar{g} + \int \bar{f}g + \int |g|^2 \\
        &\leq \|f\|^2 + 2\|f\| \|g\| + \|g\|^2 \\
        &= \left( \left\| f \right\| + \left\| g \right\| \right)^2 .
    \end{align*}
\end{proof}

\begin{myRemark}
    If we define  the distance between two functions $f$ and $g$ in
    $\mathscr{L}^2(\mu)$ to be $\left\| f-g \right\|$, 
    we see that the conditions of Definition \ref{mydef:2.15} are satisfied,
    except for the fact that $\left\| f-g \right\| = 0$ does not imply that $f(x) = g(x)$ for all $x$,
    but only for almost all $x$. 
    Thus, if we identify functions which differ only on a
    set of measure zero, $\mathscr{L}^2(\mu)$ is a metric space.
    
    We now consider $\mathscr{L}^2$ on an interval of the real line, with respect to
    Lebesgue measure.
\end{myRemark}

\begin{thm}
    \label{thm:11.38}
    The continuous functions form a dense subset of $\mathscr{L}^2$ on $[a, b]$.
\end{thm}

More explicitly, this means that for any $f \in \mathscr{L}^2$ on $[a, b]$, and any $\varepsilon > 0$,
there is a function $g$, continuous on $[a, b]$, such that
\begin{equation*}
    \left\| f - g \right\| =
    \left\{ \int_{a}^{b} \left| f - g \right|^2 \d x \right\}^{1/2} <
    \varepsilon.
\end{equation*}


\begin{mydef}
    \label{mydef:11.39}
    We say that a sequence of complex functions $\sequence{\phi_n}$ is an
    orthonormal set of functions on a measurable space $X$ if 
    \begin{equation*}
        \int_X \phi_n \bar{\phi}_m \d \mu = 
        \left\{
            \begin{array}{ll}
                0 & (n \neq m), \\
                1 & (n =    m). \\
            \end{array}
        \right.
    \end{equation*}
    In particular, we must have $\phi_n \in \mathscr{L}^2(\mu)$. 
    If $f \in \mathscr{L}^2(\mu)$ and if
    \begin{equation*}
        c_n = \int_X f \bar{\phi}_n \d \mu
        \quad (n = 1,2,3,\dots),
    \end{equation*}
    we write 
    \begin{equation*}
        f \sim \sum_{n=1}^{\infty} c_n \phi_n ,
    \end{equation*}
    as in Definition \ref{mydef:8.10}.
\end{mydef}


Parseval theorem 
\begin{thm}
    \label{thm:11.40}
    Suppose 
    \begin{equation}
        \label{eq:11.99}
        f(x) \sim \sum_{-\infty}^{\infty} c_n e^{inx} ,
    \end{equation}
    where $f in \mathscr{L}^2$ on $[-\pi, \pi]$.
    Let $s_n$ be the $n$th partial sum of (\ref{eq:11.99}).
    Then 
    \begin{align}
        \label{eq:11.100}
        \lim_{n \to \infty} \left\| f - s_n \right\| &= 0, \\
        \sum_{-\infty}^{\infty} \left| c_n \right|^2 &= 
        \frac{1}{2\pi} \int_{-\pi}^{\pi} \left| f \right|^2 \d x.
    \end{align}
\end{thm}

\begin{myCorollary*}
    If $f \in \mathscr{L}^2$ on $[-\pi, \pi]$, and if 
    \begin{equation*}
        \int_{-\pi}^{\pi} f(x) e^{-inx} \d x = 0
        \quad (n = 0, \pm 1, \pm 2, \dots),
    \end{equation*}
    then $\left\| f \right\| = 0$.
\end{myCorollary*}