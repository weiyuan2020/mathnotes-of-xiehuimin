
\section{Ordered sets}
有序集

\begin{myDefinition}\label{myDefinition:1.5}
    % 1.5 myDefinition
Let $S$ be a set. An \emph{order} on $S$ is a relation, denoted by $<$, with the following two properties:

(i) If $x\in S$ and $y\in S$ then one and only one of the statements
\begin{equation*}
    x<y, \quad
    x=y, \quad
    y<x
\end{equation*}
The statement $x<y$ may be read as 
$x$ is less than $y$, or 
$x$ is smaller than $y$, or
$x$ precedes $y$.
(It's often convenient to write $y>x$ in place of $x<y$)
(less-great, smaller-bigger, precedes-succeeds)
% form wiki,
% The relationship x precedes y is written $x ≺ y$. The relation x precedes or is equal to y is written x ≼ y.
% The relationship x succeeds (or follows) y is written x ≻ y. The relation x succeeds or is equal to y is written x ≽ y.
% ≺ \prec 
% ≼ \preccurlyeq 
% ≻ \succ 
% ≽ \succcurlyeq  

(ii) If $x,y,z\in S$, if $x<y$ and $y<z$, then $x<z$.

$x\leq y$ indicates taht $x<y$ or $x=y$, without specofying which of these two is to hold.
In other words, $x\leq y$ is the negation of $x>y$.
\end{myDefinition}


\mybox{
  偏序关系:
1. 三歧性,
2. 传递性.

建立偏序关系后, 可以使用不等式进行分析. 在后续根据极限定义计算时, 需要大量使用不等式分析数列和函数的极限计算结果. }

\begin{myDefinition}\label{myDefinition:1.6}
% 1.6 myDefinition
An \emph{ordered set} is a set $S$ in which an order is defined.
\end{myDefinition}

For Example, $\mathbb{Q}$ is an ordered set if $r<s$ is defined to mean that $s-r$ is a positive rational number.

\mybox{
    存在偏序关系的集合称为有序集
$\mathbb{Q}, \mathbb{R}$ 均是有序集, 但$\mathbb{C}$ 不是有序集. }

\begin{myDefinition}\label{myDefinition:1.7}
% 1.7 myDefinition (bounded above)
Suppose $S$ is an ordered set, and $E \subset S$. If there exists a
$\beta \in S$ such that $x \leq \beta$ for every $x \in E$, we say that $E$ is \emph{bounded above}, and call
$\beta$ an \emph{upper bound} of $E$.

Lower bounds are defined in the same way (with $\geq$ in place of $\leq$).
\end{myDefinition}

\begin{myDefinition}\label{myDefinition:1.8}
% 1.8 myDefinition (least upper bound)
Suppose $S$ is an ordered set, $E \subset S$, and $E$ is bounded above.
Suppose there exists an $a\alpha \in S$ with the following properties:

(i) $\alpha$ is an upper bound of $E$.
(ii) If $\gamma <\alpha$ then $\gamma$ is not an upper bound of $E$.

Then $\alpha$ is called the \emph{least upper bound} of $E$ [that there is at most one such
$\alpha$ is clear from (ii)] or the \emph{supremum} of $E$, and we write
\begin{equation*}
    \alpha = \sup E.
\end{equation*}

The \emph{greatest lower bound}, or \emph{infimum}, of a set $E$ which is bounded below
is defined in the same manner: The statement
\begin{equation*}
    \alpha = \inf E
\end{equation*}

means that $\alpha$ is a lower bound of $E$ and that no $\beta$ with $\beta > \alpha$ is a lower bound
of $E$.
\end{myDefinition}
\mybox{从上界引出最小上界, 没有直接定义最大下界, 而是使用对称定义引出. 
从最小上界引出的最小上界性质更为常用. Dedekind分划}

\begin{myExample}\label{Example:1.9}
    % 1.9 Example
(a) Consider the set $A, B$
    \begin{equation*}
        A = \{p|p^2 < 2\},\quad
        B = \{p|p^2 > 2\}.
    \end{equation*}
    $A$ has no least upper bound in $\mathbb{Q}$.
    $B$ has no great lower bound in $\mathbb{Q}$.
    
    (b) If $\alpha = \sup E$ exists, $\alpha$ may be or may not be a member of $E$.
    \begin{align*}
        E_1 = \{r |r\in Q, r < 0\}\\
        E_2 = \{r |r\in Q, r \leq 0\}
    \end{align*}
    \begin{equation*}
        \sup E_1 = \sup E_2 = 0,
    \end{equation*}
    and $0\not\in E_1$, $0\in E_2$.
    
    (c) $E = \{1/n | n = 1,2,3,...\}$. Then $\sup E = 1$, which is in $E$, and $\inf E = 0$, which is not in $E$.
\end{myExample}

\begin{myDefinition}\label{myDefinition:1.10}
% 1.10 myDefinition 
{\color{red}{least-upper-bound property}}

An ordered set $S$ is said to have the \emph{least-upper-bound property}
if the following is true:

If $E \subset S$, $E$ is not empty, and $E$ is bounded above, then $\sup E$ exists in $S$.
\end{myDefinition}

Example \ref{Example:1.9}(a) shows that $\mathbb{Q}$ does not have the least-upper-bound property.

We shall now show that there is a close relation between greatest lower
bounds and least upper bounds, and that every ordered set with the least-upper-bound property also has the greatest-lower-bound property.

\begin{thm}\label{thm:1.11}
    % 1.11 Theorem 
Suppose $S$ is an ordered set with the least-upper-bound property,
$B \subset S$, $B$ is not empty, and $B$ is bounded below. Let $L$ be the set of all lower
bounds of $B$. Then
\begin{equation*}
    \alpha = \sup L
\end{equation*}
exists in $S$, and $\alpha = \inf B$.

In particular, $\inf B$ exists in $S$.
\end{thm}

\begin{proof}
    % Proof 
Since $B$ is bounded below, $L$ is not empty. Since $L$ consists of
exactly those $y \in S$ which satisfy the inequality $y \leq x$ for every $x \in B$, we
see that \emph{every} $x \in B$ \emph{is an upper bound of} $L$. Thus $L$ is bounded above.
Our hypothesis about $S$ implies therefore that $L$ has a supremum in $S$;
call it $\alpha$.

If $\gamma <\alpha$ then (see Definition \ref{myDefinition:1.8}) $\gamma$ is not an upper bound of $L$,
hence $\gamma \not\in B$. It follows that $\alpha \leq x$ for every $x \in B$. Thus $\alpha \in L$.

If $\alpha < \beta$ then $\beta \not\in L$, since $\alpha$ is an upper bound of $L$.

We have shown that $\alpha \in L$ but $\beta \not\in  L$ if $\beta>\alpha$. In other words, $\alpha$
is a lower bound of $B$, but $\alpha$ is not if $\beta > \alpha$. This means that $\alpha = \inf B$.
\end{proof}

\mybox{% mynotes
这个证明第一次看比较难理清
我试着用自己的话重写梳理一下:
已知条件
$S$, ordered set + least-upper-bound property.
$B\in S$, $B\neq \varnothing $, $B$ is bounded below.
$L$ is the set of all lower bounds of $B$.
$\exists \alpha\in S$, $\alpha = \sup L$, and $\alpha = \inf B$.}

    \begin{proof}
% proof:
思路 由最小上界 $\rightarrow $ 最大下界
% \begin{align*}
%     \text{最小上界}  & \rightarrow  &\text{最大下界} \\
%     \downarrow      &               &\uparrow \\
%     L\text{最小上界}  & \rightarrow  &B\text{最大下界} \\
% \end{align*}
$L = \{y| y\in S; \forall x\in B, y\leq x\}$
    关于 $L$ 中有没有不在 $S$ 中的元素这一点我还没想明白. 定理中只是说 $L$ 是 $B$ 的下界组成的. $B$ 是 $S$ 的子集, 但 $B$ 的下界不一定全在 $S$ 中. 

$L$ 由 $B$ 在 $S$ 中的全部下界组成

$\forall x\in B$, $x$ 为 $L$ 的上界. $L\subset S$.
$S$ 有最小上界性质,
$\therefore \exists \alpha\in S$, $\alpha = \sup L$.

$\forall \gamma <\alpha$ 由 $\alpha = \sup L$ 的定义 (\ref{myDefinition:1.8})
$\gamma$ 不是 $L$ 的上界.

$\forall x \in B$, $x$ 为 $L$ 的上界, $x \geq \alpha$. $\therefore \alpha \in L$.

$\alpha < \beta$, $\alpha = \sup L$. $\therefore \beta \not\in L$.
$L$ 由 $B$ 在 $S$ 中的全部下界组成, $\beta \not\in L$.
$\beta$ 不是 $B$ 的下界.

$\therefore \alpha = \inf B$, $\inf B\in S$.
    \end{proof}

