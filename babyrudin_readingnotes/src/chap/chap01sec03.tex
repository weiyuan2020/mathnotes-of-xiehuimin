
\section{fields}
\mybox{域, 交换除环 <$\mathbb{R},+,\times$> 
<$\mathbb{R},+$>, <$\mathbb{R}\backslash\{0\},\times$>
都是交换群, 且满足分配律. 
则 <$\mathbb{R},+,\times$> 是域. }

\begin{myAxiom}\label{myAxiom:1.12}
% Axiom % 公理

(A) Axioms for addition

(Al) If $x\in F$  and $y \in F$, then their sum \(x + y\) is in F.

(A2) Addition is commutative: \(x + y=y+ x\) for all \(x, y \in F\).

(A3) Addition is associative: \((x+ y)+z = x + (y+ z)\) for all \(x, y, z \in F\).

(A4) $F$ contains an element $0$ such that $0 + x = x$ for every $x \in F$.

(A5) To every $x\in F$ corresponds an element $-x\in F$ such that
\begin{equation*}
    x+(-x)=0.
\end{equation*}

(M) Axioms for multiplication

(M1) If $x\in F$ and $x\in F$, then their product $xy$ is in $F$.

(M2) Multiplication is commutative: $xy = yx$ for all $x, y \in  F$.

(M3) Multiplication is associative: $(xy)z = x(yz)$ for all $x, y, z \in  F$.

(M4) $F$ contains an element $1 \neq 0$ such that $1x = x$ for every $x \in F$.

(M5) If $x \in F$ and $x \neq 0$ then there exists an element $1/x \in F$ such that
\begin{equation*}
    x\cdot(1/x)=1.
\end{equation*}
% 6 PRINCIPLES OF MATHEMATICAL ANALYSIS

(D) The distributive law
\begin{equation*}
    x(y+z)=xy+ xz
\end{equation*}

holds for all $x, y, z \in F$.
\end{myAxiom}

\begin{myRemark}\label{myRemark:1.13}
    % 1.13 Remark
(a) Our usual writes (in any filed)

只定义了加法和乘法, 使用逆元分别表示减法和除法.
$x-y = x+(-y)$, $x/y=x\cdot (1/y)$.

(b) The field axioms clearly hold in $\mathbb{Q}$, the set of all rational numbers, if
addition and multiplication have their customary meaning. Thus $\mathbb{Q}$ is a
field.

全体有理数的集合是一个域.

(c) Although it is not our purpose to study fields (or any other algebraic
structures) in detail, it is worthwhile to prove that some familiar properties
of $\mathbb{Q}$ are consequences of the field axioms; once we do this, we will \underline{not
need to do it} again for the real numbers and for the complex numbers.
\end{myRemark}

\begin{myProposition}\label{myProposition:1.14}
% 1.14 Proposition
The axioms for addition imply the following statements.

(a) If $x+y=x+z$ then $y=z$.

(b) If $x+y=x$ then $y=0$.

(c) If $x+y=0$ then $y= -x$.

(d) $-(-x)=x$.
\end{myProposition}

Statement (a) is a cancellation law. Note that (b) asserts the uniqueness
of the element whose existence is assumed in (A4), and that (c) does the same
for (A5).

\mybox{
    % mynotes
what is the difference between axiom and proposition?

An axiom is a proposition regarded as self-evidently true without proof. The word "axiom" is a slightly archaic synonym for postulate. Compare conjecture or hypothesis, both of which connote apparently true but not self-evident statements.
A proposition is a mathematical statement such as "3 is greater than 4," "an infinite set exists," or "7 is prime." An axiom is a proposition that is assumed to be true. With sufficient information, mathematical logic can often categorize a proposition as true or false, although there are various exceptions (e.g., "This statement is false").
\url{https://www.nutritionmodels.com/terminology.html}}


\begin{proof}
    Proof(rudin)

If $x + y =x + z$, the axioms (A) give
\begin{align*}
    y =0+y&=(-x+x)+y=-x+(x+y)\\
    &=-x+(x+z)=(-x+x)+z=0+z=z
\end{align*}

This proves (a). Take $z = 0$ in (a) to obtain (b). Take $z= -x$ in (a) to
obtain (c).
Since $-x + x = 0$, (c) (with $-x$ in place of $x$) gives (d).
\end{proof}

\mybox{mynotes 我自己证明上述四条性质时都是从定义开始的, 而 rudin 这里在后一步的证明中都利用了刚推导出的结论, 这一点需要借鉴.}
% THE REAL AND COMPLEX NUMBER SYSTEMS 7

\begin{myProposition}\label{Proposition:1.15}
% 1.15 Proposition 
The axioms for multiplication imply the following statements.

(a) If $x\neq0$ and $xy=xz$ then $y=z$.

(b) If $x\neq0$ and $xy=x$ then $y=1$.

(c) If $x\neq0$ and $xy=1$ then $y=1/x$.

(d) If $x\neq0$ then $1/(1/x) = x$.
\end{myProposition}

The proof is so similar to that of Proposition 1.14 that we omit it.


\begin{proof}
mynotes
% Proof
(a),  
\begin{align*}
    y&=1\cdot y=\left(\frac{1}{x}\cdot x\right)y =\frac{1}{x}\left( xy \right)\\
    &=\frac{1}{x}(xz) =\left(\frac{1}{x}x\right)z = z
\end{align*}

(b), (a)取  $z=1$. $y=z=1$.

(c), (a)取  $z=\frac{1}{x}$. $y=z=\frac{1}{x}$.

(d), (c)取  $x=\frac{1}{x'}$. $y=1/(1/x')$.
\end{proof}

\begin{myProposition}\label{Proposition:1.16}
    The field axioms imply the following statements, for any $x, y, z \in F$.

    (a) $0x=0$.

    (b) If $x\neq 0$ and $y\neq 0$ then $xy\neq 0$.

    (c) $(-x)y=-(xy)=x(-y)$.

    (d) $(-x)(-y)=xy$.
\end{myProposition}

\begin{proof}
    $0x+0x=(0+0)x=0x$. Hence \ref{myProposition:1.14}(b) implies that $0x=0$, and (a) holds.

    Next, assume $x \neq 0$, $y \neq 0$, but $xy =0$. Then (a) gives
    \begin{equation*}
        1=
        \left(\frac{1}{y}\right)\left(\frac{1}{x}\right)xy=
        \left(\frac{1}{y}\right)\left(\frac{1}{x}\right)0=0.
    \end{equation*}

a contradiction. Thus (b) holds.

The first equality in (c) comes from
\begin{equation*}
    (-x)y +xy=(-x+x)y=0y=0,
\end{equation*}

combined with \ref{myProposition:1.14}(c); the other half of (c) is proved in the same way.\\
Finally,
\begin{equation*}
    (-x)(-y)=-[x(-y)]=-[-(xy)]=xy
\end{equation*}
by (c) and \ref{myProposition:1.14}(d).
\end{proof}

\begin{myDefinition}\label{myDefinition:1.17}
    An ordered field is a field $F$ which is also an ordered set, such
    that
    
    (i) $x+y<x+z$ if $x,y,z\in F$ and $y<z$,
    
    (ii) $xy>0$ if $x\in F$, $y\in F$, $x>0$, and $y>0$.
\end{myDefinition}
If $x > 0$, we call $x$ positive; 
if $x < 0$, $x$ is negative.

For example, $\mathbb{Q}$ is an ordered field.

All the familiar rules for working with inequalities apply in every ordered
field: Multiplication by positive [negative] quantities preserves [reverses] inequalities, no square is negative, etc. The following proposition lists some of
these.
% 8 PRINCIPLES OF MATHEMATICAL ANALYSIS

\mybox{有序域$F$也是有序集, 由于有理数域 $\mathbb{Q}$, 实数域 $\mathbb{R}$ 都是有序域, 这里使用有理数域 $\mathbb{Q}$ 证明的有序集的性质也可以直接用于实数域. $\mathbb{R}$}

\begin{myProposition}\label{myProposition:1.18}
    The following statements are true in every ordered field.

(a) If $x>0$ then $-x <0$, and vice versa.

(b) If $x>0$ and $y<z$ then $xy <xz$.

(c) If $x<0$ and $y<z$ then $xy> xz$.

(d) If $x \neq 0$ then $x^2 > 0$. In particular, $1 > 0$.

(e) If $0<x<y$ then $0<l/y<l/x$.
\end{myProposition}

\begin{proof}
    (a) $x>0$, $-x<0$. 
    \begin{align*}
        x   &> 0=(x+-x)\\
        x+0 &> x+(-x)\\
        (-x)&<0
    \end{align*}

    (b) $x>0$, $y<z$, $xy<xz$.
    \begin{align*}
        y<z, z-y&>y-y=0\\
        x(z-y)&>0\\
        x(z-y)+xy&>0+xy\\
        xz&>xy
    \end{align*}

    (c)
    \begin{align*}
        (z-y) &>y-y=0\\
        x<0,(-x)>0.\quad (-x)(z-y)&>0 \\
        x(z-y) &<0\\
        xz<xy    
    \end{align*}

    (d)
    \begin{align*}
        x>0  && x^2    >0  \\
        x<0  &&(-x)^2 >0, (-x)^2 = -[x(-x)] = -(-(x\cdot x)) =x^2, x^2>0
    \end{align*}
    $\because 1^2=1$, $1>0$.

    (e)
    If $y>0$ and $v \leq 0$, then $yv \leq 0$. But $y \cdot (1/y)=1>0$. Hence $1/y > 0$.
    Likewise, $1/x > 0$. If we multiply both sides of the inequality $x <y$ by
    the positive quantity $(1/x)(1/y)$, we obtain $1/y <1/x$.
\end{proof}

