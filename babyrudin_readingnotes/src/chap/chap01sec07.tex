\section{Euclidean space}
\mybox{欧式空间}

\begin{myDef}
    \label{myDef:1.36}
    For each positive integer $k$, 
    let $\R^k$ be the set of all ordered $k$-tuples
    \begin{equation*}
        \mathbf{x} = \left(x_1,x_2,\dots,x_k\right),
    \end{equation*}
    where $x_1,x_2,\dots,x_k$ are real numbers, called the \emph{coordinates} of $\mathbf{x}$. 
    The elements of $\R^k$ are called points, or vectors, 
    especially when $k > 1$. We shall denote vectors
    by boldfaced letters. 
    If $\mathbf{y} = \left(y_1,y_2,\dots,y_k\right)$ 
    and if $\alpha$ is a real number, put
    \begin{align*}
        \mathbf{x} + \mathbf{y} &= \left(x_1+y_1,x_2+y_2,\dots,x_k+y_k\right),\\
        \alpha\mathbf{x}  &= \left(\alpha x_1,\alpha x_2,\dots,\alpha x_k\right)
    \end{align*}
    so that $\mathbf{x} +\mathbf{y} \in \R^k$ and $\alpha\mathbf{x} \in \R^k$. 
    This defines addition of vectors, 
    as well as multiplication of a vector by a real number (a scalar). 
    These two operations satisfy the commutative, associative, 
    and distributive laws 
    (the proof is trivial, in view of the analogous laws for the real numbers) 
    and make $\R^k$ into a vector space over the \emph{real field}. 
    The zero element of $\R^k$ (sometimes called the origin or the null vector) is the point $\mathbf{0}$, 
    all of whose coordinates are $0$.

    We also define the so-called ``inner product'' (or scalar product) of $\mathbf{x}$ and $\mathbf{y}$ by
    \begin{equation*}
        \mathbf{x}\cdot\mathbf{y} = \sum_{j=1}^{k}x_j y_j
    \end{equation*}
    and the norm of $\mathbf{x}$ by
    \begin{equation*}
        |x| = (x\cdot x)^{1/2} = \left( \sum_{1}^{k} x_j^2 \right)^{1/2}.
    \end{equation*}

    The structure now defined (the vector space $\R^k$ with the above inner product and norm) is called euclidean $k$-space.
\end{myDef}

\begin{thm}\label{thm:1.37}
    Suppose $\mathbf{x}, \mathbf{y}, \mathbf{z}\in\R^k$, and $\alpha$ is real. Then
    \begin{enumerate}[(a)]
        \item $| \mathbf{x}| \geq 0$;
        \item $| \mathbf{x}| = 0$ if and only if $\mathbf{x} =0$;
        \item $| \alpha \mathbf{x}| = | \alpha||x|$
        \item $|\mathbf{x}\cdot\mathbf{y}| \leq  |\mathbf{x}| | \mathbf{y}|$;
        \item $|\mathbf{x}+\mathbf{y}| \leq | \mathbf{x} | + | \mathbf{y}|$;
        \item $|\mathbf{x}-\mathbf{z}| \leq |\mathbf{x}-\mathbf{y}| + |\mathbf{y}-\mathbf{z}|$.
    \end{enumerate}
\end{thm}

\mybox{(f) 为欧式空间中的三角不等式. }

\begin{proof}
    Proof (a), (b), and (c) are obvious, and (d) is an immediate consequence of the Schwarz inequality\ref{thm:1.35}. 
    By (d) we have 
    \begin{align*}
        |\mathbf{x} + \mathbf{y}|^2
        &= (\mathbf{x} + \mathbf{y}) \cdot (\mathbf{x} + \mathbf{y})\\
        &= \mathbf{x} \cdot \mathbf{x} + 2\mathbf{x} \cdot \mathbf{y} + \mathbf{y} \cdot \mathbf{y}\\
        &\leq |\mathbf{x}|^2 + 2|\mathbf{x}||\mathbf{y}| + |\mathbf{y}|^2\\
        &= \left(|\mathbf{x}| + |\mathbf{y}|\right)^2.
    \end{align*}
    so that (e) is proved. Finally, 
    (f) follows from (e) if we 
    replace $\mathbf{x}$ by $\mathbf{x}-\mathbf{y}$ 
    and $\mathbf{y}$ by $\mathbf{y}-\mathbf{z}$.
\end{proof}

\begin{myRemark}
    \label{myRemark:1.38}
    Theorem \ref{thm:1.37} (a), (b), and (f) will allow us (see Chap. 2) to
    regard $\R^k$ as a metric space.
    
    $\R^1$ (the set of all real numbers) is usually called the line, 
    or the real line. 
    Likewise, $\R^2$ is called the plane, or the complex plane (compare Definitions \ref{myDef:1.24} and \ref{myDef:1.36}). 
    In these two cases the norm is just the absolute value of the corresponding real or complex number. 
\end{myRemark}
