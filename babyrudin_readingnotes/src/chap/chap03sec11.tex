% chap03sec11
\section{Summation by parts}
\mybox{部分和}
\begin{thm}
    \label{thm:3.41}
    Given two sequences $\sequence{a_n}, \sequence{b_n}$, put
    \begin{equation*}
        A_n = \sum_{k=0}^{n} a_k
    \end{equation*}
    if $n \geq 0$; put $A_{-1} = 0$. Then if $0\leq p\leq q$ , we have
    \begin{equation}
        \label{eq:3.20}
        \sum_{n=p}^{q} a_n b_n
        = \sum_{n=p}^{q-1} A_n (b_n - b_{n+1}) + A_q b_q - A_{p-1} b_p.
    \end{equation}
\end{thm}

\begin{proof}
    \begin{equation*}
        \sum_{n=p}^{q} a_n b_n
        = \sum_{n=p}^{q} (A_n - A_{n-1}) b_n
        = \sum_{n=p}^{q} A_n b_n - \sum_{n=p-1}^{q-1} A_n b_{n+1}        
    \end{equation*}
    and the last expression on the right is clearly equal to the right side of (\ref{eq:3.20}).
\end{proof}

Formula (20), the so-called ``partial summation formula,''
is useful in the investigation of series of the form $\sum a_n b_n$,
particularly when $\sequence{b_n}$ is monotonic.
We shall now give applications.

\begin{thm} 
    \label{thm:3.42}
    Suppose\\
    (a) the partial sums $A_n$ of $\sum a_n$ form a bounded sequence;\\
    (b) $b_0 \geq b_1 \geq b_2 \geq \cdots$;\\
    (c) $\lim_{n \to \infty} b_n = 0$.

Then $\sum a_n b_n$ converges.
\end{thm}

\begin{proof}
    Choose $M$ such that $|A_n| \leq M$ for all $n$.
    Given $\varepsilon > 0$, there is an integer $N$ such that $b_N \leq (\varepsilon/2M)$. For $N \leq p \leq q$, we have
    \begin{align*}
        \left|\sum_{n=p}^{q} a_n b_n\right|
        &= \left| \sum_{n=p}^{q-1} A_n (b_n - b_{n+1}) + A_q b_q - A_{p-1} b_p \right|\\
        &\leq M \left| \sum_{n=p}^{q-1} (b_n - b_{n+1}) + b_q + b_p \right|\\
        &= 2 M b_p \leq 2 M b_N \leq \varepsilon.
    \end{align*}
    Convergence now follows from the Cauchy criterion.
    We note that the first inequality in the above chain depends of course on the fact that $b_n - b_{n+1} \geq 0$.
\end{proof}

\begin{thm}
    \label{thm:3.43}
    Suppose\\
    (a) $c_1 \geq c_2 \geq c_3 \geq \cdots$;\\
    (b) $c_{2m-1} \geq 0, c_{2m} \leq 0$ $\quad$ $(m = 1,2,3,\dots);$\\
    (c) $\lim_{n \to \infty} c_n = 0$.

Then $\sum c_n$ converges.
\end{thm}

Series for which (b) holds are called ``alternating series'';
the theorem was known to Leibnitz.

\begin{proof}
    Apply Theorem \ref{thm:3.43},
    with $a_n = (-1)^{n+1}$, $b_n = |c_n|$.  
\end{proof}

\begin{thm}
    \label{thm:3.44}
    Suppose the radius of convergence of $\sum c_n z^n$ is $1$,
    and suppose $c_0 \geq c_1 \geq c_2 \geq \cdots$, $\lim_{n \to \infty} c_n = 0$.
    Then $\sum c_n z^n$ converges at every point on the circle $|z| = 1$, except possibly at $z = 1$. 
\end{thm}

\begin{proof}
    Put $a_n = z^n$, $b_n = c_n$.
    The hypotheses of Theorem \ref{thm:3.42} are then satisfied, since
    \begin{equation*}
        |A_n| 
        = \left| \sum_{m=0}^{n} z^m \right|
        = \left| \frac{1 - z^{n+1}}{1 - z} \right|
        \leq \frac{2}{|1 - z|},
    \end{equation*}  
    if $|z|=1, z \neq 1$.
\end{proof}