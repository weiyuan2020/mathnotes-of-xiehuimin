% chap06sec04
\section{Integration of vector-valued functions}
\mybox{向量函数积分}
\begin{myDef}
    \label{myDef:6.23}
    Let $f_1,\dots,f_k$ be real functions on $[a, b]$, 
    and let $\mathbf{f} = (f_1,\dots,f_k)$ be the corresponding mapping of $[a, b]$ into $\R^k$. 
    If $\alpha$ increases monotonically on $[a, b]$, 
    to say that $\mathbf{f} \in \mathscr{R}(\alpha)$ means that
    $f_j \in \mathscr{R}(\alpha)$ for $j = 1, ... , k$. 
    If this is the case, we define
    \begin{equation*}
        \int_{a}^{b} \mathbf{f} \d \alpha = 
        \left( 
            \int_{a}^{b} f_1 \d \alpha
            , \dots,
            \int_{a}^{b} f_k \d \alpha
         \right).
    \end{equation*}
    In other words, $\int \mathbf{f} \d \alpha$ is the point in $\R^k$ whose $j$th coordinate is $\int f \d \alpha$.
    
    It is clear that parts (a), (c), and (e) of Theorem \ref{thm:6.12} are valid for these vector-valued integrals; 
    we simply apply the earlier results to each coordinate. 
    The same is true of Theorems \ref{thm:6.17}, \ref{thm:6.20}, and \ref{thm:6.21}. 
    To illustrate, we state the analogue of Theorem \ref{thm:6.21}.
\end{myDef}

\begin{thm}
    \label{thm:6.24}
    If $\mathbf{f}$ and $\mathbf{F}$ map $[a, b]$ into $\R^k$ ,
    if $\mathbf{f} \in \mathscr{R}$  on $[a, b]$, 
    and if $\mathbf{F}' = \mathbf{f}$, then 
    \begin{equation*}
        \int_{a}^{b} \mathbf{f}(t) \d t = \mathbf{F}(b) - \mathbf{F}(a).
    \end{equation*}
\end{thm}
    
The analogue of Theorem \ref{thm:6.13}(b) offers some new features,
however, at least in its proof.

\begin{thm}
    \label{thm:6.25}
    If $\mathbf{f}$ maps $[a, b]$ into $\R^k$ and 
    if $\mathbf{f} \in \mathscr{R}(\alpha)$ for some monotonically
    increasing function $\alpha$ on $[a, b]$, 
    then $\left| \mathbf{f} \right| \in \mathscr{R}(\alpha)$, and
    \begin{equation}
        \label{eq:6.40}
        \left| \int_{a}^{b} \mathbf{f} \d \alpha \right| \leq
        \int_{a}^{b} \left| \mathbf{f} \right| \d \alpha .
    \end{equation}
\end{thm}

\begin{proof}
    If $f_1,\dots,f_k$ are the components of $\mathbf{f}$, then
    \begin{equation}
        \label{eq:6.41}
        \left| \mathbf{f} \right| = \left( f_1^2 + \cdots + f_k^2 \right)^{1/2}.
    \end{equation}
    By Theorem \ref{thm:6.11}, each of the functions $f_i^2$ belongs to $\mathscr{R}(\alpha)$; 
    hence so does their sum. 
    Since $x^2$ is a continuous function of $x$, 
    Theorem \ref{thm:4.17} shows that the square-root function is continuous on $[0, M]$, for every real $M$. 
    If we apply Theorem \ref{thm:6.11} once more, (\ref{eq:6.41}) shows that $\left| \mathbf{f} \right| \in \mathscr{R}(\alpha)$.

    To prove (\ref{eq:6.40}), put $y = (y_1, \dots, y_k)$, 
    where $y_j = \int f_j \d \alpha$. 
    Then we have $\mathbf{y_j} = \int \mathbf{f_j} \d \alpha$, and
    \begin{equation*}
        \left| \mathbf{f} \right|^2 = 
        \sum y_i^2 =
        \sum y_j \int f_j \d \alpha = 
        \int \left( \sum y_j f_j \right) \d \alpha .
    \end{equation*}
    By the Schwarz inequality,
    \begin{equation}
        \label{eq:6.42}
        \sum y_j f_j (t) \leq \left| \mathbf{y} \right| \left| \mathbf{f}(t) \right| 
        \quad 
        (a \leq t \leq b);
    \end{equation}
    hence Theorem \ref{thm:6.12}(b) implies
    \begin{equation}
        \label{eq:6.43}
        \left| \mathbf{y} \right|^2 \leq 
        \left| \mathbf{y} \right| \int \left| \mathbf{f} \right| \d \alpha.
    \end{equation}
    If $\mathbf{y = 0}$, (\ref{eq:6.40}) is trivial.
    If $\mathbf{y \neq 0}$, division of (\ref{eq:6.43}) gives (\ref{eq:6.40}).
\end{proof}
