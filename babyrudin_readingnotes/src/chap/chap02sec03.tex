% chap02sec03
\section{Compact sets}

\begin{myDefinition}
    \label{myDefinition:2.31 open cover}
    By an \emph{open cover} of a set $E$ in a metric space $X$ we mean a collection $\{G_{\alpha}\}$ of open subsets of $X$ such that $E \subset \cup_{\alpha} G_{\alpha}$.
\end{myDefinition}

\begin{myDefinition}
    \label{myDefinition:2.32 compact set}
    A subset $K$ of a metric space $X$ is said to be \emph{compact} if every open cover of $K$ contains a \emph{finite} subcover.
\end{myDefinition}

More explicitly, the requirement is that if $\{G_{\alpha}\}$ is an open cover of $K$, then there are finitely many indices $\alpha_1, ..., \alpha_n$ such that
\begin{equation*}
    K \subset G_{\alpha_{1}} \cup \cdots \cup G_{\alpha_{n}}.
\end{equation*}

The notion of compactness is of great importance in analysis, especially
in connection with continuity (Chap. 4).

It is clear that every finite set is compact. The existence of a large class of infinite compact sets in $\mathbb{R}^k$ will follow from Theorem 2.41.

We observed earlier (in Sec. 2.29) that if $E \subset Y \subset X$, then $E$ may be open relative to $Y$ without being open relative to $X$. The property of being open thus depends on the space in which $E$ is embedded. The same is true of the property of being closed.

Compactness, however, behaves better, as we shall now see. To formulate the next theorem, let us say, temporarily, that $K$ is compact relative to $X$ if the requirements of Definition 2.32 are met.
% BASIC TOPOLOGY 37

\begin{thm}
    \label{thm:2.33 compact relative}
    Suppose $K \subset Y \subset X$. Then $K$ is compact relative to $X$ if and only if $K$ is compact relative to $Y$.
\end{thm}

\begin{proof}
    % Proof 
    Suppose $K$ is compact relative to $X$, and let $\{V_\alpha\}$ be a collection of sets, open relative to $Y$, such that $K \subset \cup_\alpha V_\alpha$ theorem 2.30, there are sets $G_\alpha$, open relative to $X$, such that $V_\alpha = Y \cap G_\alpha$, for all $\alpha$; and since $K$ is compact relative to $X$, we have
    \begin{equation}\label{eq:2.22}
        K \subset G_{\alpha_{1}} \cup \cdots \cup G_{\alpha_{n}}.
    \end{equation}
    for some choice of finitely many indices $\alpha_1 ..., \alpha_n$. Since $K \subset Y$, \ref{eq:2.22} implies
    \begin{equation}\label{eq:2.23}
        K \subset V_{\alpha_{1}} \cup \cdots \cup V_{\alpha_{n}}.
    \end{equation}
    This proves that $K$ is compact relative to $Y$.
    
    Conversely, suppose $K$ is compact relative to $Y$, let $G_\alpha$ be a collection of open subsets of $X$ which covers $K$, and put $V_\alpha = Y \cap G_\alpha$. Then \ref{eq:2.23} will hold for some choice of $\alpha_1, ...,\alpha_n$; and since $V_\alpha = G_\alpha$, \ref{eq:2.23} implies \ref{eq:2.22}.
    
    This completes the proof.
\end{proof}

\begin{thm}
    \label{thm:2.34}
    Compact subsets of metric spaces are closed.
\end{thm}

% Proof Let K be a compact subset of a metric space X. We shall prove
% that the complement of K is an open subset of X.

% Suppose pe X, pc K. If ge K, let V, and W, be neighborhoods of p
% and g, respectively, of radius less than 4d(p, q) [see Definition 2.18(a)].
% Since K is compact, there are finitely many points gy, ..., g, in K such that

% KecW, vuuWw, =W

% If V=V, n:-nV,, then Vis a neighborhood of p which does not
% intersect W. Hence V < Kc, so that p is an interior point of K°. The
% theorem follows.

\begin{thm}\label{thm:2.34 compact-close}
    Closed subsets of compact sets are compact.
\end{thm}

% Proof Suppose F< K c X, Fis closed (relative to X), and K is compact.
% Let {V,} be an open cover of F. If Fcis adjoined to {V,}, we obtain an
% 38 PRINCIPLES OF MATHEMATICAL ANALYSIS

% open cover Q of K. Since K is compact, there is a finite subcollection ®
% of Q which covers XK, and hence F. If Fcis a member of ®, we may remove
% it from ® and still retain an open cover of F. We have thus shown that a
% finite subcollection of {V,} covers F.

% Corollary 
\begin{myCorollary}
    If $F$ is closed and $K$ is compact, then $F \cap K$ is compact.
\end{myCorollary}


% Proof Theorems 2.24(b) and 2.34 show that Fn K is closed; since
% Fn Kc K, Theorem 2.35 shows that Fn K is compact.

\begin{thm}
    % 2.36 Theorem
    If $\{K_\alpha\}$ is a collection of compact subsets of a metric space $X$ such that the intersection of every finite subcollection of $\{K_\alpha\}$ is nonempty, then $\cap K_\alpha$ is nonempty.    
\end{thm}

% Proof F ix a member K, of {K,} and put G, = K. Assume that no point
% of K; belongs to every K,. Then the sets G, form an open cover of Kj;
% and since K, is compact, there are finitely many indices a, ..., a, such
% that K; « G,, U *** U G,,. But this means that
% \begin{equation*}
%     K_1 \cap
%     K_{\alpha_1} \cap
%     \dots \cap
%     K_{\alpha_n}
% \end{equation*}
% is empty, in contradiction to our hypothesis.

\begin{myCorollary}
    % Corollary 
    If $\{K_\alpha\}$ is a sequence of nonempty compact sets such that $K_n \supset K_{n+1} (n=1,2,3,...)$, then $\cap_1^\infty K_n$ is not empty.
\end{myCorollary}

\begin{thm}
    % 2.37 Theorem 
    If $E$ is an infinite subset of a compact set $K$, then $E$ has a limit point in $K$.
\end{thm}

\begin{thm}
    % 2.38 Theorem 
    If $\{I_n\}$ is a sequence of intervals in $\mathbb{R}^1$, such that $I_n \supset I_{n+1}, (n=1,2,3,...)$, then $\cap_1^\infty I_n$ is not empty.
\end{thm}

\begin{thm}\label{thm:2.39}
    Let $k$ be a positive integer. If ${I_n}$ is a sequence of $k$-cells such that $I_n \supset I_{n+1}, (n=1,2,3,...)$, then $\cap_1^\infty I_n$ is not empty.
\end{thm}

\begin{thm}\label{thm:2.40}
    Every $k$-cell is compact.
\end{thm}

\begin{thm}\label{thm:2.41}
    If a set $E$ in $\mathbb{R}^k$ has one of the following three properties, then it has the other two:

(a) $E$ is closed and bounded.

(b) $E$ is compact.

(c) Every infinite subset of $E$ has a limit point in $E$.
\end{thm}

\begin{thm}\label{thm:2.42 Weierstrass}
    Every bounded infinite subset of $\mathbb{R}^k$ has a limit point in $\mathbb{R}^k$.
\end{thm}

\begin{proof}
    Being bounded, the set $E$ in question is a subset of a $k$-cell $I\subset \mathbb{R}^k$.
    By Theorem 2.40, $I$ is compact, 
    and so $E$ has a limit point in $I$, by Theorem 2.37.
\end{proof}