% chap02sec03
\section{Compact sets}

\mybox{
    紧集 A subset $S$ of a topological space $X$ is compact if for every open cover of $S$ there exists a finite subcover of $S$. 
    \url{https://mathworld.wolfram.com/CompactSet.html}

    紧集的任何开覆盖都有有限子覆盖

    在 $\R^{n}$中, 下面三个条件等价:
    有界闭(bounded and closed);紧(compact);列紧(sequentially compact);
}

\begin{myDef}
    \label{myDef:2.31}
    By an \emph{open cover} of a set $E$ in a metric space $X$ we mean a collection $\{G_{\alpha}\}$ of open subsets of $X$ such that $E \subset \cup_{\alpha} G_{\alpha}$.
\end{myDef}

\begin{myDef}
    \label{myDef:2.32}
    A subset $K$ of a metric space $X$ is said to be \emph{compact} if every open cover of $K$ contains a \emph{finite} subcover.
\end{myDef}
\mybox{紧集的介绍从开覆盖开始, 紧集的每个开覆盖包含一个有限子覆盖.}

More explicitly, the requirement is that if $\{G_{\alpha}\}$ is an open cover of $K$, then there are finitely many indices $\alpha_1, ..., \alpha_n$ such that
\begin{equation*}
    K \subset G_{\alpha_{1}} \cup \cdots \cup G_{\alpha_{n}}.
\end{equation*}

The notion of compactness is of great importance in analysis, especially
in connection with continuity (Chap. 4).

It is clear that every finite set is compact. The existence of a large class of infinite compact sets in $\R^k$ will follow from Theorem 2.41.

We observed earlier (in Sec. 2.29) that if $E \subset Y \subset X$, then $E$ may be open relative to $Y$ without being open relative to $X$. The property of being open thus depends on the space in which $E$ is embedded. The same is true of the property of being closed.

Compactness, however, behaves better, as we shall now see. To formulate the next theorem, let us say, temporarily, that $K$ is compact relative to $X$ if the requirements of Definition 2.32 are met.
% BASIC TOPOLOGY 37

\begin{thm}
    \label{thm:2.33}
    Suppose $K \subset Y \subset X$. Then $K$ is compact relative to $X$ if and only if $K$ is compact relative to $Y$.
\end{thm}

By virtue of this theorem we are able, in many situations, 
to regard compact sets as metric spaces in their own right, 
without paying any attention to any embedding space. 
In particular, although it makes little sense to talk of \emph{open} spaces, 
or of \emph{closed} spaces 
(every metric space $X$ is an open subset of itself,
and is a closed subset of itself), 
it does make sense to talk of \emph{compact} metric spaces.

\begin{proof}
    % Proof 
    Suppose $K$ is compact relative to $X$, and let $\{V_\alpha\}$ be a collection of sets, open relative to $Y$, such that $K \subset \cup_\alpha V_\alpha$ theorem 2.30, there are sets $G_\alpha$, open relative to $X$, such that $V_\alpha = Y \cap G_\alpha$, for all $\alpha$; and since $K$ is compact relative to $X$, we have
    \begin{equation}\label{eq:2.22}
        K \subset G_{\alpha_{1}} \cup \cdots \cup G_{\alpha_{n}}.
    \end{equation}
    for some choice of finitely many indices $\alpha_1 ..., \alpha_n$. Since $K \subset Y$, \ref{eq:2.22} implies
    \begin{equation}\label{eq:2.23}
        K \subset V_{\alpha_{1}} \cup \cdots \cup V_{\alpha_{n}}.
    \end{equation}
    This proves that $K$ is compact relative to $Y$.
    
    Conversely, suppose $K$ is compact relative to $Y$, let $G_\alpha$ be a collection of open subsets of $X$ which covers $K$, and put $V_\alpha = Y \cap G_\alpha$. Then \ref{eq:2.23} will hold for some choice of $\alpha_1, ...,\alpha_n$; and since $V_\alpha = G_\alpha$, \ref{eq:2.23} implies \ref{eq:2.22}.
    
    This completes the proof.
\end{proof}

\begin{thm}
    \label{thm:2.34}
    Compact subsets of metric spaces are closed.
\end{thm}

\begin{proof}
    Let $K$ be a compact subset of a metric space $X$. 
    We shall prove that the complement of $K$ is an open subset of $X$.

    Suppose $p \in X$, $p \not\in K$. 
    If $q \in K$, let $V_q$ and $W_q$ be neighborhoods of $p$ and $q$, 
    respectively, of radius less than $\tfrac{1}{2}d(p, q)$ [see Definition \ref{myDef:2.18}(a)].
    Since $K$ is compact, there are finitely many points $q_1, ..., q_n$ in $K$ such that
    \begin{equation*}
        K \subset 
        W_{q_1} \cup \cdots \cup
        W_{q_n}. 
    \end{equation*}

    If $V=V_{q_1} \cap \cdots \cap V_{q_1}$, 
    then $V$ is a neighborhood of $p$ which does not intersect $W$. 
    Hence $V \subset K^c$, so that $p$ is an interior point of $K^c$.
    The theorem follows.
\end{proof}

\begin{thm}
    \label{thm:2.35}
    Closed subsets of compact sets are compact.
\end{thm}
\mybox{紧集的闭子集仍是紧集}

\begin{proof}
    Suppose $F \subset K \subset X$, 
    $F$ is closed (relative to $X$), 
    and $K$ is compact. 
    Let $\sequence{V_\alpha}$ be an open cover of $F$. 
    If $F^c$ is adjoined to $\sequence{V_\alpha}$, 
    we obtain an open cover $\Omega$ of $K$. 
    Since $K$ is compact, there is a finite subcollection $\Phi$ of $\Omega$ which covers $K$, and hence $F$. 
    If $F^c$ is a member of $\Phi$, 
    we may remove it from $\Phi$ and still retain an open cover of $F$. 
    We have thus shown that a finite subcollection of $\sequence{V_\alpha}$ covers $F$.
\end{proof}

\begin{myCorollary}
    If $F$ is closed and $K$ is compact, then $F \cap K$ is compact.
\end{myCorollary}

\begin{proof}
    Theorems \ref{thm:2.24}(b) and \ref{thm:2.34} show that $F \cap K$ is closed; 
    since $F \cap K \subset K$, 
    Theorem \ref{thm:2.35} shows that $F \cap K$ is compact.
\end{proof}

\begin{thm}
    \label{thm:2.36}
    % 2.36 Theorem
    If $\{K_\alpha\}$ is a collection of compact subsets of a metric space $X$ 
    such that the intersection of every finite subcollection of $\{K_\alpha\}$ is nonempty, 
    then $\cap K_\alpha$ is nonempty.    
\end{thm}

\begin{proof}
    Fix a member $K_1$ of $\sequence{K_\alpha}$ and put $G_\alpha = K^c_\alpha$. 
    Assume that no point of $K_1$ belongs to every $K_\alpha$. 
    Then the sets $G_\alpha$ form an open cover of $K_1$; 
    and since $K_1$ is compact, there are finitely many indices $\alpha_1, ..., \alpha_n$ such that $K \subset G_{\alpha_1} \cup \cdots \cup G_{\alpha_n}$. 
    But this means that
\begin{equation*}
    K_1 \cap
    K_{\alpha_1} \cap
    \dots \cap
    K_{\alpha_n}
\end{equation*}
is empty, in contradiction to our hypothesis.
\end{proof}

\begin{myCorollary}
    If $\{K_\alpha\}$ is a sequence of nonempty compact sets such that $K_n \supset K_{n+1} (n=1,2,3,...)$, then $\cap_1^\infty K_n$ is not empty.
\end{myCorollary}

\begin{thm}
    \label{thm:2.37}
    If $E$ is an infinite subset of a compact set $K$, 
    then $E$ has a limit point in $K$.
\end{thm}

\begin{proof}
    If no point of $K$ were a limit point of $E$, 
    then each $q \in K$ would have a neighborhood $V_q$ which contains at most one point of $E$ (namely, $q$, if $q \in E$). 
    It is clear that no finite subcollection of $\sequence{V_q}$ can cover $E$;
    and the same is true of $K$, since $E \subset K$. This contradicts the compactness of $K$.
\end{proof}

\begin{thm}
    \label{thm:2.38}
    If $\{I_n\}$ is a sequence of intervals in $\R^1$, 
    such that $I_n \supset I_{n+1}, (n=1,2,3,...)$, 
    then $\cap_1^\infty I_n$ is not empty.
\end{thm}

\begin{proof}
    If $I_n = [a_n, b_n]$, let $E$ be the set of all $a_n$. 
    Then $E$ is nonempty and bounded above (by $b_1$). 
    Let $x$ be the sup of $E$. 
    If $m$ and $n$ are positive integers, then
    \begin{equation*}
        a_{n} \leq
        a_{m+n} \leq
        b_{m+n} \leq
        b_{n} .
    \end{equation*}
    so that $x \leq b_m$ for each $m$. 
    Since it is obvious that $a_m \leq x$, 
    we see that $x \in I_m$ for $m = l, 2, 3, ...$.
\end{proof}

\begin{thm}
    \label{thm:2.39}
    Let $k$ be a positive integer. 
    If ${I_n}$ is a sequence of $k$-cells such that $I_n \supset I_{n+1}, (n=1,2,3,...)$, 
    then $\cap_1^\infty I_n$ is not empty.
\end{thm}

\begin{proof}
    Let $I_n$ consist of all points $\mathbf{x} = (x_1,...,x_k)$ such that
    \begin{equation*}
        a_{n, j} \leq
        x_j \leq
        b_{n, j} 
        \quad
        (1 \leq j \leq k; n = 1,2,3,...),
    \end{equation*}
    and put $I_{n,j} = [a_{n,j}, b_{n,j}]$. 
    For each $j$, the sequence $\sequence{I_{n,j}}$ satisfies the hypotheses of Theorem \ref{thm:2.38}. 
    Hence there are real numbers $x_j^*(1 \leq j \leq k)$ such that
    \begin{equation*}
        a_{n,j}
        \leq x_j^* \leq
        b_{n,j}
        \quad
        (1 \leq j \leq k; n = 1, 2, 3, ... ).
    \end{equation*}
    Setting $\mathbf{x}* = (x_1^*, ... , x_k^*)$, 
    we see that $\mathbf{x}^* \in I_n$ for $n = 1, 2, 3, ...$.
    The theorem follows.
\end{proof}

\begin{thm}
    \label{thm:2.40}
    Every $k$-cell is compact.
\end{thm}

\begin{proof}
    Let $I$ be a $k$-cell, 
    consisting of all points $\mathbf{x} = (x_1, \dots, x_k)$
    such that $a_j \leq x_j \leq  b_j (1 \leq j \leq k)$. 
    Put
    \begin{equation*}
        \delta = 
        \left\{ \sum_{1}^{k} (b_j - a_j)^2 \right\}^{1/2}
    \end{equation*} 
    Then $ \left| \mathbf{x-y} \right| \leq \delta$, if $x \in I, y \in I$.
    
    Suppose, to get a contradiction, 
    that there exists an open cover $\sequence{G_\alpha}$ of $I$ 
    which contains no finite subcover of $I$. 
    Put $c_j = (a_j + b_j)/2$. 
    The intervals $[a_j , c_j]$ and $[c_j , b_j]$ 
    then determine $2^k$ $k$-cells $Q_i$ whose union is $I$.
    At least one of these sets $Q_i$, call it $I_1$, 
    cannot be covered by any finite subcollection of $\sequence{G_\alpha}$ 
    (otherwise $I$ could be so covered). 
    We next subdivide $I_1$ and continue the process. 
    We obtain a sequence $\sequence{I_n}$ with the following properties:
    \begin{enumerate}[(a)]
        \item $I \supset I_1 \supset I_2 \supset I_3 \supset \dots$;
        \item $I_n$ is not covered by any finite subcollection of $\sequence{G_\alpha}$;
        \item if $\mathbf{x} \in I_n$ and $\mathbf{y} \in I_n$ , then $\left| \mathbf{x-y} \right| \leq 2^{-n}\delta$.
    \end{enumerate}
    
    By (a) and Theorem \ref{thm:2.39}, there is a point $\mathbf{x}^*$ which lies in every $I_n$. 
    For some $\alpha$, $\mathbf{x}^* \in G_\alpha$. 
    Since $G_\alpha$ is open, there exists $r > 0$ such that 
    $\left| \mathbf{y-x}^* \right| < r$ implies that $\mathbf{y} \in G_\alpha$. 
    If $n$ is so large that $2^{-n}\delta < r$ 
    (there is such an $n$, for otherwise $2^n \leq \delta/r$ for all positive integers $n$, which is absurd since $\R$ is archimedean),
    then (c) implies that $I_n \subset G_\alpha$, which contradicts (b).
    
    This completes the proof.
\end{proof}

The equivalence of (a) and (b) in the next theorem is known as the Heine-Borel theorem.

\begin{thm}
    \label{thm:2.41}
    If a set $E$ in $\R^k$ has one of the following three properties, then it has the other two:
    \begin{enumerate}[(a)]
        \item $E$ is closed and bounded.
        \item $E$ is compact.
        \item Every infinite subset of $E$ has a limit point in $E$.
    \end{enumerate}
\end{thm}

\begin{proof}
    If (a) holds, then $E \subset I$ for some $k$-cell $I$, 
    and (b) follows from Theorems \ref{thm:2.40} and \ref{thm:2.35}. 
    Theorem \ref{thm:2.37} shows that (b) implies (c). 
    It remains to be shown that (c) implies (a).

    If $E$ is not bounded, then $E$ contains points $\mathbf{x}_n$ with
    \begin{equation*}
        \left| \mathbf{x}_n \right| > n
        \quad
        (n = 1, 2, 3, ... ).        
    \end{equation*}
    The set $S$ consisting of these points $\mathbf{x}_n$ is infinite and clearly has no limit point in $\R^{k}$, hence has none in $E$. 
    Thus (c) implies that $E$ is bounded.
    
    If $E$ is not closed, then there is a point $\mathbf{x}_0 \in  \R^{k}$ which is a limit point of $E$ but not a point of $E$. 
    For $n = 1, 2, 3, ... $, there are points $\mathbf{x}_n \in E$ 
    such that $\left| \mathbf{x}_n - \mathbf{x}_0 \right| < 1/n$. 
    Let $S$ be the set of these points $\mathbf{x}_n$. 
    Then $S$ is infinite (otherwise $\left| \mathbf{x}_n - \mathbf{x}_0 \right| $ would have a constant positive value, for infinitely many $n$), 
    $S$ has $\mathbf{x}_0$ as a limit point, 
    and $S$ has no other limit point in $\R^{k}$. 
    For if $\mathbf{y} \in \R^{k}$, $\mathbf{y} \neq \mathbf{x}_0$ , then
    \begin{align*}
        \left| \mathbf{x}_n - \mathbf{y} \right| 
        &\geq
        \left| \mathbf{x}_0 - \mathbf{y} \right| -
        \left| \mathbf{x}_n - \mathbf{x}_n \right| \\
        &\geq 
        \left| \mathbf{x}_0 - \mathbf{y} \right| - \frac{1}{n}
        \geq \frac{1}{2}
        \left| \mathbf{x}_0 - \mathbf{y} \right|
    \end{align*} 
    for all but finitely many $n$; 
    this shows that $\mathbf{y}$ is not a limit point of $S$ (Theorem \ref{thm:2.20}).

    Thus $S$ has no limit point in $E$; hence $E$ must be closed if (c) holds.
\end{proof}

We should remark, at this point, that (b) and (c) are equivalent in any
metric space (Exercise 26) but that (a) does not, in general, imply (b) and (c).
Examples are furnished by Exercise 16 and by the space $\mathscr{L}^2$ , which is discussed in Chap. 11.

\begin{thm}(Weierstrass)
    \label{thm:2.42}
    Every bounded infinite subset of $\R^k$ has a limit point in $\R^k$.
\end{thm}
\mybox{魏尔斯特拉斯定理 $\R^k$的有界无限子集 在$\R^k$ 中必有一极限点}
\begin{proof}
    Being bounded, the set $E$ in question is a subset of a $k$-cell $I\subset \R^k$.
    By Theorem \ref{thm:2.40}, $I$ is compact, 
    and so $E$ has a limit point in $I$, by Theorem 2.37.
\end{proof}