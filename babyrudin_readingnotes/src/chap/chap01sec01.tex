
\section{Introduction}
First we use $\sqrt{2}$ to construct real number system from integer and rational numbers.

% 1.1 Example
\begin{myExample}\label{Example:1.1}
\begin{equation}\label{eq:1.1}
    p^2=2
\end{equation}
$p$ is not a rational number.
\end{myExample}

\begin{proof}
% Proof: 
(反证法) 假设 $p$ 是有理数,  $\exists m,n \in \mathbf{N}$, s.t. $p=m/n$. $\gcd (m,n) = 1$.
Then \ref{eq:1.1}
\begin{equation}\label{eq:1.2}
    m^2 = 2n^2.
\end{equation}

$m$ is even, $m = 2k$.
那么有 $(2k)^2 = 2n^2$, $2k^2 = n^2$, $k$ is even, $\gcd (m,n)=2\neq 1$,
contrary to our choice of $m$ and $n$. Hence p can't be a rational number.
\end{proof}

After proving $\sqrt{2}$ isn't a rational number, rudin use $\sqrt{2}$ to divide the rationals
在证明 $\sqrt{2}$ 不是有理数后, 使用 $\sqrt{2}$ 将有理数集分成两部分.  引出了分划的概念? 
\begin{align*}
    A = \{p|p^2<2\}\\
    B = \{p|p^2>2\}
\end{align*}
$A$ \emph{contains no largest number},\\
$B$ \emph{contains no smallest number}.\\
$\forall p\in A$, $\exists q\in A$, s.t. $p<q$,\\
$\forall p\in B$, $\exists q\in B$, s.t. $p>q$,\\
$\forall p>0$
\begin{equation}\label{eq:1.3}
    q = p-\frac{p^2-2}{p+2} = \frac{2p+2}{p+2}
\end{equation}

Then 
\begin{equation}
    \label{eq:1.4}
    q^2 - 2 = \frac{2(p^2-2)}{(p+2)^2}
\end{equation}

If $p\in A$, $p^2<2$. \ref{eq:1.3} shows that $q>p$, \ref{eq:1.4} shows that $q^2<2$, $q\in A$.
If $p\in B$, $p^2>2$. \ref{eq:1.3} shows that $q<p$, \ref{eq:1.4} shows that $q^2>2$, $q\in B$.


\begin{myRemark}\label{Remark:1.2}
The purpose of the above discussion has been to show that the rational number system has certain gaps, 
in spite of the fact that between any two rationals there is another: If $r<s$ then $r<(r+s)/2<s$.
The real number system fills these gaps.
This is the principal reason for the fundamental role which it plays in analysis.
\end{myRemark}

\mybox{
% mynotes:
有理数的稠密性与实数的连续性. 在分析中, 考察极限等需要的是数系的连续性, 因此需要先建立实数系. 
事实上, 我们是先有微积分, 后有实数理论的. 
三次数学危机:
无理数, 微积分基础, 集合论
实数理论是极限的基础. 
}

In order to elucidate its structure, as well as that of the complex numbers, 
we start with a brief discussion of the genral concepts of \emph{ordered set} and \emph{field}.


\mybox{
% mynotes:
rudin引入复数的方法非常怪, 对初学者非常不友好, 过于抽象了. 
想起一个法国笑话, 问小学生$2+3$等于几, 回答 $2+3=3+2$ 加法是一个交换群(Abel 群)...
} 

Here is some of the standard set-theoretic terminology taht will be used throughout this book.

\mybox{接下来引入一些集合论的定义}

\begin{myDef}\label{myDef:1.3}
% 1.3 myDef
If $A$ is any set (whose elements may be numbers or any other objects), we write $x\in A$ to indicate that $x$ is a member (or an element) of $A$.

\mybox{
element 还没定义
    
object指代什么? 我个人认为集合理解的难点在于集合的集合. 这一点可以引出罗素悖论}

If $x$ is not a member of $A$, we write: $x\notin A$.

\emph{empty set} $\varnothing$ contains no element, If a set has at least one element, it is called \emph{nonempty}.

$A,B$ are sets, $\forall x\in A$, $x\in B$, we say that $A$ is a \emph{subset} of $B$, $A\subset B$ or $B\supset A$. If $\exists x\in B$, $x\notin A$, A is a \emph{proper subset} of $B$, $A \subsetneqq B$.
Note that $A\subset A$ for every set $A$.

(Bernstein) If $A\subset B$ and $B\subset A$, we write $A = B$. Otherwise $A\neq B$.
\end{myDef}
\mybox{% mynotes:
这条性质在证明集合相等时很常用
}

\begin{myDef}\label{myDef:1.4}
% 1.4 myDef
Throughout Chap. 1, the set of all rational numbers will be denoted by $\mathbb{Q}$.
\end{myDef}

有理数集$\mathbb{Q}$
