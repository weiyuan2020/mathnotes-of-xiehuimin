% chap07sec02
\section{Uniform convergence}
\mybox{一致连续}

\begin{mydef}
    \label{def:7.7}
    We say that a sequence of functions $\sequence{f_n}$, $n = 1, 2, 3,...$,
    converges \emph{uniformly} on $E$ to a function $f$ if for every $\varepsilon > 0$ there is an integer $N$ such that $n \geq N$ implies
    \begin{equation}
        \label{eq:7.12}
        \left| f_n(x) - f(x) \right| \leq \varepsilon
    \end{equation}
    for all $x \in E$.

    It is clear that every uniformly convergent sequence is pointwise convergent. 
    Quite explicitly, the difference between the two concepts is this: 
    If $\sequence{f_n}$ converges pointwise on $E$, then there exists a function $f$ such that, 
    for every $\varepsilon > 0$, and for every $x \in E$, 
    there is an integer $N$, depending one $\varepsilon$ and on $x$, 
    such that (\ref{eq:7.12}) holds if $n \geq N$; 
    if $\sequence{f_n}$ converges uniformly on $E$, 
    it is possible, for each $\varepsilon > 0$, to find \emph{one} integer $N$ which will do for all $x \in E$. 
    
    We say that the series $\sum f_n(x)$ converges uniformly on $E$ 
    if the sequence $\sequence{s_n}$ of partial sums defined by
    \begin{equation*}
        \sum_{i=1}^{n} f_i (x) = s_n (x)
    \end{equation*}
    converges uniformly on $E$. 
    
    The Cauchy criterion for uniform convergence is as follows.
\end{mydef}

\begin{thm}
    \label{thm:7.8}
    The sequence of functions $\sequence{f_n}$, defined on $E$, 
    converges uniformly on $E$ if and only if for every $\varepsilon > 0$ there exists an integer $N$ such that $n \geq N$, $x \in E$ implies
    \begin{equation}
        \label{eq:7.13}
        \left| f_n (x) - f_m (x) \right| \leq \varepsilon.
    \end{equation}
\end{thm}

% todo add proof

\begin{thm}
    \label{thm:7.9}
    Suppose 
    \begin{equation*}
        \lim_{n \to \infty} f_n (x) = f(x)
        \quad 
        (x \in E).
    \end{equation*}
    Put 
    \begin{equation*}
        M_n = \sup_{x \in E} \left| f_n (x) - f(x) \right| .
    \end{equation*}
    Then $f_n \rightarrow f$ uniformly on $E$ if and only if $M_n \rightarrow 0$ as $n \rightarrow \infty$.
\end{thm}

Since this is an immediate consequence of Definition \ref{def:7.7}, 
we omit the details of the proof.

For series, there is a very convenient test for uniform convergence, due to
Weierstrass.

\begin{thm}
    \label{thm:7.10}
    Suppose $\sequence{f_n}$ is a sequence of functions defined on $E$, 
    and suppose
    \begin{equation*}
        \left| f_n (x) \right| \leq M_n
        \quad 
        (x \in E, n = 1, 2, 3, ... ).
    \end{equation*}
    Then $\sum f_n$ converges uniformly on $E$ if $\sum M_n$ converges.
\end{thm}

Note that the converse is not asserted (and is, in fact, not true).
% todo add proof

\begin{thm}
    \label{thm:7.11}
    Suppose $f_n \rightarrow f$ uniformly on a set $E$ in a metric space. 
    Let $x$ be a limit point of $E$, and suppose that
    \begin{equation}
        \label{eq:7.15}
        \lim_{t \to x} f_n (t) = A_n
        \quad
        (n = 1, 2, 3, ... ).
    \end{equation}
    Then $\sequence{A_n}$ converges, and
    \begin{equation}
        \label{eq:7.16}
        \lim_{t \to x} f(t) = \lim_{t \to x} A_n.
    \end{equation}

    In other words, the conclusion is that
    \begin{equation}
        \label{eq:7.17}
        \lim_{t \to x} \lim_{n \to \infty} f_n (t) = 
        \lim_{n \to \infty} \lim_{t \to x} f_n (t).
    \end{equation}
\end{thm}

% todo add proof

\begin{thm}
    \label{thm:7.12}
    If $\sequence{f_n}$ is a sequence of continuous functions on $E$, 
    and if $f_n \rightarrow f$ 
    uniformly on $E$, then $f$ is continuous on $E$.
\end{thm}

This very important result is an immediate corollary of Theorem \ref{thm:7.11}.

The converse is not true; 
that is, a sequence of continuous functions may converge to a continuous function, although the convergence is not uniform.
Example \ref{myExample:7.6} is of this kind 
(to see this, apply Theorem \ref{thm:7.9}). 
But there is a case in which we can assert the converse.

\begin{thm}
    \label{thm:7.13}
    Suppose $K$ is compact, and
    \begin{enumerate}
        \item ${f_n}$ is a sequence of continuous functions on $K$,
        \item ${f_n}$ converges pointwise to a continuous function $f$ on $K$,
        \item $f_n(x)$ $f_n(x) \geq f_{n+1}(x)$ for all $x \in K$, $n = 1, 2, 3, ...$.
    \end{enumerate}
    Then $f_n \rightarrow f$ uniformly on $K$.
\end{thm}

\begin{mydef}
    \label{def:7.14}
    If $X$ is a metric space, $\mathscr{C}(X)$ will denote the set of all complex-valued, continuous, bounded functions with domain $X$.

    [Note that boundedness is redundant if $X$ is compact (Theorem \ref{thm:4.15}). 
    Thus $\mathscr{C}(X)$ consists of all complex continuous functions on $X$ if $X$ is compact.]

    We associate with each $f \in \mathscr{C}(X)$ its supremum norm
    \begin{equation*}
        \left\| f \right\| = \sup_{x \in X} \left| f(x) \right| .
    \end{equation*}
    Since 
    is assumed to be bounded, $\left\| f \right\| < \infty $. 
    It is obvious that $\left\| f \right\| = 0$ only if
    $f(x) = 0$ for every $x \in X$, 
    that is, only if $f = 0$. 
    If $h =f + g$, then
    \begin{equation*}
        \left| h(x) \right| \leq 
        \left| f(x) \right| + \left| g(x) \right| \leq
        \left\| f \right\| + \left\| g \right\| 
    \end{equation*}
    for all $x \in X$; hence
    \begin{equation*}
        \left\| f + g \right\| \leq
        \left\| f \right\| + \left\| g \right\| .
    \end{equation*}

    If we define the distance between $f \in \mathscr{C}(X)$ and $g \in \mathscr{C}(X)$ to be $\left\| f - g \right\| $,
    it follows that Axioms \ref{def:2.15} for a metric are satisfied.

    We have thus made $\mathscr{C}(X)$ into a metric space.
    
    Theorem \ref{thm:7.9} can be rephrased as follows:
    
    A sequence $\sequence{f_n}$ converges to $f$ with respect to the metric of $\mathscr{C}(X)$ if and only if $f_n \rightarrow f$ uniformly on $X$.
    
    Accordingly, closed subsets of $\mathscr{C}(X)$ are sometimes called \emph{uniformly closed}, 
    the closure of a set $\mathscr{A} \subset \mathscr{C}(X)$ is called its \emph{uniform closure}, and so on.
\end{mydef}

\begin{thm}
    \label{thm:7.15}
    The above metric makes $\mathscr{C}(X)$ into a complete metric space.
\end{thm}

% todo add proof

