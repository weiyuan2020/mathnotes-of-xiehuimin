
\section{The real field}

We now state the \emph{existence theorem} which is the core of this chapter.

\begin{thm}\label{thm:1.19}
    % 1.19 Theorem 
There exists an \emph{ordered field} $\mathbb{R}$ which has the least-upper-bound
property.

Moreover, $\mathbb{R}$ contains $\mathbb{Q}$ as a \emph{subfield}.
\end{thm}

The second statement means that $\mathbb{Q} \subset \mathbb{R}$ and that the operations of
addition and multiplication in $\mathbb{R}$, when applied to members of $\mathbb{Q}$, coincide with
the usual operations on rational numbers; also, the positive rational numbers
are positive elements of $\mathbb{R}$.

The members of $\mathbb{R}$ are called real numbers.

The proof of Theorem \ref{thm:1.19} is rather long and a bit tedious and is therefore
presented in an Appendix to Chap. 1. The proof actually constructs $\mathbb{R}$ from $\mathbb{Q}$.
% THE REAL AND COMPLEX NUMBER SYSTEMS 9

The next theorem could be extracted from this construction with very
little extra effort. However, we prefer to derive it from Theorem \ref{thm:1.19} since this
provides a good illustration of what one can do with the least-upper-bound
property.

\mybox{$\mathbb{R}$ 具有最小上界性质的有序域
least-upper-bound $\rightarrow$ upper bound in the sets.
ordered field (ordered set, field).

$\mathbb{Q} \in \mathbb{R}$ subfield

$x\in \mathbb{R}$, $x$ is a real number
}
\mybox{
    proof of theorem \ref{thm:1.19} is tedious.
    construct $\mathbb{R}$ from $\mathbb{Q}$

    tedious 乏味的, 冗长的\\
    derive 取得, 得到
    }

\begin{thm}\label{thm:1.20}(archimedean property of $\mathbb{R}$)
    (a) If $x \in \mathbb{R}$, $y \in \mathbb{R}$, and $x > 0$, then there is a positive integer $n$ such that
    \begin{equation*}
        nx > y
    \end{equation*}

    (b) If $x \in \mathbb{R}$, $y \in \mathbb{R}$, and $x < y$, then there exists a $p \in Q$ such that $x < p < y$.
\end{thm}

\begin{thm}\label{thm:1.21}
    % 1.21 Theorem 
    For every real $x > 0$ and every integer $n> 0$ there is one and only one positive real $y$ such that $y^n = x$.
\end{thm}

This number $y$ is written $\sqrt[n]{x}$ or $x^{1/n}$.

\textbf{Corollary} If $a$ and $b$ are positive real numbers and $n$ is a positive integer, then
\begin{equation*}
    (ab)^{1/n}= a^{1/n}b^{1/n}.
\end{equation*}

\begin{myDef}(Decimals)
    % 1.22 Decimals 
    We conclude this section by pointing out the relation between real numbers and decimals.
\end{myDef}

