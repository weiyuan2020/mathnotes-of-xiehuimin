
\section{The real field}

We now state the \emph{existence theorem} which is the core of this chapter.

\begin{thm}\label{thm:1.19}
    There exists an \emph{ordered field} $\R$ 
    which has the least-upper-bound property.

    Moreover, $\R$ contains $\Q $ as a \emph{subfield}.
\end{thm}

The second statement means that $\Q \subset \R$ 
and that the operations of addition and multiplication in $\R$, 
when applied to members of $\Q $, 
coincide with the usual operations on rational numbers; 
also, the positive rational numbers are positive elements of $\R$.

The members of $\R$ are called real numbers.

The proof of Theorem \ref{thm:1.19} is rather long and a bit tedious 
and is therefore presented in an Appendix to Chap. 1. The proof actually constructs $\R$ from $\Q$.

The next theorem could be extracted from this construction with very
little extra effort. 
However, we prefer to derive it from Theorem \ref{thm:1.19} since this
provides a good illustration of what one can do with the least-upper-bound property.

\mybox{$\R$ 具有最小上界性质的有序域
least-upper-bound $\rightarrow$ upper bound in the sets.
ordered field (ordered set, field).

$\Q \in \R$ subfield

$x\in \R$, $x$ is a real number
}
\mybox{
    proof of theorem \ref{thm:1.19} is tedious.
    construct $\R$ from $\Q $

    tedious 乏味的, 冗长的\\
    derive 取得, 得到
    }

\begin{thm}\label{thm:1.20}(archimedean property of $\R$)
    (a) If $x \in \R$, $y \in \R$, and $x > 0$, then there is a positive integer $n$ such that
    \begin{equation*}
        nx > y
    \end{equation*}

    (b) If $x \in \R$, $y \in \R$, and $x < y$, then there exists a $p \in Q$ such that $x < p < y$.
\end{thm}
\begin{proof}
    \begin{asparaenum}[(a)]
        \item Let $A$ be the set of all $nx$, 
        where $n$ runs through the positive integers.
        If (a) were false, then $y$ would be an upper bound of $A$. 
        But then $A$ has a least upper bound in $R$. 
        Put $\alpha = \sup A$. 
        Since $x > 0$, $\alpha - x < \alpha$, 
        and $\alpha - x$ is not an upper bound of $A$. 
        Hence $\alpha - x < mx$ for some positive integer $m$. 
        But then $\alpha < (m + l)x \in A$, 
        which is impossible, 
        since $\alpha$ is an upper bound of $A$.
        \item Since $x < y$, we have $y - x > $0, 
        and (a) furnishes a positive integer $n$ 
        such that
        \begin{equation*}
            n(y - x) > 1.
        \end{equation*}
        Apply (a) again, to obtain positive integers $m_1$ and $m_2$ 
        such that $m_1 > nx$,
        $m_2 > -nx$. Then
        $-m_2 < nx < m_1$
        Hence there is an integer $m$ (with $-m_2 \leq m \leq m_1$) 
        such that
        \begin{equation*}
            m - 1\leq 11x < m.
        \end{equation*}
        If we combine these inequalities, we obtain
        \begin{equation*}
            nx < m \leq 1 + nx < ny.
        \end{equation*}
        Since $n > 0$, it follows that
        \begin{equation*}
            x < \frac{m}{n} < y.
        \end{equation*}
        This proves (b), with $p = m/n$.
    \end{asparaenum}
\end{proof}

We shall now prove the existence of nth roots of positive reals. 
This proof will show how the difficulty pointed out in the Introduction 
(irrationality of $\sqrt{2}$) can be handled in $\R$.
\begin{thm}
    \label{thm:1.21}
    For every real $x > 0$ and every integer $n> 0$ 
    there is one and only one positive real $y$ 
    such that $y^n = x$.
\end{thm}

This number $y$ is written $\sqrt[n]{x}$ or $x^{1/n}$.

\begin{proof}
    That there is at most one such $y$ is clear, 
    since $0 < y_1 < y_2$ implies $y_1^n < y_2^n$.
    
    Let $E$ be the set consisting of all positive real numbers $t$ 
    such that $t^n < x$.
    
    If $t = x/(1 + x)$ then $0 \leq t < 1$. 
    Hence $t^{n} \leq t < x$.
    Thus $t \in E$, and $E$ is not empty.
    
    If $t > 1 + x$ then $t^{n} \geq t > x$, 
    so that $t \not\in E$. 
    Thus $1 + x$ is an upper bound of $E$.
    
    Hence Theorem \ref{thm:1.19} implies the existence of 
    \begin{equation*}
        y = \sup E.
    \end{equation*}
    To prove that $y^{n} = x$ we will show that 
    each of the inequalities $y^{n} < x$ and $y^{n} > x$ leads to a contradiction.
    
    The identity $b^{n} - a^{n}= (b - a)(b^{n-1} + b^{n}- 2a + \cdots + a^{n-1})$ yields the inequality
    \begin{equation*}
        b^{n} - a^{n} < (b - a)nb^{n-1}
    \end{equation*}
    when $0 < a < b$.
    
    Assume $y^{n} < x$. Choose $h$ so that $0 < h < 1$ and
    \begin{equation*}
        h < \frac{x - y^n}{n(y + 1)^{n-1}}.
    \end{equation*} 
    Put $a = y$, $b = y + h$. Then
    \begin{equation*}
        (y + h)^{n} - y^{n} 
        < hn(y + h)^{n-l} 
        < hn(y + l)^{n-1} 
        < x - y^{n}.
    \end{equation*}
    Thus $(y + h)^{n} < x$, and $y +h \in E$. 
    Since $y + h > y$, 
    this contradicts the fact that $y$ is an upper bound of $E$.

    Assume $y^{n} > x$. Put 
    \begin{equation*}
        k = \frac{y^n - x}{n y^{n-1}}
    \end{equation*}
    Then $0 < k < y$. 
    If $t \leq y - k$, we conclude that
    \begin{equation*}
        y^{n} - t^{n} 
        \leq y^{n} - (y - k)^{n} 
        < kny^{n-1} 
        = y^{n} - x.
    \end{equation*}
    Thus $t^{n} > x$, and $t \not\in E$. 
    It follows that $y - k$ is an upper bound of $E$.
    But $y - k < y$, 
    which contradicts the fact that $y$ is the least upper bound of $E$.
    Hence $y^{n} = x$, and the proof is complete.
\end{proof}
\mybox{
    很经典的等式证明, 
    从两边的不等式不成立出发证明有且仅有等式成立.
}

\begin{myCorollary*}
    If $a$ and $b$ are positive real numbers 
    and $n$ is a positive integer, 
    then
    \begin{equation*}
        (ab)^{1/n}= a^{1/n}b^{1/n}.
    \end{equation*}
\end{myCorollary*}

\begin{myDef}
    \label{myDef:1.22}
    (Decimals)\\
    We conclude this section by pointing out the relation between real numbers and decimals.
\end{myDef}

