% chap08sec03
\section{The trigonometric functions}
Let us define
\begin{equation}
    \label{eq:8.46}
    C(x) = \frac{1}{2} \left[ E(ix) + E(-ix) \right],
    \quad 
    S(x) = \frac{1}{2i} \left[ E(ix) - E(-ix) \right],
\end{equation}
We shall show that $C(x)$ and $S(x)$ coincide with the function $\cos x$ and $\sin x$,
whose definition is usually based on geometric considerations.
By (\ref{eq:8.25}), $E(\bar{z})=\overline{E(z)}$.
Hence (\ref{eq:8.46}) shows that $C(x)$ and $S(x)$ are real for real $x$.
Also,
\begin{equation}
    \label{eq:8.47}
    E(ix) = C(x) + iS(x).
\end{equation}
Thus $C(x)$ and $S(x)$ are the real and imaginary parts, respectively,
of $E(ix)$, if $x$ is real. By (\ref{eq:8.27}),
\begin{equation*}
    \left| E(ix) \right|^2 
    = E(ix)\overline{E(ix)}
    = E(ix)E(-ix)
    = 1,
\end{equation*}
so that 
\begin{equation}
    \label{eq:8.48}
    \left| E(ix) \right| = 1
    \quad (x \text{ real}).
\end{equation}

From (\ref{eq:8.46}) we can read off that $C(0) = 1, S(0) = 0$, 
and (\ref{eq:8.28}) shows that 
\begin{equation}
    \label{eq:8.49}
    C'(x) = -S(x), \quad 
    S'(x) = C(x).
\end{equation}

We assert that there exist positive numbers $x$ such that $C(x) = 0$. 
For suppose this is not so. 
Since $C(0) = 1$, it then follows that $C(x) > 0$ for all $x > 0$, 
hence $S'(x) > 0$, by (\ref{eq:8.49}), hence $S$ is strictly increasing; 
and since $S(0) = 0$,
we have $S(x) > 0$ if $x > 0$. 
Hence if $0 < x < y$, we have
\begin{equation}
    \label{eq:8.50}
    S(x)(y - x) 
    < \int_{y}^{x} S(t) \d t
    = C(x) - C(y)
    \leq 2.
\end{equation}
The last inequality follows from (\ref{eq:8.48}) and (\ref{eq:8.47}).
Since $S(x) > 0$, (\ref{eq:8.50}) cannot be true for large $y$,
and we have a contradiction.

Let $x_0$ be the smallest positive number such that $C(x_0) = 0$. 
This exists, since the set of zeros of a continuous function is closed, 
and $C(0) \neq 0$. 
We define the number $\pi$ by
\begin{equation}
    \label{eq:8.51}
    \pi = 2 x_0.
\end{equation}

Then $C(\pi/2) = 0$, and (\ref{eq:8.48}) shows that $S(\pi/2) = \pm 1$.
Since $C(x) > 0$ in $(0, \pi/2)$,
$S$ is increasing in $(0, \pi/2)$;
hence $S(\pi/2) = 1$. Thus 
\begin{equation*}
    E\left( \frac{\pi i}{2} \right) = i,
\end{equation*}
and the addition formula gives
\begin{equation}
    \label{eq:8.52}
    E(\pi i) = -1, \quad
    E(2 \pi i) = 1;
\end{equation}
hence 
\begin{equation}
    \label{eq:8.53}
    E(z + 2\pi i) = E(z)
    \quad (z \text{ complex}).
\end{equation}

\begin{thm}
    \label{thm:8.7}
    \begin{enumerate}[(a)]
        \item The function $E$ is periodic, with period $2 \pi i$.
        \item The functions $C$ and $S$ are periodic, with period $2 \pi$.
        \item If $0 < t < 2 \pi$, then $E(it) \neq 1$.
        \item If $z$ is a complex number with $\left| z \right| = 1$, there is a unique $t$ in $[0, 2\pi)$ such that $E(it) = z$.
    \end{enumerate}
\end{thm}

% todo add proof


It follows from (d) and (\ref{eq:8.48}) that the curve $\gamma$ defined by
\begin{equation}
    \label{eq:8.54}
    \gamma (t) = E(it)
    \quad (0 \leq t \leq 2\pi)
\end{equation}
is a simple closed curve whose range is the unit circle in the plane.
Since $\gamma'(t) = i E(it)$, the length of $\gamma$ is 
\begin{equation*}
    \int_{0}^{2\pi} \left| \gamma'(t) \right| \d t = 2\pi,
\end{equation*}
by Theorem \ref{thm:6.27}.
This is of course the expected result for the circumference of
a circle of radius 1. 
It shows that $\pi$, defined by (\ref{eq:8.51}), has the usual geometric
significance.

In the same way we see that the point $\delta(t)$ describes a circular arc of length
$t_0$ as $t$ increases from 0 to $t_0$.
Consideration of the triangle whose vertices are
\begin{equation*}
    z_1 = 0, \quad
    z_2 = \gamma (t_0), \quad
    z_2 = C(t_0)
\end{equation*}
shows that $C(t)$ and $S(t)$ are indeed identical with $\cos t$ and $\sin t$, if the latter are defined in the usual way as ratios of the sides of a right triangle.

It should be stressed that we derived the basic properties of the trigonometric functions from (\ref{eq:8.46}) and (\ref{eq:8.25}), 
without any appeal to the geometric notion of angle. 
There are other nongeometric approaches to these functions. 
The papers by W. F. Eberlein (\emph{Amer. Math. Monthly}, vol. 74, 1967, pp. 1223-1225)
and by G. B. Robison (\emph{Math. Mag.}, vol. 41, 1968, pp. 66-70) deal with these topics.

