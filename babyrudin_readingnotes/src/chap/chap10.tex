% chap10
\chapter{Integration of differential forms}
\label{chap:10}

Integration can be studied on many levels. 
In Chap. 6, the theory was developed for reasonably well-behaved functions 
on subintervals of the real line. 
In Chap. 11 we shall encounter a very highly developed theory of integration that
can be applied to much larger classes of functions, 
whose domains are more or less arbitrary sets, 
not necessarily subsets of $\R^n$. 
The present chapter is
devoted to those aspects of integration theory that are closely related to the
geometry of euclidean spaces, such as the change of variables formula, line
integrals, and the machinery of differential forms that is used in the statement
and proof of then-dimensional analogue of the fundamental theorem of calculus,
namely Stokes' theorem.

% chap10sec01

\section{Integration}
\begin{mydef}
    \label{mydef:10.1}
    Suppose $I^k$ is a $k$-cell in $\R^k$, consisting of all
    \begin{equation*}
        \mathbf{x} = (x_1,\dots,x_k)
    \end{equation*}
    such that 
    \begin{equation}
        \label{eq:10.1}
        a_i \leq x_i \leq b_i 
        \quad 
        (i = 1,\dots, k) ,
    \end{equation}
    $I^j$ is the $j$-cell in $\R^j$ defined by the first $j$ inequalities (\ref{eq:10.1}), and f is a real continuous function on $I^k$.

    Put $f = f_k$, and define $f_{k-1}$ on $I^{k-1}$ by
    \begin{equation*}
        f_{k-1}(x_1,\dots,x_{k-1}) = 
        \int_{a_k}^{b_k} f_k (x_1,\dots,x_{k-1},x_k) \d x_k .
    \end{equation*}
    The uniform continuity of $f_k$ on $I^k$ shows that $f_{k-1}$ is continuous on $I^{k-1}$.
    Hence we can repeat this process and obtain functions $f_j$, continuous on $I^j$, 
    such that $f_{j-1}$ is the integral of $f_j$, with respect to $x_j$, over $[a_j, b_j]$. 
    After $k$ steps we arrive at a \emph{number} $f_0$, 
    which we call the \emph{integral of $f$ over $I^k$}; 
    we write it in the form
    \begin{equation}
        \label{eq:10.2}
        \int_{I^k} f(\mathbf{x}) \d \mathbf{x}
        \text{  or  }
        \int_{I^k} f.
    \end{equation}
    
    A priori, this definition of the integral depends on the order in which the $k$ integrations are carried out. 
    However, this dependence is only apparent. 
    To prove this, let us introduce the temporary notation $L(f)$ for the integral (\ref{eq:10.2}) and $L'(f)$ for the result obtained by carrying out the $k$ integrations in some other order.
\end{mydef}
\mybox{
    $L(f)$ 积分 (\ref{eq:10.2}) 暂时的记号
    
    $L'(f)$ 用另外的次序求这 $k$ 个积分的结果
}

\begin{thm}
    \label{thm:10.2}
    For every $f \in \mathscr{C}(I^k)$, $L(f) = L'(f)$.
\end{thm}
\mybox{对区间上的连续函数, 积分结果与积分顺序无关}
% todo add proof
\mybox{Stone-Weierstrass 定理能够用到这些函数上}

\begin{mydef}
    \label{mydef:10.3}
    The \emph{support} of a (real or complex) function $f$ on $\R^k$ is the
    closure of the set of all points $\mathbf{x} \in R^k$ 
    at which $f(\mathbf{x}) \neq 0$. 
    If $f$ is a continuous function with compact support, 
    let $I^k$ be any $k$-cell which contains the support of $f$, 
    and define
    \begin{equation}
        \label{eq:10.3}
        \int_{R^k} f =
        \int_{I^k} f .
    \end{equation}
    The integral so defined is evidently independent of the choice of $I^k$, provided only that $I^k$ contains the support of $f$.

\end{mydef}

\mybox{support 支集}

It is now tempting to extend the definition of the integral over $R^k$ to
functions which are limits (in some sense) of continuous functions with compact support. 
We do not want to discuss the conditions under which this can be done; 
the proper setting for this question is the Lebesgue integral. 

\begin{myExample}
    \label{myexample:10.4}
    Let $Q^k$ be the $k$-simplex which consists of all points $\mathbf{x} = (x_1, \dots , x_k)$ in $\R^k$ for which $x_1 + \dots + x_k \leq 1$ and $x_i \geq 0$ for $i = 1, ... , k$. 
    If $k = 3$, for example, $Q^k$ is a tetrahedron, with vertices at $\mathbf{0, e_1, e_2, e_3}$. 
    If $f \in \mathscr{C}(Q^k)$,
    extend $f$ to a function on $I^k$ by setting $f(\mathbf{x}) = \mathbf{0}$ off $Q^k$, and define
    \begin{equation}
        \label{eq:10.4}
        \int_{Q^k} f =
        \int_{I^k} f .
    \end{equation}
    Here $I^k$ is the ``unit cube'' defined by
    \begin{equation*}
        0 \leq x_i \leq 1
        \quad 
        (1 \leq i \leq k).
    \end{equation*}

    Since $f$ may be discontinuous on $I^k$, the existence of the integral on the right of \eqref{eq:10.4} needs proof. 
    We also wish to show that this integral is independent of the order in which the $k$ single integrations are carried out.

    To do this, suppose $0 < \delta < 1$, put
    \begin{equation}
        \label{eq:10.5}
        \phi(t) = \left\{ 
            \begin{array}{ll}
                1 & (t\leq 1-\delta) \\
                \frac{(1-t)}{\delta} & (1-\delta < t \leq 1) \\
                0 & (1<t), \\
            \end{array}
         \right.
    \end{equation}
    and define 
    \begin{equation}
        \label{eq:10.6}
        F(\mathbf{x}) = 
        \phi(x_1+\cdots+x_k) f(\mathbf{x})
        \quad 
        (\mathbf{x} \in I^k).
    \end{equation}
    Then $F \in \mathscr{C}(I^k)$.

    Put $\mathbf{y} = (x_1, \dots , x_{k-1})$, $\mathbf{x} = (\mathbf{y}, x_k)$. 
    For each $\mathbf{y} \in I^{k-1}$, the set of all $x_k$ such that $F(\mathbf{y}, x_k) \neq f(\mathbf{y}; x_k)$ is either empty or is a segment whose length does not exceed $\delta$. 
    Since $0 \leq \phi \leq 1$, it follows that
    \begin{equation}
        \label{eq:10.7}
        \left| F_{k-1}(\mathbf{y})-f_{k-1}(\mathbf{y}) \right| \leq \delta \left\| f \right\| 
        \quad 
        (\mathbf{y} \in I^{k-1}),
    \end{equation}
    where $\left\| f \right\|$ has the same meaning as in the proof of Theorem \ref{thm:10.2}, 
    and $F_{k-1}$, $f_{k-1}$ are as in Definition \ref{mydef:10.1}.

    As $\delta \rightarrow 0$, \eqref{eq:10.7} exhibits $f_{k-1}$ as a uniform limit of a sequence of continuous functions. 
    Thus $f \in \mathscr{C}(I^{k-1})$, and the further integrations present no problem.

    This proves the existence of the integral \eqref{eq:10.4}. 
    Moreover, \eqref{eq:10.7} shows that
    \begin{equation}
        \label{eq:10.8}
        \left| 
            \int_{I^k} F(\mathbf{x}) \d \mathbf{x} -
            \int_{I^k} f(\mathbf{x}) \d \mathbf{x} 
         \right| \leq 
        \delta \left\| f \right\| .
    \end{equation}
    Note that \eqref{eq:10.8} is true, regardless of the order in which the $k$ single integrations are carried out. 
    Since $F \in \mathscr{C}(I^k)$, $\int F$ is unaffected by any change in this order.
    Hence \eqref{eq:10.8} shows that the same is true of $\int f$

    This completes the proof.

    Our next goal is the change of variables formula stated in Theorem \ref{thm:10.9}. 
    To facilitate its proof, we first discuss so-called primitive mappings, and partitions of unity.  
    Primitive mappings will enable us to get a clearer picture of the local action of $\mathscr{C}'$-mapping with invertible derivative, 
    and partitions of unity are a very useful device that makes it possible to use local information in a global setting.
\end{myExample}


% chap10sec02

\section{Primitive mappings}

\mybox{本原映射}

\begin{mydef}
    \label{mydef:10.5}
    If $\mathbf{G}$ maps an open set $E \subset \R^n$ into $\R^n$, 
    and if there is an integer $m$ and a real function $g$ with domain $E$ such that
    \begin{equation}
        \label{eq:10.9}
        \mathbf{G(x)} = \sum_{i \neq m} x_i \mathbf{e}_i + g(\mathbf{x}) \mathbf{e}_m
        \quad 
        (\mathbf{x} \in E) ,
    \end{equation}
    then we call $\mathbf{G}$ \emph{primitive}.
    A primitive mapping is thus one that changes at most one coordinate.
    Note that (\ref{eq:10.9}) can also be written in the form 
    \begin{equation}
        \label{eq:10.10}
        \mathbf{G(x)} = \mathbf{x} + \left[ g(\mathbf{x}) - x_m \right] \mathbf{e}_m .
    \end{equation}
    
    If $g$ is differentiable at some point $\mathbf{a} \in E$, so is $\mathbf{G}$.
    The matrix $[\alpha_{ij}]$ of the operator $\mathbf{G'(a)}$ has 
    \begin{equation}
        \label{eq:10.11}
        (D_1 g)(\mathbf{a}),...
        (D_m g)(\mathbf{a}),...
        (D_n g)(\mathbf{a})
    \end{equation}
    as ots $m$th row.
    For $j \neq m$, we have $\alpha_{jj} = 1$ and $\alpha_{ij} = 0$ if $i \neq j$.
    The Jacobian of $\mathbf{G}$ at $\mathbf{a}$ is thus given by 
    \begin{equation}
        \label{eq:10.12}
        J_{\mathbf{G}}(\mathbf{a}) = 
        \det [\mathbf{G'(a)}] = 
        (D_m g)(\mathbf{a}),
    \end{equation}
    and we see (by Theorem \ref{thm:9.36}) that $\mathbf{G'(a)}$ is 
    \emph{invertible if and only if} $(D_m g)(\mathbf{a}) \neq 0$.
\end{mydef}

\begin{mydef}
    \label{mydef:10.6}
    A linear operator $B$ on $\R^n$ that interchanges some pair of members of the standard basis and leaves the others fixed will be called a \emph{flip}.

    For example, the flip $B$ on $\R^4$ that interchanges $e_2$ and $e_4$ has the form
    \begin{equation}
        \label{eq:10.13}
        B(x_1 \mathbf{e}_1 + 
        x_2 \mathbf{e}_2 + 
        x_3 \mathbf{e}_3 + 
        x_4 \mathbf{e}_4) = 
        x_1 \mathbf{e}_1 + 
        x_2 \mathbf{e}_4 + 
        x_3 \mathbf{e}_3 + 
        x_4 \mathbf{e}_2
    \end{equation}
    or, equivalently,
    \begin{equation}
        \label{eq:10.14}
        B(x_1 \mathbf{e}_1 + 
        x_2 \mathbf{e}_2 + 
        x_3 \mathbf{e}_3 + 
        x_4 \mathbf{e}_4) = 
        x_1 \mathbf{e}_1 + 
        x_4 \mathbf{e}_2 +
        x_3 \mathbf{e}_3 + 
        x_2 \mathbf{e}_4 .
    \end{equation}
    Hence $B$ can also be thought of as interchanging two of the coordinates,
    rather that two basis vectors.

    In the proof that follows, we shall use the projections $P_0,\dots,P_n$ in $\R^n$, defined by $P_0 \mathbf{x = 0}$ and 
    \begin{equation}
        \label{eq:10.15}
        P_m \mathbf{x} = 
        x_1 \mathbf{e}_1 + \dots 
        x_m \mathbf{e}_m 
    \end{equation}
    for $1 \leq m \leq n$.
    Thus $P_m$ is the projection whose range and null space are spanned by $\{\mathbf{e_1,...,e_m}\}$ and $\{\mathbf{e_{m+1},...,e_n}\}$,
    respectively.
\end{mydef}

\begin{thm}
    \label{thm:10.7}
    Suppose $\mathbf{F}$ is a $\mathscr{C}'$-mapping of an open set $E \subset \R^n$ into $\R^n$, $\mathbf{0} \in E$, $\mathbf{F(0) = 0}$, and $\mathbf{F'(0)}$ is invertible.
    
    Then there is a neighborhood of $0$ in $\R^n$ in which a representation
    \begin{equation}
        \label{eq:10.16}
        \mathbf{F(x)} = B_1 \cdots B_{n-1} \mathbf{G_n \circ \cdots \circ G_1(x)}
    \end{equation}
    is valid.

    In (\ref{eq:10.16}), each $\mathbf{G}_i$ is a primitive $\mathscr{C}'$-mapping in some neighborhood of $\mathbf{0}$;
    $\mathbf{G}_i(\mathbf{0})=\mathbf{0}$, $\mathbf{G}'_i(\mathbf{0})$ is invertible, and each $B_i$ is either a flip or the identity operator.
\end{thm}

Briefly, (\ref{eq:10.16}) represents $\mathbf{F}$ locally as a composition of primitive mappings and flips

% todo add proof



% chap10sec03

\section{Partitions of unity}

\begin{thm}
    \label{thm:10.8}
    Suppose $K$ is a compact subset of $\R^n$,
    and $\{V_{\alpha}\}$ is an open cover of $K$.
    Then there exist functions $\psi_1, \dots, \psi_s \in \mathscr{C}(\R^n)$
    such that
    \begin{enumerate}
        \item $0 \leq \psi_i \leq 1$ for $1 \leq i \leq s$;
        \item each $\psi_i$ has its support in some $V_{\alpha}$, and
        \item $\psi_1 (\mathbf{x}) + \cdots + \psi_s(\mathbf{x}) = 1$ for every $\mathbf{x} \in K$.
    \end{enumerate}
\end{thm}

Because of (c), $\{\psi_i\}$ is called a partition of unity,
and (b) is sometimes expressed by saying that $\{\psi_i\}$ is subordinate to the cover $\{V_{\alpha}\}$.

\begin{myCorollary*}
    If $f \in \mathscr{C}(\R^n)$ and the support of $f$ lies in $K$, then
    \begin{equation}
        \label{eq:10.25}
        f = \sum_{i=1}^{s} \psi_i f .
    \end{equation}
    Each $\psi_i f$ has its support in some $V_{\alpha}$.
\end{myCorollary*}

The point of (\ref{eq:10.25}) is that it furnishes a representation of $f$ as a sum of continuous functions $\psi_i f$ with ``small'' supports.

% todo add proof


% chap10sec04
\section{Change of variables}
\mybox{变量代换}

We can now describe the effect of a change of variables on a multiple integral.
For simplicity, we confine ourselves here to continuous functions with compact
support, although this is too restrictive for many applications.
This is illustrated by Exercises \ref{ex:10.9} to \ref{ex:10.13}

\begin{thm}
    \label{thm:10.9}
    Suppose $T$ is a 1-1 $\mathscr{C}'$-mapping of an open set $E \subset \R^k$ into $\R^k$
    such that $J_T(\mathbf{x}) \neq 0$ for all $\mathbf{x} \in E$.
    If $f$ is a continuous function on $\R^k$ whose support is compact and lies in $T(E)$, then
    \begin{equation}
        \label{eq:10.31}
        \int_{\R^k} f(\mathbf{y}) \d \mathbf{y} =
        \int_{\R^k} f(T \mathbf{x}) \left| J_T(\mathbf{x}) \right| \d \mathbf{x} .
    \end{equation}
\end{thm}

% todo add words

% todo add proof


% chap10sec05
\section{Differential forms}

We shall now develop some of the machinery that is needed for the $n$-dimensional version of the fundamental theorem of calculus which is usually called \emph{Stokes' theorem}. 
The original form of Stokes' theorem arose in applications of
vector analysis to electromagnetism and was stated in terms of the curl of a
vector field. 
Green's theorem and the divergence theorem are other special cases. 
These topics are briefly discussed at the end of the chapter.
\mybox{斯托克斯定理}

It is a curious feature of Stokes' theorem that the only thing that is difficult
about it is the elaborate structure of definitions that are needed for its statement.
These definitions concern differential forms, their derivatives, boundaries, and
orientation. Once these concepts are understood, the statement of the theorem
is very brief and succinct, and its proof presents little difficulty.

Up to now we have considered derivatives of functions of several variables
only for functions defined in open sets. This was done to avoid difficulties that
can occur at boundary points. It will now be convenient, however, to discuss
differentiable functions on \emph{compact} sets. We therefore adopt the following
convention:

To say that $\mathbf{f}$ is a $\mathscr{C}'$-mapping (or a $\mathscr{C}''$-mapping) of a compact set
$D \subset \R^k$ into $\R^n$ means that there is a $\mathscr{C}'$-mapping (or a $\mathscr{C}''$-mapping) $\mathbf{g}$ of
an open set $W \subset \R^k$ into $\R^n$ such that $D \subset W$ and such that $\mathbf{g(x) = f(x)}$ for all $\mathbf{x} \in D$.

\begin{mydef}
    \label{mydef:10.10}
    Suppose $E$ is an open set in $\R^n$. 
    A $k$-surface in E is a $\mathscr{C}'$-mapping $\Phi$ from a compact set $D \subset \R^k$ into $E$.
    
    $D$ is called the \emph{parameter domain} of $\Phi$. 
    Points of $D$ will be denoted by $\mathbf{u} = (u_1, \dots , u_k)$.
\end{mydef}

We shall confine ourselves to the simple situation in which $D$ is either a $k$-cell or the $k$-simplex $Q^k$ described in Example 10.4. The reason for this is that we shall have to integrate over $D$, 
and we have not yet discussed integration over more complicated subsets of $\R^k$. 
It will be seen that this restriction on $D$ 
(which will be tacitly made from now on) entails no significant loss of generality in the resulting theory of differential forms.

We stress that $k$-surfaces in $E$ are defined to be \emph{mappings} into $E$, not subsets of $E$. 
This agrees with our earlier definition of curves (Definition \ref{mydef:6.26}).
In fact, $1$-surfaces are precisely the same as continuously differentiable curves.

\begin{mydef}
    \label{mydef:10.11}
    Suppose $E$ is an open set in $\R^n$. 
    A \emph{differential form of order} $k \geq 1$ in $E$ 
    (briefly, a $k$-form in $E$) is a function $\omega$, 
    symbolically represented by the sum 
    \begin{equation}
        \label{eq:10.34}
        \omega - \sum a_{i_1 \cdots i_k} (\mathbf{x}) 
        \d x_{i_1} \wedge \cdots \wedge 
        \d x_{i_k} 
    \end{equation}
    (the indices $i_1 , ... , i_k$ range independently from 1 to $n$), which assigns to each $k$-surface $\Phi$ in $E$ a number $\omega(\Phi) = \int_{\Phi} \omega$, 
    according to the rule
    \begin{equation}
        \label{eq:10.35}
        \int_{\Phi} \omega = 
        \int_{D} \sum a_{i_1 \cdots i_k} (\mathbf{\Phi(u)}) \frac{\partial (x_{i_1},...,x_{i_k})}{\partial (u_{1},...,u_{k})} \d \mathbf{u} ,
    \end{equation}
    where $D$ is the parameter domain of $\Phi$.
    
    The functions $a_{i_1 \dots i_k}$ are assumed to be real and continuous in $E$. 
    If $\phi_1 , ... , \phi_n$ are the components of $\Phi$, the Jacobian in (\ref{eq:10.35}) is the one determined by the mapping
    \begin{equation*}
        (u_1,..,u_k) \rightarrow 
        (\phi_{i_1}(\mathbf{u}),..,\phi_{i_k}(\mathbf{u})) .
    \end{equation*}

    Note that the right side of (\ref{eq:10.35}) is an integral over $D$, as defined in Definition \ref{mydef:10.1} (or Example 10.4) and that (\ref{eq:10.35}) is the \emph{mydefinition} of the symbol $\int_{\Phi} \omega$.
    
    A $k$-form $\omega$ is said to be of class $\mathscr{C}'$ or $\mathscr{C}''$ 
    if the functions $a_{i_1 \cdots i_k}$ in (\ref{eq:10.34}) are all of class $\mathscr{C}'$ or $\mathscr{C}''$.
    
    A 0-form in E is defined to be a continuous function in $E$.
\end{mydef}


\begin{newexample}
    % todo add examples
\end{newexample}

\begin{myRemark}
    \myKeyword{ Elementary properties}
    Let $\omega$, $\omega_1$, $\omega_2$ be $k$-forms in $E$.
    We write $\omega_1 = \omega_2$ 
    if and only if $\omega_1(\Phi) = \omega_2(\Phi)$
    for every $k$-surface $\Phi$ in $E$.
    If $c$ is a real number, 
    then $c\omega$ is the $k$-form defined by 
    \begin{equation}
        \label{eq:10.37}
        \int_{\Phi} c\omega = 
        c\int_{\Phi} \omega ,
    \end{equation}
    and $\omega = \omega_1 + \omega_2$ means that 
    \begin{equation}
        \label{eq:10.38}
        \int_{\Phi} \omega = 
        \int_{\Phi} \omega_1 + 
        \int_{\Phi} \omega_2  
    \end{equation}
    for every $k$-surface $\Phi$ in $E$.
    As a special case of (\ref{eq:10.37}), 
    note that $-\omega$ is defined so that 
    \begin{equation}
        \label{eq:10.39}
        \int_{\Phi} (-\omega) = 
        -\int_{\Phi} \omega ,
    \end{equation}

    Consider a $k$-form 
    \begin{equation}
        \label{eq:10.40}
        \omega = a(\mathbf{x}) 
        \d x_{i_1} 
        \wedge \cdots \wedge 
        \d x_{i_k} 
    \end{equation}
    and let $\overline{\omega}$ be the $k$-form obtained by interchanging some pair of subscripts in (\ref{eq:10.40})
    If (\ref{eq:10.35}) and (\ref{eq:10.39}) are combined with the fact that a determinant changes sign if two of its rows are interchanged,
    we see that 
    \begin{equation}
        \label{eq:10.41}
        \overline{\omega} = -\omega.
    \end{equation}

    As a special case of this, note that the \emph{anticommutative relation}
    \begin{equation}
        \label{eq:10.42}
        \d x_i \wedge \d x_j = 
        -\d x_j \wedge \d x_i 
    \end{equation}
    holds for all $i$ and $j$.
    In particular,
    \begin{equation}
        \label{eq:10.43}
        \d x_i \wedge \d x_i = 0
        \quad (i = 1, \dots, n).
    \end{equation}
    \mybox{外微分$\d x \wedge \d y$ 的性质, 
    与向量外积 $\mathbf{a\times b} = -\mathbf{b\times a}$ 类似}

    More generally, let us return to (\ref{eq:10.40}),
    and assume that $i_r = i_s$ for some $r \neq s$.
    If these two subscripts are interchanged,
    then $\overline{\omega} = \omega$,
    hence $\omega = 0$, by (\ref{eq:10.41}).

    In other words, \emph{if $\omega$ is given by (\ref{eq:10.40}), 
    then $\omega = 0$ unless the subscripts $i_1,\dots,i_k$ are all distinct}.

    If $\omega$ is as in (\ref{eq:10.34}), the summands with repeated subscripts can therefore be omitted without changing $\omega$.
    \mybox{summands means a part of sum 部分和 \url{https://www.dictionary.com/browse/summand}}
    \mybox{Direct Summand  直和

    Given the direct sum of additive Abelian groups $A \oplus B$, 
    $A$ and $B$ are called direct summands. 
    The map $i_1:A \rightarrow A \oplus B$ defined by $i(a)=a \oplus 0$ is called the injection of the first summand, 
    and the map $p_1:A \oplus B \rightarrow A$ defined by $p_1(a \oplus b)=a$ is called the projection onto the first summand. 
    Similar maps $i_2,p_2$ are defined for the second summand $B$. }

    It follows that 0 is the only $k$-form in any open subset of $\R^n$, if $k > n$.

    The anticommutativity expressed by (\ref{eq:10.42}) is the reason for the inordinate amount of attention that has to be paid to minus signs when studying differential forms.    
\end{myRemark}

\begin{mydef}
    \myKeyword{ Basic $k$-forms}
    If $i_1, \dots , i_k$ are integers such that 
    $1 \leq i_1 < i_2 < \cdots < i_k \leq n$, 
    and if $I$ is the ordered $k$-tuple $\{i_1, \dots , i_k\}$, 
    then we call $I$ an increasing $k$-index, 
    and we use the brief notation
    \begin{equation}
        \label{eq:10.44}
        \d x_I = 
        \d x_{i_1} \wedge \cdots \wedge
        \d x_{i_k} .
    \end{equation}
    These forms $\d x_I$ are the so-called \emph{basic $k$-forms in} $\R^n$.

    It is not hard to verify that there are precisely $n!/k!(n-k)!$ basic $k$-forms in $\R^k$;
    we shall make no use of this, however.

    Much more important is the fact that every $k$-form can be represented in terms of basic $k$-forms. 
    To see this, note that every $k$-tuple $\{j_1 , \dots ,j_k\}$ of distinct integers can be converted to an increasing $k$-index $J$ by a finite number of interchanges of pairs; 
    each of these amounts to a multiplication by $-1$, as we saw
    in Sec. 10.13; hence
    \begin{equation}
        \label{eq:10.45}
        \d x_{j_1} \wedge \cdots \wedge
        \d x_{j_k} = 
        \varepsilon (j_1, \dots , j_k) \d x_j
    \end{equation}
    where $\varepsilon(j_1, ... ,j_k)$ is $1$ or $-1$, depending on the number of interchanges that are needed. 
    In fact, it is easy to see that
    \begin{equation}
        \label{eq:10.46}
        \varepsilon (j_1, \dots , j_k) =
        s (j_1, \dots , j_k)
    \end{equation}
    where $s$ is as in Definition \ref{mydef:9.33}.
\end{mydef}

For example,
\begin{equation*}
    \d x_1 \wedge
    \d x_5 \wedge
    \d x_3 \wedge
    \d x_2 =
    - 
    \d x_1 \wedge
    \d x_2 \wedge
    \d x_3 \wedge
    \d x_5 
\end{equation*}
and 
\begin{equation*}
    \d x_4 \wedge
    \d x_2 \wedge
    \d x_3 =
    \d x_2 \wedge
    \d x_3 \wedge
    \d x_4 .
\end{equation*}

If every $k$-tuple in (\ref{eq:10.34}) is converted to an increasing $k$-index, then we
obtain the so-called standard presentation of $\omega$:
\begin{Beqnarray}
    \label{eq:10.47}
    \omega = \sum_{I} b_I (\mathbf{x}) \d x_I .
\end{Beqnarray}

The summation in (\ref{eq:10.47}) extends over all increasing $k$-indices $I$.
[Of course, every increasing $k$-index arises from many (from $k!$, to be precise) $k$-tuples. 
Each $b_I$ in (\ref{eq:10.47}) 
may thus be a sum of several of the coefficients that occur in (\ref{eq:10.34}).]

For example,
\begin{equation*}
    x_1 \d x_2 \wedge \d x_1 - 
    x_2 \d x_3 \wedge \d x_2 + 
    x_3 \d x_2 \wedge \d x_3 + 
        \d x_1 \wedge \d x_2  
\end{equation*}
is a 2-form in $\R^3$ whose standard presentation is 
\begin{equation*}
    (1-x_1) \d x_1 \wedge \d x_2 
    (x_2 + x_3) \d x_2 \wedge \d x_3 .
\end{equation*}

The following uniqueness theorem is one of the main reasons for the
introduction of the standard presentation of a $k$-form.

\begin{thm}
    \label{thm:10.15}
    Suppose 
    \begin{equation}
        \label{eq:10.48}
        \omega = \sum_{I} b_I (\mathbf{x}) \d x_I
    \end{equation}
    is the standard presentation of a $k$-form $\omega$ in an open set $E \subset \R^n$. 
    If $\omega = 0$ in $E$, 
    then $b_I(\mathbf{x}) = 0$ for every increasing $k$-index $I$ and for every $\mathbf{x} \in E$.
\end{thm}

Note that the analogous statement would be false for sums such as (\ref{eq:10.34}),
since, for example,
\begin{equation*}
    \d x_1 \wedge \d x_2 +
    \d x_2 \wedge \d x_1 = 0.
\end{equation*}

% todo add proof

\begin{mydef}
    \myKeyword{ Products of basic $k$-forms}
    Suppose 
    \begin{equation}
        \label{eq:10.51}
        I = \{i_1,...,i_p\},
        \quad 
        J = \{j_1,...,j_q\}
    \end{equation}
    where $1 \leq i_1 < \cdots < i_p \leq n$ 
    and $1 \leq j_1 < \cdots < j_q \leq n$ .
    The \emph{product} of the corresponding basic forms $\d x_I$ and $\d x_J$ in $\R^n$ is a $(p + q)$ form in $\R^n$, denoted by the symbol $\d x_I \wedge \d x_J$, and defined by 
    \begin{equation}
        \label{eq:10.52}
        \d x_I \wedge \d x_J = 
        \d x_{i_1} \wedge \cdots \wedge
        \d x_{i_p} \wedge 
        \d x_{j_1} \wedge \cdots \wedge
        \d x_{j_q}.
    \end{equation}
\end{mydef}

\begin{mydef}
    \myKeyword{ Multiplication}
    Suppose $\omega$ and $\lambda$ are $p$- and $q$-forms, respectively, in
    some open set $E \subset \R^n$, with standard presentations
    \begin{equation}
        \label{eq:10.56}
        \omega = \sum_{I} b_I (\mathbf{x}) \d x_I , \quad 
        \lambda = \sum_{J} c_J (\mathbf{x}) \d x_J
    \end{equation}
    where $I$ abd $J$ range over all increasing $p$-indices and over all increasing $q$-indices
    taken from the set $\{1, ... , n\}$.
\end{mydef}

\begin{mydef}
    \myKeyword{ Differentiation}
    We shall now define a differentiation operator $\d$ which
    associates a $(k + 1)$-form $\d \omega$ to each $k$-form $\omega$ of class $\mathscr{C}'$ in some open set $E \in \R^n$.

    A 0-form of class $\mathscr{C}'$ in $E$ is just a real function $f \in \mathscr{C}'(E)$, and we define
    \begin{equation}
        \label{eq:10.59}
        \d f = \sum_{i=1}^{n} (D_i f) (\mathbf{x}) \d x_i .
    \end{equation}
    If $\omega = \sum b_I (\mathbf{x}) \d x_I$ is the standard presentation of a $k$-form $\omega$, and $B_I \in \mathscr{C}'(E)$ for each increasing $k$-index $I$, then we define
    \begin{equation}
        \label{eq:10.60}
        \d \omega = \sum_I (\d b_I) \wedge \d x_I .
    \end{equation}
\end{mydef}

\begin{newexample}
    Suppose $E$ is open in $\R^n$, $f \in \mathscr{C}'(E)$, and $\gamma$ is a continuously differentiable curve in $E$, with domain $[0,1]$. By (\ref{eq:10.59}) and (\ref{eq:10.35}),
    \begin{equation}
        \label{eq:10.61}
        \int_{\gamma} \d f 
        = \int_{0}^{1} \sum_{i=1}^{n} 
        (D_i f)(\gamma(t)) \gamma'_i (t)
        \d t .
    \end{equation}
    By the chain rule, the last integrand is $(f \circ \gamma)'(t)$.
    Hence
    \begin{equation}
        \label{eq:10.62}
        \int_{\gamma} \d f = 
        f(\gamma(1)) - 
        f(\gamma(0)) ,
    \end{equation}
    and we see that $\int_{\gamma} \d f$ is the same for all $\gamma$ with the same initial point and the same end point, as in (a) of Example 10.12.

    Comparison with Example 10.12(b) shows therefore that the 1-form $x \d y$ is not the derivative of any 0-form $f$.
    This could also be deduced from part (b) of the following theorem, since 
    \begin{equation*}
        \d (x \d y) = \d x \wedge \d y
    \end{equation*}
\end{newexample}
\mybox{integrand 被积函数}

\begin{thm}
    \label{thm:10.20}
    \begin{asparaenum}[(a)]
        \item If $\omega$ and $\lambda$ are $k$- and $m$-forms, respectively, of class $\mathscr{C}'$ in $E$, 
        then 
        \begin{equation}
            \label{eq:10.63}
            \d (\omega \wedge \lambda) =
            (\d \omega) \wedge \lambda +
            (-1)^k \omega \wedge \d \lambda .
        \end{equation}
        \item If $\omega$ is of class $\mathscr{C}''$ in $E$, then $\d ^2 \omega = 0$.
    \end{asparaenum}
\end{thm}

Here $\d ^2 \omega$ means, of course, $\d (\d \omega)$.

% todo add proof

\begin{mydef}
    \myKeyword{Change of variables}
    Suppose $E$ is an open set in $\R^n$, 
    $T$ is a $\mathscr{C}'$-mapping of $E$ into an open set $V \subset \R^m$, 
    and $\omega$ is a $k$-form in $V$, whose standard presentation is
    \begin{equation}
        \label{eq:10.65}
        \omega = \sum_{I} b_I (\mathbf{y}) \d y_I.
    \end{equation}

    (We use $\mathbf{y}$ for points of $V$, $\mathbf{x}$ for points of $E$.)

    Let $t_1,\dots,t_m$ be the components of $T$: If 
    \begin{equation*}
        \mathbf{y} = (y_1, \dots, y_m) = T(\mathbf{x})
    \end{equation*}
    then $y_i = t_i(\mathbf{x})$.
    As in (\ref*{eq:10.59}),
    \begin{equation}
        \label{eq:10.66}
        \d t_i = \sum_{j=1}^{n}(D_j t_i)(\mathbf{x}) \d x_j
        \quad (1 \leq i \leq m).
    \end{equation}
    Thus each $\d t_i$ is a 1-form in $E$.

    The mapping $T$ transforms $\omega$ into a $k$-form $\omega_T$ in $E$,
    whose definition is 
    \begin{equation}
        \label{eq:10.67}
        \omega_T = \sum_I b_I ((T(\mathbf{x}))) 
        \d t_{i_1} \wedge \cdots \wedge
        \d t_{i_k} .
    \end{equation}
    in each summand of (\ref{eq:10.67}),
    $I = \{i_1,...,i_k\}$ is an increasing $k$-index.
\end{mydef}

Our next theorem shows that addition, multiplication, and differentiation
of forms are defined in such a way that they commute with changes of variables.

\begin{thm}
    \label{thm:10.22}
    With $E$ and $T$ as in Sec. 10.21, let $\omega$ and $A$ be $k$- and $m$-forms in $V$, respectively. Then
    \begin{enumerate}[(a)]
        \item $(\omega + \lambda)_T = \omega_T + \lambda_T$ if $k = m$;
        \item $(\omega \wedge \lambda)_T = \omega_T \wedge \lambda_T$;
        \item $\d (\omega_T) = (\d \omega)_T$ 
        if $\omega$ is of class $\mathscr{C}'$ 
        and $T$ is of class $\mathscr{C}''$.
    \end{enumerate}
\end{thm}

% todo add proof

\begin{thm}
    \label{thm:10.23}
    Suppose $T$ is a $\mathscr{C}'$-mapping of an open set $E \subset \R^n$ into an open set $V \subset \R^m$, 
    $S$ is a $\mathscr{C}'$-mapping of $V$ into an open set $W \subset \R^P$, and $w$ is a $k$-form in $W$, 
    so that $\omega$ is a $k$-form in $V$ 
    and both $(\omega_S)_T$ and $\omega_{ST}$ are $k$-forms in $E$, 
    where $ST$ is defined by $(ST)(\mathbf{x}) = S(T(\mathbf{x}))$. 
    Then
    \begin{equation}
        \label{eq:10.71}
        (\omega_S)_T = \omega_{ST} .        
    \end{equation}
\end{thm}

% todo add proof

\begin{thm}
    \label{thm:10.24}
    Suppose $\omega$ is a $k$-form in an open set $E \subset \R^n$, 
    $\Phi$ is a $k$-surface in $E$, with parameter domain $D \subset \R^k$, 
    and $\Delta$ is the $k$-surface in $\R^k$, with parameter domain $D$, defined by $\Delta(\mathbf{u}) = \mathbf{u}(\mathbf{u} \in D)$. Then
    \begin{equation*}
        \int_{\Phi} \omega = 
        \int_{\Delta} \omega_{\Phi} .
    \end{equation*}
\end{thm}

% todo add proof

\begin{thm}
    \label{thm:10.25}
    Suppose $T$ is a $\mathscr{C}'$-mapping of an open set $E \subset \R^n$ into an open set $V \subset \R^m$, 
    $\Phi$ is a $k$-surface in $E$, 
    and $\omega$ is a $k$-form in $V$.

    Then 
    \begin{equation*}
        \int_{T \Phi} \omega = 
        \int_{\Phi} \omega_T
    \end{equation*}
\end{thm}

\begin{proof}
    Let $D$ be the parameter domain of $\Phi$ (hence also of $\Phi$) and define $\Delta$ as in Theorem \ref{thm:10.24}.

    Then 
    \begin{equation*}
        \int_{T \Phi} \omega = 
        \int_{\Delta} \omega_{T \Phi} = 
        \int_{\Delta} (\omega_{T})_{\Phi} = 
        \int_{\Phi} \omega_{T} .
    \end{equation*}

    The first of these equalities is Theorem \ref{thm:10.24}, applied to $T\Phi$ in place of $\Phi$. 
    The second follows from Theorem \ref{thm:10.23}. The third is Theorem \ref{thm:10.24},
    with $\omega_T$ in place of $\omega$.
\end{proof}
% chap10sec06

\section{Simplexes and chains}

\begin{mydef}
    \myKeyword{Affine simplexes}
    A mapping $\mathbf{f}$ that carries a vector space $X$ into a
    vector space $Y$ is said to be \emph{affine} if $\mathbf{f - f(0)}$ is linear. 
    In other words, the requirement is that
    \begin{equation}
        \label{eq:10.73}
        \mathbf{f(x)} = 
        \mathbf{f(0)} + A \mathbf{x}
    \end{equation}
    for some $A \in L(X, Y)$.

    An affine mapping of $\R^k$ into $\R^n$ is thus determined if we know $\mathbf{f}(0)$ and $\mathbf{f(e_i)}$ for $1 \leq i \leq k$; 
    as usual, $\{\mathbf{e}_1, ... , \mathbf{e}_k\}$ is the standard basis of $\R^k$.
    
    We define the \emph{standard simplex} $Q^k$ to be the set of all $\mathbf{u} \in \R^k$ of the form
    \begin{equation}
        \label{eq;10.74}
        \mathbf{u} = \sum_{i=1}^{k} \alpha_i \mathbf{e}_i
    \end{equation}
    such that $\alpha \geq 0$ for $i = 1, ... , k$ and $\sum \alpha_i \leq 1$.
    % todo
\end{mydef}

\begin{thm}
    \label{thm:10.27}
    If $\delta$ is an oriented rectilinear $k$-simplex in an open set $E \subset \R^n$ 
    and if $\overline{\Delta} = \varepsilon \delta$ then
    \begin{equation}
        \label{eq:10.81}
        \int_{\overline{\delta}} \omega = 
        \varepsilon \int_{\delta} \omega 
    \end{equation}
    for every $k$-form $\omega$ in $E$.
\end{thm}

% todo add proof

\begin{mydef}
    \myKeyword{Affine chains}
    An \emph{affine $k$-chain} $\Gamma$ in an open set $E \subset \R^n$ is a collection of finitely many oriented affine $k$-simplexes $\delta_1, \dots, \delta_r$ in $E$. 
    These need not be distinct; 
    a simplex may thus occur in $\Gamma$ with a certain multiplicity.

    If $\Gamma$ is as above, and if $\omega$ is a $k$-form in $E$,
    we define 
    \begin{equation}
        \label{eq:10.82}
        \int_{\Gamma} \omega =
        \sum_{i=1}^{r} \int_{\sigma_i} \omega .
    \end{equation}
    % todo add word
\end{mydef}

\begin{mydef}
    \myKeyword{Boundaires}
    For $k \geq 1$, the \emph{boundary} of the oriented affine $k$-simplex
    \begin{equation*}
        \delta = \left[ \mathbf{p_0,p_1,\dots,p_k} \right]
    \end{equation*}
    is defined to be the affine $(k - 1)$-chain
    \begin{equation}
        \label{eq:10.85}
        \partial \delta = \sum_{j=0}^{k} (-1)^j
        \left[ \mathbf{p_0,p_1,\dots,p_{j-1},p_{j+1},\dots,p_k} \right]
    \end{equation}
\end{mydef}

% todo add word

\begin{mydef}
    \myKeyword{Differentiable simplexes and chains}
\end{mydef}

\begin{mydef}
    \myKeyword{Positively oriented boundaries}
\end{mydef}

\begin{myExample}
    For $0 \leq u \leq \pi, 0 \leq v \leq 2\pi$, define
    \begin{equation*}
        \sum(u,v) = \left( 
            \sin u \cos v,
            \sin u \sin v,
            \cos u
         \right).
    \end{equation*}
    % todo
\end{myExample}

% chap10sec07

\section{Stokes' theorem}

\begin{thm}
    \label{thm:10.33}
    If $\Psi$ is a $k$-chain of class $\mathscr{C}''$ in an open set $V \subset \R^m$ and if $\omega$ is a $(k-1)$-form of class $\mathscr{C}'$ in $V$, then 
    \begin{equation}
        \label{eq:10.91}
        \int_{\Psi} \d \omega = 
        \int_{\partial \Psi} \omega .
    \end{equation}
\end{thm}
% chap10sec08

\section{Closed forms and exact forms}

\begin{mydef}
    Let $\omega$ be a $k$-form in an open set $E \subset \R^n$. 
    If there is a $(k - 1)$- form $\lambda$ in $E$ such that $\omega = \d \lambda$, then $\omega$ is said to be exact in $E$.
    
    If $\omega$ is of class $\mathscr{C}'$ and $\d \omega = 0$, 
    then $\omega$ is said to be closed.
    
    Theorem \ref{thm:10.20}(b) shows that every exact form of class $\mathscr{C}'$ is closed.

    In certain sets $E$, for example in convex ones, the converse is true; 
    this is the content of Theorem \ref{thm:10.39} 
    (usually known as Poincare's lemma) and Theorem \ref{thm:10.40}. 
    However, Examples \ref{newexample:10.36} and \ref{newexample:10.37} will exhibit closed forms that are not exact.
\end{mydef}

\begin{myRemark}
    
\end{myRemark}

\begin{newexample}
    \label{newexample:10.36}
\end{newexample}

\begin{newexample}
    \label{newexample:10.37}
\end{newexample}

\begin{thm}
    \label{thm:10.38}
    Suppose $E$ is a convex open set in $\R^n$,
    $f \in \mathscr{C}'(E)$, $p$ is an integer, 
    $1 \leq p \leq n$, and
    \begin{equation}
        \label{eq:10.116}
        (D_j f)(\mathbf{x}) = 0
        \quad 
        (p < j \leq n, \mathbf{x} \in E).
    \end{equation}
    Then there exists an $F \in \mathscr{C}'(E)$ such that
\end{thm}

\begin{thm}
    \label{thm:10.39}
    If $E \in \R^n$ is convex and open, 
    if $k \geq 1$, 
    if $\omega$ is a $k$-form of class $\mathscr{C}'$ in $E$, a
    nd if $\d \omega = 0$, 
    then there is a $(k - 1)$-form $\lambda$ in $E$ 
    such that $\omega = \d \lambda$.
\end{thm}

\begin{thm}
    \label{thm:10.40}
    Fix $k$, $1 \leq k \leq n$. 
    Let $E \subset \R^n$ be an open set in which every closed $k$-form is exact. 
    Let $T$ be a 1-1 $\mathscr{C}''$-mapping of $E$ onto an open set $U \subset \R^n$ 
    whose inverse $S$ is also of class $\mathscr{C}''$.
    
    Then every closed $k$-form in $V$ is exact in $V$.
\end{thm}

Note that every convex open set $E$ satisfies the present hypothesis, by
Theorem \ref{thm:10.39}. 
The relation between $E$ and $V$ may be expressed by saying
that they are $\mathscr{C}''$-\emph{equivalent}.

\emph{Thus every closed form is exact in any set which is $\mathscr{C}''$-equivalent to a convex open set.}

\begin{myRemark}
    
\end{myRemark}
% chap10sec09

\section{Vector analysis}

We conclude this chapter with a few applications of the preceding material to theorems concerning vector analysis in $\R^3$. 
These are special cases of theorems about differential forms, but are usually stated in different terminology. 
We are thus faced with the job of translating from one language to another.

\begin{myDef}
    \myKeyword{Vector fileds}

\end{myDef}

\begin{thm}
    \label{thm:10.43}
    Suppose $E$ is an open set in $\R^3$, $u \in \mathscr{C}''(E)$, and $\mathbf{G}$ is a vector field in $E$, of class $\mathscr{C}''$.
    \begin{enumerate}[(a)]
        \item If $\mathbf{F} = \nabla u$, then $\nabla \times \mathbf{F} = \mathbf{0}$.
        \item If $\mathbf{F} = \nabla \times \mathbf{G}$, then $\nabla \cdot \mathbf{F} = 0$.
    \end{enumerate}

    Furthermore, if $E$ is $\mathscr{C}''$-equivalent to a convex set, 
    then (a) and (b) have converses, 
    in which we assume that $\mathbf{F}$ is a vector field in $E$, of class $\mathscr{C}'$:
    \begin{enumerate}[(a')]
        \item If $\nabla \cdot \mathbf{F} = \mathbf{0}$, then $\mathbf{F} = \nabla u$ for some $u \in \mathscr{C}''(E)$.
        \item If $\nabla \times \mathbf{F} = 0$, then $\mathbf{F} = \nabla \times \mathbf{G}$ for some vector field $\mathscr{G}$. in $E$, of class $\mathscr{C}''$
    \end{enumerate}
\end{thm}

\begin{myDef}
    \myKeyword{Volume elements}

\end{myDef}

\begin{myDef}
    \myKeyword{Green's theorem}
    
\end{myDef}


\begin{myDef}
    \myKeyword{Area elements in $\R^3$}
    
\end{myDef}

\begin{myExample}
    \label{myExample:10.47}
\end{myExample}

\begin{myDef}
    \myKeyword{Integrals of 1-forms in $\R^3$}
    
\end{myDef}

\begin{myDef}
    \myKeyword{Integrals of 2-forms in $\R^3$}
    
\end{myDef}

\begin{myDef}
    \myKeyword{Stokes' formula}
    
    \begin{equation}
        \label{eq:10.145}
        \int_{\Phi} \left( \nabla \times \mathbf{F} \right) \cdot \mathbf{n} \d A = 
        \int_{\partial \Phi} \left( \mathbf{F \cdot t} \right)  \d s
    \end{equation}
\end{myDef}

\begin{myDef}
    \myKeyword{The divergence theorem}
    
\end{myDef}


