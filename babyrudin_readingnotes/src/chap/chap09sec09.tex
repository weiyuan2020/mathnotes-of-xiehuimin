% chap09sec09

\section{Differentiation of integrals}

Suppose $\phi$ is a function of two variables which can be integrated with respect to one and which can be differentiated with respect to the other. 
Under what conditions will the result be the same if these two limit processes are carried out in the opposite order? 
To state the question more precisely: Under what
conditions on $\phi$ can one prove that the equation
\begin{equation}
    \label{eq:9.98}
    \frac{\d}{\d t} \int_{a}^{b} \phi (x,t) \d x = 
    \int_{a}^{b} \frac{\partial \phi}{\partial t}(x,t) \d x
\end{equation}
is true?
(A counter example is furnished by Exercise 28.)

It will be convenient to use the notation
\begin{equation}
    \label{eq:9.99}
    \phi^t(x) = \phi (x, t).
\end{equation}
Thus $\phi^t$ is, for each $t$, a function of one variable.

\begin{thm}
    \label{thm:9.42}
    Suppose 
    \begin{asparaenum}[(a)]
        \item $\phi(x,t)$ is defined for $a \leq x \leq b,x \leq t \leq d$;
        \item $\alpha$ is an increasing function on $[a,b]$;
        \item $\phi' \in \mathscr{R}(\alpha)$ for every $t \in [c,d]$;
        \item $c<s<d$, and to every $\varepsilon > 0$ corresponds a $\delta > 0$ such that 
        \begin{equation*}
            \left| (D_2 \phi)(x,t) - (D_2 \phi)(x,s) \right| < \varepsilon
        \end{equation*}
        for all $x \in [a,b]$ and for all $t \in (s-\delta, s+\delta)$.
    \end{asparaenum}
    Define 
    \begin{equation}
        \label{eq:9.100}
        f(t) = \int_{a}^{b} \phi(x,t) \d \alpha(x)
        \quad 
        (c \leq t \leq d) .
    \end{equation}
    Then $(D_2 \phi)^s \in \mathscr{R}(\alpha)$. $f'(s)$ exists, and 
    \begin{equation}
        \label{eq:9.101}
        f'(s) = \int_{a}^{b} (D_2 \phi)(x,s) \d \alpha(x).
    \end{equation}
\end{thm}

Note that (c) simply asserts the existence of the integrals (\ref{eq:9.100}) for all $t \in [c, d]$. 
Note also that (d) certainly holds whenever $D_2 \phi$ is continuous on the
rectangle on which $\phi$ is defined.

% todo add proof

\begin{myExample}
    One can of course prove analogues of Theorem \ref{thm:9.42} with
    $(- \infty , \infty )$ in place of $[a, b]$. Instead of doing this, let us simply look at an example. 
    % todo add example
\end{myExample}

