% chap05sec06
\section{Taylor's theorem}

\begin{thm}
    \label{thm:5.15}
    Suppose $f$ is a real function on $[a, b]$, 
    $n$ is a positive integer,
    $f^{(n-1)}$ is continuous on $[a, b]$,
    $f^{(n)}(t)$ exists for every $t \in (a, b)$. 
    Let $\alpha, \beta$ be distinct points of $[a, b]$, 
    and define
    \begin{equation}
        \label{eq:5.23}
        P(t) = \sum_{k=0}^{n-1}\frac{f^{(k)}}{}
    \end{equation}
    Then there exists a point $x$ between $\alpha$ and $\beta$ such that
    \begin{equation}
        \label{eq:5.24}
        f(\beta) = 
        P(\beta)
        + \frac{f^{(n)}(x)}{n!}\left( \beta - \alpha \right)^n.
    \end{equation}
\end{thm}

For $n = 1$, this is just the mean value theorem. 
In general, the theorem shows that $f$ can be approximated by a polynomial of degree $n - 1$, 
and that (\ref{eq:5.24}) allows us to estimate the error, 
if we know bounds on $\left| f^{(n)}(X) \right| $.
\begin{proof}
    Let $M$ be the number defined by
    \begin{equation}
        \label{eq:5.25}
        f(\beta) = P(\beta) + M(\beta - \alpha)^n
    \end{equation}
    and put
    \begin{equation}
        \label{eq:5.26}
        g(t) =f(t) -P(t) - M(t -\alpha)^n
        \quad (a \leq t \leq b).
    \end{equation}
    We have to show that $n!M = f^{(n)}(x)$ 
    for some $x$ between $\alpha$ and $\beta$. 
    By (\ref{eq:5.23}) and (\ref{eq:5.26}),
    \begin{equation}
        \label{eq:5.27}
        g^{(n)}(t) = 
        f^{(n)}(t) - n! M
        \quad (a < t < b).
    \end{equation}
    Hence the proof will be complete if we can show that $g^{(n)}(x) = 0$ for some $x$ between $\alpha$ and $\beta$.
    
    Since $p^{(k)}(\alpha) =f[^{(k)}(\alpha)$ for $k= 0, ... , n - 1$, 
    we have
    \begin{equation}
        \label{eq:5.28}
        g(\alpha) = g'(\alpha) = \cdots = g^{(n-1)}(\alpha) = 0.
    \end{equation}
    Our choice of $M$ shows that $g(\beta) = 0$, 
    so that $g'(x_1) = 0$ for some $x_1$ between $\alpha$ and $\beta$, 
    by the mean value theorem. 
    Since $g'(\alpha) = 0$, we conclude similarly that 
    $g''(x_2) = 0$ for some $x_2$ between $\alpha$ and $x_1$. 
    After $n$ steps we arrive at the conclusion that 
    $g^{(n)}(x_n) = 0$ for some $x_n$ between $\alpha$ and $x_{n-1}$, 
    that is, between $\alpha$ and $\beta$.
\end{proof}