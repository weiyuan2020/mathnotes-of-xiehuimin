% chap03sec07
\section{Series of nonnegative terms}

\begin{thm}
    \label{thm:3.26}
    If $0 \leq x < 1$, then
    \begin{equation*}
        \sum_{n=0}^{\infty} x^n = \frac{1}{1-x}.
    \end{equation*}
    If $x \geq 1$, the series diverges. 
\end{thm}

\mybox{几何级数收敛条件}


\mybox{Cauchy use ``thin'' subsequence of $\sequence{a_n}$ determines the convergence or divergence of $\sum a_n$}

\begin{thm}
    \label{thm:3.27}
    Suppose $a_1 \geq a_2 \geq a_3 \geq \dots \geq 0$.
    Then the series $\sum_{n=1}^{\infty}a_n$ converges if and only if the series
    \begin{equation}
        \sum_{k=0}^{\infty} 2^k a_{2^k}
        = a_1 + 2 a_2 + 4 a_4 + 8 a_8 + \dots
    \end{equation}
    converges.
\end{thm}

\begin{proof}
    By Theorem \ref{thm:3.24}, it suffices to consider boundedness of the partial sums.
    Let
    \begin{align*}
        s_n &= a_1 + a_2 + \dots + a_n, \\
        t_n &= a_1 + 2 a_2 + \dots + 2^{k} a_{2^k}.
    \end{align*}
    For $n < 2^k$,
    \begin{align*}
        s_n 
        &\leq a_1 + (a_2 + a_3) + \dots + (a_{2^k}+\dots+a_{2^{k+1}-1}) \\
        &\leq a_1 + 2a_2 + \dots + 2^k a_{2^k} \\
        &= t_k,
    \end{align*}
    so that
    \begin{equation}
        \label{eq:3.8}
        s_n \leq t_k.
    \end{equation}
    On the other hand, if $n > 2^k$,
    \begin{align*}
        s_n 
        &\geq a_1 + a_2 + (a_3 + a_4) + \dots + (a_{2^{k-1}+1}+\dots+a_{2^{k}}) \\
        &\geq \frac{1}{2}a_1 + a_2 + 2 a_4 + \dots + 2^{k+1} a_{2^k} \\
        &= \frac{1}{2}t_k,
    \end{align*}
    so that
    \begin{equation}
        \label{eq:3.9}
        2 s_n \geq t_k.
    \end{equation}
    $\sequence{s_n}$,
    $\sequence{t_n}$ are both bounded or both unbounded.
\end{proof}

\begin{thm}
    \label{thm:3.28}
    $\sum \frac{1}{n^p}$ converges if $p>1$ and diverges if $p\leq 1$. 
\end{thm}

\begin{thm}
    \label{thm:3.29}
    If $p > 1$ ,
    \begin{equation}
        \label{eq:3.10}
        \sum_{n=2}^{\infty} \frac{1}{n (\log n)^p}
    \end{equation}
    converges; if $p \leq 1$ , the series diverges.
\end{thm}

``$\log n$'' the logarithm of $n$ to the base $e$ (compare Exercise 7, Chap. 1);
the number $e$ will be defined in a moment (see Def 3.30). We let the series start with $n=2$ , since $\log 1 = 0$ .

\begin{proof}
    The monotonicity of the logarithmic function (which will be discussed in more detail in Chap. 8) implies that ($\log n$) increase. Hence ($1/n \log n$) decreases, and we can apply Theorem \ref{thm:3.27} to (\ref{eq:3.10}); this leads us the series
    \begin{equation}
        \label{eq:3.11}
        \sum_{k=1}^{\infty}2^k\cdot\frac{1}{2^k (\log 2^k)^p} = 
        \sum_{k=1}^{\infty}\cdot\frac{1}{(k\log 2)^p} =
        \frac{1}{(\log 2)^p}\sum_{k=1}^{\infty}\cdot\frac{1}{k^p}
    \end{equation}
    and Theorem \ref{thm:3.29} follows from Theorem \ref{thm:3.28}.
\end{proof}


This procedure may evidently be continued. For instance,
\begin{equation}
    \label{eq:3.12}
    \sum_{n=3}^{\infty}\frac{1}{n \log n \log \log n}
\end{equation}
diverges, whereas
\begin{equation}
    \label{eq:3.13}
    \sum_{n=3}^{\infty}\frac{1}{n \log n (\log \log n)^2}
\end{equation}
converges.

Series (\ref{eq:3.12}) differ very littel from (\ref{eq:3.13}). Still, one diverges, the other converges.
If we continue the process which led us from Theorem \ref{thm:3.28} to Theorem \ref{thm:3.29}, we get pairs of convergent and divergent series whose terms differ even less than those of (\ref{eq:3.12}) and (\ref{eq:3.13}).
One might thus be led to the conjecture that there is a limiting situation of some sort, a ``boundary'' with all convergent series on one side, all divergent series on the other side --- at least as far as series with monotonic coefficients are converned. 
This notion of ``boundary'' is of course quite vague.
The point we wish to make is this: No matter how we make this notion precise, the conjecture is false. Exercises 11(b) and 12(b) may serve as illustrations.

More deeper aspect of convergence theory can refer to Knopp's \emph{``Theory and Application of Infinite Series''}, Chap IX, particularly Sec. 41.
