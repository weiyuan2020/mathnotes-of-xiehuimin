% chap04sec06
\section{Monotonic functions}
We shall now study those functions which never decrease 
(or never increase) on a given segment.

\mymyDef{
    \label{myDef:4.28}
    Let $f$ be real on $(a, b)$. 
    Then $f$ is said to be \emph{monotonically increasing} on $(a, b)$ 
    if $a< x < y < b$ implies $f(x) \leq f(y)$. 
    If the last inequality is reversed, 
    we obtain the definition of a \emph{monotonically decreasing} function. 
    The class of monotonic functions consists of both the increasing and the decreasing functions.
}

\mythm{
    \label{thm:4.29}
    Let f be monotonically increasing on $(a, b)$. 
    Then $f(x+)$ and $f(x-)$ exist at every point of $x$ of $(a, b)$. 
    More precisely,
    \begin{equation}
        \label{eq:4.25}
        \sup_{a < t < x} f(t) = f(x-) 
        \leq f(x) \leq 
        f(x+) = \inf_{x < t < b} f(t).
    \end{equation}
    Furthermore, if $a < x < y < b$, then
    \begin{equation}
        \label{eq:4.26}
        f(x+) \leq f(y-).
    \end{equation}
}
Analogous results evidently hold for monotonically decreasing functions.

\myproof{
    By hypothesis, the set of numbers $f(t)$, where $a< t < x$, 
    is bounded above by the number $f(x)$, 
    and therefore has a least upper bound which we shall denote by $A$. 
    Evidently $A \leq f(x)$. We have to show that $A =f(x-)$. 
    
    Let $\varepsilon > 0$ be given. 
    It follows from the definition of $A$ 
    as a least upper bound that there exists $\delta > 0$ 
    such that $a < x - \delta < x$ and
    \begin{equation}
        \label{eq:4.27}
        A - \varepsilon < f(x - \delta) \leq A.
    \end{equation}

    Since $f$ is monotonic, we have
    \begin{equation}
        \label{eq:4.28}
        f(x-\delta) \leq f(t) \leq A
        \quad 
        (x-\delta < t < x).
    \end{equation}

    Combining (\ref{eq4.27}) and (\ref{eq:4.28}), we see that
    \begin{equation*}
        \left| f(t) - A \right| < \varepsilon
        \quad
        (x - \delta < t < x).
    \end{equation*}
    Hence $f(x-) = A$.

    The second half of (\ref{eq:4.25}) is proved in precisely the same way.

    Next, if $a < x < y < b$, we see from (\ref{eq:4.25}) that
    \begin{equation}
        \label{eq:4.29}
        f(x+) 
        = \inf_{x < t < b} f(t)
        = \inf_{x < t < y} f(t)
    \end{equation}
    The last equality is obtained by applying (\ref{eq:4.25}) to $(a, y)$ in place of $(a, b)$. Similarly,
    \begin{equation}
        \label{eq:4.30}
        f(y-) 
        = \sup_{a < t < y} f(t)
        = \sup_{x < t < y} f(t)
    \end{equation}
    Comparison of (\ref{eq:4.29}) and (\ref{eq:4.30}) gives (\ref{eq:4.26}).
}

\textbf{Corollary}: Monotonic functions have no discontinuities of the second kind.


This corollary implies that every monotonic function is discontinuous at a countable set of points at most. 
Instead of appealing to the general theorem whose proof is sketched in Exercise 17, 
we give here a simple proof which is applicable to monotonic functions.

\mythm{
    \label{thm:4.30}
    Let $f$ be monotonic on $(a, b)$. 
    Then the set of points of $(a, b)$ 
    at which $f$ is discontinuous 
    is at most countable.
}

\mymyRemark{
    It should be noted that the discontinuities of a monotonic function need not be isolated. 
    In fact, given any countable subset $E$ of $(a, b)$, 
    which may even be dense, 
    we can construct a function $f$, monotonic on $(a, b)$, 
    discontinuous at every point of $E$, and at no other point of $(a, b)$.

    To show this, let the points of $E$ be arranged in a sequence $\sequence{x_n}$, $n = 1, 2, 3,...$. 
    Let $\sequence{c_n}$ be a sequence of positive numbers such that $\sum c_n$ converges. 
    Define
    \begin{equation}
        \label{eq:4.31}
        f(x) = \sum_{x_n < x} c_n
        \quad 
        (a < x < b).
    \end{equation}
    The summation is to be understood as follows: 
    Sum over those indices $n$ for which $x_n < x$. 
    If there are no points $x_n$ to the left of $x$, the sum is empty; 
    following the usual convention, we define it to be zero. 
    Since (\ref{eq:4.31}) converges absolutely, 
    the order in which the terms are arranged is immaterial.
}

\begin{asparaenum}[(a)]
    \item $f$ is monotonically increasing on $(a, b)$;
    \item $f$ is discontinuous at every point of $E$; in fact,
    \begin{equation*}
        f(x_n+) - f(x_n-) = c_n
    \end{equation*}
    \item $f$ is continuous at every other point of $(a, b)$.
\end{asparaenum}

Moreover, it is not hard to see that $f(x-) =f(x)$ at all points of $(a, b)$. 
If a function satisfies this condition, 
we say that $f$ is \emph{continuous from the left}. 
If the summation in (\ref{eq:4.31}) were taken over all indices $n$ for which $x_n \leq x$, 
we would have $f(x+) = f(x)$ at every point of $(a, b)$; 
that is, $f$ would be \emph{continuous from the right}.

Functions of this sort can also be defined by another method; 
for an example we refer to Theorem 6.16.