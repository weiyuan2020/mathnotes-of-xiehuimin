% chap06sec05
\section{Rectifiable curves}
\mybox{可度量曲线

rectifiable 可求长的.
}

We conclude this chapter with a topic of geometric interest 
which provides an application of some of the preceding theory. 
The case $k = 2$ (i.e., the case of plane curves) 
is of considerable importance in the study of analytic functions
of a complex variable.

\begin{mydef}
    \label{mydef:6.26}
    A continuous mapping $\gamma$ of an interval $[a, b]$ into $\R^k$ is called a \emph{curve} in $\R^k$.
    To emphasize the parameter interval $[a, b]$,
    we may also say that $\gamma$ is a curve on $[a, b]$.

    If $\gamma$ is one-to-one, $\gamma$ is called an \emph{arc}.

    If $\gamma(a) = \gamma(b)$, $\gamma$ is said to be a \emph{closed curve}.
    
    It should be noted that we define a curve to be a mapping, not a point set.
    Of course, with each curve $\gamma$ in $\R^k$ there is associated a subset of $\R^k$, namely the range of $\gamma$, 
    but different curves may have the same range.

    We associate to each partition $P = \{x_0,...,x_n\}$ of $[a,b]$ and to each curve $\gamma$ on $[a,b]$ the number
    \begin{equation*}
        \Lambda (P, \gamma) = 
        \sum_{i=1}^{n} \left| \gamma(x_i) - \gamma(x_{i-1}) \right| .
    \end{equation*}
    The $i$th term in this sum is the distance (in $\R^k$) between the points $\gamma(x_{i-1})$ and $\gamma(x_i)$, 
    Hence $\Lambda(P, \gamma)$ is the length of a polygonal path with vertices at $\gamma(x_0), \gamma(x_1), ..., \gamma(x_n)$, in this order. 
    As our partition becomes finer and finer, this polygon approaches the range of $\gamma$ more and more closely. 
    This makes it seem reasonable to define the length of $\gamma$ as
    \begin{equation*}
        \Lambda(\gamma) = \sup \Lambda(P, \gamma),
    \end{equation*}
    where the supremum is taken over all partitions of $[a, b]$.

    If $\Lambda(\gamma) < \infty$, we say that $\gamma$ is \emph{rectifiable}.

    In certain cases, $\Lambda(\gamma)$ is given by a Riemann integral. 
    We shall prove this for \emph{continuously differentiable} curves, 
    i.e., for curves $\gamma$ whose derivative $\gamma'$ is continuous.
\end{mydef}

\begin{thm}
    \label{thm:6.27}
    If $\gamma'$ is continuous on $[a, b]$, then $\gamma$ is rectifiable,
    and 
    \begin{equation*}
        \Lambda(\gamma) = \int_{a}^{b} \left| \gamma'(t) \right| \d t.
    \end{equation*}
\end{thm}

% todo add proof


