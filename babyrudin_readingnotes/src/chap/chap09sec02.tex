% chap09sec02
\section{Differentiation}
\begin{myDef}
    \textbf{Preliminaries}
    In order to arrive at a definition of the derivative of a function whose domain is $\R^n$ (or an open subset of $\R^n$), 
    let us take another look at the familiar case $n = 1$, and let us see how to interpret the derivative in that case in a way which will naturally extend to $n > 1$. 

    If $f$ is a {\color{blue} real function} with domain $(a, b) \subset \R^1$ and 
    if $x \in (a, b)$, then $f'(x)$ is usually defined to be the real number
    \begin{equation}
        \label{eq:9.7}
        \lim_{h \to 0} \frac{f(x+h) - f(x)}{h} ,
    \end{equation}
    provided, of course, that this limit exists. 
    Thus
    \begin{equation}
        \label{eq:9.8}
        f(x+h) - f(x) = f'(x)h + r(h)
    \end{equation}
    where the ``remainder'' $r(h)$ is small, in the sense that 
    \begin{equation}
        \label{eq:9.9}
        \lim_{h \to 0} \frac{r(h)}{h} = 0.
    \end{equation}
    
    Note that (\ref{eq:9.8}) expresses the difference $f(x + h) - f(x)$ as the sum of the \emph{linear function} that takes $h$ to $f'(x)h$, plus a small remainder.

    We can therefore regard the derivative of $f$ at $x$, not as a real number, but as the linear operator on $\R^1$ that takes $h$ to $f'(x)h$.
    
    [Observe that every real number $\alpha$ gives rise to a linear operator on $\R^1$; 
    the operator in question is simply multiplication by $\alpha$. 
    Conversely, every linear function that carries $\R^1$ to $\R^1$ is multiplication by some real number. 
    It is this natural 1-1 correspondence between $\R^1$ and L($\R^1$) which motivates the preceding statements.]
    
    Let us next consider a function $\mathbf{f}$ that maps $(a, b) \subset \R^1$ into $\R^m$. In that case, $f'(x)$ was defined to be that vector $\mathbf{y} \in \R^m$ (if there is one) for which
    \begin{equation}
        \label{eq:9.10}
        \lim_{h \to 0} \left\{ \frac{\mathbf{f}(x+h) - \mathbf{f}(x)}{h} - \mathbf{y} \right\} = 0.
    \end{equation}
    We can again rewrite this in the form
    \begin{equation}
        \label{eq:9.11}
        \mathbf{f}(x+h) - \mathbf{f}(x) = h \mathbf{y} + \mathbf{r}(h) ,
    \end{equation}
    where $\mathbf{r}(h)/h \rightarrow \mathbf{0}$ as $h \rightarrow 0$.
    The main term on the right side of (\ref{eq:9.11}) is again a \emph{linear} function of $h$.
    Every $\mathbf{y} \in \R^m$ induces a linear transformation of $\R^1$ into $\R^m$, by associating to each $h \in \R^1$ the vector $h \mathbf{y} \in \R^m$.
    This identification of $\R^m$ with $L(\R^1, \R^m)$ allows us to regard $\mathbf{f}'(x)$ as a member of $L(\R^1,\R^m)$

    Thus, if $\mathbf{f}$ is a differentiable mapping of $(a,b) \subset \R^1$ into $\R^m$, and if $x \in (a,b)$,
    then $\mathbf{f}'(x)$ is the linear transformation of $\R^1$ into $\R^m$ that satisfies
    \begin{equation}
        \label{eq:9.12}
        \lim_{h \to 0} \frac{\mathbf{f}(x+h) - \mathbf{f}(x) - \mathbf{f}'(x)h}{h} = \mathbf{0},
    \end{equation}
    or, equivalently,
    \begin{equation}
        \label{eq:9.13}
        \lim_{h \to 0} \frac{\left| \mathbf{f}(x+h) - \mathbf{f}(x) - \mathbf{f}'(x)h \right|}{\left| h \right|} = 0,
    \end{equation}

    We are now ready for the case $n > 1$.
\end{myDef}
\mybox{研究顺序

    $\R^1 \rightarrow \R^1 \Rightarrow $ 
    $\R^1 \rightarrow \R^m \Rightarrow $ 
    $\R^n \rightarrow \R^m$
    }

\begin{myDef}
    \label{myDef:9.11}
    Suppose $E$ is an open set in $\R^n$, $f$ maps $E$ into $\R^m$, and $x \in E$.
    If there exists a linear transformation $A$ of $\R^n$ into $\R^m$ such that
    \begin{equation}
        \label{eq:9.14}
        \lim_{\mathbf{h} \to \mathbf{0}} \frac{\left| \mathbf{f}\mathbf{(x+h)} - \mathbf{f}\mathbf{(x)} - A\mathbf{(x)}h \right|}{\left| \mathbf{h} \right|} = \mathbf{0},
    \end{equation}
    then we say that $\mathbf{f}$ is \emph{differentiable at} $\mathbf{x}$, and we write
    \begin{equation}
        \label{eq:9.15}
        \mathbf{f}'(\mathbf{x}) = A.
    \end{equation}
    If $\mathbf{f}$ is differentiable at every $x \in E$, 
    we say that $\mathbf{f}$ is \emph{differentiable in} $E$.
\end{myDef}
\mybox{
    \emph{differentiable at} $\mathbf{x}$, $\mathbf{f}$ 在 $\mathbf{x}$ 可微分 \\
    \emph{differentiable in} $E$, $\mathbf{f}$ 在 $E$ 中可微分 
}

It is of course understood in (\ref{eq:9.14}) that $\mathbf{h} \in \R^n$. 
If $\left| \mathbf{h} \right|$ is small enough, 
then $\mathbf{x+h} \in E$, since $E$ is open. 
Thus $\mathbf{f(x + h)}$ is defined, $\mathbf{f(x + h)} \in \R^m$, 
and since $A \in L(\R^n, \R^m)$, $A \mathbf{h} \in \R^m$. 
Thus
\begin{equation*}
    \mathbf{f(x + h)} - \mathbf{f(x)} - A{h} \in \R^m.
\end{equation*}
The norm in the numerator of (\ref{eq:9.14}) is that of $\R^m$. 
In the denominator we have the $\R^n$-norm of $\mathbf{h}$.

There is an obvious uniqueness problem which has to be settled before we go any further.

\begin{thm}
    \label{thm:9.12}
    Suppose $E$ and $\mathbf{f}$ are as in Definition \ref{myDef:9.11}, 
    $\mathbf{x} \in E$, and (\ref{eq:9.14}) holds with $A = A_1$ and with $A = A_2$. 
    Then $A_1 = A_2$.
\end{thm}

% todo add proof

\begin{myRemark}
    \begin{asparaenum}[(a)]
        \item The relation (\ref{eq:9.14}) can be rewritten in the form 
        \begin{equation}
            \label{eq:9.17}
            \mathbf{f(x + h)} - \mathbf{f(x)} = 
            \mathbf{f('x)h} + \mathbf{r(h)}
        \end{equation}
        where the remainder $\mathbf{r(h)}$ satisfies
        \begin{equation}
            \label{eq:9.18}
            \lim_{\mathbf{h} \to \mathbf{0}} 
            \frac{\left| \mathbf{r(h)} \right|}{\left| \mathbf{h} \right|} = 0.
        \end{equation}
        We may interpret (\ref{eq:9.17}), as in Sec. 9.10, by saying that for fixed $\mathbf{x}$ and small $\mathbf{h}$, 
        the left side of (\ref{eq:9.17}) is approximately equal to $\mathbf{f'(x)h}$, 
        that is, to the value of a linear transformation applied to $\mathbf{h}$.
        \item Suppose $\mathbf{f}$ and $E$ are as in Definition \ref{myDef:9.11}, and $\mathbf{f}$ is differentiable in $E$. 
        For every $\mathbf{x} \in E$. $\mathbf{f'(x)}$ is then a function, 
        namely, a linear transformation of $\R^n$ into $\R^m$. 
        But $\mathbf{f'}$ is also a function: 
        $\mathbf{f}'$ maps $E$ into $L(\R^n, \R^m)$.
        \item A glance at (\ref{eq:9.17}) shows that $\mathbf{f}$ is continuous at any point at which $\mathbf{f}$ is differentiable.
        \item The derivative defined by (\ref{eq:9.14}) or (\ref{eq:9.17}) is often called 
        the \emph{differential} of $\mathbf{f}$ at $\mathbf{x}$, or the \emph{total derivative} of $\mathbf{f}$ at $\mathbf{x}$, 
        to distinguish it from the partial derivatives that will occur later.
    \end{asparaenum}
\end{myRemark}

\begin{myExample}
    \label{myExample:9.14}
    We have defined derivatives of functions carrying $\R^n$ to $\R^m$ to be linear transformations of $\R^n$ into $\R^m$.
    What is the derivative of such a linear transformation?
    The answer is very simple.

    \emph{If $A \in L(\R^n, \R^m)$ and if $\mathbf{x} \in \R^n$, then}
    \begin{equation}
        \label{eq:9.19}
        A'(\mathbf{x}) = A.
    \end{equation}

    Note that $\mathbf{x}$ appears on the left side of (\ref{eq:9.19}), but not on the right.
    Both sides of (\ref{eq:9.19}) are members of $L(\R^n, \R^m)$, whereas $A \mathbf{x} \in \R^m$ .

    The proof of (\ref{eq:9.19}) is a triviality, since 
    \begin{equation}
        \label{eq:9.20}
        A(\mathbf{x + h}) - A \mathbf{x} = A \mathbf{h},
    \end{equation}
    by the linearity of $A$.
    With $\mathbf{f(x)} = A \mathbf{x}$, the numerator in (\ref{eq:9.14}) is thus 0 for every $\mathbf{h} \in \R^n$. In (\ref{eq:9.17}), $\mathbf{r(h)} = \mathbf{0}$ .
\end{myExample}
\mybox{对 $\R^n$ 映射到 $\R^m$ 的函数, 定义其导数为 $\R^n$ 到 $\R^m$ 的线性变换}
We now extend the chain rule (Theorem \ref{thm:5.5}) to the present situation.
\mybox{拓展链导法则}
\begin{thm}
    \label{thm:9.15}
    Suppose $E$ is an open set in $\R^n$, $\mathbf{f}$ maps $E$ into $\R^m$, 
    $\mathbf{f}$ is differentiable at $\mathbf{x}_0 \in E$, 
    $\mathbf{g}$ maps an open set containing $\mathbf{f}(E)$ into $\R^k$, 
    and $\mathbf{g}$ is differentiable at $\mathbf{f}(\mathbf{x}_0)$. 
    Then the mapping $\mathbf{F}$ of $E$ into $\R^k$ defined by
    \begin{equation*}
        \mathbf{F(x)} = \mathbf{g(f(x))}
    \end{equation*}
    is differentiable at $\mathbf{x}_0$ , and
    \begin{equation}
        \label{eq:9.21}
        \mathbf{F'(x_0)} = \mathbf{g'(f(x_0)) f'(x_0)}.
    \end{equation}
\end{thm}

On the right side of (\ref{eq:9.21}), we have the product of two linear transformations, as defined in Sec. 9.6.

% todo add proof

\begin{myDef}
    \label{myDef:9.16}
    \textbf{Partial derivatives}
    We again consider a function $\mathbf{f}$ that maps an open set $E \subset \R^n$ into $\R^m$.
    Let $\{\mathbf{e}_1, \dots, \mathbf{e}_n\}$
    and $\{\mathbf{u}_1, \dots, \mathbf{u}_m\}$ 
    be the standard bases of $\R^n$ and $\R^m$.
    The \emph{components} of $\mathbf{f}$ are the real Functions
    $f_1, \dots, f_m$ defined by 
    \begin{equation}
        \label{eq:9.24}
        \mathbf{f(x)} = 
        \sum_{i=1}^{m} f_i(\mathbf{x}) \mathbf{u}_i
        \quad 
        (\mathbf{x} \in E).
    \end{equation}
    or, equivalently, by $f_i(\mathbf{x}) = \mathbf{f(x)}\cdot \mathbf{u}_i$, $1 \leq i \leq m$.

    For $\mathbf{x} \in E$, $1 \leq i \leq m$, $1 \leq j \leq n$, we define 
    \begin{equation}
        \label{eq:9.25}
        (D_j f_i)(\mathbf{x}) =
        \lim_{t \to 0} \frac{f_i(\mathbf{x} + t \mathbf{e}_j) - f_i (\mathbf{x})}{t},
    \end{equation}
    provided the limit exists.
    Writing $f_i(x_1 , ... , x_n)$ in place of $f_i(\mathbf{x})$, we see that $D_j f_i$ is the derivative of $f_i$ with respect to $x_j$, keeping the other variables fixed.
    The notation
    \begin{equation}
        \label{eq:9.26}
        \frac{\partial f_i}{\partial x_j}
    \end{equation}
    is therefore often used in place of $D_j f_i$, 
    and $D_j f_i$ is called a \emph{partial derivative}.
\end{myDef}

In many cases where the existence of a derivative is sufficient when dealing
with functions of one variable, continuity or at least boundedness of the partial derivatives is needed for functions of several variables. 
\mybox{
    单变量函数只需要导数存在即可,
    多变量函数需要偏导数连续, 或至少是有界.
}
For example, the functions $f$ and $g$ described in Exercise 7, Chap. 4, are not continuous, although their partial derivatives exist at every point of $\R^2$ Even for continuous functions.
the existence of all partial derivatives does not imply differentiability in the sense
of Definition \ref{myDef:9.11} ; see Exercises 6 and 14, and Theorem \ref{thm:9.21}.

However, if $\mathbf{f}$ is known to be differentiable at a point $\mathbf{x}$, then its partial derivatives exist at $\mathbf{x}$, and they determine the linear transformation $\mathbf{f'(x)}$ completely:

\begin{thm}
    \label{thm:9.17}
    Suppose $\mathbf{f}$ maps an open set $E \subset \R^n$ into $\R^m$, 
    and $\mathbf{f}$ is differentiable at a point $\mathbf{x} \in E$. 
    Then the partial derivatives $(D_j f_i)(\mathbf{x})$ exist, and
    \begin{equation}
        \label{eq:9.27}
        \mathbf{f'(x)}\mathbf{e}_j = 
        \sum_{i=1}^{m}(D_j f_i)(\mathbf{x})\mathbf{u}_i
        \quad 
        (1 \leq j \leq n).
    \end{equation}
\end{thm}

Here, as in Sec. 9.16, $\{\mathbf{e}_1, \dots , \mathbf{e}_n\}$ 
and $\{\mathbf{u}_1, \dots , \mathbf{u}_m\}$ are the standard bases
of $\R^n$ and $\R^m$.

% todo add proof

\begin{myExample}
    % todo
\end{myExample}

\begin{thm}
    \label{thm:9.19}
    Suppose $\mathbf{f}$ maps a convex open set $E \subset \R^n$ into $\R^m$, 
    $\mathbf{f}$ is differentiable in $E$, 
    and there is a real number $M$ such that 
    \begin{equation*}
        \left\| \mathbf{f'(x)} \right\| \leq M
    \end{equation*}
    for every $\mathbf{x} \in E$. Then
    \begin{equation*}
        \left| \mathbf{f(b) - f(a)} \right| \leq 
        M \left| \mathbf{b - a} \right| 
    \end{equation*}
    for all $\mathbf{a} \in E, \mathbf{b} \in E$.
\end{thm}

% todo add proof

\begin{myCorollary*}
    If, in addition, $\mathbf{f'(x) = 0}$ for all $\mathbf{x} \in E$, 
    then $\mathbf{f}$ is constant.
\end{myCorollary*}

\begin{proof}
    To prove this, note that the hypotheses of the theorem hold now
    with $M =0$.
\end{proof}

\begin{myDef}
    \label{myDef:9.20}
    \emph{continuously differentiable}
    % todo
\end{myDef}

\begin{thm}
    \label{thm:9.21}
    Suppose $\mathbf{f}$ maps an open set $E \subset \R^n$ into $\R^m$. Then $\mathbf{f} \in \mathscr{C}'(E)$ 
    if and only if the partial derivatives $D_j f_i$ exist and are continuous on $E$ for $1 \leq i \leq m$,$1 \leq j \leq n$.
\end{thm}

% todo add proof

