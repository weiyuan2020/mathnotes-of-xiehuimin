% chap08sec05
\section{Fourier series}
\mybox{傅里叶级数}
\begin{mydef}
    \label{mydef:8.9}
    A \emph{trigonomeric polynomial} isa finite sum of the from 
    \begin{equation}
        \label{eq:8.59}
        f(x) = a_0 + \sum_{n=1}^{N}(a_n \cos n x + b_n \sin n x)
        \quad (x \text{ real}),
    \end{equation}
\end{mydef}
where $a_0,\dots,a_N$, $b_0,\dots,b_N$ are complex numbers.
On account of the identities (\ref{eq:8.46}),
(\ref{eq:8.59}) can also be written in the form 
\begin{equation}
    \label{eq:8.60}
    f(x) = \sum_{-N}^{N} c_n e^{inx}
    \quad (x \text{ real}),
\end{equation}
which is more convenient for most purposes.
It is clear that every trigonometrc polynomial is periodic, with period $2\pi$.

If $n$ is a nonzero integer, $e^{inx}$ is the derivative of $e^{inx}/in$,
which also has period $2\pi$. Hence
\begin{equation}
    \label{eq:8.61}
    \frac{1}{2\pi}\int_{-\pi}^{\pi}e^{inx} \d x  =
    \left\{ 
        \begin{array}{ll}
            1 & (\text{if } n = 0), \\
            0 & (\text{if } n = \pm 1, \pm 2, \dots). \\
        \end{array}
     \right.
\end{equation}

% todo add words, p186

\begin{mydef}
    \label{mydef:8.10}
    Let $\sequence{\phi_n}$ $(n = 1,2,3,\dots)$ be a sequence of complex functions on $[a,b]$ , such that 
    \begin{equation}
        \label{eq:8.64}
        \int_{a}^{b} \phi_n (x) \overline{\phi_m (x)} \d x
        \quad (n \neq m).
    \end{equation}
\end{mydef}