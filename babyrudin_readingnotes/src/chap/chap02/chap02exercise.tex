% chap02exercise

\section*{EXERCISES}

\begin{myExercise}
    \label{ex:2.1}
    Prove that the empty set is a subset of every set.
\end{myExercise}

\begin{myExercise}
    \label{ex:2.2}
    A complex number $z$ is said to be algebraic 
    if there are integers $a_0, ... , a_n$, not all
    zero, such that
    \begin{equation*}
        a_{0} z^{n} 
        + a_{1} z^{n-1}
        +\cdots
        + a_{n-1} z
        + a_n = 0 .
    \end{equation*}
    Prove that the set of all algebraic numbers is countable. 
    
    \emph{Hint}: For every positive integer $N$ 
    there are only finitely many equations with
    \begin{equation*}
        n 
        + |a_0|
        + |a_1|
        + \cdots
        + |a_n| = N .
    \end{equation*}
\end{myExercise}

\begin{myExercise}
    \label{ex:2.3}
    Prove that there exist real numbers which are not algebraic.
\end{myExercise}

\begin{myExercise}
    \label{ex:2.4}
    Is the set of all irrational real numbers countable?
\end{myExercise}

\begin{myExercise}
    \label{ex:2.5}
    Construct a bounded set of real numbers with exactly three limit points.
\end{myExercise}

\begin{myExercise}
    \label{ex:2.6}
    Let $E'$ be the set of all limit points of a set $E$. 
    Prove that $E'$ is closed. 
    Prove that $E$ and $\bar{E}$ have the same limit points. 
    (Recall that $\bar{E} = E \cup E'$.) 
    Do $E$ and $E'$ always have the same limit points?
\end{myExercise}

\begin{myExercise}
    \label{ex:2.7}
    Let $A_1, A_2, A_3, ...$ be subsets of a metric space.
    \begin{enumerate}[(a)]
        \item If $B_n = \cup_{i=1}^n A_i$, 
        prove that $B_n = \cup_{i=1}^n \bar{A}_i$, 
        for $n = 1, 2, 3, ...$.
        \item If $B = \cup_{i=1}^{\infty} A_i$, 
        prove that $\bar{B} \supset \cup_{i=1}^{\infty}\bar{A}_i$.
    \end{enumerate}
    Show, by an example, that this inclusion can be proper.
\end{myExercise}

\begin{myExercise}
    \label{ex:2.8}
    Is every point of every open set $E \subset \R^2$ a limit point of $E$? 
    Answer the same question for closed sets in $\R^2$
\end{myExercise}

\begin{myExercise}
    \label{ex:2.9}
    Let $E^0$ denote the set of all interior points of a set $E$. 
    [See Definition \ref{mydef:2.18}(e);
    $E^0$ is called the \emph{interior} of $E$.]
    \begin{enumerate}[(a)]
        \item Prove that $E^0$ is always open.
        \item Prove that $E$ is open if and only if $E^0 = E$.
        \item If $G \subset E$ and $G$ is open, prove that $G \subset E^0$
        \item Prove that the complement of $E^0$ is the closure of the complement of $E$.
        \item Do $E$ and $\bar{E}$ always have the same interiors?
        \item Do $E$ and $E^0$ always have the same closures?
    \end{enumerate}
\end{myExercise}

\begin{myExercise}
    \label{ex:2.10}
    Let $X$ be an infinite set. 
    For $p \in X$ and $q \in X$, define
    \begin{equation*}
        d(p,q) = \left\{ 
            \begin{array}{ll}
                1 & (\text{if } p \neq q) \\
                0 & (\text{if } p =    q). \\
            \end{array}
         \right.
    \end{equation*}
    Prove that this is a metric. 
    Which subsets of the resulting metric space are open?
    Which are closed? 
    Which are compact?
\end{myExercise}

\begin{myExercise}
    \label{ex:2.11}
    For $x \in \R^1$ and $y \in \R^1$, define
    \begin{align*}
        d_1(x,y) &= (x-y)^2, \\
        d_2(x,y) &= \sqrt{|x-y|}, \\
        d_3(x,y) &= |x^2-y^2|, \\
        d_4(x,y) &= |x-2y|, \\
        d_5(x,y) &= \frac{|x-y|}{1+|x-y|}.
    \end{align*}
    Determine, for each of these, 
    whether it is a metric or not.
\end{myExercise}

\begin{myExercise}
    \label{ex:2.12}
    Let $K \subset \R^1$ consist of $0$ and the numbers $1/n$, 
    for $n = 1, 2, 3, ...$. 
    Prove that $K$ is compact directly from the definition 
    (without using the Heine-Borel theorem).
\end{myExercise}

\begin{myExercise}
    \label{ex:2.13}
    Construct a compact set of real numbers whose limit points form a countable set.
\end{myExercise}

\begin{myExercise}
    \label{ex:2.14}
    Give an example of an open cover of the segment $(0, 1)$ 
    which has no finite subcover.
\end{myExercise}

\begin{myExercise}
    \label{ex:2.15}
    Show that Theorem \ref{thm:2.36} and its Corollary become false 
    (in $\R^1$, for example) 
    if the word ``compact'' is replaced by ``closed'' or by ``bounded.''
\end{myExercise}

\begin{myExercise}
    \label{ex:2.16}
    Regard $\Q$, the set of alt rational numbers, as a metric space, 
    with $d(p, q) = |p - q|$,
    Let $E$ be the set of all $p \in \Q$ such that $2 < p^2 < 3$. 
    Show that $E$ is closed and bounded in $\Q$, 
    but that $E$ is not compact. 
    Is $E$ open in $\Q$?
\end{myExercise}

\begin{myExercise}
    \label{ex:2.17}
    Let $E$ be the set of all $x \in [0, 1]$ 
    whose decimal expansion contains only the digits $4$ and $7$. 
    Is $E$ countable? 
    Is $E$ dense in $[0, 1]$? 
    Is $E$ compact? 
    Is $E$ perfect?
\end{myExercise}

\begin{myExercise}
    \label{ex:2.18}
    Is there a nonempty perfect set in $\R^1$ which contains no rational number?
\end{myExercise}


\begin{myExercise}
    \label{ex:2.19}
    \begin{enumerate}[(a)]
        \item If $A$ and $B$ are disjoint closed sets in some metric space $X$, prove that they are separated.
        \item Prove the same for disjoint open sets. 
        \item Fix $p \in X$, $S > 0$, define $A$ to be the set of all $q \in X$ for which $d(p, q) < S$, define $B$ similarly, with $>$ in place of $<$. Prove that $A$ and Bare separated.
        \item Prove that every connected metric space with at least two points is uncountable. \emph{Hint}: Use (c).
    \end{enumerate}
\end{myExercise}

\begin{myExercise}
    \label{ex:2.20}
    Are closures and interiors of connected sets always connected? 
    (Look at subsets of $\R^2$.)
\end{myExercise}

\begin{myExercise}
    \label{ex:2.21}
    Let $A$ and $B$ be separated subsets of some $\R^t$, 
    suppose $\mathbf{a} \in A, \mathbf{b} \in B$,
    and define 
    \begin{equation*}
        \mathbf{p}(t) = (1 - t)\mathbf{a} + t\mathbf{b}
    \end{equation*}
    for $t \in \R^1$. 
    Put $A_0= p^{-1}(A), B_0= p^{-1}(B)$. 
    [Thus $t \in A_0$ if and only if $p(t) \in A$.]
    \begin{enumerate}[(a)]
        \item Prove that $A_0$ and $B_0$ are separated subsets of $\R^1$
        \item Prove that there exists $t_0 \in (0, 1)$ such that $p(t_0) <t A \cup B$.
        \item Prove that every convex subset of $\R^k$ is connected.
    \end{enumerate}
\end{myExercise}

\begin{myExercise}
    \label{ex:2.22}
    A metric space is called \myKeywordblue{separable} if it contains a countable dense subset. 
    Show that $\R^k$ is separable. 
    
    \emph{Hint}: Consider the set of points which have only rational coordinates.
\end{myExercise}

\begin{myExercise}
    \label{ex:2.23}
    A collection $\{V_a\}$ of open subsets of $X$ is said to be a base for $X$ 
    if the following is true: 
    For every $x \in X$ and every open set $G \subset X$ such that $x \in G$, 
    we have $x \in V_a \subset G$ for some $\alpha$. 
    In other words, every open set in $X$ is the union of a 
    sub-collection of $\{V_a\}$,
    
    Prove that every separable metric space has a countable base. 

    \emph{Hint}: Take all neighborhoods with rational radius 
    and center in some countable dense subset of $X$.
\end{myExercise}

\begin{myExercise}
    \label{ex:2.24}
    Let $X$ be a metric space in which every infinite subset has a limit point. 
    Prove that $X$ is separable. 
    
    \emph{Hint}: Fix $\delta > 0$, and pick $x_1 \in X$. 
    Having chosen $x_1, ... , x_j \in X$,
    choose $x_{j+1} \in X$, 
    if possible, so that $d(x_i, x_{j+1})\geq \delta$ for $i = 1, ... ,j$. 
    Show that this process must stop after a finite number of steps, 
    and that $X$ can therefore be covered by finitely many neighborhoods of radius $\delta$. 
    Take $\delta = 1/n (n = 1, 2, 3, ... )$, 
    and consider the centers of the corresponding neighborhoods.
\end{myExercise}


\begin{myExercise}
    \label{ex:2.25}
    Prove that every compact metric space $K$ has a countable base, 
    and that $K$ is therefore separable. 
    
    \emph{Hint}: For every positive integer $n$, 
    there are finitely many neighborhoods of radius $1/n$ whose union covers $K$.
\end{myExercise}


\begin{myExercise}
    \label{ex:2.26}
    Let $X$ be a metric space in which every infinite subset has a limit point. 
    Prove that $X$ is compact. 
    
    \emph{Hint}: By Exercises \ref{ex:2.23} and \ref{ex:2.24}, 
    $X$ has a countable base. 
    It follows that every open cover of $X$ has a countable subcover ${G_n}$, $n = l, 2, 3, ...$.
    If no finite subcollection of ${G_n}$ covers $X$, 
    then the complement $F_n$ of $G_1 \cup \cdots \cup G_n$
    is nonempty for each $n$, but $\cap F_n$ is empty. 
    If $E$ is a set which contains a point from each $F_n$, 
    consider a limit point of $E$, and obtain a contradiction.
\end{myExercise}


\begin{myExercise}
    \label{ex:2.27}
    Define a point $p$ in a metric space $X$ to be a condensation point of a set $E \subset X$ if every neighborhood of $p$ contains uncountably many points of $E$. 
    
    Suppose $E \subset \R^k$, $E$ is uncountable, 
    and let $P$ be the set of all condensation points of $E$.
    Prove that $P$ is perfect and that 
    at most countably many points of $E$ are not in $P$. 
    In other words, show that $P^C \cap E$ is at most countable. 
    
    \emph{Hint}: Let $\{V_n\}$ be a countable base of $\R^k$, 
    let $W$ be the union of those $V_n$ for which $E \cap V_n$
    is at most countable, and show that $P = W^C$.
\end{myExercise}

\begin{myExercise}
    \label{ex:2.28}
    Prove that every closed set in a separable metric space is the union of a (possibly empty) perfect set and a set which is at most countable. 
    (\emph{Corollary}: Every countable closed set in Rk has isolated points.) 
    
    \emph{Hint}: Use Exercise 27.
\end{myExercise}


\begin{myExercise}
    \label{ex:2.29}
    Prove that every open set in $\R^1$ is the union of an at most countable collection of disjoint segments. 

    \emph{Hint}: Use Exercise 22.
\end{myExercise}

\begin{myExercise}
    \label{ex:2.30}
    Imitate the proof of Theorem \ref{thm:2.43} to obtain the following result:
    \begin{enumerate}
        \item If $\R^k = \cup_1^{\infty} F_n$, where each $F_n$ is a closed subset of $\R^k$, then at least one $F_n$ has a nonempty interior. 
        \item Equivalent statement: If Gn is a dense open subset of $\R^k$, for $n = 1, 2, 3, ... $, then $\cap_1^{\infty} G_n$ is not empty (in fact, it is dense in $\R^k$).
    \end{enumerate}
    (This is a special case of Baire's theorem; see Exercise \ref{ex:3.22}, for the general case.)
\end{myExercise}