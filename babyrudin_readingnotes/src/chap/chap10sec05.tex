% chap10sec05
\section{Differential forms}

We shall now develop some of the machinery that is needed for the $n$-dimensional version of the fundamental theorem of calculus which is usually called \emph{Stokes' theorem}. 
The original form of Stokes' theorem arose in applications of
vector analysis to electromagnetism and was stated in terms of the curl of a
vector field. 
Green's theorem and the divergence theorem are other special cases. 
These topics are briefly discussed at the end of the chapter.
\mybox{斯托克斯定理}

It is a curious feature of Stokes' theorem that the only thing that is difficult
about it is the elaborate structure of definitions that are needed for its statement.
These definitions concern differential forms, their derivatives, boundaries, and
orientation. Once these concepts are understood, the statement of the theorem
is very brief and succinct, and its proof presents little difficulty.

Up to now we have considered derivatives of functions of several variables
only for functions defined in open sets. This was done to avoid difficulties that
can occur at boundary points. It will now be convenient, however, to discuss
differentiable functions on \emph{compact} sets. We therefore adopt the following
convention:

To say that $\mathbf{f}$ is a $\mathscr{C}'$-mapping (or a $\mathscr{C}''$-mapping) of a compact set
$D \subset \R^k$ into $\R^n$ means that there is a $\mathscr{C}'$-mapping (or a $\mathscr{C}''$-mapping) $\mathbf{g}$ of
an open set $W \subset \R^k$ into $\R^n$ such that $D \subset W$ and such that $\mathbf{g(x) = f(x)}$ for all $\mathbf{x} \in D$.

\begin{mydef}
    \label{def:10.10}
    Suppose $E$ is an open set in $\R^n$. 
    A $k$-surface in E is a $\mathscr{C}'$-mapping $\Phi$ from a compact set $D \subset \R^k$ into $E$.
    
    $D$ is called the \emph{parameter domain} of $\Phi$. 
    Points of $D$ will be denoted by $\mathbf{u} = (u_1, \dots , u_k)$.
\end{mydef}

We shall confine ourselves to the simple situation in which $D$ is either a $k$-cell or the $k$-simplex $Q^k$ described in Example 10.4. The reason for this is that we shall have to integrate over $D$, 
and we have not yet discussed integration over more complicated subsets of $\R^k$. 
It will be seen that this restriction on $D$ 
(which will be tacitly made from now on) entails no significant loss of generality in the resulting theory of differential forms.

We stress that $k$-surfaces in $E$ are defined to be \emph{mappings} into $E$, not subsets of $E$. 
This agrees with our earlier definition of curves (Definition \ref{def:6.26}).
In fact, $1$-surfaces are precisely the same as continuously differentiable curves.

\begin{mydef}
    \label{def:10.11}
    Suppose $E$ is an open set in $\R^n$. 
    A \emph{differential form of order} $k \geq 1$ in $E$ 
    (briefly, a $k$-form in $E$) is a function $\omega$, 
    symbolically represented by the sum 
    \begin{equation}
        \label{eq:10.34}
        \omega - \sum a_{i_1 \cdots i_k} (\mathbf{x}) 
        \d x_{i_1} \wedge \cdots \wedge 
        \d x_{i_k} 
    \end{equation}
    (the indices $i_1 , ... , i_k$ range independently from 1 to $n$), which assigns to each $k$-surface $\Phi$ in $E$ a number $\omega(\Phi) = \int_{\Phi} \omega$, 
    according to the rule
    \begin{equation}
        \label{eq:10.35}
        \int_{\Phi} \omega = 
        \int_{D} \sum a_{i_1 \cdots i_k} (\mathbf{\Phi(u)}) \frac{\partial (x_{i_1},...,x_{i_k})}{\partial (u_{1},...,u_{k})} \d \mathbf{u} ,
    \end{equation}
    where $D$ is the parameter domain of $\Phi$.
    
    The functions $a_{i_1 \dots i_k}$ are assumed to be real and continuous in $E$. 
    If $\phi_1 , ... , \phi_n$ are the components of $\Phi$, the Jacobian in (\ref{eq:10.35}) is the one determined by the mapping
    \begin{equation*}
        (u_1,..,u_k) \rightarrow 
        (\phi_{i_1}(\mathbf{u}),..,\phi_{i_k}(\mathbf{u})) .
    \end{equation*}

    Note that the right side of (\ref{eq:10.35}) is an integral over $D$, as defined in Definition \ref{def:10.1} (or Example 10.4) and that (\ref{eq:10.35}) is the \emph{definition} of the symbol $\int_{\Phi} \omega$.
    
    A $k$-form $\omega$ is said to be of class $\mathscr{C}'$ or $\mathscr{C}''$ 
    if the functions $a_{i_1 \cdots i_k}$ in (\ref{eq:10.34}) are all of class $\mathscr{C}'$ or $\mathscr{C}''$.
    
    A 0-form in E is defined to be a continuous function in $E$.
\end{mydef}


\begin{myExample}
    % todo add examples
\end{myExample}

\begin{myRemark}
    \myKeyword{ Elementary properties}
    Let $\omega$, $\omega_1$, $\omega_2$ be $k$-forms in $E$.
    We write $\omega_1 = \omega_2$ 
    if and only if $\omega_1(\Phi) = \omega_2(\Phi)$
    for every $k$-surface $\Phi$ in $E$.
    If $c$ is a real number, 
    then $c\omega$ is the $k$-form defined by 
    \begin{equation}
        \label{eq:10.37}
        \int_{\Phi} c\omega = 
        c\int_{\Phi} \omega ,
    \end{equation}
    and $\omega = \omega_1 + \omega_2$ means that 
    \begin{equation}
        \label{eq:10.38}
        \int_{\Phi} \omega = 
        \int_{\Phi} \omega_1 + 
        \int_{\Phi} \omega_2  
    \end{equation}
    for every $k$-surface $\Phi$ in $E$.
    As a special case of (\ref{eq:10.37}), 
    note that $-\omega$ is defined so that 
    \begin{equation}
        \label{eq:10.39}
        \int_{\Phi} (-\omega) = 
        -\int_{\Phi} \omega ,
    \end{equation}

    Consider a $k$-form 
    \begin{equation}
        \label{eq:10.40}
        \omega = a(\mathbf{x}) 
        \d x_{i_1} 
        \wedge \cdots \wedge 
        \d x_{i_k} 
    \end{equation}
    and let $\bar{\omega}$ be the $k$-form obtained by interchanging some pair of subscripts in (\ref{eq:10.40})
    If (\ref{eq:10.35}) and (\ref{eq:10.39}) are combined with the fact that a determinant changes sign if two of its rows are interchanged,
    we see that 
    \begin{equation}
        \label{eq:10.41}
        \bar{\omega} = -\omega.
    \end{equation}

    As a special case of this, note that the \emph{anticommutative relation}
    \begin{equation}
        \label{eq:10.42}
        \d x_i \wedge \d x_j = 
        -\d x_j \wedge \d x_i 
    \end{equation}
    holds for all $i$ and $j$.
    In particular,
    \begin{equation}
        \label{eq:10.43}
        \d x_i \wedge \d x_i = 0
        \quad (i = 1, \dots, n).
    \end{equation}
    \mybox{外微分$\d x \wedge \d y$ 的性质, 
    与向量外积 $\mathbf{a\times b} = -\mathbf{b\times a}$ 类似}

    More generally, let us return to (\ref{eq:10.40}),
    and assume that $i_r = i_s$ for some $r \neq s$.
    If these two subscripts are interchanged,
    then $\bar{\omega} = \omega$,
    hence $\omega = 0$, by (\ref{eq:10.41}).

    In other words, \emph{if $\omega$ is given by (\ref{eq:10.40}), 
    then $\omega = 0$ unless the subscripts $i_1,\dots,i_k$ are all distinct}.

    If $\omega$ is as in (\ref{eq:10.34}), the summands with repeated subscripts can therefore be omitted without changing $\omega$.
    \mybox{summands means a part of sum 部分和 \url{https://www.dictionary.com/browse/summand}}
    \mybox{Direct Summand  直和

    Given the direct sum of additive Abelian groups $A \oplus B$, 
    $A$ and $B$ are called direct summands. 
    The map $i_1:A \rightarrow A \oplus B$ defined by $i(a)=a \oplus 0$ is called the injection of the first summand, 
    and the map $p_1:A \oplus B \rightarrow A$ defined by $p_1(a \oplus b)=a$ is called the projection onto the first summand. 
    Similar maps $i_2,p_2$ are defined for the second summand $B$. }

    It follows that 0 is the only $k$-form in any open subset of $\R^n$, if $k > n$.

    The anticommutativity expressed by (\ref{eq:10.42}) is the reason for the inordinate amount of attention that has to be paid to minus signs when studying differential forms.    
\end{myRemark}

\begin{mydef}
    \myKeyword{ Basic $k$-forms}
    If $i_1, \dots , i_k$ are integers such that 
    $1 \leq i_1 < i_2 < \cdots < i_k \leq n$, 
    and if $I$ is the ordered $k$-tuple $\{i_1, \dots , i_k\}$, 
    then we call $I$ an increasing $k$-index, 
    and we use the brief notation
    \begin{equation}
        \label{eq:10.44}
        \d x_I = 
        \d x_{i_1} \wedge \cdots \wedge
        \d x_{i_k} .
    \end{equation}
    These forms $\d x_I$ are the so-called \emph{basic $k$-forms in} $\R^n$.

    It is not hard to verify that there are precisely $n!/k!(n-k)!$ basic $k$-forms in $\R^k$;
    we shall make no use of this, however.

    Much more important is the fact that every $k$-form can be represented in terms of basic $k$-forms. 
    To see this, note that every $k$-tuple $\{j_1 , \dots ,j_k\}$ of distinct integers can be converted to an increasing $k$-index $J$ by a finite number of interchanges of pairs; 
    each of these amounts to a multiplication by $-1$, as we saw
    in Sec. 10.13; hence
    \begin{equation}
        \label{eq:10.45}
        \d x_{j_1} \wedge \cdots \wedge
        \d x_{j_k} = 
        \varepsilon (j_1, \dots , j_k) \d x_j
    \end{equation}
    where $\varepsilon(j_1, ... ,j_k)$ is $1$ or $-1$, depending on the number of interchanges that are needed. 
    In fact, it is easy to see that
    \begin{equation}
        \label{eq:10.46}
        \varepsilon (j_1, \dots , j_k) =
        s (j_1, \dots , j_k)
    \end{equation}
    where $s$ is as in Definition \ref{def:9.33}.
\end{mydef}

For example,
\begin{equation*}
    \d x_1 \wedge
    \d x_5 \wedge
    \d x_3 \wedge
    \d x_2 =
    - 
    \d x_1 \wedge
    \d x_2 \wedge
    \d x_3 \wedge
    \d x_5 
\end{equation*}
and 
\begin{equation*}
    \d x_4 \wedge
    \d x_2 \wedge
    \d x_3 =
    \d x_2 \wedge
    \d x_3 \wedge
    \d x_4 .
\end{equation*}

If every $k$-tuple in (\ref{eq:10.34}) is converted to an increasing $k$-index, then we
obtain the so-called standard presentation of $\omega$:
\begin{Beqnarray}
    \label{eq:10.47}
    \omega = \sum_{I} b_I (\mathbf{x}) \d x_I .
\end{Beqnarray}

The summation in (\ref{eq:10.47}) extends over all increasing $k$-indices $I$.
[Of course, every increasing $k$-index arises from many (from $k!$, to be precise) $k$-tuples. 
Each $b_I$ in (\ref{eq:10.47}) 
may thus be a sum of several of the coefficients that occur in (\ref{eq:10.34}).]

For example,
\begin{equation*}
    x_1 \d x_2 \wedge \d x_1 - 
    x_2 \d x_3 \wedge \d x_2 + 
    x_3 \d x_2 \wedge \d x_3 + 
        \d x_1 \wedge \d x_2  
\end{equation*}
is a 2-form in $\R^3$ whose standard presentation is 
\begin{equation*}
    (1-x_1) \d x_1 \wedge \d x_2 
    (x_2 + x_3) \d x_2 \wedge \d x_3 .
\end{equation*}

The following uniqueness theorem is one of the main reasons for the
introduction of the standard presentation of a $k$-form.

\begin{thm}
    \label{thm:10.15}
    Suppose 
    \begin{equation}
        \label{eq:10.48}
        \omega = \sum_{I} b_I (\mathbf{x}) \d x_I
    \end{equation}
    is the standard presentation of a $k$-form $\omega$ in an open set $E \subset \R^n$. 
    If $\omega = 0$ in $E$, 
    then $b_I(\mathbf{x}) = 0$ for every increasing $k$-index $I$ and for every $\mathbf{x} \in E$.
\end{thm}

Note that the analogous statement would be false for sums such as (\ref{eq:10.34}),
since, for example,
\begin{equation*}
    \d x_1 \wedge \d x_2 +
    \d x_2 \wedge \d x_1 = 0.
\end{equation*}

% todo add proof

\begin{mydef}
    \myKeyword{ Products of basic $k$-forms}
    Suppose 
    \begin{equation}
        \label{eq:10.51}
        I = \{i_1,...,i_p\},
        \quad 
        J = \{j_1,...,j_q\}
    \end{equation}
    where $1 \leq i_1 < \cdots < i_p \leq n$ 
    and $1 \leq j_1 < \cdots < j_q \leq n$ .
    The \emph{product} of the corresponding basic forms $\d x_I$ and $\d x_J$ in $\R^n$ is a $(p + q)$ form in $\R^n$, denoted by the symbol $\d x_I \wedge \d x_J$, and defined by 
    \begin{equation}
        \label{eq:10.52}
        \d x_I \wedge \d x_J = 
        \d x_{i_1} \wedge \cdots \wedge
        \d x_{i_p} \wedge 
        \d x_{j_1} \wedge \cdots \wedge
        \d x_{j_q}.
    \end{equation}
\end{mydef}

\begin{mydef}
    \myKeyword{ Multiplication}
    Suppose $\omega$ and $\lambda$ are $p$- and $q$-forms, respectively, in
    some open set $E \subset \R^n$, with standard presentations
    \begin{equation}
        \label{eq:10.56}
        \omega = \sum_{I} b_I (\mathbf{x}) \d x_I , \quad 
        \lambda = \sum_{J} c_J (\mathbf{x}) \d x_J
    \end{equation}
    where $I$ abd $J$ range over all increasing $p$-indices and over all increasing $q$-indices
    taken from the set $\{1, ... , n\}$.
\end{mydef}

\begin{mydef}
    \myKeyword{ Differentiation}
    We shall now define a differentiation operator $\d$ which
    associates a $(k + 1)$-form $\d \omega$ to each $k$-form $\omega$ of class $\mathscr{C}'$ in some open set $E \in \R^n$.

    A 0-form of class $\mathscr{C}'$ in $E$ is just a real function $f \in \mathscr{C}'(E)$, and we define
    \begin{equation}
        \label{eq:10.59}
        \d f = \sum_{i=1}^{n} (D_i f) (\mathbf{x}) \d x_i .
    \end{equation}
    If $\omega = \sum b_I (\mathbf{x}) \d x_I$ is the standard presentation of a $k$-form $\omega$, and $B_I \in \mathscr{C}'(E)$ for each increasing $k$-index $I$, then we define
    \begin{equation}
        \label{eq:10.60}
        \d \omega = \sum_I (\d b_I) \wedge \d x_I .
    \end{equation}
\end{mydef}

\begin{myExample}
    Suppose $E$ is open in $\R^n$, $f \in \mathscr{C}'(E)$, and $\gamma$ is a continuously differentiable curve in $E$, with domain $[0,1]$. By (\ref{eq:10.59}) and (\ref{eq:10.35}),
    \begin{equation}
        \label{eq:10.61}
        \int_{\gamma} \d f 
        = \int_{0}^{1} \sum_{i=1}^{n} 
        (D_i f)(\gamma(t)) \gamma'_i (t)
        \d t .
    \end{equation}
    By the chain rule, the last integrand is $(f \circ \gamma)'(t)$.
    Hence
    \begin{equation}
        \label{eq:10.62}
        \int_{\gamma} \d f = 
        f(\gamma(1)) - 
        f(\gamma(0)) ,
    \end{equation}
    and we see that $\int_{\gamma} \d f$ is the same for all $\gamma$ with the same initial point and the same end point, as in (a) of Example 10.12.

    Comparison with Example 10.12(b) shows therefore that the 1-form $x \d y$ is not the derivative of any 0-form $f$.
    This could also be deduced from part (b) of the following theorem, since 
    \begin{equation*}
        \d (x \d y) = \d x \wedge \d y
    \end{equation*}
\end{myExample}
\mybox{integrand 被积函数}

\begin{thm}
    \label{thm:10.20}
    \begin{asparaenum}[(a)]
        \item If $\omega$ and $\lambda$ are $k$- and $m$-forms, respectively, of class $\mathscr{C}'$ in $E$, 
        then 
        \begin{equation}
            \label{eq:10.63}
            \d (\omega \wedge \lambda) =
            (\d \omega) \wedge \lambda +
            (-1)^k \omega \wedge \d \lambda .
        \end{equation}
        \item If $\omega$ is of class $\mathscr{C}''$ in $E$, then $\d ^2 \omega = 0$.
    \end{asparaenum}
\end{thm}

Here $\d ^2 \omega$ means, of course, $\d (\d \omega)$.

% todo add proof

\begin{mydef}
    \myKeyword{Change of variables}
    Suppose $E$ is an open set in $\R^n$, 
    $T$ is a $\mathscr{C}'$-mapping of $E$ into an open set $V \subset \R^m$, 
    and $\omega$ is a $k$-form in $V$, whose standard presentation is
    \begin{equation}
        \label{eq:10.65}
        \omega = \sum_{I} b_I (\mathbf{y}) \d y_I.
    \end{equation}

    (We use $\mathbf{y}$ for points of $V$, $\mathbf{x}$ for points of $E$.)

    Let $t_1,\dots,t_m$ be the components of $T$: If 
    \begin{equation*}
        \mathbf{y} = (y_1, \dots, y_m) = T(\mathbf{x})
    \end{equation*}
    then $y_i = t_i(\mathbf{x})$.
    As in (\ref*{eq:10.59}),
    \begin{equation}
        \label{eq:10.66}
        \d t_i = \sum_{j=1}^{n}(D_j t_i)(\mathbf{x}) \d x_j
        \quad (1 \leq i \leq m).
    \end{equation}
    Thus each $\d t_i$ is a 1-form in $E$.

    The mapping $T$ transforms $\omega$ into a $k$-form $\omega_T$ in $E$,
    whose definition is 
    \begin{equation}
        \label{eq:10.67}
        \omega_T = \sum_I b_I ((T(\mathbf{x}))) 
        \d t_{i_1} \wedge \cdots \wedge
        \d t_{i_k} .
    \end{equation}
    in each summand of (\ref{eq:10.67}),
    $I = \{i_1,...,i_k\}$ is an increasing $k$-index.
\end{mydef}

Our next theorem shows that addition, multiplication, and differentiation
of forms are defined in such a way that they commute with changes of variables.

\begin{thm}
    \label{thm:10.22}
    With $E$ and $T$ as in Sec. 10.21, let $\omega$ and $A$ be $k$- and $m$-forms in $V$, respectively. Then
    \begin{enumerate}[(a)]
        \item $(\omega + \lambda)_T = \omega_T + \lambda_T$ if $k = m$;
        \item $(\omega \wedge \lambda)_T = \omega_T \wedge \lambda_T$;
        \item $\d (\omega_T) = (\d \omega)_T$ 
        if $\omega$ is of class $\mathscr{C}'$ 
        and $T$ is of class $\mathscr{C}''$.
    \end{enumerate}
\end{thm}

% todo add proof

\begin{thm}
    \label{thm:10.23}
    Suppose $T$ is a $\mathscr{C}'$-mapping of an open set $E \subset \R^n$ into an open set $V \subset \R^m$, 
    $S$ is a $\mathscr{C}'$-mapping of $V$ into an open set $W \subset \R^P$, and $w$ is a $k$-form in $W$, 
    so that $\omega$ is a $k$-form in $V$ 
    and both $(\omega_S)_T$ and $\omega_{ST}$ are $k$-forms in $E$, 
    where $ST$ is defined by $(ST)(\mathbf{x}) = S(T(\mathbf{x}))$. 
    Then
    \begin{equation}
        \label{eq:10.71}
        (\omega_S)_T = \omega_{ST} .        
    \end{equation}
\end{thm}

% todo add proof

\begin{thm}
    \label{thm:10.24}
    Suppose $\omega$ is a $k$-form in an open set $E \subset \R^n$, 
    $\Phi$ is a $k$-surface in $E$, with parameter domain $D \subset \R^k$, 
    and $\Delta$ is the $k$-surface in $\R^k$, with parameter domain $D$, defined by $\Delta(\mathbf{u}) = \mathbf{u}(\mathbf{u} \in D)$. Then
    \begin{equation*}
        \int_{\Phi} \omega = 
        \int_{\Delta} \omega_{\Phi} .
    \end{equation*}
\end{thm}

% todo add proof

\begin{thm}
    \label{thm:10.25}
    Suppose $T$ is a $\mathscr{C}'$-mapping of an open set $E \subset \R^n$ into an open set $V \subset \R^m$, 
    $\Phi$ is a $k$-surface in $E$, 
    and $\omega$ is a $k$-form in $V$.

    Then 
    \begin{equation*}
        \int_{T \Phi} \omega = 
        \int_{\Phi} \omega_T
    \end{equation*}
\end{thm}

\begin{proof}
    Let $D$ be the parameter domain of $\Phi$ (hence also of $\Phi$) and define $\Delta$ as in Theorem \ref{thm:10.24}.

    Then 
    \begin{equation*}
        \int_{T \Phi} \omega = 
        \int_{\Delta} \omega_{T \Phi} = 
        \int_{\Delta} (\omega_{T})_{\Phi} = 
        \int_{\Phi} \omega_{T} .
    \end{equation*}

    The first of these equalities is Theorem \ref{thm:10.24}, applied to $T\Phi$ in place of $\Phi$. 
    The second follows from Theorem \ref{thm:10.23}. The third is Theorem \ref{thm:10.24},
    with $\omega_T$ in place of $\omega$.
\end{proof}