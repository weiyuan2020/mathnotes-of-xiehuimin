% chap10sec05
\section{Differential forms}

We shall now develop some of the machinery that is needed for the $n$-dimensional version of the fundamental theorem of calculus which is usually called \emph{Stokes' theorem}. 
The original form of Stokes' theorem arose in applications of
vector analysis to electromagnetism and was stated in terms of the curl of a
vector field. 
Green's theorem and the divergence theorem are other special cases. 
These topics are briefly discussed at the end of the chapter.
\mybox{斯托克斯定理}

It is a curious feature of Stokes' theorem that the only thing that is difficult
about it is the elaborate structure of definitions that are needed for its statement.
These definitions concern differential forms, their derivatives, boundaries, and
orientation. Once these concepts are understood, the statement of the theorem
is very brief and succinct, and its proof presents little difficulty.

Up to now we have considered derivatives of functions of several variables
only for functions defined in open sets. This was done to avoid difficulties that
can occur at boundary points. It will now be convenient, however, to discuss
differentiable functions on \emph{compact} sets. We therefore adopt the following
convention:

To say that $\mathbf{f}$ is a $\mathscr{C}'$-mapping (or a $\mathscr{C}''$-mapping) of a compact set
$D \subset \R^k$ into $\R^n$ means that there is a $\mathscr{C}'$-mapping (or a $\mathscr{C}''$-mapping) $\mathbf{g}$ of
an open set $W \subset \R^k$ into $\R^n$ such that $D \subset W$ and such that $\mathbf{g(x) = f(x)}$ for all $\mathbf{x} \in D$.

\begin{myDef}
    \label{myDef:10.10}
    Suppose $E$ is an open set in $\R^n$. 
    A $k$-surface in E is a $\mathscr{C}'$-mapping $\Phi$ from a compact set $D \subset \R^k$ into $E$.
    
    $D$ is called the \emph{parameter domain} of $\Phi$. 
    Points of $D$ will be denoted by $\mathbf{u} = (u_1, \dots , u_k)$.
\end{myDef}

We shall confine ourselves to the simple situation in which $D$ is either a $k$-cell or the $k$-simplex $Q^k$ described in Example 10.4. The reason for this is that we shall have to integrate over $D$, 
and we have not yet discussed integration over more complicated subsets of $\R^k$. 
It will be seen that this restriction on $D$ 
(which will be tacitly made from now on) entails no significant loss of generality in the resulting theory of differential forms.

We stress that $k$-surfaces in $E$ are defined to be \emph{mappings} into $E$, not subsets of $E$. 
This agrees with our earlier definition of curves (Definition \ref{myDef:6.26}).
In fact, $1$-surfaces are precisely the same as continuously differentiable curves.

\begin{myDef}
    \label{myDef:10.11}
    Suppose $E$ is an open set in $\R^n$. 
    A \emph{differential form of order} $k \geq 1$ in $E$ 
    (briefly, a $k$-form in $E$) is a function $\omega$, 
    symbolically represented by the sum 
    \begin{equation}
        \label{eq:10.34}
        \omega - \sum a_{i_1 \cdots i_k} (\mathbf{x}) 
        \d x_{i_1} \wedge \cdots \wedge 
        \d x_{i_k} 
    \end{equation}
    (the indices $i_1 , ... , i_k$ range independently from 1 to $n$), which assigns to each $k$-surface $\Phi$ in $E$ a number $\omega(\Phi) = \int_{\Phi} \omega$, 
    according to the rule
    \begin{equation}
        \label{eq:10.35}
        \int_{\Phi} \omega = 
        \int_{D} \sum a_{i_1 \cdots i_k} (\mathbf{\Phi(u)}) \frac{\partial (x_{i_1},...,x_{i_k})}{\partial (u_{1},...,u_{k})} \d \mathbf{u} ,
    \end{equation}
    where $D$ is the parameter domain of $\Phi$.
    
    The functions $a_{i_1 \dots i_k}$ are assumed to be real and continuous in $E$. 
    If $\phi_1 , ... , \phi_n$ are the components of $\Phi$, the Jacobian in (\ref{eq:10.35}) is the one determined by the mapping
    \begin{equation*}
        (u_1,..,u_k) \rightarrow 
        (\phi_{i_1}(\mathbf{u}),..,\phi_{i_k}(\mathbf{u})) .
    \end{equation*}

    Note that the right side of (\ref{eq:10.35}) is an integral over $D$, as defined in Definition \ref{myDef:10.1} (or Example 10.4) and that (\ref{eq:10.35}) is the \emph{definition} of the symbol $\int_{\Phi} \omega$.
    
    A $k$-form $\omega$ is said to be of class $\mathscr{C}'$ or $\mathscr{C}''$ 
    if the functions $a_{i_1 \cdots i_k}$ in (\ref{eq:10.34}) are all of class $\mathscr{C}'$ or $\mathscr{C}''$.
    
    A 0-form in E is defined to be a continuous function in $E$.
\end{myDef}


\begin{myExample}
    % todo add examples
\end{myExample}

\begin{myRemark}
    \textbf{{\color{blue} Elementary properties}}
    Let $\omega$, $\omega_1$, $\omega_2$ be $k$-forms in $E$.
    We write $\omega_1 = \omega_2$ 
    if and only if $\omega_1(\Phi) = \omega_2(\Phi)$
    for every $k$-surface $\Phi$ in $E$.
    If $c$ is a real number, 
    then $c\omega$ is the $k$-form defined by 
    \begin{equation}
        \label{eq:10.37}
        \int_{\Phi} c\omega = 
        c\int_{\Phi} \omega ,
    \end{equation}
    and $\omega = \omega_1 + \omega_2$ means that 
    \begin{equation}
        \label{eq:10.38}
        \int_{\Phi} \omega = 
        \int_{\Phi} \omega_1 + 
        \int_{\Phi} \omega_2  
    \end{equation}
    for every $k$-surface $\Phi$ in $E$.
    As a special case of (\ref{eq:10.37}), 
    note that $-\omega$ is defined so that 
    \begin{equation}
        \label{eq:10.39}
        \int_{\Phi} (-\omega) = 
        -\int_{\Phi} \omega ,
    \end{equation}

    Consider a $k$-form 
    \begin{equation}
        \label{eq:10.40}
        \omega = a(\mathbf{x}) 
        \d x_{i_1} 
        \wedge \cdots \wedge 
        \d x_{i_k} 
    \end{equation}
    and let $\bar{\omega}$ be the $k$-form obtained by interchanging some pair of subscripts in (\ref{eq:10.40})
    If (\ref{eq:10.35}) and (\ref{eq:10.39}) are combined with the fact that a determinant changes sign if two of its rows are interchanged,
    we see that 
    \begin{equation}
        \label{eq:10.41}
        \bar{\omega} = -\omega.
    \end{equation}

    As a special case of this, note that the \ref{anticommutative relation}
    \begin{equation}
        \label{eq:10.42}
        \d x_i \wedge \d x_j = 
        -\d x_j \wedge \d x_i 
    \end{equation}
    holds for all $i$ and $j$.
    In particular,
    \begin{equation}
        \label{eq:10.43}
        \d x_i \wedge \d x_i = 0
        \quad (i = 1, \dots, n).
    \end{equation}
    \mybox{外微分$\d x \wedge \d y$ 的性质, 
    与向量外积 $\mathbf{a\times b} = -\mathbf{b\times a}$ 类似}

    More generally, let us return to (\ref{eq:10.40}),
    and assume that $i_r = i_s$ for some $r \neq s$.
    If these two subscripts are interchanged,
    then $\bar{\omega} = \omega$,
    hence $\omega = 0$, by (\ref{eq:10.41}).

    In other words, \emph{if $\omega$ is given by (\ref{eq:10.40}), 
    then $\omega = 0$ unless the subscripts $i_1,\dots,i_k$ are all distinct}.

    If $\omega$ is as in (\ref{eq:10.34}), the summands with repeated subscripts can therefore be omitted without changing $\omega$.
    \mybox{summands means a part of sum 部分和 \url{https://www.dictionary.com/browse/summand}}
    \mybox{Direct Summand  直和

    Given the direct sum of additive Abelian groups $A \oplus B$, 
    $A$ and $B$ are called direct summands. 
    The map $i_1:A \rightarrow A \oplus B$ defined by $i(a)=a \oplus 0$ is called the injection of the first summand, 
    and the map $p_1:A \oplus B \rightarrow A$ defined by $p_1(a \oplus b)=a$ is called the projection onto the first summand. 
    Similar maps $i_2,p_2$ are defined for the second summand $B$. }

    It follows that 0 is the only $k$-form in any open subset of $\R^n$, if $k > n$.

    The anticommutativity expressed by (\ref{eq:10.42}) is the reason for the inordinate amount of attention that has to be paid to minus signs when studying differential forms.    
\end{myRemark}