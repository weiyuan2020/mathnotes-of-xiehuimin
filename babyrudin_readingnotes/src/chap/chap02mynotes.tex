% chap02def
\section*{mynotes}
自己的笔记还是需要有自己的思考在里面

2.18 中的定义结合2.21 的例子是非常重要的。

\begin{myDef}neighborhood
    % (2.18)    

    neighborhood $N_r(p)$, $\forall q, \exists r>0$ s.t. $d(p,r)<r$.
\end{myDef}

\begin{myDef}limit point, isolated point, interior point

    limit point, $p\in E$ , $\forall N_r(p)$, $\exists q\in N_r(p)$, $q\neq p$ s.t. $q\in E$.

    isolated point, $p\in E$, $p$ is not a limit point.

    interior point, for a point $p \in E$, $\exists N_r(p)\subset E$.
\end{myDef}

\begin{myDef} closed

    $E$ is \emph{closed} if every limit point of $E$ is a point of $E$.
\end{myDef}

\begin{myDef} open

    $E$ is \emph{open} if every point of $E$ is an interior point of $E$.
\end{myDef}

\begin{mynewthm}
    Every neighborhood is an open set.
\end{mynewthm}

\begin{myDef} complement
    
    The \emph{complement} of $E$ (denoted by $E^c$)is the set of all points $p \in X$ such that $p \not\in E$.
\end{myDef}

\begin{myDef} perfect
    
    (h) $E$ is \emph{perfect} if $E$ is closed and if every point of $E$ is a limit point of $E$.
\end{myDef}
    
\begin{myDef} bounded
    
    (i) $E$ is \emph{bounded} if there is a real number $M$ and a point $q \in X$ such that $d(p,q)< M$ for all $p \in E$.
\end{myDef}    

\begin{myDef} dense

    (j) $E$ is \emph{dense} in $X$ if every point of $X$ is a limit point of $E$, or a point of $E$ (or both).
\end{myDef}

定理 \ref{thm:2.23} 表明,虽然使用了看似不相关的定义,这里得到的开集与闭集仍然满足``开集的补集是闭集,闭集的补集是开集''这样直观的定理
\begin{mynewthm}
    A set $E$ is open if and only if its complement is closed.
\end{mynewthm}

\begin{myDef} open relative

    $E$ is open relative to $Y$, $\forall p \in E$, $\exists r>0$, s.t. $q \in E$, $d(p, q)<r$, $q \in Y$.
\end{myDef}
任给 $E$ 中一点, 存在邻域 $N$, $N$ 是 $Y$ 的子集. 称 $E$ 对 $Y$ 而言是开集.


2022.11.10

\begin{myDef}
    \emph{Derived Set}

    The limit points of a set ${P}$, denoted $P^{'}$. 
\end{myDef}

\begin{myDef}
    \emph{Perfect Set}

    A set $P$ is called perfect if $P=P^{'}$, where $P^{'}$ is the derived set of $P$.
\end{myDef}
\url{https://mathworld.wolfram.com/PerfectSet.html}

\begin{myDef}
    \emph{Complete Space}

    A space of functions comprising a complete biorthogonal system.
\end{myDef}

\begin{myDef}
    \emph{Complete Metric Space}

    A complete metric space is a metric space in which every Cauchy sequence is convergent.

    Examples include the real numbers with the usual metric, the complex numbers, finite-dimensional real and complex vector spaces, the space of square-integrable functions on the unit interval $L^2([0,1])$, and the $p$-adic numbers.
\end{myDef}