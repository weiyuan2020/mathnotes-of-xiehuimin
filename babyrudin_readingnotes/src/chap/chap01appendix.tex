% chap01appendix.tex
\section*{appendix}
Theorem \ref{thm:1.19} will be proved in this appendix 
by constructing $\R$ from $\Q$. 
We shall divide the construction into several steps.

\myKeyword{Step 1} 
The members of $\R$ will be certain subsets of $\Q$, called \emph{cuts}. 
A cut is, by definition, 
any set $\alpha \subset \Q$ with the following three properties.
\begin{enumerate}[(I)]
    \item $\alpha$ is not empty, and $\alpha \neq \Q$.
    \item If $p\in \alpha$,$q \in \Q$, and $q <p$, then $q \in \alpha$.
    \item If $p \in \alpha$, then $p <r$ for some $r\in \alpha$.
\end{enumerate}
\mybox{分划 $\alpha$ 是一个集合}

The letters $p, q, r, ...$ will always denote rational numbers, 
and $\alpha, \beta, \gamma, ...$ will denote cuts.
\mybox{建立分划定义的这三条性质说明了有理数集是稠密而不是连续的}

Note that (III) simply says that a has no largest member: 
(II) implies two facts which will be used freely:

If $p\in\alpha$ and $q\not\in\alpha$ then $p<q$.

If $r\not\in \alpha$ and $r<s$ then $s\not\in \alpha$.

\myKeyword{Step 2}
Define ``$\alpha < \beta$'' to mean: 
$\alpha$ is a proper subset of $\beta$.
\mybox{这里使用真子集关系定义了分划(集合)间的序}

Let us check that this meets the requirements of Definition \ref{mydef:1.5}.

If $\alpha < \beta$ and $\beta < \gamma$ it is clear that $\alpha < \gamma$. 
(A proper subset of a proper subset is a proper subset.) 
It is also clear that at most one of the three relations
\begin{equation*}
    \alpha < \beta, \qquad
    \alpha = \beta, \qquad
    \beta < \alpha.
\end{equation*}
can hold for any pair $\alpha, \beta$. 
To show that at least one holds, assume that the first two fail. 
Then $\alpha$ is not a subset of $\beta$. 
Hence there is a $p \in \alpha$ with $p \not\in \beta$. 
If $q \in \beta$, it follows that $q <p$ (since $p \not\in \beta$), 
hence $q \in \alpha$, by (II). 
Thus $\beta \subset \alpha$. 
Since $\beta \neq \alpha$, we conclude: $\beta < \alpha$.

Thus $\R$ is now an ordered set.

\mybox{利用集合关系定义的偏序关系具有传递性和三歧性}

\myKeyword{Step 3}
The ordered set $\R$ has the least-upper-bound property.

To prove this, let $A$ be a nonempty subset of $\R$, 
and assume that $\beta \in \R$ is an upper bound of $A$. 
Define $\gamma$ to be the union of all $\alpha \in A$. 
In other words, $p \in \gamma$ if and only if $p \in \alpha$ for some $\alpha \in A$. 
We shall prove that $\gamma \in \R$ and that $\gamma = \sup A$.

Since $A$ is not empty, there exists an $a_0 \in A$. 
This $\alpha_0$ is not empty. 
Since $\alpha_0 \in \gamma$, $\gamma$ is not empty. 
Next, $\gamma \subset \beta$ 
(since $\alpha \subset \beta$ for every $\alpha \in A$), 
and therefore $\gamma \neq \Q$. 
Thus $\gamma$ satisfies property (I). 
To prove (II) and (III), pick $p \in \gamma$. 
Then $p \in \alpha_1$ for some $\alpha_1 \in A$. 
If $q <p$, then $q \in \alpha_1$, hence $q \in \gamma$; this proves (II). 
If $r \in \alpha_1$ is so chosen that $r > p$, 
we see that $r\in \gamma$ (since $\alpha_1 \subset \gamma$), 
and therefore $\gamma$ satisfies (III).

Thus $\gamma \in \R$.

It is clear that $\alpha \leq \gamma$ for every $\alpha \in A$.

Suppose $\delta < \gamma$. 
Then there is an $s \in \gamma$ and that $s \not\in \delta$. 
Since $s \in \gamma, s \in \alpha$ for some $\alpha \in A$. 
Hence $\delta <a$, and $\delta$ is not an upper bound of $A$.

This gives the desired result: $\gamma = \sup A$.

\mybox{
    这里分划之间所用的 $\in$ 让我很费解, 
    上文对分划的定义是集合, 那么应该用子集形式而不是元素形式来描述偏序关系. 
    分划是不是集合的一个元素? 
    
    查询原始pdf文件发现确实是子集形式描述的!
    }

\myKeyword{Step 4} 
If $\alpha \in \R$ and $\beta \in \R$ 
we define $\alpha + \beta$ to be the set of all sums $r + s$, 
where $r \in \alpha$ and $s \in \beta$.

We define $0^*$ to be the set of all negative rational numbers. 
It is clear that $0^*$ is a cut. 
We verify that the axioms for addition 
(see Definition \ref{mydef:1.12}) 
hold in $\R$, with $0^*$ playing the role of $0$.

\myKeyword{Step 5} 
Having proved that the addition defined in Step 4 satisfies Axioms (A) of Definition \ref{mydef:1.12}, 
it follows that Proposition \ref{myProposition:1.14} is valid in $\R$, and we can
prove one of the requirements of Definition \ref{mydef:1.17}:

If $\alpha, \beta, \gamma \in \R$ and $\beta < \gamma$, then $\alpha + \beta < \alpha + \gamma$.

Indeed, it is obvious from the definition of $+$ in $\R$ that $\alpha + \beta \subset \alpha + \gamma$; 
if we had $\alpha + \beta = \alpha + \gamma$, 
the cancellation law (Proposition \ref{myProposition:1.14}) would imply $\beta = \gamma$.

It also follows that $\alpha > 0^*$ if and only if $-\alpha < 0^*$.

\myKeyword{Step 6} 
Multiplication is a little more bothersome than addition in the present context, 
since products of negative rationals are positive. 
For this reason we confine ourselves first to $\R^+$, 
the set of all $\alpha \in \R$ with $\alpha > 0^*$.

If $\alpha \in \R^+$ and $\beta \in \R^+$, 
we define $\alpha\beta$ to be the set of all $p$ such that $p \leq rs$
for some choice of $r \in \alpha$, $s \in \beta$, $r>0$, $s>0$.

We define $1^*$ to be the set of all $q < 1$.

Then the axioms (M) and (D) of Definition \ref{mydef:1.12} hold, 
with $\R^+$ in place of $F$, and with $1^*$ in the role of $1$.

The proofs are so similar to the ones given in detail in Step 4 that we omit them.

Note, in particular, that the second requirement of Definition \ref{mydef:1.17} holds:
If $\alpha > 0^*$ and $\beta > 0^*$ then $\alpha\beta > 0^*$.

\myKeyword{Step 7} 
We complete the definition of multiplication 
by setting $\alpha 0^* = 0^* \alpha = 0^*$,
and by setting
\begin{equation*}
    \alpha\beta = \left\{
        \begin{array}{ll}
            (-\alpha)(-\beta) & \text{if } \alpha < 0^*, \beta < 0^*,\\
            -[(-\alpha)\beta] & \text{if } \alpha < 0^*, \beta > 0^*,\\
            -[\alpha\cdot(-\beta)] & \text{if } \alpha > 0^*, \beta < 0^*,\\
        \end{array}
    \right.
\end{equation*}
The products on the right were defined in Step 6.

Having proved (in Step 6) that the axioms (M) hold in $\R^+$, 
it is now perfectly simple to prove them in$\R^+$, 
by repeated application of the identity $\gamma = -(-\gamma)$ 
which is part of Proposition \ref{myProposition:1.14}. (See Step 5.)

The proof of the distributive law
\begin{equation*}
    \alpha(\beta + \gamma) = \alpha\beta + \alpha\gamma
\end{equation*}
breaks into cases. 
For instance, suppose $\alpha> 0^*$, $\beta <0^*$, $\beta + \gamma > 0^*$ 
Then $\gamma = (\beta + \gamma) + (- \beta)$, 
and (since we already know that the distributive law holds in $\R^+$)
\begin{equation*}
    \alpha\gamma = \alpha(\beta+\gamma) + \alpha \cdot (-\beta).
\end{equation*}
But $\alpha \cdot (-\beta) = -(\alpha\beta)$. Thus
\begin{equation*}
    \alpha\beta + \alpha\gamma = \alpha(\beta + \gamma).
\end{equation*}
The other cases are handled in the same way.

We have now completed the proof that 
$\R$ is an ordered-field with the least-upper-bound property.

\myKeyword{Step 8} 
We associate with each $r\in \Q$ the set $r^*$ 
which consists of all $p \in \Q$ such that $p < r$. 
It is clear that each $r^*$ is a cut; 
that is, $r^* \in \R$. 
These cuts satisfy the following relations:
\begin{enumerate}[(a)]
    \item $r^* + s^* = (r+s)^*$,
    \item $r^* s^* = (rs)^*$,
    \item $r^* < s^*$ if and only if $r < s$.
\end{enumerate}

To prove (a), choose $p \in r^* + s^*$. Then $p=u+v$, where $u<r$, $v<s$.
Hence $p < r +s$, which says that $p \in (r + s)^*$.

Conversely, suppose $p \in (r+s)^*$. Then $p < r + s$. Choose $t$ so that
$2t = r + s - p$, put
\begin{equation*}
    r' = r - t, 
    s' = s - t.
\end{equation*}
Then $r' \in r^*$, $s' \in s^*$, and $p = r' + s'$, so that $p \in r^* + s^*$

This proves (a). The proof of (b) is similar.

If $r < s$ then $r \in s*$, but $r \not\in r^*$; 
hence $r^* < s^*$.

If $r^* <s^*$ then there is a $p \in s^*$ 
such that $p \not\in r^*$ 
Hence $r < p < s$, so that $r < s$.

This proves (c).

\myKeyword{Step 9} 
We saw in Step 8 that the replacement of the rational numbers $r$ 
by the corresponding ``rational cuts'' $r^* \in \R$ preserves sums, products, and order. 
This fact may be expressed by saying that 
the ordered field $\Q$ is isomorphic to the ordered field $\Q^*$ 
whose elements are the rational cuts. 
Of course, $r^*$ is by no means the same as $r$, 
but the properties we are concerned with (arithmetic and order) are the same in the two fields.

It is this identification of $\Q$ with $\Q^*$ which allows us to regard $\Q$ as a subfield of $\R$.

The second part of Theorem \ref{thm:1.19} is to be understood in terms of this identification. 
Note that the same phenomenon occurs when the real numbers are regarded as a subfield of the complex field, 
and it also occurs at a much more elementary level, when the integers are identified with a certain subset of $\Q$.

It is a fact, which we will not prove here, that any two ordered-fields with the least-upper-bound property are isomorphic. 
The first part of Theorem \ref{thm:1.19} therefore characterizes the real field $\R$ completely.

The books by Landau and Thurston cited in the Bibliography are entirely devoted to number systems. 
Chapter 1 of Knopp's book contains a more leisurely description of how $\R$ can be obtained from $\Q$. 
Another construction, 
in which each real number is defined to be an equivalence class of Cauchy sequences of rational numbers (see Chap. 3), 
is carried out in Sec. 5 of the book by Hewitt and Stromberg.

The cuts in $\Q$ which we used here were invented by Dedekind. 
The construction of $\R$ from $\Q$ by means of Cauchy sequences is due to Cantor. 
Both Cantor and Dedekind published their constructions in 1872.
