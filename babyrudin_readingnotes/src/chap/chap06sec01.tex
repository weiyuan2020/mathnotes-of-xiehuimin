% chap06sec01
\section{Definition and existence of the integral}
\begin{mydef}
    \label{def:6.1}
    Let $[a, b]$ be a given interval. 
    By a \emph{partition} $P$ of $[a, b]$ 
    we mean a finite set of points $x_0, x_1, ... , x_n$, where
    \begin{equation*}
        a = x_0 \leq x_1 \leq \dots \leq x_{n-1} \leq x_n = b.
    \end{equation*}
    \mybox{partition 分划}
    We write
    \begin{equation*}
        \Delta x_i = x_i - x_{i-1}
        \quad    (i=1, ... ,n).
    \end{equation*}
    Now suppose $f$ is a bounded real function defined on $[a, b ]$. 
    Corresponding to each partition $P$ of $[a, b]$ we put
    \begin{align*}
        M_i &= \sup f(x) \quad (x_{i-1} \leq x \leq x_i), \\
        m_i &= \inf f(x) \quad (x_{i-1} \leq x \leq x_i), \\
        U(P,f) &= \sum_{i=1}^{n} M_i \Delta x_i,\\
        L(P,f) &= \sum_{i=1}^{n} m_i \Delta x_i,
    \end{align*}
    and finally
    \begin{align}
        \label{eq:6.1}
        \overline{\int_{a}^{b}} f \d x &= \inf U(P, f), \\
        \label{eq:6.2}
        \underline{\int_{a}^{b}} f \d x &= \sup L(P, f),
    \end{align}
    where the inf and the sup are taken over all partitions $P$ of $[a, b]$. 
    The left members of (\ref{eq:6.1}) and (\ref{eq:6.2}) are called 
    the \emph{upper} and \emph{lower Riemann integrals} of $f$
    over $[a, b]$, respectively.
    \mybox{利用黎曼上下和定义黎曼积分}
    If the upper and lower integrals are equal, 
    we say that $f$ is Riemann-integrable on $[a, b]$, 
    we write $f \in \mathscr{R}$ 
    (that is, $\mathscr{R}$ denotes the set of Riemann-integrable functions), 
    and we denote the common value of (\ref{eq:6.1}) and (\ref{eq:6.2}) by
    \begin{equation}
        \label{eq:6.3}
        \int_{a}^{b} f \d x,
    \end{equation}
    or by 
    \begin{equation}
        \label{eq:6.4}
        \int_{a}^{b} f(x) \d x,
    \end{equation}
    This is the \emph{Riemann integral} of $f$ over $[a, b]$. 
    Since $f$ is bounded, there exist two numbers, $m$ and $M$, such that
    \begin{equation*}
        m \leq f(x) \leq M \quad 
        (a \leq x \leq b).
    \end{equation*}
    Hence, for every $P$,
    \begin{equation*}
        m(b - a) \leq L(P,f) \leq U(P,f) \leq M(b - a),
    \end{equation*}
    so that the numbers $L(P,f)$ and $U(P,f)$ form a bounded set. 
    This shows \emph{that the upper and lower integrals} are defined for every bounded function $f$.
    The question of their equality, 
    and hence the question of the integrability of $f$, is a more delicate one. 
    Instead of investigating it separately for the Riemann integral,
    we shall immediately consider a more general situation.
\end{mydef}

\begin{mydef}
    \label{def:6.2}
    Let $\alpha$ be a monotonically increasing function on $[a, b]$ 
    (since $\alpha (a)$ and $\alpha (b)$ are finite, 
    it follows that $\alpha$ is bounded on $[a, b]$). 
    Corresponding to each partition $P$ of $[a, b]$, 
    we write
    \begin{equation*}
        \Delta \alpha_t = \alpha (x_{i}) - \alpha (x_{i-1}),        
    \end{equation*}
    It is clear that $\Delta \alpha \geq 0$. 
    For any real function $f$ which is bounded on $[a, b]$
    we put
    \begin{align*}
        U(P, f, \alpha) &= \sum_{i=1}^{n} M_1 \Delta \alpha_i, \\
        L(P, f, \alpha) &= \sum_{i=1}^{n} m_1 \Delta \alpha_i. 
    \end{align*}
    where $M_1$, $m_1$ have the same meaning as in Definition \ref{def:6.1}, 
    and we define
    \begin{align}
        \label{eq:6.5}
        \overline{\int_{a}^{b}} f \d \alpha &= \inf U(P, f, \alpha), \\
        \label{eq:6.6}
        \underline{\int_{a}^{b}} f \d \alpha &= \sup L(P, f, \alpha),
    \end{align}
    the inf and sup again being taken over all partitions.

    If the left members of (\ref{eq:6.5}) and (\ref{eq:6.6}) are equal, 
    we denote their common value by
    \begin{equation}
        \label{eq:6.7}
        \int_{a}^{b} f \d \alpha
    \end{equation}
    or sometimes by 
    \begin{equation}
        \label{eq:6.8}
        \int_{a}^{b} f(x) \d \alpha (x).
    \end{equation}
    This is the \emph{Riemann-Stieltjes integral} 
    ( or simply the \emph{Stieltjes integral}) of $f$ 
    with respect to $\alpha$, over $[a, b]$.

    If (\ref{eq:6.7}) exists, i.e., 
    if (\ref{eq:6.5}) and (\ref{eq:6.6}) are equal, 
    we say that $f$ is integrable with respect to $\alpha$, 
    in the Riemann sense, and write $f \in \mathscr{R}(\alpha)$.
    
    By taking $\alpha(x) = x$, 
    the Riemann integral is seen to be a special case of
    the Riemann-Stieltjes integral. 
    Let us mention explicitly, however, that in the
    general case $\alpha$ need not even be continuous.

    A few words should be said about the notation. 
    We prefer (\ref{eq:6.7}) to (\ref{eq:6.8}), since
    the letter $x$ which appears in (\ref{eq:6.8}) 
    adds nothing to the content of (\ref{eq:6.7}). 
    It is immaterial which letter we use to represent the so-called 
    ``variable of integration.''
    For instance, (\ref{eq:6.8}) is the same as
    \begin{equation*}
        \int_{a}^{b} f(y) \d \alpha (y).
    \end{equation*}
    The integral depends on $f$, $\alpha$, $a$ and $b$, 
    but not on the variable of integration, 
    which may as well be omitted.

    The role played by the variable of integration is quite analogous to that
    of the index of summation: 
    \begin{equation*}
        \sum_{i=1}^{n} c_i , \quad
        \sum_{k=1}^{n} c_k 
    \end{equation*}
    The two symbols mean the same thing. 
    since each means $c_1 + c_2 + \cdots + c_n$.
    
    Of course, no harm is done by inserting the variable of integration, 
    and in many cases it is actually convenient to do so.
    
    We shall now investigate the existence of the integral (\ref{eq:6.7}). 
    Without saying so every time, 
    $f$ will be assumed real and bounded, 
    and $\alpha$ monotonically increasing on $[a, b]$; 
    and, when there can be no misunderstanding, 
    we shall write $\int$ in place of $\int_{a}^{b}$.
\end{mydef}

\begin{mydef}
    \label{def:6.3}
    We say that the partition $P^*$ is a \emph{refinement} of $P$ 
    if $P^* \supset P$
    (that is, if every point of $P$ is a point of $P^*$). 
    Given two partitions, $P_1$ and $P_2$ ,
    we say that $P^*$ is their common refinement if $P^* = P_1 \cup P_2$ .
\end{mydef}
\mybox{refinement 加细, 细分}

\begin{thm}
    \label{thm:6.4}
    If $P^*$ is a refinement of $P$. then
    \begin{equation}
        \label{eq:6.9}
        L(P,   f, \alpha) \leq 
        L(P^*, f, \alpha) 
    \end{equation}
    and 
    \begin{equation}
        \label{eq:6.10}
        U(P^*, f, \alpha) \leq 
        U(P,   f, \alpha) .
    \end{equation}
\end{thm}
\mybox{加细使得不等式范围变小, 结果更接近真实积分值.}
\begin{proof}
    To prove (\ref{eq:6.9}), 
    suppose first that $P^*$ contains just one point more than $P$. 
    Let this extra point be $x^*$, 
    and suppose $x_{i-1} < x^{*} < x_{i}$, 
    where $x_{i-1}$ and $x_{i}$, are two consecutive points of $P$. 
    Put
    \begin{align*}
        w_1 &= \inf f(x) \quad (x_{i-1} \leq x \leq x^{*}) \\
        w_2 &= \inf f(x) \quad (x^{*} \leq x \leq x_{i})        
    \end{align*} 
    Clearly $w_1 \geq m_i$ and $w_2 \geq m_i$, 
    where, as before,
    \begin{equation*}
        m_i = \inf f(x) \quad (x_{i-1} \leq x \leq x_{i})
    \end{equation*}
    Hence
    \begin{align*}
        L(P^*,f, \alpha ) - L(P,f, \alpha )
        &= w_1[\alpha (x^*) - \alpha (x_{i-1})] 
         + w_2[\alpha (x_i) - \alpha (x^*)] 
         - m_i[\alpha (x_i) - \alpha (x_{i-1})] \\
        &= (w_1 - m_i)[\alpha (x^*) - \alpha (x_{i-1})] 
         + (w_2 - m_i)[\alpha (x_i) - \alpha (x*)] \geq 0.
    \end{align*}
    
    If $P^*$ contains $k$ points more than $P$, 
    we repeat this reasoning $k$ times, 
    and arrive at (\ref{eq:6.9}). 
    The proof of (\ref{eq:6.10}) is analogous.
\end{proof}


\begin{thm}
    \label{thm:6.5}
    $\underline{\int_{a}^{b}} f \d \alpha \leq
    \overline{\int_{a}^{b}} f \d \alpha .$
\end{thm}
\begin{proof}
    Let $P^*$ be the common refinement of two partitions $P_1$ and $P_2$.
    By Theorem \ref{thm:6.4},
    \begin{equation*}
        L(P_1, f, \alpha) \leq
        L(P^*, f, \alpha) \leq
        U(P^*, f, \alpha) \leq
        U(P_2, f, \alpha) .
    \end{equation*}
    Hence 
    \begin{equation}
        \label{eq:6.11}
        L(P_1, f, \alpha) \leq
        U(P_2, f, \alpha) .
    \end{equation}
    If $P_2$ is fixed and the sup is taken over all $P_1$,
    (\ref{eq:6.11}) gives
    \begin{equation}
        \label{eq:6.12}
        \underline{\int} f \d \alpha \leq
        U(P_2, f, \alpha) .
    \end{equation}
    The theorem follows by taking the inf over all $P_2$ in (\ref{eq:6.12})
\end{proof}

\begin{thm}
    \label{thm:6.6}
    $f \in \mathscr{R} (\alpha)$ on $[a, b]$ if and only if 
    for every $\varepsilon > 0$ there exists a partition $P$
    such that
    \begin{equation}
        \label{eq:6.13}
        U(P, f, \alpha) -
        L(P, f, \alpha) < \varepsilon.
    \end{equation}
\end{thm}

\begin{proof}
    For every $P$ we have 
    \begin{equation*}
        L(P, f, \alpha)  \leq
        \underline{\int} f \d \alpha \leq
        \overline{\int}  f \d \alpha \leq
        U(P, f, \alpha) .
    \end{equation*}
    Thus (\ref{eq:6.13}) implies 
    \begin{equation*}
        0 \leq
        \underline{\int} f \d \alpha \leq
        \overline{\int}  f \d \alpha \leq
        \varepsilon .
    \end{equation*}
    Hence, if (\ref{eq:6.13}) can be satisfied for every $\varepsilon >0$,
    we have 
    \begin{equation*}
        \underline{\int} f \d \alpha =
        \overline{\int}  f \d \alpha 
    \end{equation*}
    that is, $f \in \mathscr{R}(\alpha)$.

    Conversely, suppose $f \in \mathscr{R}(\alpha)$, and let $\varepsilon > 0$,
    Then there exist partition $P_1$ and $P_2$ such that
    \begin{align}
        \label{eq:6.14}
        U(P_2, f, \alpha) - \int f \d \alpha &< \frac{\varepsilon}{2}, \\
        \label{eq:6.15}
        \int f \d \alpha - L(P_1, f, \alpha) &< \frac{\varepsilon}{2}.
    \end{align}
    We choose $P$ to be the common refinement of $P_1$ and $P_2$ 
    Then Theorem \ref{thm:6.4}, together with (\ref{eq:6.14}) and (\ref{eq:6.15}),
    shows that

    so that (\ref{eq:6.13}) holds for this partition $P$.
\end{proof}

Theorem \ref{thm:6.6} furnishes a convenient criterion for integrability. 
Before we apply it, we state some closely related facts.

\begin{thm}
    \label{thm:6.7}
    \begin{asparaenum}[(a)]
        \item If (\ref{eq:6.13}) holds for some $P$ and some $\varepsilon$,
        then (\ref{eq:6.13}) holds (with the same $\varepsilon$) 
        for every refinement of $P$.
        \item If (\ref{eq:6.13}) holds for $P = \{x_0, ...,x_n\}$ and 
        if $s_i, t_i$ are arbitrary points in $[x_{i-1}, x_i]$, then 
        \begin{equation*}
            \sum_{i=1}^{n} \left| f(s_i) - f(t_i) \right| \Delta \alpha_i < \varepsilon.
        \end{equation*}
        \item If (\ref{eq:6.13}) and the hypotheses of (b) hold, then 
        \begin{equation*}
            \left| \sum_{i=1}^{n}f(t_i)\Delta \alpha_i - \int_{a}^{b} f \d \alpha \right| 
            < \varepsilon.
        \end{equation*}
    \end{asparaenum}
\end{thm}

\begin{proof}
    Theorem \ref{thm:6.4} implies (a). 
    Under the assumptions made in (b),
    both $f(s_i)$ and $f(t_i)$ lie in $[m_i, M_i]$, 
    so that $\left| f(s_i) - f(t_i) \right| \leq M_i - m_i$. 
    Thus
    \begin{equation*}
        \sum_{i=1}^{n} \left| f(s_i) - f(t_i) \right| \Delta \alpha_i \leq
        U(P, f, \alpha) -
        L(P, f, \alpha).
    \end{equation*}
    which proves (b). 
    The obvious inequalities
    \begin{equation*}
        L(P, f, \alpha) \leq
        \sum f(t_i) \Delta \alpha_i \leq
        U(P, f, \alpha)
    \end{equation*}
    and
    \begin{equation*}
        L(P, f, \alpha) \leq
        \int f \d \alpha \leq
        U(P, f, \alpha)
    \end{equation*}
    prove (c).
\end{proof}

\begin{thm}
    \label{thm:6.8}
    If $f$ is continuous on $[a, b]$ then $f \in \mathscr{R}(\alpha)$ on $[a, b]$.
\end{thm}

\begin{proof}
    Let $\varepsilon > 0$ be given.
    Choose $\eta > 0$ so that 
    \begin{equation*}
        [\alpha(b) - \alpha(a)]\eta < \varepsilon.
    \end{equation*}
    Since $f$ is uniformly continuous on $[a, b]$ (Theorem \ref{thm:4.19}).
    there exists a $\delta > 0$ such that 
    \begin{equation}
        \label{eq:6.16}
        \left| f(x) - f(t) \right| < \eta
    \end{equation}
    if $x \in [a, b]$, $t \in [a, b]$, and $\left| x-t \right| < \delta$.

    If $P$ is any partition of $[a, b]$ 
    such that $\Delta x_i < \delta$ for all $i$,
    then (\ref{eq:6.16}) implies that 
    \begin{equation}
        \label{eq:6.17}
        M_i - m_i \leq \eta 
        \quad
        (i = 1,\dots,n)
    \end{equation}
    \mybox{原书这里写错了, 不是 $i-1,...,n$}
    and therefore 
    \begin{align*}
        U(P, f, \alpha) - L(P, f, \alpha)
        &= \sum_{i=1}^{n} \left( M_i - m_i \right) \Delta \alpha_i \\
        \leq \eta \sum_{i=1}^{n} \Delta \alpha_i 
        &= \eta \left[ \alpha (b) - \alpha (a) \right] < \varepsilon.
    \end{align*}
    By Theorem \ref{thm:6.6}, $f \in \mathscr{R}(\alpha)$.
\end{proof}

\begin{thm}
    \label{thm:6.9}
    If $f$ is monotonic on $[a, b]$,
    and if $\alpha$ is continuous on $[a, b]$,
    then $f \in \mathscr{R}(\alpha)$.
    (We still assume, of course, that $\alpha$ is monotonic.)
\end{thm}
% todo add proof

\begin{thm}
    \label{thm:6.10}
    Suppose $f$ is bounded on $[a, b]$, 
    $f$ has only finitely many points of discontinuity on $[a, b]$, 
    and $\alpha$ is continuous at every point 
    at which $f$ is discontinuous. 
    Then $f \in \mathscr{R}(\alpha)$.
\end{thm}
% todo add proof

\begin{thm}
    \label{thm:6.11}
    Suppose $f \in \mathscr{R}(\alpha)$ on $[a, b]$, $m \leq f \leq M$, 
    $\phi$ is continuous on $[m, M]$, 
    and $h(x) = \phi(f(x))$ on $[a, b]$. 
    Then $h \in \mathscr{R}(\alpha)$ on $[a, b]$.
\end{thm}
% todo add proof

% \begin{proof}
%     \begin{equation}
%         \label{eq:6.18}
%         U(P, f, \alpha) -
%         L(P, f, \alpha) < \delta^2.
%     \end{equation}
%     \begin{equation}
%         \delta \sum_{i \in B} \Delta \leq
%         \sum_{i \in B} \left( M_i - m_i \right) \Delta \alpha_i
%         < \delta^2.
%     \end{equation}
% \end{proof}

\emph{Remark: } This theorem suggests the question: 
Just what functions are Riemann-integrable?
The answer is given by Theorem 11.33(b).