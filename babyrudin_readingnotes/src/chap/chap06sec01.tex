% chap06sec01
\section{Definition and existence of the integral}
\mymyDef{
    Let $[a, b]$ be a given interval. 
    By a \emph{partition} $P$ of $[a, b]$ 
    we mean a finite set of points $x_0, x_1, ... , x_n$, where
    \begin{equation*}
        a = x_0 \leq x_1 \leq \dots \leq x_{n-1} \leq x_n = b.
    \end{equation*}
    We write
    \begin{equation*}
        \Delta x_i = x_i - x_{i-1}
        \quad    (i=1, ... ,n).
    \end{equation*}
    Now suppose $f$ is a bounded real function defined on $[a, b ]$. 
    Corresponding to each partition $P$ of $[a, b]$ we put
    \begin{align*}
        M_i &= \sup f(x) \quad (x_{i-1} \leq x \leq x_i), \\
        m_i &= \inf f(x) \quad (x_{i-1} \leq x \leq x_i), \\
        U(P,f) &= \sum_{i=1}^{n} M_i \Delta x_i,\\
        L(P,f) &= \sum_{i=1}^{n} m_i \Delta x_i,
    \end{align*}
    and finally
    \begin{align}
        \label{eq:6.1}
        \overline{\int_{a}^{b}} f \text{d} x &= \inf U(P, f), \\
        \label{eq:6.2}
        \underline{\int_{a}^{b}} f \text{d} x &= \sup L(P, f),
    \end{align}
    where the inf and the sup are taken over all partitions $P$ of $[a, b]$. 
    The left members of (\ref{eq:6.1}) and (\ref{eq:6.2}) are called 
    the \emph{upper} and \emph{lower Riemann integrals} of $f$
    over $[a, b]$, respectively.

    If the upper and lower integrals are equal, 
    we say that $f$ is Riemann-integrable on $[a, b]$, 
    we write $f \in \mathscr{R}$ 
    (that is, $\mathscr{R}$ denotes the set of Riemann-integrable functions), 
    and we denote the common value of (\ref{eq:6.1}) and (\ref{eq:6.2}) by
    \begin{equation}
        \label{eq:6.3}
        \int_{a}^{b} f \text{d}x,
    \end{equation}
    or by 
    \begin{equation}
        \label{eq:6.4}
        \int_{a}^{b} f(x) \text{d}x,
    \end{equation}
    This is the \emph{Riemann integral} of $f$ over $[a, b]$. 
    Since $f$ is bounded, there exist two numbers, $m$ and $M$, such that
    \begin{equation*}
        m \leq f(x) \leq M \quad 
        (a \leq x \leq b).
    \end{equation*}
    Hence, for every $P$,
    \begin{equation*}
        m(b - a) \leq L(P,f) \leq U(P,f) \leq M(b - a),
    \end{equation*}
    so that the numbers $L(P,f)$ and $U(P,f)$ form a bounded set. 
    This shows \emph{that the upper and lower integrals} are defined for every bounded function $f$.
    The question of their equality, 
    and hence the question of the integrability of $f$, is a more delicate one. 
    Instead of investigating it separately for the Riemann integral,
    we shall immediately consider a more general situation.
}

