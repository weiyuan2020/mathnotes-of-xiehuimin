% chap05sec05
\section{Derivatives of higher order}

\begin{mydef}
    \label{mydef:5.14}
    If $f$ has a derivative $f'$ on an interval,
    and if $f'$ is itself differentiable, 
    we denote the derivative of $f'$ by $f''$ and 
    call $f''$ the second derivative of $f$.
    Continuing in this manner, we obtain functions
    \begin{equation*}
        f,f',f'', f^{(3)}, \dots ,f^{(n)}, 
    \end{equation*} 
    each of which is the derivative of the preceding one.
    $f^{(n)}$ is called the $n$th derivative, 
    or the derivative of order $n$, of $f$.
    
    In order for $f^{(n)}(x)$ to exist at a point $x$, 
    $f^{(n-1)}(t)$ must exist in a neighborhood of $x$ 
    (or in a one-sided neighborhood, 
    if $x$ is an endpoint of the interval on which $f$ is defined), 
    and $f^{(n-1)}$ must be differentiable at $x$. 
    Since $f^{(n-1)}$ must exist in a neighborhood of $x$,
    $f^{(n-2)}$ must be differentiable in that neighborhood.
\end{mydef}
\mybox{高阶导数 这里关于高阶导数的内容不多 
其他课本给出二阶导数符号与函数图像(如果存在的话)的关系: 
二阶导为正, 函数下凸, 二阶导为负, 函数上凸}