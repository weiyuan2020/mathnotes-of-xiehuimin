% chap03sec14
\section{Rearrangements}
\mybox{重排 

绝对收敛级数, 重排后收敛结果不变.
而条件收敛级数, 重排后可以收敛至 $[-\infty , +\infty ]$ 的任意值.
}
\begin{mydef}
    \label{mydef:3.52}
    Let $\sequence{k_n}, n=1,2,3,\dots,$ be a sequence in which every positive integer appears once and only once 
    (that is, $\sequence{k_n}$ is a 1-1 function from $J$ onto $J$, in the notation of Definition \ref{mydef:2.2}).
    Putting
    \begin{equation*}
        a'_n = a_{k_n} \quad (n = 1,2,3,\dots),
    \end{equation*}
    we say that $\sum a'_n$ is a rearrangements of $\sum a_n$.
\end{mydef}

If 
$\sequence{s_n}$ ,
$\sequence{s'_n}$ 
are the sequences of partial sums of 
$\sequence{a_n}$ ,
$\sequence{a'_n}$ ,
it is easily seen that, in general,
these two sequences consist of entirely different numbers.
We are thus led to the problem of determining under what conditions all rearrangements of a convergent series will converge and whether the sums are necessarily the same.

\begin{myExample}
    Consider the convergent series
    \begin{equation}
        \label{eq:3.22}
        1
        -\frac{1}{2}
        +\frac{1}{3}
        -\frac{1}{4}
        +\frac{1}{5}
        -\frac{1}{6}
        +\cdots
    \end{equation}
    and one of its rearrangements
    \begin{equation}
        \label{eq:3.23}
        1
        +\frac{1}{3}
        -\frac{1}{2}
        +\frac{1}{5}
        +\frac{1}{7}
        -\frac{1}{4}
        +\frac{1}{9}
        +\frac{1}{11}
        -\frac{1}{6}
        +\cdots
    \end{equation}
    in which two positive terms are always followed by one negative.
    If $s$ is the sum of (\ref{eq:3.22}), then
    \begin{equation*}
        s < 1 - \frac{1}{2} + \frac{1}{3} = \frac{5}{6}.
    \end{equation*}
    Since
    \begin{equation*}
        \frac{1}{4k-3} + 
        \frac{1}{4k-1} - 
        \frac{1}{2k} > 0
    \end{equation*}
    for $k \geq 1$ , we see that $s'_3 < s'_6 < s'_9 < \cdots$ , where $s'_n$ is $n$th partial sum of (\ref{eq:3.23}).\\    
    Hence
    \begin{equation*}
        \limsup_{n \to \infty} s'_n > s'_3 = \frac{5}{6},
    \end{equation*}
    so that (\ref{eq:3.23}) certainly does not converge to $s$.
    [verigy (\ref{eq:3.23}) converge]
    my:
    \begin{align*}
        \frac{1}{4k-3} + 
        \frac{1}{4k-1} - 
        \frac{1}{2k} 
        &= \frac{(4k-3) + (4k-1)}{(4k-3)(4k-1)} - \frac{1}{2k} \\
        &= \frac{(8k-4)}{(4k-3)(4k-1)} - \frac{1}{2k} \\
        &= \frac{(8k-4)2k - (4k-3)(4k-1)}{(4k-3)(4k-1)2k} \\
        &= \frac{8k-3}{(4k-3)(4k-1)2k} \\
        &< \frac{4}{(4k-3)(4k-1)} \\
        &< \frac{4}{(4(k-1))^2}
        = \frac{}{4(k-1)^2}
    \end{align*}
    because $\sum 1/n^2$ converge, (\ref{eq:3.23}) converge.
\end{myExample}

This example illustrates the following theorem, due to Riemann.

\begin{thm}
    \label{thm:3.54}
    Let $\sum a_n$ be a series of real numbers which converges, but not absolutely.
    Suppose
    \begin{align*}
        -\infty 
        \leq \alpha 
        \leq \beta 
        \leq \infty.
    \end{align*}
    Then there exists a rearrangement $\sum a'_n$ with partial sum $\sum s'_n$ such that 
    \begin{equation}
        \label{eq:3.24}
        \liminf_{n \to \infty} s'_n = \alpha,\quad
        \limsup_{n \to \infty} s'_n = \beta.
    \end{equation}
\end{thm}

\begin{proof}
    Let
    \begin{equation*}
        p_n = \frac{\left| a_n \right| + a_n}{2}, \quad
        q_n = \frac{\left| a_n \right| - a_n}{2}, \quad
        (n = 1,2,3,\dots).
    \end{equation*}
    Then $p_n - q_n = a_n$ , $p_n + q_n = a_n$ , $p_n \geq 0$ , $q_n \geq 0$ .
    The series $\sum p_n, \sum q_n$ must both diverge.

    For if both were convergent, then 
    \begin{equation*}
        \sum (p_n + q_n) = \sum \left| a_n \right| 
    \end{equation*}
    would converge, contrary to hypothesis. Since
    \begin{equation*}
        \sum_{n=1}^{N} a_n = 
        \sum_{n=1}^{N} (p_n - q_n) = 
        \sum_{n=1}^{N} p_n -
        \sum_{n=1}^{N} q_n ,
    \end{equation*}
    divergence of $\sum p_n$ and convergence of $\sum q_n$ (or vice versa) implies divergence of $\sum a_n$ , again contrary to hypothesis.

    Now let $P_1, P_2, P_3, \dots$ denote the nonnegative terms of $\sum a_n$, in the order in which they occur,
    and let $Q_1, Q_2, Q_3, \dots$ be the absolute values of the negative terms of $\sum a_n$, also in their original order.

    The series $\sum P_n$ , $\sum Q_n$ differ from $\sum p_n$ , $\sum q_n$ only by zero terms, and are therefore divergent.

    We shall construct sequences $\sequence{m_n}$, $\sequence{k_n}$ , such that the series 
    \begin{equation}
        \label{eq:3.25}
        \begin{aligned}
            P_1 
            &+ \cdots + P_{m_1} - Q_1 - \cdots - Q_{k_1} \\
            &+ P_{m_1 + 1} + \cdots + P_{m_2} - Q_{k_1 + 1} - \cdots - Q_{k_2} + \cdots,
        \end{aligned}
    \end{equation} 
    which clearly is a rearrangement of $\sum a_n$ , satisfies (\ref{eq:3.24}).

    Choose real-valued sequence $\sequence{\alpha_n}, \sequence{\beta_n}$ such that $\alpha_n \rightarrow \alpha$ , $\beta_n \rightarrow \beta$ ,
    $\alpha_n < \beta_n$ , $\beta_1 > 0$.
    
    Let $m_1, k_1$ be the smallest integers such that
    \begin{align*}
        &P_1 + \cdots + P_{m_1} > \beta_1,\\
        &P_1 + \cdots + P_{m_1} - Q_1 - \cdots - Q_{k_1} < \alpha_1;
    \end{align*}
    let $m_2, k_2$ be the smallest integers such that
    \begin{align*}
        &P_1 + \cdots + P_{m_1} - Q_1 - \cdots - Q_{k_1} + P_{m_1 + 1} + \cdots + P_{m_2} > \beta_2,\\
        &P_1 + \cdots + P_{m_1} - Q_1 - \cdots - Q_{k_1} + P_{m_1 + 1} + \cdots + P_{m_2} - Q_{k_1 + 1} - \cdots - Q_{k_2} < \alpha_2;
    \end{align*}
    and continues in this way. This is possible since 
    $\sum P_n$ and 
    $\sum Q_n$ diverge.

    If $x_n, y_n$ denote the partial sums of (\ref{eq:3.25}) whose last terms are $P_{m_n}$ , $-Q_{k_n}$ , then 
    \begin{equation*}
        \left| x_n - \beta_n \right| \leq P_{m_n}, \quad
        \left| y_n - \alpha_n \right| \leq Q_{k_n}.
    \end{equation*}
    Since $P_n \rightarrow 0$ and $Q_n \rightarrow 0$ as $n \rightarrow \infty$ , we see that $x_n \rightarrow \beta, y_n \rightarrow \alpha$ .

    Finally, it is clearly that no number less than $\alpha$ or greater than $\beta$ can be a subsequential limit of the partial sums of (\ref{eq:3.25}).
\end{proof}

\begin{thm}
    \label{thm:3.55}
    If $\sum a_n$ is a series of complex numbers which converges absolutely,
    then every rearrangement of $\sum a_n$ converges,
    and they all converge to the same sum.
\end{thm}

\begin{proof}
    Let $\sum a'_n$ be a rearrangement, with partial sums $\sum s'_n$ .
    Given $\varepsilon > 0$ ,
    there exists an integer $N$ such that $m \geq n \geq N$ implies
    \begin{equation}
        \label{eq:3.26}
        \sum_{i=n}^{m} \left| a_i \right| \leq \varepsilon.
    \end{equation}
    Now choose $p$ such that the integers $1,2,\dots,N$ are all contained in the set $k_1, k_2, \dots, k_p$ 
    (we use the notation of Definition \ref{mydef:3.52}).
    Then if $n>p$ , the numbers $a_1, a_2, \dots, a_N$ will cancel in the difference $s_n - s'_n$ , 
    so that $\left| s_n - s'_n \right| \leq \varepsilon$ , by (\ref{eq:3.26}).
    Hence $\sequence{s'_n}$ converges to the same sum as $\sequence{s_n}.$  
\end{proof}