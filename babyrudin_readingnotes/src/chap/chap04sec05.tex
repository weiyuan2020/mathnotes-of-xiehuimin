% chap04sec05
\section{Discontinuities}

If $x$ is a point in the domain of definition of the function $f$ at which $f$ is not continuous, 
we say that $f$ is \emph{discontinuous} at $x$, 
or that $f$ has a \emph{discontinuity} at $x$.
If $f$ is defined on an interval or on a segment, 
it is customary to divide discontinuities into two types. 
Before giving this classification, 
we have to define the right-hand and the left-hand limits of $f$ at $x$, 
which we denote by $f(x+)$ and $f(x-)$, respectively.

\begin{myDef}
    \label{myDef:4.25}
    Let $f$ be defined on $(a, b)$. 
    Consider any point $x$ such that $a \leq x < b$. 
    We write
    \begin{equation*}
        f(x+) = q        
    \end{equation*}
    if $f(t_n) \rightarrow q$ as $n \rightarrow \infty$, 
    for all sequences $\sequence{t_n}$ in $(x, b)$ such that $t_n \rightarrow x$. 
    To obtain the definition of $f(x-)$, 
    for $a < x \leq b$, 
    we restrict ourselves to sequences $\sequence{t_n}$ in $(a, x)$.
    It is clear that any point $x$ of $(a, b)$, 
    $\lim_{t \to x} f(t)$ exists if and only if
    \begin{equation*}
        f(x+) = f(x-) = \lim_{t \to x} f(t).
    \end{equation*}
\end{myDef}

\begin{myDef}
    \label{myDef:4.26}
    Let $f$ be defined on $(a, b)$. 
    If $f$ is discontinuous at a point $x$,
    and if $f(x +)$ and $f (x-)$ exist, 
    then $f$ is said to have a discontinuity of the \emph{first kind}, 
    or a \emph{simple discontinuity}, at $x$. 
    Otherwise the discontinuity is said to be of the \emph{second kind}.

    There are two ways in which a function can have a simple discontinuity:
    either $f(x+) \neq f(x-)$ [in which case the value $f(x)$ is immaterial], 
    or $f(x+) = f(x-) \neq f(x)$.
\end{myDef}
\mybox{
    第一类间断点和第二类间断点, \\
    第一类间断点也称为可去间断点, 跳跃间断点.\\ 
    第二类间断点也称为无穷间断点, 震荡间断点.
}

\begin{myExample}
    \begin{asparaenum}[(a)]
        \item Define 
    \begin{equation*}
        f(x) = \left\{
            \begin{array}{lc}
                1 & (x \text{ rational}),\\
                0 & (x \text{ irrational}).
            \end{array}
        \right.
    \end{equation*}
    Then $f$ has a discontinuity of the second kind at every point $x$. 
    since neither $f(x+)$ nor $f(x-)$ exists.
    \item Define
    \begin{equation*}
        f(x) = \left\{
            \begin{array}{lc}
                x & (x \text{ rational}),\\
                0 & (x \text{ irrational}).
            \end{array}
        \right.
    \end{equation*}
    Then $f$ is continuous at $x = 0$ and has a discontinuity of the second kind at every other point.

    \item Define 
    \begin{equation*}
        f(x) = \left\{
            \begin{array}{lc}
                 x + 2  & (-3 <    x < -2),\\
                -x - 2  & (-2 \leq x <  0),\\
                 x + 2  & ( 0 \leq x <  1).
            \end{array}
        \right.
    \end{equation*}
    Then $f$ has a simple discontinuity at $x = 0$ and is continuous at every other point of $(-3, 1)$.
    \item Define
    \begin{equation*}
        f(x) = \left\{
            \begin{array}{lc}
                \sin \frac{1}{x} & (x \neq 0),\\
                0 & (x = 0).
            \end{array}
        \right.
    \end{equation*}
    Since neither $f(0+)$ nor $f(0-)$ exists,
    $f$ has a discontinuity of the second kind at $x = 0$. 
    We have not yet shown that $\sin x$ is a continuous function. 
    If we assume this result for the moment, 
    Theorem \ref{thm:4.7} implies that $f$ is continuous at every point $x \neq 0$.
    \end{asparaenum}
\end{myExample}