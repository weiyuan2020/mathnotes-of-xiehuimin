chap10sec02

\section{Primitive mappings}

\mybox{本原映射}

\begin{mydef}
    \label{def:10.5}
    If $\mathbf{G}$ maps an open set $E \subset \R^n$ into $\R^n$, 
    and if there is an integer $m$ and a real function $g$ with domain $E$ such that
    \begin{equation}
        \label{eq:10.9}
        \mathbf{G(x)} = \sum_{i \neq m} x_i \mathbf{e}_i + g(\mathbf{x}) \mathbf{e}_m
        \quad 
        (\mathbf{x} \in E) ,
    \end{equation}
    then we call $\mathbf{G}$ \emph{primitive}.
    A primitive mapping is thus one that changes at most one coordinate.
    Note that (\ref{eq:10.9}) can also be written in the form 
    \begin{equation}
        \label{eq:10.10}
        \mathbf{G(x)} = \mathbf{x} + \left[ g(\mathbf{x}) - x_m \right] \mathbf{e}_m .
    \end{equation}
    
    If $g$ is differentiable at some point $\mathbf{a} \in E$, so is $\mathbf{G}$.
    The matrix $[\alpha_{ij}]$ of the operator $\mathbf{G'(a)}$ has 
    \begin{equation}
        \label{eq:10.11}
        (D_1 g)(\mathbf{a}),...
        (D_m g)(\mathbf{a}),...
        (D_n g)(\mathbf{a})
    \end{equation}
    as ots $m$th row.
    For $j \neq m$, we have $\alpha_{jj} = 1$ and $\alpha_{ij} = 0$ if $i \neq j$.
    The Jacobian of $\mathbf{G}$ at $\mathbf{a}$ is thus given by 
    \begin{equation}
        \label{eq:10.12}
        J_{\mathbf{G}}(\mathbf{a}) = 
        \det [\mathbf{G'(a)}] = 
        (D_m g)(\mathbf{a}),
    \end{equation}
    and we see (by Theorem \ref{thm:9.36}) that $\mathbf{G'(a)}$ is 
    \emph{invertible if and only if} $(D_m g)(\mathbf{a}) \neq 0$.
\end{mydef}

\begin{mydef}
    \label{def:10.6}
    A linear operator $B$ on $\R^n$ that interchanges some pair of members of the standard basis and leaves the others fixed will be called a \emph{flip}.

    For example, the flip $B$ on $\R^4$ that interchanges $e_2$ and $e_4$ has the form
    \begin{equation}
        \label{eq:10.13}
        B(x_1 \mathbf{e}_1 + 
        x_2 \mathbf{e}_2 + 
        x_3 \mathbf{e}_3 + 
        x_4 \mathbf{e}_4) = 
        x_1 \mathbf{e}_1 + 
        x_2 \mathbf{e}_4 + 
        x_3 \mathbf{e}_3 + 
        x_4 \mathbf{e}_2
    \end{equation}
    or, equivalently,
    \begin{equation}
        \label{eq:10.14}
        B(x_1 \mathbf{e}_1 + 
        x_2 \mathbf{e}_2 + 
        x_3 \mathbf{e}_3 + 
        x_4 \mathbf{e}_4) = 
        x_1 \mathbf{e}_1 + 
        x_4 \mathbf{e}_2 +
        x_3 \mathbf{e}_3 + 
        x_2 \mathbf{e}_4 .
    \end{equation}
    Hence $B$ can also be thought of as interchanging two of the coordinates,
    rather that two basis vectors.

    In the proof that follows, we shall use the projections $P_0,\dots,P_n$ in $\R^n$, defined by $P_0 \mathbf{x = 0}$ and 
    \begin{equation}
        \label{eq:10.15}
        P_m \mathbf{x} = 
        x_1 \mathbf{e}_1 + \dots 
        x_m \mathbf{e}_m 
    \end{equation}
    for $1 \leq m \leq n$.
    Thus $P_m$ is the projection whose range and null space are spanned by $\{\mathbf{e_1,...,e_m}\}$ and $\{\mathbf{e_{m+1},...,e_n}\}$,
    respectively.
\end{mydef}

\begin{thm}
    \label{thm:10.7}
    Suppose $\mathbf{F}$ is a $\mathscr{C}'$-mapping of an open set $E \subset \R^n$ into $\R^n$, $\mathbf{0} \in E$, $\mathbf{F(0) = 0}$, and $\mathbf{F'(0)}$ is invertible.
    
    Then there is a neighborhood of $0$ in $\R^n$ in which a representation
    \begin{equation}
        \label{eq:10.16}
        \mathbf{F(x)} = B_1 \cdots B_{n-1} \mathbf{G_n \circ \cdots \circ G_1(x)}
    \end{equation}
    is valid.

    In (\ref{eq:10.16}), each $\mathbf{G}_i$ is a primitive $\mathscr{C}'$-mapping in some neighborhood of $\mathbf{0}$;
    $\mathbf{G}_i(\mathbf{0})=\mathbf{0}$, $\mathbf{G}'_i(\mathbf{0})$ is invertible, and each $B_i$ is either a flip or the identity operator.
\end{thm}

Briefly, (\ref{eq:10.16}) represents $\mathbf{F}$ locally as a composition of primitive mappings and flips

% todo add proof


