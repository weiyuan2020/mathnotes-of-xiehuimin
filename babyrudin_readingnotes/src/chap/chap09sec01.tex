% chap09sec01
\section{Linear transformations}
\mybox{线性变换}
We begin this chapter with a discussion of sets of vectors in euclidean $n$-space $\R^n$.
The algebraic facts presented here extend without change to finite-dimensional vector spaces over any field of scalars. 
However, for our purposes it is quite sufficient to stay within the familiar framework provided by the euclidean spaces.
\mybox{有限维向量空间}

\begin{myDef}
    \label{myDef:9.1}
    \begin{asparaenum}[(a)]
        \item A nonempty set $X \subset \R^n$ is a \emph{vector space} if $\mathbf{x + y} \in X$ and $c \mathbf{x} \in X$ for all $x \in X$, $y \in X$, and for all scalars $c$. 
        \item If $\mathbf{x}_1, ... , \mathbf{x}_k$ $\in \R^n$ and $c_1, ... , c_k$ are scalars, the vector
        \begin{equation*}
            c_1 \mathbf{x}_1 + \cdots
            c_k \mathbf{x}_k 
        \end{equation*}
        is called a \emph{linear combination} of $\mathbf{x}_1, ... , \mathbf{x}_k$ .
        If $S \subset \R^n$ and if $E$ is the set of all linear combinations of elements of $S$, we say that $S$ \emph{spans} $E$, or that $E$ \emph{is the span of} $S$.

        Observe that every span is a vector space 
        \item A set consisting of vectors  $\mathbf{x}_1, ... , \mathbf{x}_k$  
        (we shall use the notation $\{\mathbf{x}_1, ... , \mathbf{x}_k\}$  for such a set) 
        is said to be \emph{independent} if the relation
        $c_1 \mathbf{x}_1 + \cdots + c_k \mathbf{x}_k = \mathbf{0}$
        implies that $c_1 = \dots = c_k = 0$. 
        Otherwise  $\{\mathbf{x}_1, ... , \mathbf{x}_k\}$ is said to be \emph{dependent}.

        Observe that no independent set contains the null vector.
        \item If a vector space $X$ contains an independent set of $r$ vectors but contains no independent set of $r+1$ vectors, we say that X has \emph{dimension} $r$, and write: $\dim X = r$.
        
        The set consisting of $\mathbf{0}$ alone is a vector space its dimension is $0$.
        \item An independent subset of a vector space $X$ which spans $X$ is called a \emph{basis} of $X$. 
        
        Observe that if $B = \{\mathbf{x}_1, ... , \mathbf{x}_r\}$ is a basis of $X$, then every $\mathbf{x} \in X$ has a unique representation of the form $\mathbf{x} = \sum c_j \mathbf{x}_j$.
        Such a representation exists since $B$ spans $X$,
        and it is unique since $B$ is independent.
        The numbers $c_1, \dots, c_r$ are called the \emph{coordinates of} $\mathbf{x}$ with respect to the basis $B$.

        The most familiar example of a basis is the set $\{\mathbf{e}_1, \dots, \mathbf{e}_n\}$,
        where $\mathbf{e}_j$ is the vector in $\R^n$ whose $j$th coordinate is $1$ and whose other coordinates are all $0$.
        If $\mathbf{x} \in \R^n$, $\mathbf{x} = (x_1, \dots ,x_n)$,
        then $\mathbf{x} = \sum x_j \mathbf{e}_j$.
        We shall call 
        \begin{equation*}
            \{\mathbf{e}_1, \dots , \mathbf{e}_n\}
        \end{equation*}
        the \emph{standard basis} of $\R^n$.
    \end{asparaenum}

\end{myDef}

\mybox{
    部分名词对应汉语
    \begin{enumerate}[(a)]
        \item vector space 向量空间
        \item linear combination 线性组合, span 张成的空间
        \item independent 线性无关, 线性独立; dependent 线性相关
        \item dimension 维度
        \item basis 基; coordinate 坐标; standard basis 标准基
    \end{enumerate}
}

\begin{thm}
    \label{thm:9.2}
    Let $r$ be a positive integer. 
    If a vector space $X$ is spanned by a set of $r$ vectors, 
    then $\dim X \leq r$.
\end{thm}

% todo add proof

\begin{myCorollary*}
    $\dim \R^n = n$.
\end{myCorollary*}

\begin{proof}
    Since $\{\mathbf{e}_1, \dots , \mathbf{e}_n\}$ spans $\R^n$,
    the theorem shows that $\dim \R^n \leq n$.
    Since $\{\mathbf{e}_1, \dots , \mathbf{e}_n\}$ is independent,
    $\dim \R^n \geq n$.
\end{proof}

\begin{thm}
    \label{thm:9.3}
    Suppose $X$ is a vector space, and $\dim X = n$.
    \begin{enumerate}[(a)]
        \item A set $E$ pf $n$ vectors in $X$ spans $X$ if and only if $E$ is independent.
        \item $X$ has a basis, and every basis consists of $n$ vectors.
        \item If $1 \leq r \leq n$ and $\{\mathbf{y}_1, \dots , \mathbf{y}_r\}$ is an independent set in $X$, then $X$ has a basis containing $\{\mathbf{y}_1, \dots , \mathbf{y}_r\}$.
    \end{enumerate}
\end{thm}

% todo add proof

\begin{myDef}
    A mapping $A$ of a vector space $X$ into a vector space $Y$ is said to be a \emph{linear transformation} if 
    \begin{equation*}
        A(\mathbf{x}_1 + \mathbf{x}_2) =
        A\mathbf{x}_1 + A\mathbf{x}_2 ,
        \quad
        A(c\mathbf{x}) = 
        cA(\mathbf{x})
    \end{equation*}
    for all $\mathbf{x}$, $\mathbf{x}_1, \mathbf{x}_2 \in X$ and all scalars $c$.
    Note that one often writes $A \mathbf{x}$ instead of $A(\mathbf{x})$ if $A$ is linear.

    Observe that $A \mathbf{0} = \mathbf{0}$ if $A$ is linear.
    Observe also that a linear transformation $A$ of $X$ into $Y$ is completely determined by its action on any basis:
    If $\{\mathbf{x}_1, \dots , \mathbf{x}_n\}$ is a basis of $X$,
    then every $\mathbf{x} \in X$ has a unique representation of the form 
    \begin{equation*}
        \mathbf{x} = \sum_{i=1}^{n} c_i \mathbf{x}_i ,
    \end{equation*}
    and the linearity of $A$ allows us to compute $A \mathbf{x}$ from the vectors $A \mathbf{x}_1, \dots , A \mathbf{x}_n$ and the coordinates $c_1, \dots, c_n$ by the formula 
    \begin{equation*}
        A \mathbf{x} = \sum_{i=1}^{n} c_i A \mathbf{x}_i .
    \end{equation*}

    Linear transformations of $X$ into $X$ are often called \emph{linear operators} on $X$.
    If $A$ is a linear operator on $X$ which 
    \begin{inparaenum}[(i)]
        \item is one-to-one and 
        \item maps $X$ onto $X$,
    \end{inparaenum}
    we say that $A$ is \emph{invertible}.
    In this case we can define an operator $A^{-1}$ on $X$ by requiring that $A^{-1}(A \mathbf{x}) = \mathbf{x}$ for all $\mathbf{x} \in X$.
    It is trivial to verify that we then also have $A(A^{-1}\mathbf{x}) = \mathbf{x}$, for all $\mathbf{x} \in X$, and that $A^{-1}$ is linear.
\end{myDef}

An important fact about linear operators on finite-dimensional vector spaces is that each of the above conditions (i) and (ii) implies the other:

\begin{thm}
    \label{thm:9.5}
    A linear operator $A$ on a finite-dimensional vector space $X$ is
    one-to-one if and only if the range of $A$ is all of $X$.
\end{thm}

% todo add proof

\begin{myDef}
    \label{myDef:9.6}
    \begin{asparaenum}[(a)]
        \item Let $L(X,Y)$ be the set of all linear transformations of the vector space $X$ into the vector space $Y$. 
        Instead of $L(X, X)$, we shall simply write $L(X)$.
        If $A_1, A_2 \in L(X, Y)$ and if $c_1, c_2$ are scalars, define $c_1 A_1 + c_2 A_2$ by 
        \begin{equation*}
            (c_1 A_1 + c_2 A_2) \mathbf{x} =
            c_1 A_1 \mathbf{x} + c_2 A_2 \mathbf{x} 
            \quad
            (\mathbf{x} \in X).
        \end{equation*}
        It is then clear that $c_1 A_1 + c_2 A_2 \in L(X, Y)$.
        \item If $X, Y, Z$ are vector spaces, and if $A \in L(X, Y)$ and $B \in L(Y, Z)$, we define their \emph{product} $BA$ to be the composition of $A$ and $B$:
        \begin{equation*}
            (BA)\mathbf{x} = 
            B(A\mathbf{x}) 
            \quad 
            (\mathbf{x} \in X).
        \end{equation*}
        Then $BA \in L(X, Z)$.

        Note that $BA$ need not be the same as $AB$, even if $X = Y = Z$.
        \item For $A \in L(\R^n, \R^m)$, define the \emph{norm} $\left\| A \right\| $ of $A$ to be the sup of all numbers $\left| A \mathbf{x} \right| $ , where $\mathbf{x}$ ranges over all vectors in $\R^n$ with $\left| x \right| \leq 1$.
        
        Observe that the inequality 
        \begin{equation*}
            \left| A \mathbf{x} \right| \leq
            \left\| A \right\| \left| \mathbf{x} \right| 
        \end{equation*}
        holds for all $\mathbf{x} \in \R^n$. 
        Also, if $\lambda$ is such that $\left| A \left| x \right| \right| \leq \lambda \left| \mathbf{x} \right|$ for all $\mathbf{x} \in \R^n$, then $\left\| A \right\| \leq \lambda$.
    \end{asparaenum}
\end{myDef}

\mybox{
    norm 范数 \url{https://mathworld.wolfram.com/Norm.html} 

    The norm of a mathematical object is a quantity that in some (possibly abstract) sense describes the length, size, or extent of the object. 
    Norms exist for complex numbers (the complex modulus, sometimes also called the complex norm or simply "the norm"), Gaussian integers (the same as the complex modulus, but sometimes unfortunately instead defined to be the absolute square), quaternions (quaternion norm), vectors (vector norms), and matrices (matrix norms). 
    A generalization of the absolute value known as the $p$-adic norm is also defined.

    Norms are variously denoted $|x|, |x|_p, ||x||$, or $||x||_p$. 
    In this work, single bars are used to denote the complex modulus, quaternion norm, $p$-adic norms, and vector norms, while the double bar is reserved for matrix norms.

    The term "norm" is often used without additional qualification to refer to a particular type of norm (such as a matrix norm or vector norm). Most commonly, the unqualified term "norm" refers to the flavor of vector norm technically known as the $L2$-norm. This norm is variously denoted $||x||_2$, $||x||$, or $|x|$, and gives the length of an $n$-vector $x=(x_1,x_2,...,x_n)$. 
    It can be computed as 
    \begin{equation*}
        |x|=\sqrt{(x_1^2+x_2^2+...+x_n^2)}.
    \end{equation*}

    The norm of a complex number, 2-norm of a vector, or 2-norm of a (numeric) matrix is returned by Norm[expr]. Furthermore, the generalized p-norm of a vector or (numeric) matrix is returned by Norm$[expr, p]$.

    The norm (length) of a vector should not be confused with a normal vector (a vector perpendicular to a surface).
}

\begin{thm}
    \label{thm:9.7}
    \begin{asparaenum}[(a)]
        \item If $A \in L(\R^n, \R^m)$, then $\left\| A \right\| < \infty$ and $A$ is a uniformly continuous mapping of $\R^n$ into $\R^m$.
        \item If $A, B \in L(\R^n, \R^m)$ and $c$ is a scalar, then 
        \begin{equation*}
            \left\| A + B \right\| \leq
            \left\| A \right\| + \left\| B \right\| ,
            \quad 
            \left\| cA \right\| = \left| c \right| \left\| A \right\| .
        \end{equation*}
        With the distance between $A$ and $B$ defined as $\left\| A - B \right\|$, $L(\R^n, \R^m)$ is a metric space.
        \item If $A \in L(\R^n, \R^m)$, and $B \in L(\R^m, \R^k)$, then 
        \begin{equation*}
            \left\| BA \right\| \leq \left\| B \right\| \left\| A \right\| .
        \end{equation*}
    \end{asparaenum}
\end{thm}

% todo add proof

Since we now have metrics in the spaces $L(\R^n, \R^m)$, the concepts of open set, continuity, etc., make sense for these spaces. 
Our next theorem utilizes these concepts.
\mybox{utilizes 使用, 利用}

\begin{thm}
    \label{thm:9.8}
    Let $\Omega$ be the set of all invertible linear operators on $\R^n$.
    \begin{asparaenum}[(a)]
        \item If $A \in \Omega, B \in L(\R^n)$, and 
        \begin{equation*}
            \left\| B - A \right\| \cdot \left\| A^{-1} \right\| < 1,
        \end{equation*}
        then $B \in \Omega$.
        \item $\Omega$ is an open subset of $L(\R^n)$, 
        and the mapping $A \rightarrow A^{-1}$ is continuous on $\Omega$.
    \end{asparaenum}
    (This mapping is also obviously a 1-1 mapping of $\Omega$ onto $\Omega$, which is its own inverse.)
\end{thm}

% todo add proof

\begin{myDef}
    \textbf{Matrices} 
    Suppose $\{\mathbf{x}_1, \dots , \mathbf{x}_n\}$ 
    and $\{\mathbf{y}_1, \dots , \mathbf{y}_n\}$ are bases if vector spaces $X$ and $Y$, respectively.
    Then every $A \in L(X, Y)$ determines a set of numbers $a_{ij}$ such that 
    \begin{equation}
        \label{eq:9.3}
        A \mathbf{x}_j = 
        \sum_{i=1}^{m} a_{ij} \mathbf{y}_i
        \quad 
        (1 \leq j \leq n).
    \end{equation}
    It is convenient to visualize these numbers in a rectangular array of $m$ rows and $n$ columns, called an $m$ by $n$ \emph{matrix}:
    \begin{equation*}
        \left[ A \right] = 
        \begin{bmatrix}
            a_{11} & a_{12} & \cdots & a_{1n} \\
            a_{21} & a_{22} & \cdots & a_{2n} \\
            \cdots & \cdots & \cdots & \cdots \\
            a_{m1} & a_{m2} & \cdots & a_{mn} \\
        \end{bmatrix}
    \end{equation*}
    Observe that the coordinates $a_{ij}$ of the vector $A \mathbf{x}_j$ 
    (with respect to the basis $\{y_1, ... , y_m\}$) appear in the $j$th column of $[A]$. 
    The vectors $A \mathbf{x}_j$ are therefore sometimes called the \emph{column vectors} of $[A]$. 
    With this terminology, the \emph{range of} $A$ \emph{is spanned by the column vectors of} $[A]$.

    If $\mathbf{x} = \sum c_j \mathbf{x}_j$, the linearity of $A$ , combined with (\ref{eq:9.3}), shows that 
    \begin{equation}
        \label{eq:9.4}
        A \mathbf{x} = 
        \sum_{i=1}^{m} \left( 
            \sum_{j=1}^{n} a_{ij} c_j
         \right) 
         \mathbf{y}_i .
    \end{equation}
    Thus the coordinates of $A \mathbf{x}$ are $\sum_j a_{ij} c_j$ 
    Note that in (\ref{eq:9.3}) the summation ranges over the first subscript of $a_{11}$ , 
    but that we sum over the second subscript when computing coordinates.

    Suppose next that an $m$ by $n$ matrix is given, with real entries $a_{ij}$ . 
    If $A$ is then defined by (\ref{eq:9.4}), 
    it is clear that $A \in L(X, Y)$ and that $[A]$ is the given matrix.
    Thus there is a natural 1-1 correspondence between $L(X, Y)$ and the set of all real $m$ by $n$ matrices. 
    We emphasize, though, that $[A]$ depends not only on $A$ but also on the choice of bases in $X$ and $Y$. 
    The same $A$ may give rise to many different matrices if we change bases, and vice versa. 
    We shall not pursue this observation any further, since we shall usually work with fixed bases. 
    (Some remarks on this may be found in Sec. 9.37.)
    If $Z$ is a third vector space, with basis $\{z_1, ... , z_p\}$, 
    if $A$ is given by (\ref{eq:9.3}), and if 
    \begin{equation*}
        B \mathbf{y}_i = \sum_k b_{ki} \mathbf{z}_k, \quad
        (BA) \mathbf{x}_j = \sum_k c_{kj} \mathbf{z}_k, 
    \end{equation*}
    then $A \in L(X, Y)$, $B \in L(Y, Z)$, $BA \in L(X, Z)$, and since
    \begin{align*}
        B(A \mathbf{x}_j)
        &= B \sum_i a_{ij} \mathbf{y}_i 
        = \sum_i a_{ij} B \mathbf{y}_i \\
        &= \sum a_{ij} \sum_k b_{ki} \mathbf{z}_k 
        = \sum_k \left( \sum_i b_{ki} a_{ij} \right) \mathbf{z}_k,
    \end{align*}
    the independence of $\{\mathbf{z}_1,...,\mathbf{z}_p\}$ implies that 
    \begin{equation}
        \label{eq:9.5}
        c_{kj} = \sum_i b_{ki} a_{ij}
        \quad 
        (1 \leq k \leq p,1 \leq j \leq n).
    \end{equation}
    This shows how to compute the $p$ by $n$ matrix $[BA]$ from $[B]$ and $[A]$.
    If we define the product $[B][A]$ to be $[BA]$ , then (\ref{eq:9.5}) describes the usual rule of matrix multiplication.
    \mybox{矩阵乘法运算}
    Finally, suppose $\{\mathbf{x}_1,...,\mathbf{x}_n\}$ and $\{\mathbf{y}_1,...,\mathbf{y}_n\}$ are standard bases of $\R^n$ and $\R^m$, and $A$ is given by (\ref{eq:9.4}).
    The Schwarz inequality shows that 
    \begin{equation*}
        \left| A \mathbf{x} \right|^2 = 
        \sum_i \left( \sum_j a_{ij} c_j \right)^2 \leq
        \sum_i \left( \sum_j a_{ij}^2 c_j^2 \right) =
        \sum_{i, j} a_{ij}^2 \left| \mathbf{x} \right|^2 .
    \end{equation*}
    Thus 
    \begin{equation}
        \label{eq:9.6}
        \left\| A \right\| \leq 
        \left\{ \sum_{i, j} a_{ij}^2 \right\}^{1/2} .
    \end{equation}

    If we apply (\ref{eq:9.6}) to $B - A$ in place of $A$, 
    where $A, B \in L(\R^n, \R^m)$, 
    we see that if the matrix elements $a_{ij}$ are continuous functions of a parameter, then the same is true of $A$. More precisely:
\end{myDef}

\emph{If $S$ is a metric space, if $\{a_{11}, \dots, a_{mn}\}$ are real continuous functions on $S$,
and if, for each $p \in S$, $A_P$ is the linear transformation of $\R^n$ into $\R^m$ whose matrix has entries $a_{ij}(p)$, 
then the mapping $p \rightarrow A_P$ is a continuous mapping of $S$ into
$L(\R^n, \R^m)$.}