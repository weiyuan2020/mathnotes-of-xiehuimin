% chap02sec05
\section{connected sets}

% 2.45 Definition 
Two subsets A and B of a metric space X are said to be
separated if both A n B and A n B are empty, i.e., if no point of A lies in the
closure of B and no point of B lies in the closure of A.

A set E < Xis said to be connected if E is not a union of two nonempty
separated sets.

2.46 Remark Separated sets are of course disjoint, but disjoint sets need not
be separated. For example, the interval [0,1] and the segment (1, 2) are not
separated, since 1 is a limit point of (I, 2). However, the segments (0, /) and
(1. 2) are separated.

The connected subsets of the line have a particularly simple structure:

2.47 Theorem A subset E of the real line R" is connected if and only if it has the
following property: If xe E, ye E, and x < z <Y, then Z \in  E.

Proof If there exist x \in  E. y \in E, and some z \in (x, y) such that z c E, then
E = A, u B, where

A,= En (-o,z), B,= En (z, ©).
BASIC TOPOLOGY 43

Since x \in A, and y \in B,, A and B are nonempty. Since A, = (-0, z)and
B, = (z, ©), they are separated. Hence E is not connected.

To prove the converse, suppose Eis not connected. Then there are
nonempty separated sets A and B such that A u B=E. Pick x \in  A, y \in  B,
and assume (without loss of generality) that x <y. Define

z=sup(A nx, y]).

By Theorem 2.28, 7 \in A; hence z c B. In particular, x < 7 <y.

If 7 c A, it follows that Xx < z < y and z c E.

If z \in  A, then 7cB, hence there exists z, such that z<z, <y and
72, c B. Thenx<z, <yandz cE.