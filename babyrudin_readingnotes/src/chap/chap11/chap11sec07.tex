% chap11sec07

\section{Comparison with the Riemann integral}

Our next theorem will show that every function which is Riemann-integrable
on an interval is also Lebesgue-integrable, 
and that Riemann-integrable functions are subject to rather stringent continuity conditions. 
Quite apart from the fact that the Lebesgue theory therefore enables us to integrate a much larger class of functions, 
its greatest advantage lies perhaps in the ease with which many limit operations can be handled; 
from this point of view, 
Lebesgue's convergence theorems may well be regarded as the core of the Lebesgue theory.


One of the difficulties which is encountered in the Riemann theory is
that limits of Riemann-integrable functions 
(or even continuous functions)
may fail to be Riemann-integrable. 
This difficulty is now almost eliminated,
since limits of measurable functions are always measurable.

Let the measure space $X$ be the interval $[a, b]$ of the real line, with $\mu = m$
(the Lebesgue measure), and $\mathfrak{M}$ the family of Lebesgue-measurable subsets
of $[a, b]$. Instead of
\begin{equation*}
    \int_X f \d m
\end{equation*}
it is customary to use the familiar notation
\begin{equation*}
    \int_{a}^{b} f \d x
\end{equation*}
for the Lebesgue integral of $f$ over $[a, b]$. 
To distinguish Riemann integrals from Lebesgue integrals, 
we shall now denote the former by
\begin{equation*}
    \mathfrak{R} \int_{a}^{b} f \d x .
\end{equation*}

\begin{thm}
    \label{thm:11.33}
    \begin{asparaenum}[(a)]
        \item If $f \in \mathscr{R}$ on $[a,b]$, then $f \in \mathscr{L}$ on $[a,b]$, 
        and 
        \begin{equation}
            \label{eq:11.87}
            \int_{a}^{b} f \d x = 
            \mathscr{R} \int_{a}^{b} f \d x .
        \end{equation}
        \item Suppose $f$ is bounded on $[a,b]$. 
        Then $f \in \mathscr{R}$ on $[a,b]$ 
        if and only if $f$ is continuous almost everywhere on $[a,b]$.
    \end{asparaenum}
\end{thm}