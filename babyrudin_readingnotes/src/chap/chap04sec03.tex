% chap04sec03

\section{Continuity and compactness}

\begin{myDef}
    \label{myDef:4.13}
    A mapping $\mathbf{f}$ of a set $E$ into $\R{k}$ is said to be \emph{bounded} 
    if there is a real number $M$ such that $\left| f(x) \right| \leq M$ for all $x \in E$.
\end{myDef}

\begin{thm}
    \label{thm:4.14}
    Suppose $f$ is a continuous mapping of a compact metric space $X$ into a metric space $Y$. Then $f(X)$ is compact.
\end{thm}

\myproof{
    Let $\sequence{V_\alpha}$ be an open cover of $f(X)$. Since $f$ is continuous, 
    Theorem \ref{thm:4.8} shows that each of the sets $f^{-1}(V_{\alpha})$ is open. 
    Since $X$ is compact, there are finitely many indices, say $\alpha_1,  , \alpha_n$, such that
    \begin{equation}
        \label{eq:4.12}
        X \subset 
        f^{-1} (V_{\alpha_1})
        \cup \cdots \cup
        f^{-1} (V_{\alpha_n}).    
    \end{equation}
    Since $f(f^{-1}(E)) \subset E$ for every $E \subset Y$, 
    (\ref{eq:4.12}) implies that 
    \begin{equation}
        \label{eq:4.13}
        f(X) \subset 
        (V_{\alpha_1})
        \cup \cdots \cup
        (V_{\alpha_n}).
    \end{equation}
    
    This completes the proof
}

Note: We have used the relation $f(f^{- 1}(E)) \subset E$, valid for $E \subset Y$. 
If $E \subset X$, then $f^{- 1}(f(E)) \supset E$; equality need not hold in either case.

We shall now deduce some consequences of Theorem \ref{thm:4.14}

\mythm{
    \label{thm:4.15}
    If $\mathbf{f}$ is a continuous mapping of a compact metric space $X$ into $\R{k}$, 
    then $\mathbf{f}(X)$ is closed and bounded. 
    Thus, $\mathbf{f}$ is bounded.
}

This follows from Theorem \ref{thm:2.41}. 
The result is particularly important when $f$ is real:

\mythm{
    \label{thm:4.16}
    Suppose $f$ is a continuous real function on a compact metric space $X$, and
    \begin{equation}
        \label{eq:4.14}
        M = \sup_{p\in X} f(p), \quad
        m = \inf_{p\in X} f(p).
    \end{equation}
    Then there exist points $p, q \in X$ 
    such that $f(p) = M$ and $f(q) = m$.
}

The notation in (\ref{eq:4.14}) means that 
$M$ is the least upper bound of the set of all numbers $f(p)$, 
where $p$ ranges over $X$, 
and that $m$ is the greatest lower bound of this set of numbers.

The conclusion may also be stated as follows: 
\emph{There exist points $p$ and $q$
in $X$ such that $f(q) \leq f(x) \leq f(p)$ for all $x \in X$;} 
that is, $f$ attains its maximum (at $p$) and its minimum (at $q$).

\myproof{
    By Theorem \ref{thm:4.15}, 
    $f(X)$ is a closed and bounded set of real numbers; 
    hence $f(X)$ contains
    \begin{equation*}
        M = \sup f(X), \quad
        m = \inf f(X).        
    \end{equation*}
    By Theorem \ref{thm:2.28}
}

\mythm{
    \label{thm:4.17}
    Suppose $f$ is a continuous 1-1 mapping of a compact metric space $X$ onto a metric space $Y$. 
    Then the inverse mapping 1-1 defined on $Y$ by 
    \begin{equation*}
        f^{-1}(f(x)) = x \quad
        (x \in X)
    \end{equation*}
    is a continuous mapping of $Y$ onto $X$.
}

\mymyDef{
    \label{myDef:4.18}
    Let $f$ be a mapping of a metric space $X$ into a metric space $Y$.
    We say that $f$ is \emph{uniformly continuous} on $X$ 
    if for every $\varepsilon > 0$ there exists $\delta > 0$
    such that
    \begin{equation}
        \label{eq:4.15}
        d_Y(f(p),f(q)) < \varepsilon
    \end{equation}
    for all $p$ and $q$ in $X$ for which $d_X(p, q) < \delta$.
}

Let us consider the differences between the concepts of continuity and of
uniform continuity. 
First, uniform continuity is a property of a function on a set, 
whereas continuity can be defined at a single point. 
To ask whether a given function is uniformly continuous at a certain point is meaningless. 
Second, if $f$ is continuous on $X$, 
then it is possible to find, 
for each $\varepsilon > 0$ and for each point $p$ of $X$, 
a number $\delta > 0$ having the property specified in Definition \ref{myDef:4.5}. 
This $\delta$ depends one $\varepsilon$ \emph{and} on $p$. 
If $f$ is, however, uniformly continuous on $X$, 
then it is possible, for each $\varepsilon > 0$, 
to find \emph{one} number $\delta > 0$ which will do for \emph{all} points $p$ of $X$.

Evidently, every uniformly continuous function is continuous. 
That the two concepts are equivalent on compact sets follows from the next theorem. 

\mythm{
    \label{mythm:4.19}
    Let $f$ be a continuous mapping of a compact metric space $X$ into a metric space $Y$. 
    Then $f$ is uniformly continuous on $X$.
}