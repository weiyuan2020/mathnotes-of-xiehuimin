% chap04sec03

\section{Continuity and compactness}
\mybox{连续性与紧致性}
\begin{mydef}
    \label{def:4.13}
    A mapping $\mathbf{f}$ of a set $E$ into $\R^{k}$ is said to be \emph{bounded} 
    if there is a real number $M$ such that $\left| f(x) \right| \leq M$ for all $x \in E$.
\end{mydef}
\mybox{映射有界 代表其定义域上所有点的映射组成的集合有界.}
\begin{thm}
    \label{thm:4.14}
    Suppose $f$ is a continuous mapping of a compact metric space $X$ into a metric space $Y$. Then $f(X)$ is compact.
\end{thm}
\mybox{$f$为紧度量空间 $X$ 到度量空间 $Y$ 的连续映射, $f(X)$是紧的}
\begin{proof}
    Let $\sequence{V_\alpha}$ be an open cover of $f(X)$. Since $f$ is continuous, 
    Theorem \ref{thm:4.8} shows that each of the sets $f^{-1}(V_{\alpha})$ is open. 
    Since $X$ is compact, there are finitely many indices, say $\alpha_1,  , \alpha_n$, such that
    \begin{equation}
        \label{eq:4.12}
        X \subset 
        f^{-1} (V_{\alpha_1})
        \cup \cdots \cup
        f^{-1} (V_{\alpha_n}).    
    \end{equation}
    Since $f(f^{-1}(E)) \subset E$ for every $E \subset Y$, 
    (\ref{eq:4.12}) implies that 
    \begin{equation}
        \label{eq:4.13}
        f(X) \subset 
        (V_{\alpha_1})
        \cup \cdots \cup
        (V_{\alpha_n}).
    \end{equation}
    
    This completes the proof
\end{proof}

Note: We have used the relation $f(f^{- 1}(E)) \subset E$, valid for $E \subset Y$. 
If $E \subset X$, then $f^{- 1}(f(E)) \supset E$; equality need not hold in either case.

We shall now deduce some consequences of Theorem \ref{thm:4.14}

\begin{thm}
    \label{thm:4.15}
    If $\mathbf{f}$ is a continuous mapping of a compact metric space $X$ into $\R^{k}$, 
    then $\mathbf{f}(X)$ is closed and bounded. 
    Thus, $\mathbf{f}$ is bounded.
\end{thm}

This follows from Theorem \ref{thm:2.41}. 
The result is particularly important when $f$ is real:

\begin{thm}
    \label{thm:4.16}
    Suppose $f$ is a continuous real function on a compact metric space $X$, and
    \begin{equation}
        \label{eq:4.14}
        M = \sup_{p\in X} f(p), \quad
        m = \inf_{p\in X} f(p).
    \end{equation}
    Then there exist points $p, q \in X$ 
    such that $f(p) = M$ and $f(q) = m$.
\end{thm}

The notation in (\ref{eq:4.14}) means that 
$M$ is the least upper bound of the set of all numbers $f(p)$, 
where $p$ ranges over $X$, 
and that $m$ is the greatest lower bound of this set of numbers.

The conclusion may also be stated as follows: 
\emph{There exist points $p$ and $q$
in $X$ such that $f(q) \leq f(x) \leq f(p)$ for all $x \in X$;} 
that is, $f$ attains its maximum (at $p$) and its minimum (at $q$).

\begin{proof}
    By Theorem \ref{thm:4.15}, 
    $f(X)$ is a closed and bounded set of real numbers; 
    hence $f(X)$ contains
    \begin{equation*}
        M = \sup f(X), \quad
        m = \inf f(X).        
    \end{equation*}
    By Theorem \ref{thm:2.28}
\end{proof}

\begin{thm}
    \label{thm:4.17}
    Suppose $f$ is a continuous 1-1 mapping of a compact metric space $X$ onto a metric space $Y$. 
    Then the inverse mapping 1-1 defined on $Y$ by 
    \begin{equation*}
        f^{-1}(f(x)) = x \quad
        (x \in X)
    \end{equation*}
    is a continuous mapping of $Y$ onto $X$.
\end{thm}
\mybox{1-1映射---逆映射}

\begin{mydef}
    \label{def:4.18}
    Let $f$ be a mapping of a metric space $X$ into a metric space $Y$.
    We say that $f$ is \emph{uniformly continuous} on $X$ 
    if for every $\varepsilon > 0$ there exists $\delta > 0$
    such that
    \begin{equation}
        \label{eq:4.15}
        d_Y(f(p),f(q)) < \varepsilon
    \end{equation}
    for all $p$ and $q$ in $X$ for which $d_X(p, q) < \delta$.
\end{mydef}
\mybox{一致连续}
Let us consider the differences between the concepts of continuity and of
uniform continuity. 
First, uniform continuity is a property of a function on a set, 
whereas continuity can be defined at a single point. 
To ask whether a given function is uniformly continuous at a certain point is meaningless. 
Second, if $f$ is continuous on $X$, 
then it is possible to find, 
for each $\varepsilon > 0$ and for each point $p$ of $X$, 
a number $\delta > 0$ having the property specified in Definition \ref{def:4.5}. 
This $\delta$ depends one $\varepsilon$ \emph{and} on $p$. 
If $f$ is, however, uniformly continuous on $X$, 
then it is possible, for each $\varepsilon > 0$, 
to find \emph{one} number $\delta > 0$ which will do for \emph{all} points $p$ of $X$.

Evidently, every uniformly continuous function is continuous. 
That the two concepts are equivalent on compact sets follows from the next theorem. 

\begin{thm}
    \label{thm:4.19}
    Let $f$ be a continuous mapping of a compact metric space $X$ into a metric space $Y$. 
    Then $f$ is uniformly continuous on $X$.
\end{thm}

\begin{proof}
    Let $\varepsilon > 0$ be given.
    Since $f$ is continuous, we can associate to each point $p \in X$ a positive number $\phi(p)$ such that 
    \begin{equation}
        \label{eq:4.16}
        q\in X, d_X(p, q) < \phi(p)
        \text{ implies }
        d_Y (f(p), f(q)) < \frac{\varepsilon}{2}.
    \end{equation}

    Let $J(p)$ be the set of all $q \in X$ for which
    \begin{equation}
        \label{eq:4.17}
        d_X(p, q) < \frac{1}{2}\phi(p).
    \end{equation}

    Since $p \in J(p)$, the collection of all sets $J(p)$ is an open cover of $X$;
    and since $X$ is compact, there is a finite set of points $p_1,...,p_n$ in $X$, such that 
    \begin{equation}
        \label{eq:4.18}
        X \subset J(p_1) \cup \cdots \cup J(p_n).
    \end{equation}
    We put 
    \begin{equation}
        \delta = \frac{1}{2} \min [\phi(p_1), ..., \phi(p_n)].
    \end{equation}
    Then $\delta > 0$ .
    (This is one point where the finiteness of the covering,
    inherent in the definition of compactness, is essential.
    The minimum of a finite set of positive numbers is positive,
    whereas the inf of an infinite set of positive numbers may very well be 0.)

    Now let $q$ and $p$ be points of $X$, such that $d_X(p, q) < \delta$,
    By (\ref{eq:4.18}), there is an integer $m$, $1 \leq m \leq n$, 
    such that $p \in J(p_m)$; hence 
    \begin{equation}
        \label{eq:4.20}
        d_X(p, p_m) < \frac{1}{2}\phi(p_m),
    \end{equation}
    and we also have 
    \begin{equation*}
        d_X(q, p_m) \leq
        d_X(p, q) +
        d_X(p, p_m) <
        \delta + \frac{1}{2}\phi(p_m) \leq
        \phi(p_m).
    \end{equation*}
    Finally, (\ref{eq:4.16}) shows that therefore 
    \begin{equation*}
        d_Y(f(p), f(q)) \leq
        d_Y(f(p), f(p_m)) +
        d_Y(f(q), f(p_m)) <
        \varepsilon .
    \end{equation*}
    This complete the proof.
\end{proof}

An alternative proof is sketched in Exercise 10.

We now proceed to show that compactness is essential in the hypotheses
of Theorems \ref{thm:4.14}, \ref{thm:4.15}, \ref{thm:4.16}, and \ref{thm:4.19}.

\begin{thm}
    \label{thm:4.20}
    Let $E$ be a noncompact set in $\R^{1}$ Then
    \begin{enumerate}[(a)]
    \item there exists a continuous function on $E$ which is not bounded,     
    \item there exists a continuous and bounded function on $E$ which has no maximum.\\
    If, in addition, $E$ is bounded, then     
    \item there exists a continuous function on $E$ which is not uniformly continuous.
    \end{enumerate}
\end{thm}
\begin{proof}
    Suppose first that $E$ is bounded, 
    so that there exists a limit point $x_0$ of $E$ 
    which is not a point of $E$. 
    Consider
    \begin{equation}
        \label{eq:4.21}
        f(x) = \frac{1}{x - x_0}
        \quad
        (x \in E).
    \end{equation}
    This is continuous on $E$ (Theorem 4.9), but evidently unbounded. 
    To see that (\ref{eq:4.21}) is not uniformly continuous, 
    let $\varepsilon > 0$ and $\delta > 0$ be arbitrary, 
    and choose a point $x \in E$ such that $\left| x - x_0 \right| < \delta$.
    Taking $t$ close enough to $x_0$ , 
    we can then make the difference $\left| f(t) - f(x) \right|$ greater than $\varepsilon$, although $\left| t-x \right| < \delta$.
    Since this is true for every $\delta > 0$, 
    $f$ is not uniformly continuous on $E$.

    The function $g$ given by
    \begin{equation}
        \label{eq:4.22}
        g(x) = \frac{1}{1+(x-x_0)^2}
        \quad
        (x \in E)
    \end{equation}
    is continuous on $E$, and is bounded, since $0 < g(x) < 1$. 
    It is clear that 
    \begin{equation*}
        \sup_{x \in E} g(x) = 1,
    \end{equation*}
    whereas $g(x) < l$ for all $x \in E$. Thus $g$ has no maximum on $E$.

    Having proved the theorem for bounded sets $E$, 
    let us now suppose that $E$ is unbounded. 
    Then $f(x) = x$ establishes (a), whereas
    \begin{equation}
        \label{eq:4.23}
        h(x) = \frac{x^2}{1 + x^2}
        \quad 
        (x \in E)
    \end{equation}
    establishes (b), since
    \begin{equation*}
        \sup_{x \in E} h(x) = 1
    \end{equation*}
    and $h(x) < 1$ for all $x \in E$.

    Assertion (c) would be false if boundedness were omitted from the
    hypotheses. 
    For, let $E$ be the set of all integers. 
    Then every function defined on $E$ is uniformly continuous on $E$. 
    To see this, we need merely take $\delta < 1$ in Definition \ref{def:4.18}.
\end{proof}

We conclude this section by showing that compactness is also essential in
Theorem \ref{thm:4.17}.


\begin{myExample}
    Let $X$ be the half-open interval $[0, 2\pi)$ on the real line, 
    and let $f$ be the mapping of $X$ onto the circle $Y$ consisting of all points 
    whose distance from the origin is $1$, given by
    \begin{equation}
        \label{eq:4.24}
        f(t) = (\cos t, \sin t)
        \quad
        (0 \leq t < 2\pi).
    \end{equation}
    The continuity of the trigonometric functions cosine and sine, 
    as well as their periodicity properties, will be established in Chap. 8. 
    These results show that $f$ is a continuous 1-1 mapping of $X$ onto $Y$.
    
    However, the inverse mapping 
    (which exists, since $f$ is one-to-one and onto) 
    fails to be continuous at the point $(1, 0) = \mathbf{f}(0)$. 
    Of course, $X$ is not compact in this example. 
    (It may be of interest to observe that 
    $\mathbf{f}^{-1}$ fails to be continuous in spite of the fact that $Y$ \emph{is} compact!)
\end{myExample}