\section{Metric space}
\mybox{metirc space 度量空间}
\begin{myDef}\label{myDef:2.15}
    set $X$ metric space $p\in X$, $p$ point.

    $\forall p,q \in X$ associate a real number $d(p,q)$ (distance)
    \begin{enumerate}[a.]
        \item $d(p,q) > 0$ if $p \neq q$; $d(p,p)=0$,
        \item $d(p,q) = d(q,p)$.
        \item $d(p,q) \leq d(p,r) + d(r,q)$, $\forall r\in X$
    \end{enumerate}
\end{myDef}
\mybox{对称性,正定性,三角不等式。}

\begin{myExample}
    the distance of the euclidean space $\R^k$ is defined by
    \begin{equation}\label{eq:2.19}
        d(\mathbf{x}, \mathbf{y}) = |\mathbf{x} - \mathbf{y}|
        \qquad (\mathbf{x}, \mathbf{y}\in \R^k)
    \end{equation}
\end{myExample}

It's important to observe that every subset $Y$ of metric space $X$ is a metric space in its own right, with the same distance function. For it is clear that if conditions (a) to (c) of Definition \ref{myDef:2.15} hold for $p, q, r \in X$, they also hold if we restrict $p, q, r$ to lie in $Y$.

Thus every subset of a euclidean space is a metric space. Other examples
are the spaces $l(K)$ and $L^2 (\mu)$, 
which are discussed in Chaps. 7 and 11, respectively.

\mybox{
    使用 mathescr 命令输入英文花体字(报错,改用 mathcal)
    $\mathscr{l} (K)$ and $\mathscr{l}^2 (\mu)$

    Ralph Smith’s Formal Script Font (rsfs): Use the “mathrsfs” package.
    usepackage{mathrsfs}
    ...
    $\mathscr{ABCDEFGHIJKLMNOPQRSTUVWXYZ}$
}

\begin{myDef}\label{myDef:2.17}
% 2.17 Definition 
By the \emph{segment} $(a, b)$ we mean the set of all real numbers $x$
such that $a < x <b$.

By the \emph{interval} $[a. b]$ we mean the set of all real numbers $x$ such that $a \leq x \leq b$

Occasionally we shall also encounter ``half-open intervals'' $[a, b)$ and $(a, b]$; the first consists of all $x$ such that $a \leq x < b$, the second of all $x$ such that $a < x \leq b$
\end{myDef}

If $a_i <b_i$ for $i=1,...,k$, the set of all points $\mathbf{x} =(x_1, ..., x_k)$ in $\R^k$ whose coordinates satisfy the inequalities $a_i \leq x_i \leq _i (1 \leq i \leq k)$ is called a \emph{$k$-cell}.\\
Thus a $1$-cell is an interval, a $2$-cell is a rectangle, etc.

If $\mathbf{x}\in \R^k$ and $r > 0$. the \emph{open (or closed) ball} $B$ with center at $\mathbf{x}$ and radius $r$ is defined to be the set of all $\mathbf{y} \in \R^k$ such that $|\mathbf{y} - \mathbf{x}| <r$ (or $|\mathbf{y} - \mathbf{x}| \leq r$).

We call a set $E \subset \R^k$ \emph{convex} if

\begin{equation*}
    \lambda\mathbf{x} + (1 - \lambda)\mathbf{y} \in E
\end{equation*}

whenever $\mathbf{x} \in E$, $\mathbf{y} \in E$, and $0 < \lambda < 1$.

For example, \emph{balls are convex}. For if |y -x| <r, |z-x| <r, and
$0 < \lambda <1$, we have
\begin{align*}
    |\lambda \mathbf{y} + (1-\lambda) \mathbf{z} - \mathbf{x}|
    & = |\lambda (\mathbf{y} - \mathbf{x}) + (1 - \lambda)(\mathbf{z} - \mathbf{x})\\
    & \leq \lambda |\mathbf{y} - \mathbf{x}| + (1 - \lambda)|\mathbf{z} - \mathbf{x}| < \lambda r + (1 - \lambda)r\\
    & = r.
\end{align*}

The same proof applies to closed balls. It is also easy to see that $k$-cells are convex.
% 32 PRINCIPLES OF MATHEMATICAL ANALYSIS

\mybox{这里给出了 开区间, 闭区间, 半开区间以及凸集 convex 的定义

凸性 Jensen 不等式.
}

\begin{myDef}\label{myDef:2.18}
    Definition Let $X$ be a metric space. All points and sets mentioned below are understood to be elements and subsets of $X$.
    \begin{enumerate}[(a)]
        \item A \emph{neighborhood} of $p$ is a set $N_r(p)$ consisting of all $q$ such that $d(p, q) < r$,for some $r > 0$. The number $r$ is called the \emph{radius} of $N_r(p)$.
        \item A point $p$ is a \emph{limit point} of the set $E$ if \emph{every} neighborhood of $p$ contains a point $q \neq p$ such that $q \in E$.
        \item If $p \in E$ and $p$ is not a limit point of $E$, then $p$ is called an \emph{isolated point} of $E$.
        \item $E$ is \emph{closed} if every limit point of $E$ is a point of $E$.
        \item A point $p$ is an \emph{interior} point of $E$ if there is a neighborhood $N$ of $p$ such that $N \subset E$.
        \item $E$ is \emph{open} if every point of $E$ is an interior point of $E$.
        \item The \emph{complement} of $E$ (denoted by $E^c$)is the set of all points $p \in X$ such that $p \not\in E$.
        \item $E$ is \emph{perfect} if $E$ is closed and if every point of $E$ is a limit point of $E$.
        \item $E$ is \emph{bounded} if there is a real number $M$ and a point $q \in X$ such that $d(p,q)< M$ for all $p \in E$.
        \item $E$ is \emph{dense} in $X$ if every point of $X$ is a limit point of $E$, or a point of $E$ (or both).
    \end{enumerate}
\end{myDef}

Let us note that in $\R^1$ neighborhoods are segments, whereas in $\R^2$ neighborhoods are interiors of circles.

\mybox{
    neighbhoods 邻域, \\
    limit point 极限点, 
    isolated point 孤立点, 
    interior point 内点. \\
    closed set 闭集, open set 开集, closure set 补集.\\
    perfect set 完全集, complete set 完备集, 
    dense 稠密 (有理数集 $\Q $ 是稠密的, 与之对应 实数集 $\R$ 是连续的.)
}

\begin{thm}
    \label{thm:2.19}
    Every neighborhood is un open set.
\end{thm}
\myproof{
    Consider a neighborhood $E = N_r(p)$, 
    and let $q$ be any point of $E$.
    Then there is a positive real number $h$ such that
    \begin{equation*}
        d(p, q) = r - h.
    \end{equation*}
    For all points $s$ such that $d(q, s) < h$, we have then
    \begin{equation*}
        d(p, s) \leq d(p, q) + d(q, s) < r - h + h = r,
    \end{equation*}
    so that $s \in E$. 
    Thus $q$ is an interior point of $E$.
}

\begin{thm}
    \label{thm:2.20}
    If $p$ is a limit point of a set $E$, then every neighborhood of $p$ contains infinitely many points of $E$.
\end{thm}

\myproof{
    Suppose there is a neighborhood $N$ of $p$ 
    which contains only a finite number of points of $E$. 
    Let $q_1, ... , q_n$ be those points of $N \cap E$,
    which are distinct from $p$, and put
    \begin{equation*}
        r = \min_{1 \leq m \leq n} d(p, q_m)
    \end{equation*}
    [we use this notation to denote the smallest of the numbers 
    $d(p, q_1), ..., d(p, q_n)$]. 
    The minimum of a finite set of positive numbers is clearly positive, 
    so that $r > 0$.

    The neighborhood $N_r(p)$ contains no point $q$ of $E$ such that $q \neq p$,
    so that $p$ is not a limit point of $E$. 
    This contradiction establishes the theorem.
}

\begin{myCorollary}
    A finite point set has no limit points.
\end{myCorollary}

\begin{myExample}
    \label{myExample:2.21}
    Let us consider the following subsets of $\R^{2}$:
    \begin{enumerate}[(a)]
        \item The set of all complex $z$ such that $|z| < 1$.
        \item The set of all complex $z$ such that $|z| \leq 1$.
        \item A nonempty finite set.
        \item The set of all integers.
        \item The set consisting of the numbers $1/n(n=1,2,3,...)$. Let us note that this set $E$ has a limit point (namely, $z =0$) but that no point of $E$ is a limit point of $E$; we wish to stress the difference between having a limit point and containing one.
        \item The set of all complex numbers (that is, $\R^{2}$).
        \item The segment $(a,b)$.
    \end{enumerate}
\end{myExample}

Let us note that (d),(e),(g)can be regarded also as subsets of $\R^{1}$.
Some properties of these sets are tabulated below:

\begin{center}
    \begin{tabular}{lcccc}
        & Closed & Open & Perfect & Bounded \\
    (a) & No     & Yes  & No      & Yes     \\
    (b) & Yes    & No   & Yes     & Yes     \\
    (c) & Yes    & No   & No      & Yes     \\
    (d) & Yes    & No   & No      & No      \\
    (e) & No     & No   & No      & Yes     \\
    (f) & Yes    & Yes  & Yes     & No      \\
    (g) & No     &      & No      & Yes    
    \end{tabular}
\end{center}

In (g), we left the second entry blank. 
The reason is that the segment $(a,b)$ is not open 
if we regard it as a subset of $\R^2$, 
but it is an open subset of $\R^1$.

\mybox{根据定义, 复数集既是闭集又是开集...}

\begin{thm}
    \label{thm:2.22}
    Let $\{E_\alpha\}$ be a (finite or infinite) collection of sets $E_\alpha$. Then
    \begin{equation}
        \label{eq:2.20}
        \left(\bigcup_{\alpha} E_{\alpha} \right)^c = \bigcap_{\alpha}( E_{\alpha}^c )
    \end{equation}
\end{thm}

\myproof{
    Let $A$ and $B$ be the left and right members of (\ref{eq:2.20}). 
    If $x \in A$, then $X \not\in \cup_\alpha E_\alpha$, 
    hence $x \not\in E_\alpha$ for any $\alpha$, 
    hence $x \in E_\alpha^c$ for every $\alpha$, 
    so that $x \in \cap E_\alpha^c$.
    Thus $A \subset B$.

    Conversely, if $x \in B$, then $x \in E_\alpha^c$ for every $\alpha$,
    hence $x \not\in E_\alpha$ for any $\alpha$,
    hence $x \not\in \cup_\alpha E_\alpha$, 
    so that $x \in (\cup_\alpha E_\alpha)^c$. 
    Thus $B \subset A$.

    It follows that $A = B$.
}

\begin{thm}
    \label{thm:2.23}
    A set $E$ is open if and only if its complement is closed.
\end{thm}

\begin{proof}
    % Proof 
    First, suppose $E^c$ is closed. Choose $x \in E$. Then $x \not\in E^c$, and $x$ is not a limit point of $E^c$. Hence there exists a neighborhood $N$ of $x$ such that $E^c \cap N$ is empty, that is, $N \subset E$. Thus $x$ is an interior point of $E$, and $E$ is open.
    
    Next, suppose $E$ is open. Let $x$ be a limit point of $E^c$. Then every neighborhood of $x$ contains a point of $E^c$, so that $x$ is not an interior point of $E$. Since $E$ is open, this means that $x \in E^c$. It follows that $E$ is closed.
\end{proof}

\begin{myCorollary}
    A set $F$ is closed if and only if its complement is open.
\end{myCorollary}

\mybox{这里使用新的定义得到的开集与闭集保持了原有的性质: 开集的补集是闭集, 闭集的补集是开集}

\begin{thm}
    \label{thm:2.24}
    \begin{enumerate}[(a)]
        \item For any collection $\{G_\alpha\}$ of open sets,  $\cup_\alpha G_\alpha$ is open.
        \item For any collection $\{F_\alpha\}$ of closed sets, $\cap_\alpha F_\alpha$ is closed.
        \item For any finite collection $G_1, ..., G_n$ of open sets, $\cap_{i=1}^n G_i$ is open.
        \item For any finite collection $F_1, ..., F_n$ of closed sets, $\cup_{i=1}^n F_i$ is closed.
    \end{enumerate}
\end{thm}

\myproof{
    Put 

    By Theorem \ref{thm:2.22}
    \begin{equation}
        \label{eq:2.21}
        \left( \bigcap_\alpha F_\alpha \right)^c = 
        \bigcup_\alpha \left( F_\alpha^c \right),
    \end{equation}
}

\mybox{
    无限开集的并仍是开集, 有限开集的交仍是开集

    无限闭集的交仍是闭集, 有限闭集的并仍是闭集

    下面给出一个反例
}

\begin{myExample}
    In parts (c)and (d) of the preceding theorem, the finiteness of the collections is essential.
    \begin{equation*}
        G_n = \left(-\frac{1}{n}, \frac{1}{n} \; (n=1,2,3,\dots). \right)
    \end{equation*}
    $G = \cap_{n=1}^\infty G_n$
    Then $G$ consists of a single point (namely, $x = 0$) and is therefore not an open subset of $\R$.
    
    Thus the intersection of an infinite collection of open sets need not be open. Similarly, the union of an infinite collection of closed sets need not be closed.
\end{myExample}

\begin{myDef}
    \label{myDef:2.26}
    If $X$ is a metric space, if $E \subset X$, and if $E'$ denotes the set of all limit points of $E$ in $X$, then the \emph{closure} of $E$ is the set $\bar{E}=E \cup E'$.
\end{myDef}

\begin{thm}
    \label{thm:2.27}
    If $X$ is a metric space and $E \subset X$, then
    \begin{enumerate}[(a)]
        \item $E$ is closed,
        \item $E = \bar{E}$ if and only if $E$ is closed,
        \item $E \subset F$ for every closed set $F \subset X$ such that $E \subset F$.
    \end{enumerate}
\end{thm}
By (a) and (c), $E$ 1s the smallest closed subset of $X$ that contains $E$.
\myproof{
    \begin{enumerate}[(a)]
        \item If $p \in X$ and $p \not\in \bar{E}$ then $p$ is neither a point of $E$ nor a limit point of $E$. Hence $p$ has a neighborhood which does not intersect $E$. The complement of $E$ is therefore open. Hence $E$ is closed.
        \item If $E = \bar{E}$, (a) implies that $E$ is closed. If $E$is closed, then $E' \subset E$ [by Definitions \ref{myDef:2.18}(d) and \ref{myDef:2.26}], hence $\bar{E} = E$.
        \item If $F$ is closed and $F \supset E$, then $F \supset F'$, hence $F \supset E'$. Thus $F \supset \bar{E}$.
    \end{enumerate}
}


\begin{thm}
    \label{thm:2.28}
    Let $E$ be a nonempty set of real numbers which is bounded above.     Let $y = \sup E$. Then $y \in \bar{E}$. Hence $y \in E$ if $E$ is closed.
\end{thm}
Compare this with the examples in Sec. 1.9.
\myproof{
    If $y \in E$ then $y \in \bar{E}$. 
    Assume $y \not\in E$. 
    For every $h > 0$ there exists then a point $x \in E$ 
    such that $y - h < x < y$, 
    for otherwise $y - h$ would be an upper bound of $E$. 
    Thus $y$ is a limit point of $E$. Hence $y \in \bar{E}$.
}

\begin{myRemark}
    Suppose $E \subset Y \subset X$, where $X$ is a metric space. To say that $E$ is an open subset of $X$ means that to each point $p \in E$ there is associated a positive number $r$ such that the conditions $d(p,q) < r$ , $g \in X$ imply that $q \in E$. But we have already observed (Sec. 2.16) that $Y$ is also a metric space, so that our definitions may equally well be made within $Y$. To be quite explicit, let us say that 
    
    $E$ is \emph{open relative to $Y$} if to each $p \in E$ there is associated an $r > 0$ such that $q \in E$ whenever $d(p,q) <r$ and $g \in Y$. 
    
    Example 2.21(g) showed that a set may be open relative to $Y$ without being an open subset of $X$. However, there is a simple relation between these concepts, which we now state.
\end{myRemark}
\mybox{open relative to $Y$ 关于Y是开的} 

\begin{thm}
    \label{thm:2.30}
    Suppose $Y \subset X$. A subset $E$ of $Y$ is open relative to $Y$ if and only if $E = Y \cap G$ for some open subset $G$ of $X$.
\end{thm}
\myproof{
    Suppose $E$ is open relative to $Y$. 
    To each $p \in E$ there is a positive number $r_P$ 
    such that the conditions $d(p, q) < r_P$, 
    $q \in Y$ imply that $q \in E$. 
    Let $V_P$ be the set of all $q \in X$ 
    such that $d(p, q) < r_P$, and define
    \begin{equation*}
        G = \bigcup_{p \in E} V_P.
    \end{equation*}
    Then $G$ is an open subset of $X$, by Theorems \ref{thm:2.19} and \ref{thm:2.24}. 

    Since $p \in V_P$ for all $p \in E$, 
    it is clear that $E \subset G \cap Y$.
    
    By our choice of $V_P$, 
    we have $V_P \cap Y \subset E$ for every $p \in E$, 
    so that $G \cap Y \subset E$. 
    Thus $E = G \cap Y$, and one half of the theorem is proved.
    
    Conversely, if $G$ is open in $X$ and $E = G \cap Y$, 
    every $p \in E$ has a neighborhood $V_P \subset G$. 
    Then $V_P \cap Y \subset E$, 
    so that $E$ is open relative to $Y$.
}