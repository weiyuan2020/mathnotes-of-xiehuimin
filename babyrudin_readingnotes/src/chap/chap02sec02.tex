\section{Metric space}
\begin{myDefinition}\label{myDefinition:2.15 metric space}
    set $X$ metric space\\
    $p\in X$, $p$ point.

    $\forall p,q \in X$ associate a real number $d(p,q)$ (distance)

    a. $d(p,q) > 0$ if $p \neq q$; $d(p,p)=0$,

    b. $d(p,q) = d(q,p)$.

    c. $d(p,q) \leq d(p,r) + d(r,q)$, $\forall r\in X$
\end{myDefinition}
对称性,正定性,三角不等式。

\begin{myExample}
    the distance of the euclidean space $\mathbb{R}^k$ is defined by
    \begin{equation}\label{eq:2.19 distance in eulidean space}
        d(\mathbf{x}, \mathbf{y}) = |\mathbf{x} - \mathbf{y}|
        \qquad (\mathbf{x}, \mathbf{y}\in \mathbb{R}^k)
    \end{equation}
\end{myExample}

It's important to observe that every subset $Y$ of metric space $X$ is a metric space in its own right, with the same distance function. For it is clear that if conditions (a) to (c) of Definition \ref{myDefinition:2.15 metric space} hold for $p, q, r \in X$, they also hold if we restrict $p, q, r$ to lie in $Y$.

Thus every subset of a euclidean space is a metric space. Other examples
are the spaces $l(K)$ and $L^2 (\mu)$, 
% $\mathscr{l} (K)$ and $\mathscr{l}^2 (\mu)$
% \footnote{使用 mathescr 命令输入英文花体字(报错,改用 mathcal)
% Ralph Smith’s Formal Script Font (rsfs): Use the “mathrsfs” package.
% usepackage{mathrsfs}
% ...
% \mathscr{ABCDEFGHIJKLMNOPQRSTUVWXYZ}
% }
, which are discussed in Chaps. 7 and 11, respectively.

\begin{myDefinition}\label{myDefinition:2.17 segment}
% 2.17 Definition 
By the \emph{segment} $(a, b)$ we mean the set of all real numbers $x$
such that $a < x <b$.

By the \emph{interval} $[a. b]$ we mean the set of all real numbers $x$ such that $a \leq x \leq b$

Occasionally we shall also encounter ``half-open intervals'' $[a, b)$ and $(a, b]$; the first consists of all $x$ such that $a \leq x < b$, the second of all $x$ such that $a < x \leq b$
\end{myDefinition}

If $a_i <b_i$ for $i=1,...,k$, the set of all points $\mathbf{x} =(x_1, ..., x_k)$ in $\mathbb{R}^k$ whose coordinates satisfy the inequalities $a_i \leq x_i \leq _i (1 \leq i \leq k)$ is called a \emph{$k$-cell}.\\
Thus a $1$-cell is an interval, a $2$-cell is a rectangle, etc.

If $\mathbf{x}\in \mathbb{R}^k$ and $r > 0$. the \emph{open (or closed) ball} $B$ with center at $\mathbf{x}$ and radius $r$ is defined to be the set of all $\mathbf{y} \in \mathbb{R}^k$ such that $|\mathbf{y} - \mathbf{x}| <r$ (or $|\mathbf{y} - \mathbf{x}| \leq r$).

We call a set $E \subset \mathbb{R}^k$ \emph{convex} if

\begin{equation*}
    \lambda\mathbf{x} + (1 - \lambda)\mathbf{y} \in E
\end{equation*}

whenever $\mathbf{x} \in E$, $\mathbf{y} \in E$, and $0 < \lambda < 1$.

For example, \emph{balls are convex}. For if |y -x| <r, |z-x| <r, and
$0 < \lambda <1$, we have
\begin{align*}
    |\lambda \mathbf{y} + (1-\lambda) \mathbf{z} - \mathbf{x}|
    & = |\lambda (\mathbf{y} - \mathbf{x}) + (1 - \lambda)(\mathbf{z} - \mathbf{x})\\
    & \leq \lambda |\mathbf{y} - \mathbf{x}| + (1 - \lambda)|\mathbf{z} - \mathbf{x}| < \lambda r + (1 - \lambda)r\\
    & = r.
\end{align*}

The same proof applies to closed balls. It is also easy to see that $k$-cells are convex.
% 32 PRINCIPLES OF MATHEMATICAL ANALYSIS

\mybox{这里给出了 开区间, 闭区间, 半开区间以及凸集 convex 的定义}

\begin{myDefinition}\label{myDefinition:2.18 important}
    Definition Let $X$ be a metric space. All points and sets mentioned below are understood to be elements and subsets of $X$.

    (a) A \emph{neighborhood} of $p$ is a set $N_r(p)$ consisting of all $q$ such that $d(p, q) < r$,for some $r > 0$. The number $r$ is called the \emph{radius} of $N_r(p)$.

    (b) A point $p$ is a \emph{limit point} of the set $E$ if \emph{every} neighborhood of $p$ contains a point $q \neq p$ such that $q \in E$.
    
    (c) If $p \in E$ and $p$ is not a limit point of $E$, then $p$ is called an \emph{isolated point} of $E$.
    
    (d) $E$ is \emph{closed} if every limit point of $E$ is a point of $E$.
    
    (e) A point $p$ is an \emph{interior} point of $E$ if there is a neighborhood $N$ of $p$ such that $N \subset E$.
    
    (f) $E$ is \emph{open} if every point of $E$ is an interior point of $E$.
    
    (g) The \emph{complement} of $E$ (denoted by $E^c$)is the set of all points $p \in X$
    such that $p \not\in E$.
    
    (h) $E$ is \emph{perfect} if $E$ is closed and if every point of $E$ is a limit point
    of $E$.
    
    (i) $E$ is \emph{bounded} if there is a real number $M$ and a point $q \in X$ such that $d(p,q)< M$ for all $p \in E$.
    
    (j) $E$ is \emph{dense} in $X$ if every point of $X$ is a limit point of $E$, or a point of $E$ (or both).
\end{myDefinition}

Let us note that in $\mathbb{R}^1$ neighborhoods are segments, whereas in $\mathbb{R}^2$ neighborhoods are interiors of circles.

\begin{thm}\label{thm:2.19 neighbhood is open}
    Every neighborhood is un open set.
\end{thm}

\begin{thm}\label{thm:2.20}
    If $p$ is a limit point of a set $E$, then every neighborhood of $p$ contains infinitely many points of $E$.
\end{thm}

\begin{myCorollary}
    A finite point set has no limit points.
\end{myCorollary}

\begin{myExample}
    Let us consider the following subsets of $\mathbb{R}^2$:

(a) The set of all complex $z$ such that $|z| < 1$.

(b) The set of all complex $z$ such that $|z| \leq 1$.

(c) A nonempty finite set.

(d) The set of all integers.

(e) The set consisting of the numbers $1/n(n=1,2,3,...)$. Let us note that this set $E$ has a limit point (namely, $z =0$) but that no point of $E$ is a limit point of $E$; we wish to stress the difference between having a limit point and containing one.

(f) The set of all complex numbers (that is, $\mathbb{R}^2$).

(g) The segment $(a,b)$.
\end{myExample}

Let us note that (d),(e),(g)can be regarded also as subsets of R'.
Some properties of these sets are tabulated below:


\begin{table}[htbp]
    \begin{center}
    \begin{tabular}{lcccc}
        & Closed & Open & Perfect & Bounded \\
    (a) & No     & Yes  & No      & Yes     \\
    (b) & Yes    & No   & Yes     & Yes     \\
    (c) & Yes    & No   & No      & Yes     \\
    (d) & Yes    & No   & No      & No      \\
    (e) & No     & No   & No      & Yes     \\
    (f) & Yes    & Yes  & Yes     & No      \\
    (g) & No     &      & No      & Yes    
    \end{tabular}
    \end{center}
\end{table}

In (g), we left the second entry blank. The reason is that the segment $(a,b)$ is not open if we regard it as a subset of $\mathbb{R}^2$, but it is an open subset of $\mathbb{R}^1$.

\mybox{根据定义, 复数集既是闭集又是开集...}

\begin{thm}\label{thm:2.22}
    Let $\{E_\alpha\}$ be a (finite or infinite) collection of sets $E_\alpha$. Then
    \begin{equation}
        \left(\bigcup_{\alpha} E_{\alpha} \right)^c = \bigcap_{\alpha}( E_{\alpha}^c )
    \end{equation}
\end{thm}

\begin{thm}\label{thm:2.23}
    A set $E$ is open if and only if its complement is closed.
\end{thm}

\begin{proof}
    % Proof 
    First, suppose $E^c$ is closed. Choose $x \in E$. Then $x \not\in E^c$, and $x$ is not a limit point of $E^c$. Hence there exists a neighborhood $N$ of $x$ such that $E^c \cap N$ is empty, that is, $N \subset E$. Thus $x$ is an interior point of $E$, and $E$ is open.
    
    Next, suppose $E$ is open. Let $x$ be a limit point of $E^c$. Then every neighborhood of $x$ contains a point of $E^c$, so that $x$ is not an interior point of $E$. Since $E$ is open, this means that $x \in E^c$. It follows that $E$ is closed.
\end{proof}

\begin{myCorollary}
    A set $F$ is closed if and only if its complement is open.
\end{myCorollary}

\mybox{这里使用新的定义得到的开集与闭集保持了原有的性质: 开集的补集是闭集, 闭集的补集是开集}

\begin{thm}\label{thm:2.24}
    (a) For any collection $\{G_\alpha\}$ of open sets,  $\cup_\alpha G_\alpha$ is open.

    (b) For any collection $\{F_\alpha\}$ of closed sets, $\cap_\alpha F_\alpha$ is closed.

    (c) For anyfinite collection $G_1, ..., G_n$ of open sets, $\cap_{i=1}^n G_i$ is open.

    (d) For anyfinite collection $F_1, ..., F_n$ of closed sets, $\cup_{i=1}^n F_i$ is closed.
\end{thm}

\mybox{
    无限开集的并仍是开集, 有限开集的交仍是开集

    无限闭集的交仍是闭集, 有限闭集的并仍是闭集

    下面给出一个反例
}

\begin{myExample}
    In parts (c)and (d) of the preceding theorem, the finiteness of the collections is essential.
    \begin{equation*}
        G_n = \left(-\frac{1}{n}, \frac{1}{n} \; (n=1,2,3,\dots). \right)
    \end{equation*}
    $G = \cap_{n=1}^\infty G_n$
    Then $G$ consists of a single point (namely, $x = 0$) and is therefore not an open subset of $\mathbb{R}$.
    
    Thus the intersection of an infinite collection of open sets need not be open. Similarly, the union of an infinite collection of closed sets need not be closed.
\end{myExample}

\begin{myDefinition}\label{myDefinition:2..26 closure}
    If $X$ is a metric space, if $E \subset X$, and if $E'$ denotes the set of all limit points of $E$ in $X$, then the \emph{closure} of $E$ is the set $\bar{E}=E \cup E'$.
\end{myDefinition}

\begin{thm}
    If $X$ is a metric space and $E \subset X$, then
    
    (a) $E$ is closed,
    
    (b) $E = \bar{E}$ if and only if $E$ is closed,
    
    (c) $E \subset F$ for every closed set $F \subset X$ such that $E \subset F$.
\end{thm}
By (a) and (c), $E$ 1s the smallest closed subset of $X$ that contains $E$.

\begin{thm}
    Let $E$ be a nonempty set of real numbers which is bounded above.     Let $y = \sup E$. Then $y \in \bar{E}$. Hence $y \in E$ if $E$ is closed.
\end{thm}

\begin{myRemark}
    Suppose $E \subset Y \subset X$, where $X$ is a metric space. To say that $E$ is an open subset of $X$ means that to each point $p \in E$ there is associated a positive number $r$ such that the conditions $d(p,q) < r$ , $g \in X$ imply that $q \in E$. But we have already observed (Sec. 2.16) that $Y$ is also a metric space, so that our definitions may equally well be made within $Y$. To be quite explicit, let us say that 
    
    $E$ is \emph{open relative} to $Y$ if to each $p \in E$ there is associated an $r > 0$ such that $q \in E$ whenever $d(p,q) <r$ and $g \in Y$. 
    
    Example 2.21(g) showed that a set may be open relative to $Y$ without being an open subset of $X$. However, there is a simple relation between these concepts, which we now state.
\end{myRemark}

\begin{thm}
    Suppose $Y \subset X$. A subset $E$ of $Y$ is open relative to $Y$ if and only if $E = Y \cap G$ for some open subset $G$ of $X$.
\end{thm}