% chap06sec03
\section{Integration and differentiation}

We still confine ourselves to real functions in this section. 
We shall show that integration and differentiation are, in a certain sense, inverse operations.

\begin{thm}
    \label{thm:6.20}
    Let $f \in \mathscr{R}$ on $[a,b]$ .
    For $a \leq x \leq b$ , put
    \begin{equation*}
        F(x) = \int_{a}^{x} f(t) \d t.
    \end{equation*}
    Then $F$ is continuous on $[a, b ]$;
    furthermore, if $f$ is continuous at a point $x_0$ of $[a, b ]$, 
    then $F$ is differentiable at $x_0$ , and
    \begin{equation*}
        F'(x_0) = f(x_0).
    \end{equation*}
\end{thm}
\mybox{变上限积分}
% todo add proof

\begin{thm}
    \label{thm:6.21}
    \textbf{The fundamental theorem of calculus}
    If $f \in \mathscr{R}$ on $[a,b]$ 
    and if there is a differentiable function $F$
    such that $F' = f$, then
    \begin{equation*}
        \int_{a}^{b} f(x) \d x = F(b) - F(a).
    \end{equation*}
\end{thm}
\mybox{牛顿-莱布尼茨公式(Newton-Leibniz formula),通常也被称为微积分基本定理}
\begin{proof}
    Let $\varepsilon > 0$ be given.
    Choose a partition $P = \{x_0,...,x_n\}$ of $[a,b]$ 
    so that $U(P,f) - L(P,f)<\varepsilon$.
    The mean value theorem furnishes points $t_i \in [x_{i-1}, x_i]$
    such that
    \begin{equation*}
        F(x_i) - F(x_{i-1}) = f(t_i) \Delta x_i
    \end{equation*}
    for $i = 1,...,n$ . Thus 
    \begin{equation*}
        \sum_{i=1}^{n} f(t_i) \Delta x_i = F(b) - F(a).
    \end{equation*}

    It now follows from Theorem 6.7(c) that
    \begin{equation*}
        \left| F(b) - F(a) - \int_{a}^{b} f(x) \d x \right|  < \varepsilon.
    \end{equation*}
    Since this holds for every $\varepsilon > 0$,
    the proof is complete.
\end{proof}

\begin{thm}
    \label{thm:6.22}
    \textbf{integration by parts}
    Suppose $F$ and $G$ are differentiable functions on $[a,b]$,
    $F' = f \in \mathscr{R}$, and $G' = g \in \mathscr{R}$.
    Then 
    \begin{equation*}
        \int_{a}^{b} F(x)g(x) \d x = 
        F(b)G(b) - F(a)G(a) -
        \int_{a}^{b} f(x)G(x) \d x.
    \end{equation*}
\end{thm}
\mybox{分部积分
\begin{equation*}
    \int_{a}^{b} F(x)g(x) \d x = 
    F(x)G(x) \Bigg|_a^b
    \int_{a}^{b} f(x)G(x) \d x.
\end{equation*}
}

\begin{proof}
    Put $H(x) = F(x)G(x)$ and apply Theorem \ref{thm:6.21} to $H$
    and its derivative. 
    Note that $H' \in \mathscr{R}$, by Theorem \ref{thm:6.13}.
\end{proof}