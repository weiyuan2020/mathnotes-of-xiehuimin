% chap01 exercise
\section*{EXERCISES}

Unless the contrary is explicitly stated, all numbers that are mentioned in these exercises are understood to be real.

\begin{myExercise}
    \label{ex:1.1}
    $r \in \Q $, $r \neq 0$, $x \not\in \Q $, $x \in \R$
    $r+x, rx$ $\not\in \Q , \in \R$
\end{myExercise}


\mySolve

if $r+x \in \Q $, there exists $m, n \in \mathbb{N}, n \neq 0$, s.t. $r+x = \frac{m}{n}$.
    $\because r\in \Q $, $r = \frac{p}{q}, p,q \in \mathbb{N}, q \neq 0$.
    \begin{align*}
        r + x &= \frac{m}{n}\\
        \frac{p}{q} + x &= \frac{m}{n}
    \end{align*}
    \begin{equation*}
        x = \frac{m}{n} - \frac{p}{q} = \frac{mq-np}{nq}
    \end{equation*}
    then $x \in \Q $ contradict to the supposition that $x \not\in \Q $

    If $rx \in \Q $, then $rx = \frac{m}{n}, m,n\in \mathbb{N}$, $x = \frac{qm}{pn} \in \Q $, contradictory!

\mybox{supposition 假设}


\begin{myExercise}
    \label{ex:1.2}
    prove that there is no rational number whose square is $12$. 
\end{myExercise}


\mySolve

If $\left(p/q\right)^2 = 12$, $p^2/q^2 = 12$. $p$ must be even, $p = 2m$.
    $(2m)^2/q^2 = 12$, $m^2/q^2=3$. 
    $3$ is a prime number, $m = 3n$, $(3n)^2/q^2 = 3$, $3n^2 = q^2$, $q$ have a factor $3$,
    $\gcd(p,q) = \gcd(m,q) = \gcd(n,q) = 3 \neq 1$, contradict to the fact that $p,q$ are coprime.


\begin{myExercise}
    \label{ex:1.3}
    Prove Proposition \ref{myProposition:1.15}.
\end{myExercise}


\mySolve

\begin{asparaenum}[(a)]
        \item $x \neq 0$, $xy \neq xz$. $x \neq 0$, $\exists 1/x$, $1/x\cdot x = 1$.
        \begin{align*}
            y & = \left(\frac{1}{}x \cdot x\right) y = \frac{1}{x}(xy)\\
            &=\frac{1}{x}(xz) = \left(\frac{1}{}x \cdot x\right) z = z.
        \end{align*}
        \item $x \neq 0$, $xy = x$ then $y = 1$.   
        Let $z = 1$ in (a).
        \item $x \neq 0$, $xy = 1$ then $y = 1/x$.     
        Let $z = 1/x$ in (a).
        \item $x \neq 0$, $1/(1/x) = x$.     
        $x\cdot \frac{1}{x} = 1$, $\frac{1}{x} \cdot \frac{1}{\frac{1}{x}} = 1.$
        then $x\cdot \frac{1}{x} = \frac{1}{x} \cdot \frac{1}{\frac{1}{x}}$. 
        so $1/(1/x) = x$.
    \end{asparaenum}


\begin{myExercise}
    \label{ex:1.4}
    $E = \varnothing$, $E$ 为有序的非空子集.
    $\alpha$ 是 $E$ 的下界
    $\beta$ 是 $E$ 的上界
    Prove that $\alpha \leq \beta$.
\end{myExercise}


\mySolve

$\forall x\in E$, $\alpha \leq x$, $x\leq \beta$.
   $\alpha \leq x \leq \beta$, $\alpha \leq \beta$.


\begin{myExercise}
    \label{ex:1.5}
    $A$ 为 $\R$ 的非空子集, $A$ 有下界
    \begin{equation*}
        -A = \{-x|x\in A\}
    \end{equation*}
    Prove that $\inf A = -\sup (-A)$
\end{myExercise}


\mySolve

(rudin)
    $\beta = \inf A$, $\alpha = \sup (-A)$.
    \begin{enumerate}[(1)]
        \item $\beta < -\alpha$, $\exists x\in A$, 
        $\beta \leq x < \alpha$, $-x > \alpha$. 矛盾.
        \item $\beta > -\alpha$, $\exists x\in A$, 
        $\alpha \geq -x > -\beta$, $x < \beta$. 矛盾.
    \end{enumerate}
    $\therefore \beta = -\alpha$.


\begin{myExercise}
    \label{ex:1.6}
    Fix $b>1$,
    \begin{asparaenum}[(a)]
        \item If $m, n, p, q$ are integers, $n > 0, q > 0$, and $r = m/n = p/q$, prove that
        \begin{equation*}
            \left(b^m\right)^{1/n} = 
            \left(b^p\right)^{1/q}
        \end{equation*}
        Hence it makes sense to define $b^r = \left(b^m\right)^{1/n}$.
        \item Prove that $b^{r+s} = b^r b^s$ if $r$ and $s$ are rational.
        \item If $x$ is real, define $B(x)$ to be the set of all numbers $b^t$, where $t$ is rational and $t \leq x$. Prove that
        \begin{equation*}
            b^r = \sup B(r)
        \end{equation*}
        when $r$ is rational. Hence it makes sense to define
        \begin{equation*}
            b^x = \sup B(x)
        \end{equation*}
        for every real $x$.
        \item Prove that $b^{x+y} = b^x b^y$ for all real $x$ and $y$.
    \end{asparaenum}
\end{myExercise}

\mySolve

\begin{asparaenum}[(a)]
        \item $\frac{m}{n} = \frac{p}{q}$, $m,n,p,q \in \N$, $n>0,q>0$.
        \begin{equation*}
            b>1, \quad
            b^{\frac{1}{n}}>1, \quad
            b^{\frac{1}{1}}>1.
        \end{equation*}
        Let $(b^m)^{\frac{1}{n}} = c$, $c^n = b^m$.\\
        Let $(b^p)^{\frac{1}{q}} = d$, $d^q = b^p$.\\
        \begin{align*}
            (b^m)^p &= c^{np}\\
            (b^p)^m &= d^{qm}
        \end{align*}
        $qm = np$, $(b^m)^p = b^{mp} = (b^p)^m$,
        $\therefore c^{np} = c^{mq} = d^{np}$.
        $\therefore c = d$. $b^{\frac{m}{n}} = b^{\frac{p}{q}}$.
        \item $b^{r+s} = b^r b^s$, $r, s \in \Q$.
        \begin{equation*}
            b^{m+n} = b^m b^n. 
            \quad m,n \in \N
        \end{equation*}
        $r = \frac{m}{n}$, $c = \frac{p}{q}$.
        \begin{align*}
            b^{r+s} = b^{\frac{m}{n} + \frac{p}{q}}
            &= b^{\frac{mq+np}{nq}} \\
            &= \left( b^{mq+np} \right)^{\frac{1}{nq}} \\
            &= \left( b^{mq} b^{np} \right)^{\frac{1}{nq}} \\
            &= b^{\frac{m}{n}} b^{\frac{p}{q}}  = b^r b^s.
        \end{align*}
        \item $b^r = \sup B(r)$, $r \in \Q$.
        \begin{equation*}
            B(r) = \{b^t | t \leq r\}.
        \end{equation*}
        $\forall x \in B(r)$, $x = b^t \leq b^r$. $(t \leq r, b > 1)$.\\
        $b^r = \sup B(r)$
        \item $b^{x+y} = \sup B(x+y)$.
        \begin{align*}
            b^x &= \sup B(x) \\
            b^y &= \sup B(y) \\
            b^x b^y &= \left( \sup B(x) \right) \left( \sup B(y) \right)
        \end{align*}
        $\{b^r| r\leq x\}$
        $\{b^s| s\leq y\}$
        $\{b^r b^s| r+s\leq x+y\}$
        \begin{equation*}
            \left\{ \begin{array}{ll}
                b^x b^y &\geq b^{x+y} \\
                b^x b^y &\leq b^{x+y}
            \end{array} \right.
        \end{equation*}
        $b^{x+y} = b^x b^y$.
        \item 
    \end{asparaenum}

\mybox{使用大于等于和小于等于同时成立证明等式}

\begin{myExercise}
    \label{ex:1.7}
    Fix $b>1, y>0$, and prove that there is a unique real $x$ such that $b^x =y$, by
    completing the following outline. (This $x$ is called the logarithm of $y$ to the base $b$.)
    \begin{enumerate}[(a)]
        \item For any positive integer $n$, $b^n - 1 \geq n(b- 1)$.
        \item Hence $b- 1 \geq n(b^{1/n}-1)$.
        \item If $t>1$ and $n> (b-1)/(t-1)$, then $b^{1/n} < t$.
        \item If $w$ is such that $b^w < y$, then $b^{w+(1/n)} < y$ for sufficiently large $n$; to see this, apply part (c) with $t =y \cdot b^{-w}$.
        \item If $b^w > y$, then $b^{w-(1/n)} > y$ for sufficiently large $n$.
        \item Let $A$ be the set of all $w$ such that $b^w < y$, and show that $x = \sup A$ satisfies $b^x =y$.
        \item Prove that this $x$ is unique.
    \end{enumerate}
\end{myExercise}

\mySolve

\begin{asparaenum}[(a)]
        \item $b^n - 1 = (b-1)(b^{(n-1)} + b^{(n-2)} + \cdots + 1) > n(b-1)$.
        \item $b \rightarrow b^{1/n}$, $b - 1 = \left( b^{1/n} \right)^n -1 > n\left( b^{1/n}-1 \right)$.
        \item $b-1 > n(b^{1/n}-1) > \frac{b-1}{t-1}(b^{1/n}-1)$. $\because b-1 > 0$, $1 > \frac{b^{1/n}-1}{t-1}$, $\because t-1 >0$, $t-1 > b^{1/n}-1$, $b^{1/n} < t$.
        \item $t = y \cdot b^{-w} > 1$, $n > (b-1)/(t-1)$ is sufficiently large, $b^{1/n} < t = y \cdot b^{-w} $, $b^{w+(1/n)}<y$.
        \item let $t = b^{w}/y > 1$, $n > (b-1)/(t-1)$ is sufficiently large, $b^{(1/n)} < t$, $b^{w-(1/n)}>y$.
        \item $A = \{w|b^w<y\}$. let $x = \sup A$. if $b^x<y$, by (d) there exist $n$ s.t. $^{x+(1/n)}<y$, $x+(1/n) \in A$, $x \neq \sup A$. Else if $b^x>y$, there exist $n$ (large enough) s.t. $b^{x-(1/n)}>y$, $x-(1/n) \notin A$, $x \neq \sup A$. Therefore $b^x = y$ when $x = \sup A$.
        \item Suppose there are two different number $x_1 \neq x_2$, $x_1 = \sup A$, $x_2 = \sup A$, let $x_1 > x_2$, there exists $x_1 > y >x_2$, $\because x_1 = \sup A$, $y \in A$, but we also have $x_2 = \sup A$, $y \not\in A$. therefore $x_1 \nleq x_2$ and vice versa. So that $x = \sup A$ is unique.
    \end{asparaenum}



\begin{myExercise}
    \label{ex:1.8}
    Prove that no order can be defined in the complex field that turns it into an ordered field. \\
    Hint: $-1$ is a square.
\end{myExercise}

\mybox{
    这里序的定义的原则就是要使复数域成为 ordered field, 即满足 ordered field 的几条性质。
没必要穷举, 只需根据 ordered field 的定义性质去推出矛盾。
因为是全序, 所以有$i > 0$或$i < 0$。
若$i > 0$, 则有$-1 > 0$。由此可得..., 矛盾。
若$i < 0$, 则..., 矛盾。
省略部分你来补全吧。

A1:  与 ordered field 第二条性质矛盾。2,则$-i> 0$,有$(-i)^2=-1<0$,同上与第二条性质矛盾。对吧?

Q2: 有点问题,因为$-1$与0的大小关系并不是已知的。要由$-1 > 0$得到$(-1)² = 1 > 0$,又$0 = -1+1 > 0+1 = 1$,矛盾。
\url{https://tieba.baidu.com/p/4026030436}
}

\mySolve
$-1+1=0, 1>0, \therefore -1<0$, $\N is an ordered set$, We already known that $\Q, \R$ are ordered field. (整数集, 有理数域, 实数域)

If all complex number can made up an ordered field, there must exist an order in it. In this order, $i \neq 0$. if $i > 0$, $-1 = (i)^2 > 0$, contradict to the fact in $\R$. Else if $i < 0$, $-i > 0$, $(-i)^2 = -1 > 0$, still wrong. So the complex field can't be an ordered set.


\begin{myExercise}
    \label{ex:1.9}
    Suppose $z=a+ bi$, $w=c+di$. Define $z<w$ if $a<c$, and also if $a=c$ but
    $b < d$. Prove that this turns the set of all complex numbers into an ordered set.
    (This type of order relation is called a \emph{dictionary order}, or \emph{lexicographic order}, for
    obvious reasons.) Does this ordered set have the least-upper-bound property?
\end{myExercise}

\mySolve

This ordered set doesn't have the least-upper-bound property.\\
Suppose $S = {x+iy| x<a \text{ or } x=a, y<b}$,
Let $E = {x+iy|x<a,y<b}$ $\sup E = a+ib \not\in S$.
the least-upper-bound property : $E \subset S, E \neq \varnothing$, $E$ is bounded above, $\sup E \in S$.
Therefore this ordered set doesn't have the least-upper-bound property.

\begin{myExercise}    
    \label{ex:1.10}
    Suppose $z = a + bi$, $w =u + iv$, and
    \begin{equation*}
        a = \left(\frac{|w|+u}{2}\right)^{1/2},\qquad
        b = \left(\frac{|w|-u}{2}\right)^{1/2}.
    \end{equation*}
    Prove that $z^2 = w$ if $v \geq 0$ and that $(\bar{z})^2 = w$ if $v \leq 0$. Conclude that every complex
    number (with one exception!) has two complex square roots.
\end{myExercise}

\mySolve
\begin{align*}
    z^2 &= (a+bi)^2 = a^2 + 2abi - b^2 \\
    &= \frac{|w|+u}{2} + 2\left( \frac{|w|^2-u^2}{4} \right)^{1/2}i - \frac{|w|-u}{2}\\
    &= u + 2 \sqrt{\frac{u^2 + v^2 - u^2}{4}}i \\
    &= u + |v| i 
\end{align*}
$v \geq 0, |v| = v$, $x^2 = w$.

$\bar{z}^2 = u - |v| i = w$ if $v \leq 0$(0 的情况是否重复了?).

$\forall w$, there exist $z$, s.t. $z^2 = w$, $(-z)^2 = w$.
If $w = 0$, $z = -z = 0$, other complex number have two complex square roots.


\begin{myExercise}
    \label{ex:1.11}
    If $z$ is a complex number, prove that there exists an $r \geq 0$ and a complex number
    $w$ with $|w| = 1$ such that $z =rw$. Are $w$ and $r$ always uniquely determined by $z$?
\end{myExercise}

\mySolve 
$z = rw$, $r \in \R$, $r \geq 0$, $|w| = 1$.
$|z| = |r||w| = |r|$, $w = \frac{z}{r} = \frac{z}{|z|}$,
therefore $r,w$ are uniquely determined by $z$.


\begin{myExercise}
    \label{ex:1.12}
    If $z_1 ,..., z_n$ are complex, prove that
    \begin{equation*}
        |z_1 + z_2 ...+ z_n| \leq 
        |z_1| + |z_2| ...+ |z_n|.
    \end{equation*}
\end{myExercise}

\mySolve 
\begin{align*}
    |z_1 + z_2| &\leq |z_1| + |z_2| \\
    |z_1 + z_2 + z_3| &\leq |z_1| + |z_2 + z_3| \leq |z_1| + |z_2| + |z_3| \\
    \text{Suppose } |z_1 + \cdots + z_{n-1}| &\leq |z_1| + \cdots |z_{n-1}| \\ 
    |z_1 + \cdot + z_{n-1} + z_n| &\leq |z_1| + \cdots |z_{n-1} + z_n| 
    &\leq |z_1| + \cdots |z_{n-1}| + |z_n| \\ 
\end{align*}


\begin{myExercise}
    \label{ex:1.13}
    If $x, y$ are complex, prove that
    \begin{equation*}
        ||x|-|y|| \leq |x-y|.
    \end{equation*}
\end{myExercise}

\mySolve 
\begin{align*}
    x &= x - y + y \\
    |x| &\leq |x-y| + |y| 
\end{align*}
\begin{align*}
    |x| - |y| &\leq  |x-y| \\
    |y| - |x| &\leq  |y-x| = |x-y| \\
    \left| |x| - |y| \right| &\leq |x - y|
\end{align*}


\begin{myExercise}
    \label{ex:1.14}
    If $z$ is a complex number such that $|z| = 1$, that is, such that $z\bar{z} = 1$, compute
    \begin{equation*}
        |1+z|^2 + |1-z|^2
    \end{equation*}
\end{myExercise}

\mySolve
\begin{align*}
    |1+z|^2 &= (1+z)\overline{(1+z)} = (1+z)(1+\overline{z}) \\
    &= 1+z+\overline{z}+z\overline{z}
\end{align*}
\begin{align*}
    |1-z|^2 &= (1-z)\overline{(1-z)} = (1-z)(1-\overline{z}) \\
    &= 1-z-\overline{z}+z\overline{z}
\end{align*}
\begin{equation*}
    |1+z|^2 + |1-z|^2 = 2(1+z\overline{z}) = 2\times 2 = 4
\end{equation*}


\begin{myExercise}
    \label{ex:1.15}
    Under what conditions does equality hold in the Schwarz inequality?
\end{myExercise}

\mySolve
\begin{equation*}
    \left|\sum a_j \overline{b_j}\right| \leq \sum |a_j|^2 \sum |b_j|^2 
\end{equation*}
\begin{align*}
    |a_j + \lambda b_j|^2 
    &= (a_j + \lambda b_j) \overline{(a_j + \lambda b_j)} \\
    &= (a_j + \lambda b_j) (\overline{a_j} + \lambda \overline{b_j}) \\
    &= |a_j|^2 + \lambda (a_j \overline{b_j} + \overline{a_j} b_j) + |b_j|^2 \lambda^2 
\end{align*}
$\Delta = (a_j \overline{b_j} + \overline{a_j} b_j)^2 - 4|a_j|^2|b_j|^2 \leq 0$ \\
$\Delta = 0$, $b_j = k a_j$, $k \in \R$.


\begin{myExercise}
    \label{ex:1.16}
    Suppose $k \geq 3$, $\mathbf{x}, \mathbf{y} \in \R^k$, $|\mathbf{x} - \mathbf{y}| =d>0$, and $r >0$. Prove:
    \begin{asparaenum}[(a)]
        \item If $2r > d$, there are infinitely many $\mathbf{z} \in \R^k$ such that
        \begin{equation*}
            |z-x| =|z-y| =r.
        \end{equation*}        
        \item If $2r = d$, there is exactly one such $\mathbf{z}$,
        \item If $2r < d$, there is no such $\mathbf{z}$.
    \end{asparaenum}
    How must these statements be modified if $k$ is $2$ or $1$?
\end{myExercise}

\mySolve
$2r = |z-x|+|z-y| \geq |x-z+z-y| = |x-y| = d$.\\
$2r > d$, there is infinitely $z$ s.t. $|z-x| =|z-y| =r$.\\
$2r = d$, there is only one $z$ s.t. $|z-x| =|z-y| =r$, $z = \frac{x+y}{2}$.\\
$2r < d$, there is no such $z$.


\begin{myExercise}
    \label{ex:1.17}
    Prove that
    \begin{equation*}
        |\mathbf{x} + \mathbf{y}|^2 + 
        |\mathbf{x} - \mathbf{y}|^2 =
        2|\mathbf{x}|^2 + 2|\mathbf{y}|^2
    \end{equation*}
    if $\mathbf{x} \in \R^k$ and $\mathbf{y} \in \R^k$. Interpret this geometrically, as a statement about parallelograms.
\end{myExercise}

\mySolve
$|x+y|^2 = (x+y)\overline{(x+y)} = x\bar{x} + x\bar{y} + y\bar{x} + y\bar{y}$, \\
$|x-y|^2 = (x-y)\overline{(x-y)} = x\bar{x} - x\bar{y} - y\bar{x} + y\bar{y}$, \\
$|x+y|^2 + |x-y|^2 = 2(x\bar{x} + y\bar{y})$. \\
$|x|^2 + |y|^2 = x\bar{x} + y\bar{y}$.


\begin{myExercise}
    \label{ex:1.18}
    If $k >2$ and $\mathbf{x}\in \R^k$, prove that there exists $\mathbf{y} \in \R^k$ such that $\mathbf{y} \neq 0$ but $\mathbf{x}\cdot\mathbf{y} =0$.
    Is this also true if $k =1$?
\end{myExercise}

\mySolve
$x = a+ai, y = a-ai$, $x \neq 0, y \neq 0$, $x\cdot y = a^2-a^2=0$.
When $k=1$, it's false.


\begin{myExercise}
    \label{ex:1.19}
    Suppose $\mathbf{a} \in \R^k$, $\mathbf{b} \in\R^k$. Find $\mathbf{c} \in \R^k$ and $r > 0$ such that
    \begin{equation*}
        |\mathbf{x} - \mathbf{a}| = 2|\mathbf{x} - \mathbf{b}|
    \end{equation*}
    if and only if $|\mathbf{x} - \mathbf{c}| = r$.
    (Solution: $3\mathbf{c} =4\mathbf{b}-\mathbf{a}$, $3r =2|\mathbf{b}-\mathbf{a}|$.)
\end{myExercise}

\mySolve
$|x-a|^2 = 4|x-b|^2$.\\
$x\bar{x} - x\bar{a} - a\bar{x} + a\bar{a} = 4(x\bar{x} - x\bar{b} - b\bar{x} + b\bar{b})$.\\
$|x-c|^2 = r^2$.\\
$x\bar{x} - x\bar{c} - c\bar{x} + c\bar{c} = r^2$.\\
$x\overline{3c-4b+a}+\bar{x}(3c-4b+a)+3r^2-3c\bar{c}+4b\bar{b}-a\bar{a} = 0$.
\begin{align*}
    3c-4b+a&=0\\
    3r^2 &= 3|c|^2 - 4|b|^2 + |a|^2 \\
\end{align*}
$3c = 4b-a$, $3r = 2|b-a|$.


\begin{myExercise}
    \label{ex:1.20}
    With reference to the Appendix, suppose that property (III) were omitted from the
    definition of a cut. Keep the same definitions of order and addition. Show that
    the resulting ordered set has the least-upper-bound property, that addition satisfies
    axioms (A1) to (A4) (with a slightly different zero-element!) but that (A5) fails.
\end{myExercise}

\mySolve


