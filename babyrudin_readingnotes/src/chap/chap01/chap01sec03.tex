
\section{Fields}
\mybox{
    域, 交换除环 <$\R,+,\times$> 
    <$\R,+$>, <$\R\backslash\{0\},\times$>
    都是交换群, 且满足分配律. 

    则 <$\R,+,\times$> 是域. 
}

\begin{mydef}\label{mydef:1.12}
% Axiom % 公理

(A) Axioms for addition
\begin{enumerate}[(\text{A}1)]
    \item If $x\in F$  and $y \in F$, then their sum \(x + y\) is in F.
    \item Addition is commutative: \(x + y=y+ x\) for all \(x, y \in F\).
    \item Addition is associative: \((x+ y)+z = x + (y+ z)\) for all \(x, y, z \in F\).
    \item $F$ contains an element $0$ such that $0 + x = x$ for every $x \in F$.
    \item To every $x\in F$ corresponds an element $-x\in F$ such that
\end{enumerate}
\begin{equation*}
    x+(-x)=0.
\end{equation*}

(M) Axioms for multiplication
\begin{enumerate}[(\text{M}1)]
    \item If $x\in F$ and $x\in F$, then their product $xy$ is in $F$.
    \item Multiplication is commutative: $xy = yx$ for all $x, y \in  F$.
    \item Multiplication is associative: $(xy)z = x(yz)$ for all $x, y, z \in  F$.
    \item $F$ contains an element $1 \neq 0$ such that $1x = x$ for every $x \in F$.
    \item If $x \in F$ and $x \neq 0$ then there exists an element $1/x \in F$ such that
\end{enumerate}
\begin{equation*}
    x\cdot(1/x)=1.
\end{equation*}


(D) The distributive law
\begin{equation*}
    x(y+z)=xy+ xz
\end{equation*}

holds for all $x, y, z \in F$.
\end{mydef}

\begin{myRemark}
    \label{myRemark:1.13}
    \begin{enumerate}[(a)]
        \item Our usual writes (in any filed) $x-y = x+(-y)$, $x/y=x\cdot (1/y)$.
        \item The field axioms clearly hold in $\Q $, 
        the set of all rational numbers, 
        if addition and multiplication have their customary meaning. 
        Thus $\Q $ is a field.
        \item Although it is not our purpose to study fields 
        (or any other algebraic structures) in detail, 
        it is worthwhile to prove that some familiar properties of $\Q $ are consequences of the field axioms; 
        once we do this, we will \underline{not need to do it} again for the real numbers and for the complex numbers.
    \end{enumerate}
\end{myRemark}
 
\mybox{
    \begin{inparaenum}
        \item 只定义了加法和乘法, 使用逆元分别表示减法和除法.
        \item 全体有理数的集合是一个域.
    \end{inparaenum}
}


\begin{myProposition}
    \label{myProposition:1.14}
    The axioms for addition imply the following statements.
    \begin{enumerate}[(a)]
        \item If $x+y=x+z$ then $y=z$.
        \item If $x+y=x$ then $y=0$.
        \item If $x+y=0$ then $y= -x$.
        \item $-(-x)=x$.
    \end{enumerate}
\end{myProposition}

Statement (a) is a cancellation law. 
Note that (b) asserts the uniqueness of the element 
whose existence is assumed in (A4), 
and that (c) does the same for (A5).

\mybox{
    what is the difference between axiom and proposition?

    An axiom is a proposition regarded as self-evidently true without proof. 
    The word ``axiom'' is a slightly archaic synonym for postulate. 
    Compare conjecture or hypothesis, 
    both of which connote apparently true but not self-evident statements.
    
    A proposition is a mathematical statement such as 
    ``3 is greater than 4,'' 
    ``an infinite set exists,''
    or 
    ``7 is prime.''
    
    An axiom is a proposition that is assumed to be true. 
    With sufficient information, m
    athematical logic can often categorize a proposition as true or false, 
    although there are various exceptions 
    (e.g., 
    ``This statement is false''
    ).
    
    \url{https://www.nutritionmodels.com/terminology.html}
}


\begin{proof}
    Proof(rudin)
    If $x + y =x + z$, the axioms (A) give
    \begin{align*}
        y =0+y&=(-x+x)+y=-x+(x+y)\\
        &=-x+(x+z)=(-x+x)+z=0+z=z
    \end{align*}

    This proves (a). Take $z = 0$ in (a) to obtain (b). 
    Take $z= -x$ in (a) to obtain (c).
    Since $-x + x = 0$, (c) (with $-x$ in place of $x$) gives (d).
\end{proof}

\mybox{
    我自己证明上述四条性质时都是从定义开始的, 
    而 rudin 这里在后一步的证明中都利用了刚推导出的结论, 
    这一点需要借鉴.
    }
% THE REAL AND COMPLEX NUMBER SYSTEMS 7

\begin{myProposition}
    \label{myProposition:1.15}
    The axioms for multiplication imply the following statements.
    \begin{enumerate}[(a)]
        \item If $x\neq0$ and $xy=xz$ then $y=z$.
        \item If $x\neq0$ and $xy=x$ then $y=1$.
        \item If $x\neq0$ and $xy=1$ then $y=1/x$.
        \item If $x\neq0$ then $1/(1/x) = x$.
    \end{enumerate}
\end{myProposition}

The proof is so similar to that of Proposition \ref{myProposition:1.14} that we omit it.


\begin{proof}
    \begin{asparaenum}[(a)]
        \item ,
        \begin{align*}
            y&=1\cdot y=\left(\frac{1}{x}\cdot x\right)y =\frac{1}{x}\left( xy \right)\\
            &=\frac{1}{x}(xz) =\left(\frac{1}{x}x\right)z = z
        \end{align*}
        \item (a)Let $z=1$. $y=z=1$.
        \item (a)Let $z=\frac{1}{x}$. $y=z=\frac{1}{x}$.
        \item (c)Let $x=\frac{1}{x'}$. $y=1/(1/x')$.
    \end{asparaenum}
\end{proof}

\begin{myProposition}
    \label{Proposition:1.16}
    The field axioms imply the following statements, for any $x, y, z \in F$.
    \begin{enumerate}[(a)]
        \item $0x=0$.
        \item If $x\neq 0$ and $y\neq 0$ then $xy\neq 0$.
        \item $(-x)y=-(xy)=x(-y)$.
        \item $(-x)(-y)=xy$.
    \end{enumerate}
\end{myProposition}

\begin{proof}
    $0x+0x=(0+0)x=0x$. Hence \ref{myProposition:1.14}(b) implies that $0x=0$, and (a) holds.

    Next, assume $x \neq 0$, $y \neq 0$, but $xy =0$. Then (a) gives
    \begin{equation*}
        1=
        \left(\frac{1}{y}\right)\left(\frac{1}{x}\right)xy=
        \left(\frac{1}{y}\right)\left(\frac{1}{x}\right)0=0.
    \end{equation*}

    a contradiction. Thus (b) holds.

    The first equality in (c) comes from
    \begin{equation*}
        (-x)y +xy=(-x+x)y=0y=0,
    \end{equation*}

    combined with \ref{myProposition:1.14}(c); 
    the other half of (c) is proved in the same way.\\
    Finally,
    \begin{equation*}
        (-x)(-y)=-[x(-y)]=-[-(xy)]=xy
    \end{equation*}
    by (c) and \ref{myProposition:1.14}(d).
\end{proof}

\begin{mydef}
    \label{mydef:1.17}
    An ordered field is a field $F$ which is also an ordered set, 
    such that
    \begin{enumerate}[(i)]
        \item $x+y<x+z$ if $x,y,z\in F$ and $y<z$,
        \item $xy>0$ if $x\in F$, $y\in F$, $x>0$, and $y>0$.
    \end{enumerate}
\end{mydef}

If $x > 0$, we call $x$ positive; 

if $x < 0$, $x$ is negative.

For example, $\Q $ is an ordered field.

All the familiar rules for working with inequalities apply in every ordered
field: 
Multiplication by positive [negative] quantities preserves [reverses] inequalities, no square is negative, etc. 
The following proposition lists some of these.

\mybox{
    有序域$F$也是有序集, 
    由于有理数域 $\Q $, 实数域 $\R$ 都是有序域, 
    这里使用有理数域 $\Q $ 证明的有序集的性质也可以直接用于实数域 $\R$.}

\begin{myProposition}
    \label{myProposition:1.18}
    The following statements are true in every ordered field.
    \begin{enumerate}[(a)]
        \item If $x>0$ then $-x <0$, and vice versa.
        \item If $x>0$ and $y<z$ then $xy <xz$.
        \item If $x<0$ and $y<z$ then $xy> xz$.
        \item If $x \neq 0$ then $x^2 > 0$. In particular, $1 > 0$.
        \item If $0<x<y$ then $0<l/y<l/x$.
    \end{enumerate}
\end{myProposition}

\begin{proof}
    \begin{asparaenum}[(a)]
    \item $x>0$, $-x<0$. 
    \begin{align*}
        x   &> 0=(x+-x)\\
        x+0 &> x+(-x)\\
        (-x)&<0
    \end{align*}

    \item $x>0$, $y<z$, $xy<xz$.
    \begin{align*}
        y<z, z-y&>y-y=0\\
        x(z-y)&>0\\
        x(z-y)+xy&>0+xy\\
        xz&>xy
    \end{align*}

    \item
    \begin{align*}
        (z-y) &>y-y=0\\
        x<0,(-x)>0.\quad (-x)(z-y)&>0 \\
        x(z-y) &<0\\
        xz<xy    
    \end{align*}

    \item
    \begin{align*}
        x>0  && x^2    >0  \\
        x<0  &&(-x)^2 >0, (-x)^2 = -[x(-x)] = -(-(x\cdot x)) =x^2, x^2>0
    \end{align*}
    $\because 1^2=1$, $1>0$.

    \item
    If $y>0$ and $v \leq 0$, then $yv \leq 0$. 
    But $y \cdot (1/y)=1>0$. Hence $1/y > 0$.
    Likewise, $1/x > 0$. 
    If we multiply both sides of the inequality $x <y$ by
    the positive quantity $(1/x)(1/y)$, we obtain $1/y <1/x$.
    \end{asparaenum}
\end{proof}

