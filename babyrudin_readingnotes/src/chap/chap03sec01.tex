% chap03sec01.tex
\section{Convergent sequences}

\begin{mydef}\label{def:3.1}
    A sequences 
    $\{p_n\}$ 
    in metric space $X$ is said to converge if there is a point $p \in X$ with the following property:
    
    For every $\varepsilon >0$ there is an integer $N$ s.t. $n \geq N$ implies that $d(p_n, p) < \varepsilon$. (Here $d$ denotes the distance in $X$.)

    In this case we also say that $\{p_n\}$ converges to $p$, or that $p$ is the limit of $\{p_n\}$. [see Th 3.2(b)], and we write $p_n \rightarrow p$, or

    \begin{equation*}
        \lim_{n \to \infty} p_n = p.
    \end{equation*}

    if $\{p_n\}$ does not converge, it is said to diverge.
\end{mydef}

our definition of ``convergent sequence'' depends not only on $\{p_n\}$ but also on $X$. For instance, the sequence $\{1/n\}$ converges in $\R^1$(to $0$), but fails to converge in the set of all positive real numbers [with $d(x,y) = |x-y|$]. 
In cases of possible ambiguity, we can be more
precise and specify ``convergent in $X$'' rather than ``convergent''.

we recall that the set of all points $p_n (n=1,2, 3,...)$ is the range of 
$\{p_n\}$.
The range of a sequence may be a finite set, or it may be infinite. The sequence
$\{p_n\}$ is said to be bounded if its range is bounded.

As examples, consider the following sequences of complex numbers
(that is, $X = \R^2$):

\begin{enumerate}[(a)]
    \item If $s_n=1/n$, then $\lim_{n \to \infty} s_n = 0$; the range is infinite, and the sequence is bounded.
    \item If $s_n=n^2$ the sequence $\{s_n\}$ is unbounded, is divergent, and has infinite range.
    \item If $s_n = 1+[(- 1)^n/n]$, the sequence $\{s_n\}$ converges to $1$, is bounded, and has infinite range.
    \item If $s_n =i^n$ the sequence $\{s_n\}$ is divergent, is bounded, and has finite range.
    \item If $s_n = 1(n=1,2,3,...)$, then $\{s_n\}$ converges to $1$, is bounded, and has finite range.
\end{enumerate}

% We now summarize some important properties of convergent sequences in metric spaces.

\begin{thm}\label{thm:3.2}
    Let$\{p_n\}$ be a sequence in a metric space $X$.
    \begin{enumerate}[(a)]
        \item $\{p_n\}$ converges to $p \in X$ if and only if every neighborhood of $p$ contains $p_n$ for all but finitely many $n$.
        \item If $p\in X$, $p^\prime \in X$, and if $\{p_n\}$ converges to $p$ and to $p'$, then $p^\prime =p$.
        \item If $\{p_n\}$ converges, then $\{p_n\}$ is bounded.
        \item If $E \subset X$ and if $p$ is a limit point of $E$, then there is a sequence$\{p_n\}$ in $E$ such that $p = \lim_{n \to \infty} p_n$.
    \end{enumerate}
\end{thm}

\begin{proof}
    (d) For each positive integer $n$, there is a point $p_n \in E$ such that $d(p_n,p) <1/n$. Given $\varepsilon > 0$, choose $N$ so that $N \varepsilon >1$. If $n>N$, it follows that $d(p_n, p) <\varepsilon$. Hence $p_n \rightarrow p$.
\end{proof}

\begin{thm}\label{thm:3.3}
    Suppose $\{s_n\}, \{t_n\}$ are complex sequences, and 
    $\lim_{n \to \infty} s_n = s$,
    $\lim_{n \to \infty} t_n = t$.
    Then
    \begin{enumerate}[(a)]
        \item $\lim_{n \to \infty} (s_n + t_n) = s + t$;
        \item $\lim_{n \to \infty} c s_n = cs$, $\lim_{n \to \infty} (c + s_n) = c + s$, for any number $c$;
        \item $\lim_{n \to \infty} s_n t_n = st$;
        \item $\lim_{n \to \infty} \frac{1}{s_n} = \frac{1}{s}$, provided $s_n \neq 0(n = 1,2,3,\dots)$, and $s \neq 0$.
    \end{enumerate}
\end{thm}

\begin{proof}
    \begin{equation}
        \label{eq:3.1}
        s_n t_n - st = (s_n - s)(t_n - t) + s(t_n - t) + t(s_n - s).
    \end{equation}
\end{proof}

\begin{thm} 
    \label{thm:3.4}
    \begin{asparaenum}[(a)]
    \item Suppose $\mathbf{x}_n \in R^k (n = 1,2,3,\dots)$ and
    \begin{equation*}
        \mathbf{x_n} = (
            \alpha_{1,n},\dots
            \alpha_{k,n}
        ).
    \end{equation*}
    Then $\{\mathbf{x}_n\}$ converges to $\mathbf{x} = (\alpha_1, \dots, \alpha_k)$ if and only if
    \begin{equation}
        \lim_{n \to \infty} \alpha_{j,n} = \alpha_j \qquad (1\leq j\leq k).
    \end{equation}

    \item Suppose $\{\mathbf{x}_n\}$, $\{\mathbf{y}_n\}$ are sequences in $\R^k$, $\{\beta_n\}$ is a sequence of real numbers, and 
    $\mathbf{x}_n \rightarrow \mathbf{x}$,
    $\mathbf{y}_n \rightarrow \mathbf{y}$,
    $\beta_n \rightarrow \beta$. Then
    \begin{equation*}
        \lim_{n \to \infty} (\mathbf{x_n} + \mathbf{y_n}) = \mathbf{x} + \mathbf{y}, \quad
        \lim_{n \to \infty} \mathbf{x_n} \cdot \mathbf{y_n} = \mathbf{x} \cdot \mathbf{y}, \quad
        \lim_{n \to \infty} \beta_n \mathbf{x_n} = \beta \mathbf{x}.
    \end{equation*}
    \end{asparaenum}
\end{thm}
