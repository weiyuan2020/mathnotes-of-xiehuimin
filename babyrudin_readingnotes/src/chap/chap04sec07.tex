% chap04sec07
\section{Infinite limits and limits at infinity}
To enable us to operate in the extended real number system, 
we shall now enlarge the scope of Definition \ref{myDef:4.1}, 
by reformulating it in terms of neighborhoods.

For any real number $x$, we have already defined a neighborhood of $x$ to
be any segment $(x - \delta, x + \delta)$.

\mymyDef{
    \label{myDef:4.32}
    For any real $c$, the set of real numbers $x$ such that $x > c$ 
    is called a neighborhood of $+\infty$ and is written $(c, +\infty)$. Similarly, the set $(-\infty , c)$ is a neighborhood of $-\infty$ .
}

\mymyDef{
    \label{myDef:4.33}
    Let $f$ be a real function defined on $E \subset R$. 
    We say that 
    \begin{equation*}
        f(t) \rightarrow A \text{ as } t \rightarrow x,
    \end{equation*}
    where $A$ and $x$ are in the extended real number system, 
    if for every neighborhood $U$ of $A$ 
    there is a neighborhood $V$ of $x$ 
    such that $V \cap E$ is not empty, 
    and such that $f(t) \in U$ for all $t \in V \cap E$, $t \neq x$.
    
    A moment's consideration will show that this coincides with Definition \ref{myDef:4.1} when $A$ and $x$ are real.
    The analogue of Theorem \ref{thm:4.4} is still true, 
    and the proof offers nothing new. 
    We state it, for the sake of completeness.
}
\mybox{sake 目的、理由、缘故...等意思,经常以for the sake of 的形式出现}

\begin{thm}
    \label{thm:4.34}
    Let $f$ and $g$ be defined on $E \in \R$. Suppose
    \begin{equation*}
        f(t) \rightarrow A, \quad
        g(t) \rightarrow B, \quad
        \text{ as } t \rightarrow x.
    \end{equation*}
    Then
\begin{enumerate}[(a)]
    \item $f(t) \rightarrow A'$ implies $A' = A$.
    \item $(f + g)(t) \rightarrow A + B$,
    \item $(fg)(t) \rightarrow AB$,
    \item $(f /g)(t) \rightarrow A/B$,
\end{enumerate}
provided the right members of (b), (c), and (d) are defined.
\end{thm}

Note that $\infty  - \infty$ , $0 \cdot \infty$ , $\infty /\infty$ , $A/0$ are not defined (see Definition 1.23).