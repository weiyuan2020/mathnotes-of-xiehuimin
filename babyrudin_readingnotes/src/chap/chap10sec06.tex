% chap10sec06

\section{Simplexes and chains}

\begin{mydef}
    \myKeyword{Affine simplexes}
    A mapping $\mathbf{f}$ that carries a vector space $X$ into a
    vector space $Y$ is said to be \emph{affine} if $\mathbf{f - f(0)}$ is linear. 
    In other words, the requirement is that
    \begin{equation}
        \label{eq:10.73}
        \mathbf{f(x)} = 
        \mathbf{f(0)} + A \mathbf{x}
    \end{equation}
    for some $A \in L(X, Y)$.

    An affine mapping of $\R^k$ into $\R^n$ is thus determined if we know $\mathbf{f}(0)$ and $\mathbf{f(e_i)}$ for $1 \leq i \leq k$; 
    as usual, $\{\mathbf{e}_1, ... , \mathbf{e}_k\}$ is the standard basis of $\R^k$.
    
    We define the \emph{standard simplex} $Q^k$ to be the set of all $\mathbf{u} \in \R^k$ of the form
    \begin{equation}
        \label{eq;10.74}
        \mathbf{u} = \sum_{i=1}^{k} \alpha_i \mathbf{e}_i
    \end{equation}
    such that $\alpha \geq 0$ for $i = 1, ... , k$ and $\sum \alpha_i \leq 1$.
    % todo
\end{mydef}

\begin{thm}
    \label{thm:10.27}
    If $\delta$ is an oriented rectilinear $k$-simplex in an open set $E \subset \R^n$ 
    and if $\bar{\Delta} = \varepsilon \delta$ then
    \begin{equation}
        \label{eq:10.81}
        \int_{\bar{\delta}} \omega = 
        \varepsilon \int_{\delta} \omega 
    \end{equation}
    for every $k$-form $\omega$ in $E$.
\end{thm}

% todo add proof

\begin{mydef}
    \myKeyword{Affine chains}
    An \emph{affine $k$-chain} $\Gamma$ in an open set $E \subset \R^n$ is a collection of finitely many oriented affine $k$-simplexes $\delta_1, \dots, \delta_r$ in $E$. 
    These need not be distinct; 
    a simplex may thus occur in $\Gamma$ with a certain multiplicity.

    If $\Gamma$ is as above, and if $\omega$ is a $k$-form in $E$,
    we define 
    \begin{equation}
        \label{eq:10.82}
        \int_{\Gamma} \omega =
        \sum_{i=1}^{r} \int_{\sigma_i} \omega .
    \end{equation}
    % todo add word
\end{mydef}

\begin{mydef}
    \myKeyword{Boundaires}
    For $k \geq 1$, the \emph{boundary} of the oriented affine $k$-simplex
    \begin{equation*}
        \delta = \left[ \mathbf{p_0,p_1,\dots,p_k} \right]
    \end{equation*}
    is defined to be the affine $(k - 1)$-chain
    \begin{equation}
        \label{eq:10.85}
        \partial \delta = \sum_{j=0}^{k} (-1)^j
        \left[ \mathbf{p_0,p_1,\dots,p_{j-1},p_{j+1},\dots,p_k} \right]
    \end{equation}
\end{mydef}

% todo add word

\begin{mydef}
    \myKeyword{Differentiable simplexes and chains}
\end{mydef}

\begin{mydef}
    \myKeyword{Positively oriented boundaries}
\end{mydef}

\begin{myExample}
    For $0 \leq u \leq \pi, 0 \leq v \leq 2\pi$, define
    \begin{equation*}
        \sum(u,v) = \left( 
            \sin u \cos v,
            \sin u \sin v,
            \cos u
         \right).
    \end{equation*}
    % todo
\end{myExample}
