% chap09sec04
\section{The inverse function theorem}

The inverse function theorem states, roughly speaking, that a continuously
differentiable mapping $\mathbf{f}$ is invertible in a neighborhood of any point $\mathbf{x}$ at which
the linear transformation $\mathbf{f'(x)}$ is invertible:

\begin{thm}
    \label{thm:9.24}
    Suppose $\mathbf{f}$ is a $\mathscr{C}'$-mapping of an open set $E \subset \R^n$ into $\R^n$, 
    $\mathbf{f'(a)}$ is invertible for some $\mathbf{a} \in E$, 
    and $\mathbf{b = f(a)}$. Then
% (a)
% there exist open sets U and Vin Rn such that a e U, be V, f is one-to-
% one on U, and f(U) = V;
% (b) if g is the inverse off [which exists, by (a)], defined in V by
% g(f(x)) = X
% (x e U),
% then g e <fl'(V).
\end{thm}

\begin{thm}
    \label{thm:9.25}
%     /ff is a <67'-mapping of an open set E c Rn into Rn and if f'(x)
% is invertible for every x e E, then f ( W) is an open subset of Rn for every open set
% WcE.
\end{thm}