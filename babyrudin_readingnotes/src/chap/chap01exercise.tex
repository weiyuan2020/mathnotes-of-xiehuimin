% chap01 exercise
\section*{EXERCISES}

Unless the contrary is explicitly stated, all numbers that are mentioned in these exercises are understood to be real.

\begin{myExercise}
    $r \in \mathbb{Q}$, $r \neq 0$, $x \not\in \mathbb{Q}$, $x \in \mathbb{R}$
    $r+x, rx$ $\not\in \mathbb{Q}, \in \mathbb{R}$
\end{myExercise}

\begin{mySolve}
if $r+x \in \mathbb{Q}$, there exists $m, n \in \mathbb{N}, n \neq 0$, s.t. $r+x = \frac{m}{n}$.
$\because r\in \mathbb{Q}$, $r = \frac{p}{q}, p,q \in \mathbb{N}, q \neq 0$.
\begin{align*}
    r + x &= \frac{m}{n}\\
    \frac{p}{q} + x &= \frac{m}{n}
\end{align*}
\begin{equation*}
    x = \frac{m}{n} - \frac{p}{q} = \frac{mq-np}{nq}
\end{equation*}
then $x \in \mathbb{Q}$ contradict to the supposion that $x \not\in \mathbb{Q}$

If $rx \in \mathbb{Q}$, then $rx = \frac{m}{n}, m,n\in \mathbb{N}$, $x = \frac{qm}{pn} \in \mathbb{Q}$, contradictory!
\end{mySolve}



\begin{myExercise}
   prove that there is no rational number whose square is $12$. 
\end{myExercise}

\begin{mySolve}
    If $\left(p/q\right)^2 = 12$, $p^2/q^2 = 12$. $p$ must be even, $p = 2m$.
    $(2m)^2/q^2 = 12$, $m^2/q^2=3$. 
    $3$ is a prime number, $m = 3n$, $(3n)^2/q^2 = 3$, $3n^2 = q^2$, $q$ have a factor $3$,
    $\gcd(p,q) = \gcd(m,q) = \gcd(n,q) = 3 \neq 1$, contradict to the fact that $p,q$ are coprime.
\end{mySolve}

\begin{myExercise}
    Prove Proposition \ref{Proposition:1.15}.
\end{myExercise}

\begin{mySolve}
    (a) $x \neq 0$, $xy \neq xz$. $x \neq 0$, $\exists 1/x$, $1/x\cdot x = 1$.
    \begin{align*}
        y & = \left(\frac{1}{}x \cdot x\right) y = \frac{1}{x}(xy)\\
        &=\frac{1}{x}(xz) = \left(\frac{1}{}x \cdot x\right) z = z.
    \end{align*}

    (b) $x \neq 0$, $xy = x$ then $y = 1$.\\
    Let $z = 1$ in (a).

    (c) $x \neq 0$, $xy = 1$ then $y = 1/x$.\\
    Let $z = 1/x$ in (a).

    (d) $x \neq 0$, $1/(1/x) = x$.\\
    $x\cdot \frac{1}{x} = 1$, $\frac{1}{x} \cdot \frac{1}{\frac{1}{x}} = 1$.
    then $x\cdot \frac{1}{x} = \frac{1}{x} \cdot \frac{1}{\frac{1}{x}}$.
    so $1/(1/x) = x$.
\end{mySolve}

\begin{myExercise}
    $E = \varnothing$, $E$ 为有序的非空子集.
    $\alpha$ 是 $E$ 的下界
    $\beta$ 是 $E$ 的上界
    Prove that $\alpha \leq \beta$.
\end{myExercise}

\begin{mySolve}
   $\forall x\in E$, $\alpha \leq x$, $x\leq \beta$.
   $\alpha \leq x \leq \beta$, $\alpha \leq \beta$.
\end{mySolve}

\begin{myExercise}
    $A$ 为 $\mathbb{R}$ 的非空子集, $A$ 有下界
    \begin{equation*}
        -A = \{-x|x\in A\}
    \end{equation*}
    Prove that $\inf A = -\sup (-A)$
\end{myExercise}

\begin{mySolve}(rudin)
    $\beta = \inf A$, $\alpha = \sup (-A)$.

    (1) $\beta < -\alpha$, $\exists x\in A$, $\beta \leq x < \alpha$, $-x > \alpha$. 矛盾.
    
    (2) $\beta > -\alpha$, $\exists x\in A$, $\alpha \geq -x > -\beta$, $x < \beta$. 矛盾.

    $\therefore \beta = -\alpha$.
\end{mySolve}


