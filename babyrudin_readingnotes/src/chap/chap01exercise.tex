% chap01 exercise
\section*{EXERCISES}

Unless the contrary is explicitly stated, all numbers that are mentioned in these exercises are understood to be real.

\begin{myExercise}
    $r \in \mathbb{Q}$, $r \neq 0$, $x \not\in \mathbb{Q}$, $x \in \mathbb{R}$
    $r+x, rx$ $\not\in \mathbb{Q}, \in \mathbb{R}$
\end{myExercise}

\begin{mySolve}
if $r+x \in \mathbb{Q}$, there exists $m, n \in \mathbb{N}, n \neq 0$, s.t. $r+x = \frac{m}{n}$.
$\because r\in \mathbb{Q}$, $r = \frac{p}{q}, p,q \in \mathbb{N}, q \neq 0$.
\begin{align*}
    r + x &= \frac{m}{n}\\
    \frac{p}{q} + x &= \frac{m}{n}
\end{align*}
\begin{equation*}
    x = \frac{m}{n} - \frac{p}{q} = \frac{mq-np}{nq}
\end{equation*}
then $x \in \mathbb{Q}$ contradict to the supposion that $x \not\in \mathbb{Q}$

If $rx \in \mathbb{Q}$, then $rx = \frac{m}{n}, m,n\in \mathbb{N}$, $x = \frac{qm}{pn} \in \mathbb{Q}$, contradictory!
\end{mySolve}



\begin{myExercise}
   prove that there is no rational number whose square is $12$. 
\end{myExercise}

\begin{mySolve}
    If $\left(p/q\right)^2 = 12$, $p^2/q^2 = 12$. $p$ must be even, $p = 2m$.
    $(2m)^2/q^2 = 12$, $m^2/q^2=3$. 
    $3$ is a prime number, $m = 3n$, $(3n)^2/q^2 = 3$, $3n^2 = q^2$, $q$ have a factor $3$,
    $\gcd(p,q) = \gcd(m,q) = \gcd(n,q) = 3 \neq 1$, contradict to the fact that $p,q$ are coprime.
\end{mySolve}

\begin{myExercise}
    Prove Proposition \ref{Proposition:1.15}.
\end{myExercise}

\begin{mySolve}
    \begin{asparaenum}[(a)]
        \item $x \neq 0$, $xy \neq xz$. $x \neq 0$, $\exists 1/x$, $1/x\cdot x = 1$.
        \begin{align*}
            y & = \left(\frac{1}{}x \cdot x\right) y = \frac{1}{x}(xy)\\
            &=\frac{1}{x}(xz) = \left(\frac{1}{}x \cdot x\right) z = z.
        \end{align*}
        \item $x \neq 0$, $xy = x$ then $y = 1$.\\    Let $z = 1$ in (a).
        \item $x \neq 0$, $xy = 1$ then $y = 1/x$.\\     Let $z = 1/x$ in (a).
        \item $x \neq 0$, $1/(1/x) = x$.\\     $x\cdot \frac{1}{x} = 1$, $\frac{1}{x} \cdot \frac{1}{\frac{1}{x}} = 1$.     then $x\cdot \frac{1}{x} = \frac{1}{x} \cdot \frac{1}{\frac{1}{x}}$.     so $1/(1/x) = x$.
    \end{asparaenum}
\end{mySolve}

\begin{myExercise}
    $E = \varnothing$, $E$ 为有序的非空子集.
    $\alpha$ 是 $E$ 的下界
    $\beta$ 是 $E$ 的上界
    Prove that $\alpha \leq \beta$.
\end{myExercise}

\begin{mySolve}
   $\forall x\in E$, $\alpha \leq x$, $x\leq \beta$.
   $\alpha \leq x \leq \beta$, $\alpha \leq \beta$.
\end{mySolve}

\begin{myExercise}
    $A$ 为 $\mathbb{R}$ 的非空子集, $A$ 有下界
    \begin{equation*}
        -A = \{-x|x\in A\}
    \end{equation*}
    Prove that $\inf A = -\sup (-A)$
\end{myExercise}

\begin{mySolve}(rudin)
    $\beta = \inf A$, $\alpha = \sup (-A)$.
    \begin{enumerate}[(1)]
        \item $\beta < -\alpha$, $\exists x\in A$, $\beta \leq x < \alpha$, $-x > \alpha$. 矛盾.
        \item $\beta > -\alpha$, $\exists x\in A$, $\alpha \geq -x > -\beta$, $x < \beta$. 矛盾.
    \end{enumerate}
    $\therefore \beta = -\alpha$.
\end{mySolve}

\begin{myExercise}
    Fix $b>1$,
    \begin{asparaenum}[(a)]
        \item If $m, n, p, q$ are integers, $n > 0, q > 0$, and $r = m/n = p/q$, prove that
        \begin{equation*}
            \left(b^m\right)^{1/n} = 
            \left(b^p\right)^{1/q}
        \end{equation*}
        Hence it makes sense to define $b^r = \left(b^m\right)^{1/n}$.
        \item Prove that $b^{r+s} = b^r b^s$ if $r$ and $s$ are rational.
        \item If $x$ is real, define $B(x)$ to be the set of all numbers $b^t$, where $t$ is rational and $t \leq x$. Prove that
        \begin{equation*}
            b^r = \sup B(r)
        \end{equation*}
        when $r$ is rational. Hence it makes sense to define
        \begin{equation*}
            b^x = \sup B(x)
        \end{equation*}
        for every real $x$.
        \item Prove that $b^{x+y} = b^x b^y$ for all real $x$ and $y$.
    \end{asparaenum}
\end{myExercise}

\begin{myExercise}
    Fix $b>1, y>0$, and prove that there is a unique real $x$ such that $b^x =y$, by
    completing the following outline. (This $x$ is called the logarithm of $y$ to the base $b$.)
    \begin{enumerate}[(a)]
        \item For any positive integer $n$, $b^n - 1 \geq n(b- 1)$.
        \item Hence $b- 1 \geq n(b^{1/n}- 1)$.
        \item If $t>1$ and $n> (b-1)/(t- 1)$, then $b^{1/n} < t$.
        \item If $w$ is such that $b^w < y$, then $b^{w+(1/n)} < y$ for sufficiently large $n$; to see this, apply part (c) with $t =y \cdot b^{-w}$.
        \item If $b^w > y$, then $b^{w-(1/n)} > y$ for sufficiently large $n$.
        \item Let $A$ be the set of all $w$ such that $b^w < y$, and show that $x = \sup A$ satisfies $b^x =y$.
    \end{enumerate}
Prove that this $x$ is unique.    
\end{myExercise}

\begin{myExercise}
    Prove that no order can be defined in the complex field that turns it into an ordered field. \\
    Hint: $-1$ is a square.
\end{myExercise}

\begin{myExercise}
    Suppose $z=a+ bi$, $w=c+di$. Define $z<w$ if $a<c$, and also if $a=c$ but
    $b < d$. Prove that this turns the set of all complex numbers into an ordered set.
    (This type of order relation is called a \emph{dictionary order}, or \emph{lexicographic order}, for
    obvious reasons.) Does this ordered set have the least-upper-bound property?
\end{myExercise}

\begin{myExercise}    
    Suppose $z = a + bi$, $w =u + iv$, and
    \begin{equation*}
        a = \left(\frac{|w|+u}{2}\right)^{1/2},\qquad
        b = \left(\frac{|w|-u}{2}\right)^{1/2}.
    \end{equation*}
    Prove that $z^2 = w$ if $v \geq 0$ and that $(\bar{z})^2 = w$ if $v \leq 0$. Conclude that every complex
    number (with one exception!) has two complex square roots.
\end{myExercise}

\begin{myExercise}
    If $z$ is a complex number, prove that there exists an $r \geq 0$ and a complex number
    $w$ with $|w| = 1$ such that $z =rw$. Are $w$ and $r$ always uniquely determined by $z$?
\end{myExercise}

\begin{myExercise}
    If $z_1 ,..., z_n$ are complex, prove that
    \begin{equation*}
        |z_1 + z_2 ...+ z_n| \leq 
        |z_1| + |z_2| ...+ |z_n|.
    \end{equation*}
\end{myExercise}

\begin{myExercise}
    If $x, y$ are complex, prove that
    \begin{equation*}
        ||x|-|y|| \leq |x-y|.
    \end{equation*}
\end{myExercise}

\begin{myExercise}
    If $z$ is a complex number such that $|z| = 1$, that is, such that $z\bar{z} = 1$, compute
    \begin{equation*}
        |1+z|^2 + |1-z|^2
    \end{equation*}
\end{myExercise}

\begin{myExercise}
    Under what conditions does equality hold in the Schwarz inequality?
\end{myExercise}

\begin{myExercise}
    Suppose $k \geq 3$, $\mathbf{x}, \mathbf{y} \in \mathbb{R}^k$, $|\mathbf{x} - \mathbf{y}| =d>0$, and $r >0$. Prove:
    \begin{asparaenum}[(a)]
        \item If $2r > d$, there are infinitely many $\mathbf{z} \in \mathbb{R}^k$ such that
        \begin{equation*}
            |z-x| =|z-y| =r.
        \end{equation*}        
        \item If $2r = d$, there is exactly one such $\mathbf{z}$,
        \item If $2r < d$, there is no such $\mathbf{z}$.
    \end{asparaenum}
    How must these statements be modified if $k$ is $2$ or $1$?
\end{myExercise}

\begin{myExercise}
    Prove that
    \begin{equation*}
        |\mathbf{x} + \mathbf{y}|^2 + 
        |\mathbf{x} - \mathbf{y}|^2 =
        2|\mathbf{x}|^2 + 2|\mathbf{y}|^2
    \end{equation*}
    if $\mathbf{x} \in \mathbb{R}^k$ and $\mathbf{y} \in \mathbb{R}^k$. Interpret this geometrically, as a statement about parallelograms.
\end{myExercise}

\begin{myExercise}
    If $k >2$ and $\mathbf{x}\in \mathbb{R}^k$, prove that there exists $\mathbf{y} \in \mathbb{R}^k$ such that $\mathbf{y} \neq 0$ but $\mathbf{x}\cdot\mathbf{y} =0$.
    Is this also true if $k =1$?
\end{myExercise}

\begin{myExercise}
    Suppose $\mathbf{a} \in \mathbb{R}^k$, $\mathbf{b} \in\mathbb{R}^k$. Find $\mathbf{c} \in \mathbb{R}^k$ and $r > 0$ such that
    \begin{equation*}
        |\mathbf{x} - \mathbf{a}| = 2|\mathbf{x} - \mathbf{b}|
    \end{equation*}
    if and only if $|\mathbf{x} - \mathbf{a}| = r$.
    (Solution: $3\mathbf{c} =4\mathbf{b}-\mathbf{a}$, $3r =2|\mathbf{b}-\mathbf{a}|$.)
\end{myExercise}

\begin{myExercise}
    With reference to the Appendix, suppose that property (III) were omitted from the
    definition of a cut. Keep the same definitions of order and addition. Show that
    the resulting ordered set has the least-upper-bound property, that addition satisfies
    axioms (A1) to (A4) (with a slightly different zero-element!) but that (A5) fails.
\end{myExercise}
