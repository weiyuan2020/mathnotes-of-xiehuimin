% chap11sec02

\section{Constriction of the lebesgue measure}

\begin{mydef}
    \label{mydef:11.4}
    Let $\R^p$ denote $p$-dimensional euclidean space. 
    By an \emph{interval} in $\R^p$ we mean the set of points $\mathbf{x} = (x_1 , ... , x_p)$ such that
    \begin{equation}
        \label{eq:11.10}
        a_i \leq x_i \leq b_i
        \quad 
        (i = 1,\dots,p),
    \end{equation}
    or the set of points which is characterized by (\ref{eq:11.10}) with any or all of the $\leq$ signs replaced by $<$. 
    The possibility that $a_i = b_i$ for any value of $i$ is not ruled out; 
    in particular, the empty set is included among the intervals.

    If $A$ is the union of a finite number of intervals, 
    $A$ is said to be an \emph{elementary set}.

    If $I$ is an interval, we define
    \begin{equation*}
        m(I) = \prod_{i=1}^{p} (b_i - a_i) ,
    \end{equation*}
    no matter whether equality is included or excluded in any of the inequalities (\ref{eq:11.10}).

    If $A = I_1 \cup \cdots \cup I_n$, and if these intervals are pairwise disjoint, we set 
    \begin{equation}
        \label{eq:11.11}
        m(A) = 
        m(I_1) + \cdots + 
        m(I_n) . 
    \end{equation}

    We let $\mathscr{E}$ denote the family of all elementary subsets of $\R^p$.

    At this point, the following properties should be verified:
    \begin{enumerate}[(a)]
        \item $\mathscr{E}$ is a ring, but not a $\sigma$-ring.
        \item If $A \in \mathscr{E}$, then $A$ is the union of a finite number of \emph{disjoint} intervals.
        \item If $A \in \mathscr{E}$, $m(A)$ is well defined by (\ref{eq:11.11}); that is. if two different decompositions of $A$ into disjoint intervals are used, each gives rise to the same value of $m(A)$.
        \item $m$ is additive on $\mathscr{E}$
    \end{enumerate}

    Note that if $p = 1,2,3$, then $m$ is length, area, and volume, respectively.
\end{mydef}
\mybox{原书这里的列表项使用公式编号记录...}
\begin{mydef}
    \label{mydef:11.5}
    A nonnegative additive set function $\phi$ defined on $\mathscr{E}$ is said to be regular if the following is true: 
    To every $A \in \mathscr{E}$ and to every $\varepsilon > 0$ 
    there exist sets $F \in \mathscr{E}$, $G \in \mathscr{E}$ 
    such that $F$ is closed, $G$ is open, $F \subset A \subset G$, 
    and
    \begin{equation}
        \label{eq:11.16}
        \phi(G) - \varepsilon \leq 
        \phi(A) \leq 
        \phi(F) + \varepsilon . 
    \end{equation}
\end{mydef}

\begin{newexample}
    \begin{asparaenum}[(a)]
        \item \emph{The set function $m$ is regular.}
        If $A$ is an interval, it is trivial that the requirements of Definition \ref{mydef:11.5} are satisfied. The general case follows from \ref{mydef:11.4} property (b).
        \item Take $\R^p = \R^1$, and let $\alpha$ be a monotonically increasing function, defined for all real $x$. Put 
        \begin{align*}
            \mu([a,b]) &= \alpha(b-)-\alpha(a-),\\
            \mu([a,b]) &= \alpha(b+)-\alpha(a+),\\
            \mu([a,b]) &= \alpha(b+)-\alpha(a+),\\
            \mu([a,b]) &= \alpha(b-)-\alpha(a-).
        \end{align*}
        Here $[a,b)$ is the set $a \leq x < b$, etc.
        Because of the possible discontinuities of $\alpha$, these cases have to be distinguished.
        If $\mu$ is defined for elementary sets as in (\ref{eq:11.11}), $\mu$ is regular on $\mathscr{E}$.
        The proof is just like that of (a)
    \end{asparaenum}
\end{newexample}

Our next objective is to show that every regular set function on $\mathscr{E}$ can be
extended to a countably additive set function on a $\sigma$-ring which contains $\mathscr{E}$.

\begin{mydef}
    \label{mydef:11.7}
    outer measure $\mu^*(E)$
    % Let µ be additive, regular, nonnegative, and finite on 8.
% Consider countable coverings of any set E c RP by open elementary sets An:
\end{mydef}

\begin{thm}
    \label{thm:11.8}
    \begin{asparaenum}[(a)]
        \item For every $A \in \mathscr{E}$, $\mu^* (A) = \mu (A)$.
        \item If $E = \cup_1^{\infty} E_n$, then 
        \begin{equation}
            \label{eq:11.19}
            \mu^* (E) \leq 
            \sum_{n=1}^{\infty} \mu^* (E_n) .
        \end{equation}
    \end{asparaenum}
\end{thm}

Note that (a) asserts that $\mu^*$ is an extension of $\mu$ from $\mathscr{E}$ to the family of all subsets of $\R^P$. 
The property (\ref{eq:11.19}) is called \emph{subadditivity}.

% todo add proof 

\begin{mydef}
    \label{mydef:11.9}
    finitely $\mu$-measurable
    $A \in \mathfrak{M}_F(\mu)$
\end{mydef}

\begin{thm}
    \label{thm:11.10}
    $\mathfrak{M}(\mu)$ is a $\sigma$-ring, and $\mu^*$ is countably additive on $\mathfrak{M}(\mu)$.
\end{thm}

properties of $S(A, B)$ and $d(A, B)$.

\begin{myremark}
    \label{myremark:11.11}
\end{myremark}