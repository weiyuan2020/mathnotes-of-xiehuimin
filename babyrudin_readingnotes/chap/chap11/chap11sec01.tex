% chap11sec01

\section{Set functions}

If $A$ and $B$ are any two sets, 
we write $A - B$ for the set of all elements $x$ such that 
$x \in A, x \not\in B$. 
The notation $A - B$ does not imply that $B \subset A$. 
We denote
the empty set by 0, 
and say that $A$ and $B$ are disjoint if $A \cap B = 0$.


\begin{mydef}
    A family $\mathscr{R}$ of sets is called a ring if A e $\mathscr{R}$ and Be $\mathscr{R}$ implies
    \begin{equation}
        \label{eq:11.1}
        A \cup B \in \mathscr{R}, \quad 
        A - B \in \mathscr{R}.
    \end{equation}
    Since $A n B = A - (A - B)$, we also have $A \cap B \in \mathscr{R}$ if $\mathscr{R}$ is a ring.

    A ring $\mathscr{R}$ is called a $\sigma$-\emph{ring} if 
    \begin{equation}
        \label{eq:11.2}
        \bigcup_{n=1}^{\infty} A_n \in \mathscr{R}
    \end{equation}
    whenever $A_n \in \mathscr{R} (n = 1,2,3,\dots)$. 
    Since 
    \begin{equation*}
        \bigcap_{n=1}^{\infty} A_n 
        = A_1 - \bigcup_{n=1}^{\infty} (A_1 - A_n),
    \end{equation*}
    we also have 
    \begin{equation*}
        \bigcap_{n=1}^{\infty} A_n \in \mathscr{R}
    \end{equation*}
    if $\mathscr{R}$ is a $\sigma$-ring.
\end{mydef}

\begin{mydef}
    \label{mydef:11.2}
    We say that $\phi$ is a set function defined on $\mathscr{R}$ if $\phi$ assigns to every $A \in \mathscr{R}$ a number $@f(A)$ of the extended real number system.
    $\phi$ is \emph{additive} if $A \cap B = 0$ implies
    \begin{equation}
        \label{eq:11.3}
        \phi \left( A \cup B \right) = 
        \phi (A) + \phi (B),
    \end{equation}
    and $\phi$ is \emph{countably additive} if $A_i \cap A_j = 0 (i \neq j)$ implies
    \begin{equation}
        \label{eq:11.4}
        \phi\left( \bigcup_{n=1}^{\infty} A_n \right) =
        \sum_{n=1}^{\infty} \phi\left( A_n \right) .
    \end{equation}
    We shall always assume that the range of $\phi$ does not contain both $+ \infty$ and $- \infty$; 
    for if it did, the right side of (\ref{eq:11.3}) could become meaningless. 
    Also, we exclude set functions whose only value is $+ \infty$ or $- \infty$.
    
    It is interesting to note that the left side of (\ref{eq:11.4}) is independent of the order in which the $A_n$'s are arranged. 
    Hence the rearrangement theorem shows that the right side of (\ref{eq:11.4}) converges absolutely if it converges at all; 
    if it does not converge, the partial sums tend to $+ \infty$, or to $- \infty$.

    If $\phi$ is additive, the following properties are easily verified:
    \begin{align}
        \phi(0) &= 0 \label{eq:11.5}\\
        \phi \left( A_1 \cup \cdots \cup A_n \right) 
        &= \phi(A_1) + \cdots + \phi(A_n) \label{eq:11.6}
    \end{align}
    if $A_i \cap A_j = 0$ whenever $i \neq j$.
    \begin{equation}
        \label{eq:11.7}
        \phi \left( A_1 \cup A_2 \right) +
        \phi \left( A_1 \cap A_2 \right) =
        \phi (A_1) + \phi (A_2).
    \end{equation}

    If $\phi(A) \geq 0$ for all $A$, and $A_1 \subset A_2$, then 
    \begin{equation}
        \label{eq:11.8}
        \phi(A_1) \leq \phi(A_2) .
    \end{equation}

    Because of (\ref{eq:11.8}), nonnegative additive set functions are often called monotonic.
    \begin{equation}
        \label{eq:11.9}
        \phi\left( A - B \right) = 
        \phi\left( A \right) -
        \phi\left( B \right) 
    \end{equation}
    if $B \subset A$, and $\left| \left( \phi B \right) \right| < +\infty $.
\end{mydef}

\begin{thm}
    \label{thm:11.3}
    Suppose $\phi$ is countably additive on a ring $\mathscr{R}$.
    Suppose $A_n \in \mathscr{R} (n = 1,2,3,\dots)$,
    $A_1 \subset A_2 \subset A_3 \subset \cdots$, $A \in \mathscr{R}$, and 
    \begin{equation*}
        A = \bigcup_{n=1}^{\infty} A_n .
    \end{equation*}
    Then, as $n \rightarrow \infty$,
    \begin{equation*}
        \phi(A_n) \rightarrow \phi(A) .
    \end{equation*}
\end{thm}

