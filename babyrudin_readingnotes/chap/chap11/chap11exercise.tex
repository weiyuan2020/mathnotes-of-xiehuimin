% chap11exercise

\section{Exercises}

\begin{myexercise}
    \label{ex:11.1}
    If $f \geq 0$ and $\int_E f \d = 0$, prove that $f(x) = 0$ almost everywhere on $E$.

    \emph{Hint:} Let $E_n$ be the subset of $E$ on which $f(x) > 1/n$.
    Write $A = \bigcup E_n$.
    Then $\mu(A)= 0$ if and only if $\mu(E_n)= 0$ for every $n$.
\end{myexercise}


\begin{myexercise}
    \label{ex:11.2}
    If $\int_A f d \mu = 0$ for every measurable subset $A$ of a measurable set $E$, then $f(x) = 0$ almost everywhere on $E$.
\end{myexercise}


\begin{myexercise}
    \label{ex:11.3}
    If $\{f_n\}$ is a sequence of measurable functions, prove that the set of points $x$ at which $\{f_n(x)\}$ converges is measurable.
\end{myexercise}


\begin{myexercise}
    \label{ex:11.4}
    If $f \in \mathscr{L}(\mu)$ on $E$ and $g$ is bounded and measurable on $E$, then $fg \in \mathscr{L}(\mu)$ on $E$.
\end{myexercise}


\begin{myexercise}
    \label{ex:11.5}
    Put
    \begin{align*}
        g(x)             & =
        \begin{array}{l}
            0 \\
            1 \\
        \end{array} &
        \begin{array}{l}
            (0 \leq x \leq \frac{1}{2}) \\
            (\frac{1}{2} \leq x \leq 1) \\
        \end{array}                     \\
        f_{2k}(x)        & = g(x)   & (0 \leq x \leq 1) \\
        f_{2k+1}(x)      & = g(1-x) & (0 \leq x \leq 1)
    \end{align*}
    Show that
    \begin{equation*}
        \liminf_{n \to \infty} f_n(x) = 0
        \quad
        (0 \leq x \leq 1),
    \end{equation*}
    but
    \begin{equation*}
        \int_{0}^{1} f_n(x) \d x = \frac{1}{2}.
    \end{equation*}
    [Compare with \eqref{eq:11.77}.]
\end{myexercise}


\begin{myexercise}
    \label{ex:11.6}
    \begin{equation*}
        f_n(x) = \begin{array}{ll}
            \frac{1}{n} & (|x| \leq n), \\
            0           & (|x|    > n).
        \end{array}
    \end{equation*}
    Then $f_n(x) \rightarrow 0$ uniformly on $\R^1$, but
    \begin{equation*}
        \int_{-\infty}^{\infty} f_n \d x = 2
        \quad
        (n = 1,2,3,\dots).
    \end{equation*}
    (We write $\int_{-\infty}^{\infty}$ in place of $\int_{\R 1}$.)
    Thus uniform convergence does not imply dominated convergence in the sense of Theorem \ref{thm:11.32}.
    However, on sets of finite measure, uniformly convergent sequences of bounded functions do satisfy Theorem \ref{thm:11.32}.
\end{myexercise}


\begin{myexercise}
    \label{ex:11.7}
    Find a necessary and sufficient condition that $f \in \mathscr{R}(\alpha)$ on $[a, b]$.
    \emph{Hint:} Consider Example \ref{neqexample:11.6}(b) and Theorem \ref{thm:11.33}.
\end{myexercise}


\begin{myexercise}
    \label{ex:11.8}
    If $f \in \mathscr{R}$ on $[a, b]$
    and if $F(x) = \int_{a}^{b} f(t) \d t$,
    prove that $F'(x) =f(x)$ almost everywhere on $[a, b]$.
\end{myexercise}


\begin{myexercise}
    \label{ex:11.9}
    Prove that the function $F$ given by \eqref{eq:11.96} is continuous on $[a, b]$.
\end{myexercise}



\begin{myexercise}
    \label{ex:11.10}
\end{myexercise}


\begin{myexercise}
    \label{ex:11.11}
\end{myexercise}


\begin{myexercise}
    \label{ex:11.12}
\end{myexercise}


\begin{myexercise}
    \label{ex:11.13}
\end{myexercise}


\begin{myexercise}
    \label{ex:11.14}
\end{myexercise}


\begin{myexercise}
    \label{ex:11.15}
\end{myexercise}


\begin{myexercise}
    \label{ex:11.16}
\end{myexercise}


\begin{myexercise}
    \label{ex:11.17}
\end{myexercise}


\begin{myexercise}
    \label{ex:11.18}
\end{myexercise}

