% chap06exercise

\section{Exercises}

\begin{myExercise}
    \label{ex:6.1}
    Suppose $\alpha$ increases on $[a, b]$, $a \leq x_0 \leq b$, 
    $\alpha$ is continuous at $x_0$, $f(x_0) = 1$, and
    $f(x) = 0$ if $x \neq x_0$. 
    Prove that $f \in \mathscr{R}(\alpha)$ and that $\int f \d \alpha = 0$.
\end{myExercise}


\begin{myExercise}
    \label{ex:6.2}
    Suppose $f \geq 0$, 
    $f$ is continuous on $[a, b]$, and
    $\int_{a}^{b} \d x = 0$. 
    Prove that $f(x) = 0$
    for all $x \in [a, b]$. 
    (Compare this with Exercise \ref{ex:6.1}.)
\end{myExercise}


\begin{myExercise}
    \label{ex:6.3}
    Define three functions $\beta_1$, $\beta_2$, $\beta_3$ as follows: 
    $\beta_j(x) = 0$ if $x < 0$, 
    $\beta_j(x) = 1$ if $x > 0$ 
    for $j = 1, 2, 3$; 
    and $\beta_1(0) = 0$, $\beta_2(0) =1$, $\beta_3(0) = \frac{1}{2}$. Let $f$ be a bounded function on $[-1,1]$.
    \begin{asparaenum}[(a)]
        \item Prove that $f \in \mathscr{R}(\beta_1)$ if and only if $f(0+) = f(0)$ and that then 
        \begin{equation*}
            \int f \d \beta_1 = f(0)
        \end{equation*}
        \item State and prove a similar result for $\beta_2$.
        \item Prove that $f \in \mathscr{R}(\beta_3)$ if and only if $f$ is continuous at 0.
        \item If $f$ is continuous at 0 prove that 
        \begin{equation*}
            \int f \d \beta_1 = 
            \int f \d \beta_2 = 
            \int f \d \beta_3 = f(0).
        \end{equation*}
    \end{asparaenum}
\end{myExercise}


\begin{myExercise}
    \label{ex:6.4}
    If $f(x) = 0$ for all irrational $x$,
    $f(x) = 1$ for all rational $x$, 
    prove that $f \in \mathscr{R}$ on $[a, b]$ for any $a < b$.
\end{myExercise}


\begin{myExercise}
    \label{ex:6.5}
    Suppose $f$ is a bounded real function on $[a, b]$, 
    and $f^2 \in \mathscr{R}$ on $[a, b]$. 
    Does it follow that $f \in \mathscr{R}$? 
    Does the answer change if we assume that $f^3 \in \mathscr{R}$?
\end{myExercise}


\begin{myExercise}
    \label{ex:6.6}
    Let $P$ be the Cantor set constructed in Sec. \ref{mydef:2.44}. 
    Let $f$ be a bounded real function on $[0, 1]$ which is continuous at every point outside $P$. 
    Prove that $f \in \mathscr{R}$ on $[0, 1]$.
    
    \emph{Hint:} $P$ can be covered by finitely many segments whose total length can be made as small as desired. 
    Proceed as in Theorem \ref{thm:6.10}.
\end{myExercise}


\begin{myExercise}
    \label{ex:6.7}
    Suppose $f$ is a real function on $(0, 1]$ and $f \in \mathscr{R}$ on $[c, 1]$ for every $c > 0$. 
    Define
    \begin{equation*}
        \int_{0}^{1} f(x) \d x =
        \lim_{c \to 0} \int_{c}^{1} f(x) \d x
    \end{equation*}
    if this limit exists (and is finite).
    \begin{enumerate}[(a)]
        \item If $f \in \mathscr{R}$ on $[0,1]$, show that this definition of the integral agree with the old one.
        \item Construct a function $f$ such that the above limit exists, although it fails to exist with $|f|$ in place of $f$.
    \end{enumerate}
\end{myExercise}


\begin{myExercise}
    \label{ex:6.8}
    Suppose $f \in \mathscr{R}$ on $[a, b]$ for every $b > a$ where $a$ is fixed. 
    Define
    \begin{equation*}
        \int_{a}^{\infty} f(x) \d x = 
        \lim_{b \to \infty} \int_{a}^{b} f(x) \d x
    \end{equation*}
    if this limit exists (and is finite). 
    In that case, we say that the integral on the left \myKeywordblue{converges}. 
    If it also converges after $f$ has been replaced by $|f|$, it is said to converge \myKeywordblue{absolutely}.

    Assume that $f(x) \geq 0$ and that $f$ decreases monotonically on $[1, \infty)$. 
    Prove that
    \begin{equation*}
        \int_{1}^{\infty} f(x) \d x
    \end{equation*}
    converges if and only if 
    \begin{equation*}
        \sum_{n=1}^{\infty} f(n)
    \end{equation*}
    converges.
    (This is the so-called ``integral test'' for convergence of series.)
\end{myExercise}


\begin{myExercise}
    \label{ex:6.9}
    Show that integration by parts can sometimes be applied to the ``improper'' integrals defined in Exercises \ref{ex:6.7} and \ref{ex:6.8}. 
    (State appropriate hypotheses, formulate a theorem, and prove it.) 
    For instance show that
    \begin{equation*}
        \int_{0}^{\infty} \frac{\cos x}{1+x} \d x = 
        \int_{0}^{\infty} \frac{\sin x}{(1+x)^2} \d x .
    \end{equation*}
    Show that one of these integrals converges \myKeywordblue{absolutely}, but that the other does not.
\end{myExercise}


\begin{myExercise}
    \label{ex:6.10}
    Let $p$ and $q$ be positive real numbers such that
    \begin{equation*}
        \frac{1}{p} + \frac{1}{q} = 1 .
    \end{equation*}
    Prove the following statements.
    \begin{asparaenum}[(a)]
        \item If $u \geq 0$ and $v \geq 0$, then
        \begin{equation*}
            uv \leq \frac{u^p}{p} + \frac{v^q}{q} .
        \end{equation*}
        Equality holds if and only of $u^p = v^q$.
        \item If $f \in \mathscr{R}$, $g \in \mathscr{R}$, $f \geq 0$, $g \geq 0$, and 
        \begin{equation*}
            \int_{a}^{b} f^p \d \alpha = 1
            \int_{a}^{b} g^q \d \alpha ,
        \end{equation*}
        then 
        \begin{equation*}
            \int_{a}^{b} fg \d \alpha \leq 1.
        \end{equation*}
        \item if $f$ and $g$ are complex functions in $\mathscr{R}$, then 
        \begin{equation*}
            \left| \int_{a}^{b} fg \d \alpha \right| \leq
            \left\{ \int_{a}^{b} \left| f \right|^p \d \alpha \right\}^{1/p} 
            \left\{ \int_{a}^{b} \left| g \right|^q \d \alpha \right\}^{1/q} .
        \end{equation*}
        This is \myKeywordblue{H\"{o}lder's inequality}.
        When $p = q = 2$ it is usually called the Schwarz
        inequality. 
        (Note that Theorem \ref{thm:1.35} is a very special case of this.)
        \item Show that H\"{o}lder's inequality is also true fir the ``proper'' integrals described in Exercises \ref{ex:6.7} and \ref{ex:6.8} 
    \end{asparaenum}
\end{myExercise}


\begin{myExercise}
    \label{ex:6.11}
    Let $\alpha$ be a fixed increasing function on $[a, b]$. 
    For $u \in \mathscr{R}(\alpha)$, define
    \begin{equation*}
        \left\| u \right\|_2 = \left\{ \int_a^b \left| u \right|^2 \d \alpha \right\}^{1/2} .
    \end{equation*}
    Suppose $f, g, h \in \mathscr{R}(\alpha)$, and prove the triangle inequality
    \begin{equation*}
        \left\| f-h \right\|_2 \leq
        \left\| f-g \right\|_2 +
        \left\| g-h \right\|_2 
    \end{equation*}
    as a consequence of the Schwarz inequality, as in the proof of Theorem \ref{thm:1.37}.
\end{myExercise}


\begin{myExercise}
    \label{ex:6.12}
    With the notations of Exercise \ref{ex:6.11}, 
    suppose $f \in \mathscr{R}(\alpha)$ and $\varepsilon > 0$. 
    Prove that there exists a continuous function $g$ on $[a, b]$ such that $\left\| f-g \right\|_2 < \varepsilon$.

    \emph{Hint:} Let $P = \{x_0,...,x_n\}$ be a suitable partition of $[a, b]$, define
    \begin{equation*}
        g(t) = 
        \frac{x_i-t}{\Delta x_i} f(x_{i-1}) +
        \frac{t-x_{i-1}}{\Delta x_i} f(x_{i})
    \end{equation*}
\end{myExercise}


\begin{myExercise}
    \label{ex:6.13}
    Define 
    \begin{equation*}
        f(x) = \int_{x}^{x+1} \sin (t^2) \d t.
    \end{equation*}
    \begin{asparaenum}[(a)]
        \item Prove that $\left| f(x) \right| < 1/x$ if x>0.
        \emph{Hint:} Put $t^2 = u$ and integrate by parts, to show that $f(x)$ is equal to 
        \begin{equation*}
            \frac{\cos(x^2)}{2x} - \frac{\cos [(x+1)^2]}{2(x+1)} - \int_{x^2}^{(x+1)^2}\frac{\cos u}{4 u^{3/2}} \d u.
        \end{equation*}
        Replace $\cos u$ by $-1$.
        \item Prove that 
        \begin{equation*}
            2xf(x) = \cos(x^2) - \cos [(x+1)^2] + r(x)
        \end{equation*}
        where $\left| r(x) \right| < c/x$ and $c$ is a constant.
        \item Find the upper and lower limits of $x f(x)$, as $x \rightarrow \infty$.
        \item Does $\int_{0}^{\infty} \sin (t^2) \d t$ converge?
    \end{asparaenum}
\end{myExercise}


\begin{myExercise}
    \label{ex:6.14}
    Deal similarly with
    \begin{equation*}
        f(x) = \int_{x}^{x+1} \sin (e^t) \d t . 
    \end{equation*}
    Show that 
    \begin{equation*}
        e^x \left| f(x) \right| < 2
    \end{equation*}
    and that 
    \begin{equation*}
        e^x f(x) = \cos (e^x) - e^{-1} \cos (e^{x+1}) + r(x),
    \end{equation*}
    where $\left| r(x) \right| < C e^{-x}$, for some constant $C$.
\end{myExercise}


\begin{myExercise}
    \label{ex:6.15}
    Suppose $f$ is a real, continuously differentiable function on $[a, b]$,$f(a) =f(b) = 0$, and
    \begin{equation*}
        \int_{a}^{b} f^2 (x) \d x = 1.
    \end{equation*}
    Prove that 
    \begin{equation*}
        \int_{a}^{b} x f(x) f'(x) \d x = -\frac{1}{2}
    \end{equation*}
    and that 
    \begin{equation*}
        \int_{a}^{b}[f'(x)]^2 \d x \cdot 
        \int_{a}^{b}x^2 f^2(x) \d x >
        \frac{1}{4} .
    \end{equation*}
\end{myExercise}


\begin{myExercise}
    \label{ex:6.16}
    For $1<s<\infty$ , define 
    \begin{equation*}
        \zeta(s) = \sum_{n=1}^{\infty} \frac{1}{n^2} .
    \end{equation*}
    (This is Riemann's zeta function, of great importance in the study of the distribution of prime numbers.) 
    Prove that
    \begin{asparaenum}[(a)]
        \item $\zeta(s) = s \int_{1}^{\infty} \frac{[x]}{x^{s+1}}\d x$ 
        and that 
        \item $\zeta(s) = \frac{s}{s-1} - s \int_{1}^{\infty} \frac{x - [x]}{x^{s+1}} \d x$ ,
        where $[x]$ denotes the greatest integer $\leq x$.
    \end{asparaenum}

    Prove that the integral in (b) converges for all $s>0$.

    \emph{Hint:} To prove (a), compute the difference between the integral over $[1,N]$ and the $N$th partial sum of the series that defines $\zeta(s)$.
\end{myExercise}


\begin{myExercise}
    \label{ex:6.17}
    Suppose $\alpha$ increases monotonically on $[a, b]$, 
    $g$ is continuous, and $g(x) = G'(x)$ 
    for $a \leq x \leq b$. 
    Prove that
    \begin{equation*}
        \int_{a}^{b} \alpha(x) g(x) \d x =
        G(b)\alpha(b) - G(a)\alpha(a) - \int_{a}^{b}G \d \alpha .
    \end{equation*}

    \emph{Hint:} Take $g$ real, without loss of generality.
    Given $P = \{x_0,x_1,...,x_n\}$, choose $t_1 \in (x_{i-1},x_i)$ so that $g(t_i)\Delta x_i = G(x_i) - G(x_{i-1})$.
    Show that 
    \begin{equation*}
        \sum_{i=1}^{n} \alpha(x_i) g(t_i) \Delta x_i =
        G(b)\alpha(b) - G(a)\alpha(a) - \sum_{i=1}^{n}G(x_{i-1}) \Delta \alpha_i .
    \end{equation*}
\end{myExercise}


\begin{myExercise}
    \label{ex:6.18}
    Let $\gamma_1, \gamma_2, \gamma_3$ be curves in the complex plane, defined on $[0, 2\pi]$ by
    \begin{equation*}
        \gamma_1(t) = e^{it} , \quad
        \gamma_1(t) = e^{2it} , \quad
        \gamma_1(t) = e^{2\pi it \sin (1/t)} .
    \end{equation*}
    Show that these three curves have the same range, 
    that $\gamma_1$ and $\gamma_2$ are rectifiable, 
    that the length of $\gamma_1$ is $2\pi$, 
    that the length of $\gamma_2$ is $4\pi$, 
    and that $\gamma_3$ is not rectifiable.
\end{myExercise}


\begin{myExercise}
    \label{ex:6.19}
    Let $\gamma_1$ be a curve in $\R^k$, defined on $[a, b]$; 
    let $\phi$ be a continuous 1-1 mapping of $[c, d]$ onto $[a, b]$, such that $\phi(c) = a$; 
    and define $\gamma_2(s) = \gamma_1(\phi(s))$. 
    Prove that $\gamma_2$ is an arc, a closed curve, or a rectifiable curve if and only if the same is true of $\gamma_1$.
    Prove that $\gamma_2$ and $\gamma_1$ have the same length.
\end{myExercise}