% chap08sec01
\section{Power series}
In this section we shall derive some properties of functions which are represented by power series, i.e., functions of the form
\begin{equation}
    \label{eq:8.1}
    f(x) = \sum_{n=0}^{\infty} c_n x^n
\end{equation}
or, more generally,
\begin{equation}
    \label{eq:8.2}
    f(x) = \sum_{n=0}^{\infty} c_n (x - a)^n.
\end{equation}

These are called \emph{analytic functions}.

We shall restrict ourselves to real values of $x$. 
Instead of circles of convergence (see Theorem \ref{thm:3.39}) 
we shall therefore encounter intervals of convergence.

If (\ref{eq:8.1}) converges for all $x$ in $(-R, R)$, 
for some $R > 0$ ($R$ may be $+ \infty$),
we say that $f$ is expanded in a power series about the point $x = 0$. 
Similarly, if (\ref{eq:8.2}) converges for $\left| x - a \right|  < R$, 
$f$ is said to be expanded in a power series about the point $x = a$. 
As a matter of convenience, we shall often take $a = 0$ without
any loss of generality.

\begin{thm}
    \label{thm:8.1}
    Suppose the series
    \begin{equation}
        \label{eq:8.3}
        \sum_{n=0}^{\infty} c_n x^n
    \end{equation}
    converges for $\left| x \right| < R$, and define
    \begin{equation}
        \label{eq:8.4}
        f(x) = 
        \sum_{n=0}^{\infty} c_n x^n
        \quad 
        (\left| x \right| < R).
    \end{equation}

    Then (\ref{eq:8.3}) converges uniformly on $[-R+\varepsilon, R-\varepsilon]$, no matter which $\varepsilon > 0$ is chosen. 
    The function $f$ is continuous and differentiable in $(-R, R)$, and
    \begin{equation}
        \label{eq:8.5}
        f'(x) = 
        \sum_{n=1}^{\infty} n c_n x^{n-1}
        \quad 
        (\left| x \right| < R).
    \end{equation}
\end{thm}

% todo add proof

\begin{myCorollary*}
    Under the hypotheses of Theorem \ref{thm:8.1}, 
    $f$ has derivatives of all orders in $(-R, R)$, 
    which are given by
    \begin{equation}
        \label{eq:8.6}
        f^{(k)} (x) = 
        \sum_{n=k}^{\infty} n(n-1)\cdots(n-k+1) c_n x^{n-k}.
    \end{equation}
    
    In particular
    \begin{equation}
        \label{eq:8.7}
        f^{(k)} (0) = k! c_k
        \quad 
        (k = 0,1,2,\dots).
    \end{equation}
    (Here $f^{(0)}$ means $f$, and $f^{(k)}$ is the $k$th derivatives of $f$,
    for $k = 1,2,3,\dots$).
\end{myCorollary*}

% todo add proof

\begin{thm}
    \label{thm:8.2}
    Suppose $\sum c_n$ converges.
    Put 
    \begin{equation*}
        f(x) = \sum_{n=0}^{\infty} c_n x^n 
        \quad (-1 < x < 1).
    \end{equation*}
    Then
    \begin{equation}
        \label{eq:8.8}
        \lim_{x \to 1} f(x) = 
        \sum_{n=0}^{\infty} c_n.
    \end{equation}
\end{thm}

% todo add proof

\begin{thm}
    \label{thm:8.3}
    Given a double sequence $\sequence{a_{ij}}$, 
    $i = 1, 2, 3, ...$, 
    $j = 1, 2, 3, ...$,
    suppose that
    \begin{equation}
        \label{eq:8.12}
        \sum_{j=1}^{\infty} \left| a_{ij} \right| = b_i
        \quad (i = 1,2,3,\dots)
    \end{equation}
    and $\sum b_i$ converges. Then
    \begin{equation}
        \label{eq:8.13}
        \sum_{i=1}^{\infty} 
        \sum_{j=1}^{\infty} a_{ij} =
        \sum_{j=1}^{\infty} 
        \sum_{i=1}^{\infty} a_{ij} .
    \end{equation}
\end{thm}

% todo add proof

\begin{thm}
    \label{thm:8.4}
    Suppose
    \begin{equation*}
        f(x) = \sum_{n=0}^{\infty} c_n x^n ,
    \end{equation*}
    the series converging in $\left| x \right| < R$. 
    If $-R < a < R$, then $f$ can be expanded in a power series about the point $x = a$ which converges in $| x - a | < R - | a |$ , and
    \begin{equation}
        \label{eq:8.17}
        f(x) = 
        \sum_{i=1}^{\infty} 
        \frac{f^{(n)}(a)}{n!}\left( x - a \right)^n
        \quad
        (\left| x - a \right| < R - \left| a \right|) .
    \end{equation}
    This is an extension of Theorem \ref{thm:5.15} and is also known as \emph{Taylor's theorem}.
\end{thm}

% todo add proof

\begin{thm}
    \label{thm:8.5}
    Suppose the series $\sum a_n x^n$ and $\sum b_n x^n$ converge in the segment $S = (-R, R)$. 
    Let $E$ be the set of all $x \in S$ at which
    \begin{equation}
        \label{eq:8.20}
        \sum_{n=0}^{\infty} a_n x^n =
        \sum_{n=0}^{\infty} b_n x^n .
    \end{equation}
    If $E$ has a limit point in $S$, 
    then $a_n = b_n$ for $n = 0, 1, 2, ...$. 
    Hence (\ref{eq:8.20}) holds for all $x \in S$.
\end{thm}

% todo add proof

