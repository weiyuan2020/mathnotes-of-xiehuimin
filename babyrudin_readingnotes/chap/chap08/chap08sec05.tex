% chap08sec05
\section{Fourier series}
\mybox{傅里叶级数}
\begin{mydef}
    \label{mydef:8.9}
    A \emph{trigonomeric polynomial} isa finite sum of the from 
    \begin{equation}
        \label{eq:8.59}
        f(x) = a_0 + \sum_{n=1}^{N}(a_n \cos n x + b_n \sin n x)
        \quad (x \text{ real}),
    \end{equation}
\end{mydef}
where $a_0,\dots,a_N$, $b_0,\dots,b_N$ are complex numbers.
On account of the identities (\ref{eq:8.46}),
(\ref{eq:8.59}) can also be written in the form 
\begin{equation}
    \label{eq:8.60}
    f(x) = \sum_{-N}^{N} c_n e^{inx}
    \quad (x \text{ real}),
\end{equation}
which is more convenient for most purposes.
It is clear that every trigonometrc polynomial is periodic, with period $2\pi$.

If $n$ is a nonzero integer, $e^{inx}$ is the derivative of $e^{inx}/in$,
which also has period $2\pi$. Hence
\begin{equation}
    \label{eq:8.61}
    \frac{1}{2\pi}\int_{-\pi}^{\pi}e^{inx} \d x  =
    \left\{ 
        \begin{array}{ll}
            1 & (\text{if } n = 0), \\
            0 & (\text{if } n = \pm 1, \pm 2, \dots). \\
        \end{array}
     \right.
\end{equation}

% todo add words, p186
\begin{equation}
    \label{eq:8.62}
    c_m = \frac{1}{2\pi} \int_{-\pi}^{\pi} f(x) e^{-imx} \d x
\end{equation}

\begin{equation}
    \label{eq:8.63}
    \sum_{-\infty}^{\infty} c_n e^{inx} 
    \quad 
    (x \text{ real});
\end{equation}


\begin{mydef}
    \label{mydef:8.10}
    Let $\sequence{\phi_n}$ $(n = 1,2,3,\dots)$ be a sequence of complex functions on $[a,b]$ , such that 
    \begin{equation}
        \label{eq:8.64}
        \int_{a}^{b} \phi_n (x) \overline{\phi_m (x)} \d x
        \quad (n \neq m).
    \end{equation}
    Then $\{\phi n\}$ is said to be an \myKeywordblue{orthogonal system of functions} on $[a, b]$. 
    If, in addition,
    \begin{equation}
        \label{eq:8.65}
        \int_{a}^{b} \left| \phi_n(x) \right|^2 \d x = 1
    \end{equation}
    for all $n, \{\phi_n\}$ is said to be \myKeywordblue{orthonormal}.

    For example, the functions $(2\pi)^{-\frac{1}{2}}e^{inx}$ form an orthonormal system on $[-n, n]$. 
    So do the real functions
    \begin{equation*}
        \frac{1}{\sqrt{2\pi}},
        \frac{\cos x}{\sqrt{\pi}},
        \frac{\sin x}{\sqrt{\pi}},
        \frac{\cos 2x}{\sqrt{\pi}},
        \frac{\sin 2x}{\sqrt{\pi}}, \cdots ,
    \end{equation*}
    If $\{\phi_n\}$ is orthonormal on $[a,b]$ and if 
    \begin{equation}
        \label{eq:8.66}
        c_n = \int_{a}^{b} f(t) \overline{\phi_n(t)} \d t
        \quad 
        (n=1,2,3,\dots),
    \end{equation}
    we call $c_n$ the $n$th Fourier coefficient of $f$ relative to $\{\phi_n\}$.
    We write 
    \begin{equation}
        \label{eq:8.67}
        f(x) \sim \sum_{1}^{\infty} c_n \phi_n (x)
    \end{equation}
    and call this series the Fourier series off (relative to $\{ \phi_n\}$ ).
    
    Note that the symbol $\sim$ used in (\ref{eq:8.67}) implies nothing about the convergence of the series; it merely says that the coefficients are given by (\ref{eq:8.66}).

    The following theorems show that the partial sums of the Fourier series of $f$ have a certain minimum property. 
    We shall assume here and in the rest of this chapter that
    $f \in \mathscr{R}$, although this hypothesis can be weakened.
\end{mydef}


\begin{thm}
    \label{thm:8.11}
    Let $\{\phi_n\}$ be orthonormal on $[a,b]$. 
    Let 
    \begin{equation}
        \label{eq:8.68}
        s_n(x) = \sum_{m=1}^{n} c_m \phi_m (x)
    \end{equation}
    be the $n$th partial sum of the Fourier series of $f$, and suppose 
    \begin{equation}
        \label{eq:8.69}
        t_n(x) = \sum_{m=1}^{n} \gamma_m \phi_m (x).
    \end{equation}
    Then 
    \begin{equation}
        \label{eq:8.70}
        \int_{a}^{b} \left| f-s_n \right|^2 \d x \leq
        \int_{a}^{b} \left| f-t_n \right|^2 \d x ,
    \end{equation}
    and equality holds if and only if 
    \begin{equation}
        \label{eq:8.71}
        \gamma_m = c_m 
        \quad 
        (m=1,\dots,n).
    \end{equation}
\end{thm}

That is to say, among all functions $t_n$, $s_n$ gives the best possible mean square approximation to $f$.

% todo add proof

\begin{equation}
    \label{eq:8.72}
    \int_{a}^{b} \left| s_n(x) \right|^2 \d x = 
    \sum_{1}^{n} \left| c_m \right|^2 \leq
    \int_{a}^{b} \left| f(x) \right|^2 \d x ,
\end{equation}
since $\int \left| f-t_n \right|^2 \geq 0$.


\begin{thm}
    \label{thm:8.12}
    If $\{\phi_n\}$ is orthonormal on $[a,b]$, and if 
    \begin{equation*}
        f(x) \sim \sum_{n=1}^{\infty} c_n \phi_n (x),
    \end{equation*}
    then 
    \begin{equation}
        \label{eq:8.73}
        \sum_{n=1}^{\infty} \left| c_n \right|^2 \leq 
        \int_{a}^{b} \left| f(x) \right|^2 \d x .
    \end{equation}
    In particular, 
    \begin{equation}
        \label{eq:8.74}
        \lim_{n \to \infty} c_n = 0 .
    \end{equation}
\end{thm}

\begin{proof}
    Letting $n \rightarrow \infty $ in (\ref{eq:8.72}), we obtain (\ref{eq:8.73}), the so-called ``Bessel inequality.''
\end{proof}


\begin{thm}
    \label{thm:8.13}
    \myKeywordblue{Trigonometric series}
    From now on we shall deal only with the trigonometric system. 
    We shall consider functions $f$ that have period $2\pi$ and that are Riemann-integrable on $[-\pi, \pi]$ (and hence on every bounded interval). 
    The Fourier series of $f$ is then the series (\ref{eq:8.63}) whose coefficients en are given by the integrals (\ref{eq:8.62}), and
    \begin{equation}
        \label{eq:8.75}
        s_N(x)=s_N(f;x)=\sum_{-N}^{N}c_n e^{inx}
    \end{equation}
    is the $N$th partial sum of the Fourier series off The inequality (\ref{eq:8.72}) now takes the form
    \begin{equation}
        \label{eq:8.76}
        \frac{1}{2\pi} \int_{-\pi}^{\pi} \left| s_N(x) \right|^2 \d x = 
        \sum_{-N}^{N} \left| c_n \right|^2 \leq
        \frac{1}{2\pi} \int_{-\pi}^{\pi} \left| f(x) \right|^2 \d x.
    \end{equation}
    In order to obtain an expression for $s_N$ that is more manageable than (\ref{eq:8.75})
    we introduce the \myKeywordblue{Dirichlet kernel}
    \begin{equation}
        \label{eq:8.77}
        D_N(x) = \sum_{n=-N}^{N} e^{inx} = \frac{\sin(N+\frac{1}{2})x}{\sin(x/2)} .
    \end{equation}
    The first of these equalities is the definition of $D_N(x)$. 
    The second follows if both sides of the identity
    \begin{equation*}
        (e^{ix}-1)D_N(x) = e^{i(N+1)x} - e^{-iNx}
    \end{equation*}
    are multiplied by $e^{-ix/2}$.

    By (\ref{eq:8.62}) and (\ref{eq:8.75}), we have 
    \begin{align*}
        s_N(f;x) 
        &= \sum_{-N}^{N} \frac{1}{2\pi} \int_{-\pi}^{\pi} f(t) e^{-int} \d t e^{inx} \\
        &= \frac{1}{2\pi} \int_{-\pi}^{\pi} f(t) \sum_{-N}^{N} e^{in(x-t)} \d t ,
    \end{align*}
    so that 
    \begin{equation}
        \label{eq:8.78}
        s_N(f;x)=\frac{1}{2\pi}\int_{-\pi}^{\pi}f(t)D_N(x-t)\d t 
        =\frac{1}{2\pi} \int_{-\pi}^{\pi} f(x-t) D_N(t) \d t.
    \end{equation}
    The periodicity of all functions involved shows that it is immaterial over which interval we integrate, as long as its length is $2\pi$. 
    This shows that the two integrals in (\ref{eq:8.78}) are equal. 
    
    We shall prove just one theorem about the pointwise convergence of Fourier series.
\end{thm}


\begin{thm}
    \label{thm:8.14}
    If, for some $x$, there are constants $\delta > 0$ and $M < \infty $ such that 
    \begin{equation}
        \label{eq:8.79}
        \left| f(x+t)-f(x) \right| \leq M \left| t \right| 
    \end{equation}
    for all $t \in (-\delta,\delta)$, then 
    \begin{equation}
        \label{eq:8.80}
        \lim_{N \to \infty} s_N(f;x) = f(x).
    \end{equation}
\end{thm}

\begin{proof}
    Define 
    \begin{equation}
        \label{eq:8.81}
        g(t) = \frac{f(x-t)-f(x)}{\sin(t/2)}
    \end{equation}
    for $0<|t|\leq \pi$, and put $g(0)=0$.
    By the definition (\ref{eq:8.77}),
    \begin{equation*}
        \frac{1}{2\pi} \int_{-\pi}^{\pi} D_N(x) \d x = 1.
    \end{equation*}
    Hence (\ref{eq:8.78}) shows that 
    \begin{align*}
        s_N(f;x)-f(x)
        &= \frac{1}{2\pi} \int_{-\pi}^{\pi} g(t) \sin \left( N+\frac{1}{2} \right) t \d t \\
        &= \frac{1}{2\pi} \int_{-\pi}^{\pi} \left[ g(t)\cos \frac{t}{2} \right] \sin N t \d t \\
        &+ \frac{1}{2\pi} \int_{-\pi}^{\pi} \left[ g(t)\sin \frac{t}{2} \right] \cos N t \d t .
    \end{align*}
    By (\ref{eq:8.79}) and (\ref{eq:8.81}), $g(t) \cos (t/2)$ and $g(t) \sin (t/2)$ are bounded. 
    The last two integrals thus tend to $0$ as $N > \infty$, by (\ref{eq:8.74}). 
    This proves (\ref{eq:8.80}).
\end{proof}


\begin{myCorollary*}
    If $f(x) = 0$ for all $x$ in some segment $J$, 
    then $\lim s_N(f; x) = 0$ for every $x \in J$.
\end{myCorollary*}

Here is another formulation of this corollary:
If $f (t) = g(t)$ for all $t$ in some neighborhood of $x$, then
\begin{equation*}
    s_N(f; x) - s_N(g; x) = s_N(f - g ; x) \rightarrow 0 
    \text{ as }
    N \rightarrow \infty .
\end{equation*}

This is usually called the \myKeywordblue{localization theorem}. 
It shows that the behavior of the sequence $\{s_N(f; x)\}$, as far as convergence is concerned, depends only on the values of $f$ in some (arbitrarily small) neighborhood of $x$. 
Two Fourier series may thus have the same behavior in one interval, but may behave in entirely different ways in some other interval. 
We have here a very striking contrast between Fourier series and power series (Theorem \ref{thm:8.5}).

We conclude with two other approximation theorems.


\begin{thm}
    \label{thm:8.15}
    If $f$ is continuous (with period $2\pi$) and if $\varepsilon > 0$, then there is a trigonometric polynomial $P$ such that 
    \begin{equation*}
        |P(x) - f(x) | < e
    \end{equation*}
    for all real $x$.
\end{thm}

\begin{proof}
    If we identify $x$ and $x + 2\pi$, we may regard the $2n$-periodic functions on $\R^1$ as functions on the unit circle $T$, by means of the mapping $x \rightarrow e^{ix}$. 
    The trigonometric polynomials, i.e., the functions of the form (\ref{eq:8.60}), form a self-adjoint algebra $\mathscr{A}$, which separates points on $T$, and which vanishes at no point of $T$. 
    Since $T$ is compact, Theorem \ref{thm:7.33} tells us that $\mathscr{A}$ is dense in $\mathscr{C}(T)$. 
    This is exactly what the theorem asserts.
\end{proof}

A more precise form of this theorem appears in Exercise \ref{ex:8.15}.


\begin{thm}
    \label{thm:8.16}
    \myKeywordblue{Parseval's theorem} 
    Suppose $f$ and $g$ are Riemann-integrable functions with period $2\pi$, and
    \begin{equation}
        \label{eq:8.82}
        f(x) \sim \sum_{-\infty}^{\infty} c_n e^{inx}, 
        \quad 
        g(x) \sim \sum_{-\infty}^{\infty} \gamma_n e^{inx}, 
    \end{equation}
    Then 
    % \begin{equation}
    %     \label{eq:8.83}
    %     \lim_{N \to \infty} \frac{1}{2\pi} \int_{-\pi}^{\pi} \left| f(x)-s_N(f;x) \right|^2 \d x = 0,
    % \end{equation}
    \begin{align}
        \lim_{N \to \infty} \frac{1}{2\pi} \int_{-\pi}^{\pi} \left| f(x)-s_N(f;x) \right|^2 \d x &= 0, \label{eq:8.83} \\
        \frac{1}{2\pi} \int_{-\pi}^{\pi} f(x)\overline{g(x)} \d x &= \sum_{-\infty}^{\infty} c_n \bar{\gamma}_n , \label{eq:8.84} \\
        \frac{1}{2\pi} \int_{-\pi}^{\pi} \left| f(x) \right|^2 \d x &= \sum_{-\infty}^{\infty} \left| c_n \right|^2 . \label{eq:8.85}
    \end{align}
\end{thm}


\begin{proof}
    % todo
    \begin{equation}
        \label{eq:8.86}
        \left\| h \right\|_2 = 
        \left\{ \frac{1}{2\pi} \int_{-\pi}^{\pi} \left| h(x) \right|^2 \d x 
        \right\}^{1/2} .
    \end{equation}

    \begin{equation}
        \label{eq:8.87}
        \left\| f-h \right\|_2 < \varepsilon .
    \end{equation}

    \begin{equation}
        \label{eq:8.88}
        \left\| h-s_N(h) \right\|_2 \leq
        \left\| h-P \right\|_2 < \varepsilon 
    \end{equation}

    \begin{equation}
        \label{eq:8.89}
        \left\| s_N(h)-s_N(f) \right\|_2 =
        \left\| s_N(h-f) \right\|_2 \leq
        \left\| h-f \right\|_2 < \varepsilon .
    \end{equation}

    \begin{equation}
        \label{eq:8.90}
        \left\| f-s_N(f) \right\|_2 < 3\varepsilon 
        \quad 
        (N \geq N_0) .
    \end{equation}

    \begin{equation}
        \label{eq:8.91}
        \frac{1}{2\pi} \int_{-\pi}^{\pi} s_N(f)\bar{g} \d x =
        \sum_{-N}^{N} c_n \frac{1}{2\pi} \int_{-\pi}^{\pi} e^{inx} \overline{g(x)} \d x =
        \sum_{-N}^{N} c_n \bar{\gamma}_n ,
    \end{equation}

    \begin{equation}
        \label{eq:8.92}
        \left| \int f \bar{g} - \int s_N(f)\bar{g} \right| \leq 
        \int \left| f - \int s_N(f) \right| \left| g \right| \leq
        \left\{ \int \left| f-s_N \right|^2 \int \left| g \right|^2 \right\}^{1/2} ,
    \end{equation}
\end{proof}