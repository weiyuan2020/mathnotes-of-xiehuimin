% chap08exercise
\section{Exercises}

\begin{myExercise}
    \label{ex:8.1}
    Define 
    \begin{equation*}
        f(x) = \left\{ 
            \begin{array}{ll}
                e^{-1/x^2} & (x \neq 0) , \\
                0 & (x = 0) . \\
            \end{array}
         \right.
    \end{equation*}
    Prove that $f$ has derivatives of all orders at $x=0$,
    and that $f^{(n)}(0)=0$ for $n=1,2,3,...$.
\end{myExercise}


\begin{myExercise}
    \label{ex:8.2}
    Let $a_{ij}$ be the number in the $i$th row and $j$th column of the array 
    \begin{equation*}
        \begin{array}{ccccc}
            -1 & 0 & 0 & 0 & \dots \\
            \frac{1}{2} & -1 & 0 & 0 & \dots \\
            \frac{1}{4} & \frac{1}{2} & -1 & 0 & \dots \\
            \frac{1}{8} & \frac{1}{4} & \frac{1}{2} & -1 & \dots \\
            \dots & \dots & \dots & \dots & \dots \\
        \end{array}
    \end{equation*}
    so that 
    \begin{equation*}
        a_{ij} = \left\{ 
            \begin{array}{ll}
                0 & (i<j), \\
                1 & (i=j), \\
                2^{j-i} & (i>j). \\
            \end{array}
        \right.
    \end{equation*}
    Prove that 
    \begin{equation*}
        \sum_i \sum_j a_{ij} = -2, \quad 
        \sum_j \sum_i a_{ij} = 0.
    \end{equation*}
\end{myExercise}


\begin{myExercise}
    \label{ex:8.3}
    Prove that 
    \begin{equation*}
        \sum_i \sum_j a_{ij} = 
        \sum_j \sum_i a_{ij} 
    \end{equation*}
    if $a_{ij} \geq 0$ for all $i$ and $j$ (the case $+\infty=+\infty$ may occur).
\end{myExercise}


\begin{myExercise}
    \label{ex:8.4}
    Prove the following limit relations:
    \begin{enumerate}[(a)]
        \item $\lim_{x \to 0} \frac{b^x-1}{x}=\log b$  $(b>0)$.
        \item $\lim_{x \to 0} \frac{\log(1+x)}{x}=1$.
        \item $\lim_{x \to 0} (1+x)^{1/x}=e$.
        \item $\lim_{n \to \infty} \left( 1+\frac{x}{n} \right)^n=e^x$.
    \end{enumerate}
\end{myExercise}


\begin{myExercise}
    \label{ex:8.5}
    Find the following limits 
    \begin{enumerate}[(a)]
        \item $\lim_{x \to 0} \frac{e-(1+x)^{1/x}}{x}$.
        \item $\lim_{n \to \infty} \frac{n}{\log n}\left[ n^{1/n}-1 \right]$.
        \item $\lim_{x \to 0} \frac{\tan x-x}{x(1-\cos x)}$.
        \item $\lim_{x \to 0} \frac{x-\sin x}{\tan x-x}$.
    \end{enumerate}
\end{myExercise}


\begin{myExercise}
    \label{ex:8.6}
    Suppose  $f(x) f(y) = f(x + y)$ for all real $x$ and $y$.
    \begin{asparaenum}[(a)]
        \item Assuming that $f$ is differentiable and not zero, prove that 
        \begin{equation*}
            f(x) = e^{cx}
        \end{equation*}
        where $c$ is a constant.
        \item Prove the same thing, assuming only that $f$ is continuous.
    \end{asparaenum}
\end{myExercise}


\begin{myExercise}
    \label{ex:8.7}
    If $0<x<\frac{\pi}{2}$, prove that 
    \begin{equation*}
        \frac{2}{\pi}<\frac{\sin x}{x}<1.
    \end{equation*}
\end{myExercise}


\begin{myExercise}
    \label{ex:8.8}
    For $n=0,1,2,\dots$, and $x$ real, prove that 
    \begin{equation*}
        \left| \sin nx \right| \leq 
        n \left| \sin x \right| .
    \end{equation*}
    Note that this inequality may be false for other values of $n$.
    For instance,
    \begin{equation*}
        \left| \sin \frac{1}{2}\pi \right| >
        \frac{1}{2}\left| \sin \pi \right| .
    \end{equation*}
\end{myExercise}


\begin{myExercise}
    \label{ex:8.9}
    \begin{asparaenum}[(a)]
        \item Put $s_N=1+\left( \frac{1}{2} \right)+\cdots+\left( 1/N \right)$.
        Prove that 
        \begin{equation*}
            \lim_{N \to \infty} \left( s_N - \log N \right)
        \end{equation*}
        exists.
        (The limit, often denoted by $\gamma$, is called Euler's constant. 
        Its numerical value is $0.5772\dots$. 
        It is not known whether $\gamma$ is rational or not.)
        \item Roughly how large must $m$ be so that $N=10^m$ satisfies $s_N>100$?
    \end{asparaenum}
\end{myExercise}


\begin{myExercise}
    \label{ex:8.10}
    Prove that $\sum 1/p$ diverges; 
    the sum extends over all primes.

    (This shows that the primes form a fairly substantial subset of the positive integers.)

    \emph{Hint:} Given $N$, let $p_1,\dots,p_k$ be those primes that divide at least one integer $\leq N$.
    Then 
    \begin{align*}
        \sum_{n=1}^{N}\frac{1}{n} 
        &\leq \prod_{j=1}^{k} \left( 1+\frac{1}{p_j}+\frac{1}{p_j^2}+\cdots \right) \\
        &= \prod_{j=1}^{k} \left( 1-\frac{1}{p_j} \right)^{-1} \\
        &\leq \exp \sum_{j=1}^{k} \frac{2}{p_j} .
    \end{align*}
    The last inequality holds because
    \begin{equation*}
        (1-x)^{-1} \leq e^{2x}
    \end{equation*}
    if $0 \leq x \leq \frac{1}{2}$.

    (There are many proofs of this result. See, for instance, the article by I. Niven in \emph{Amer. Math. Monthly}, vol. 78, 1971, pp. 272-273, and the one by R. Bellman in \emph{Amer. Math. Monthly}, vol. 50, 1943, pp. 318-319.)
\end{myExercise}


\begin{myExercise}
    \label{ex:8.11}
    Suppose $f \in \mathscr{R}$ on $[0,A]$ for all $A<\infty$, and $f(x) \rightarrow 1$ as $x \rightarrow +\infty$.
    Prove that 
    \begin{equation*}
        \lim_{t \to 0} t \int_{0}^{\infty} e^{-tx} f(x) \d x = 1
        \quad 
        (t>0).
    \end{equation*}
\end{myExercise}


\begin{myExercise}
    \label{ex:8.12}
    Suppose $0 < \delta < \pi$, $f(x) = 1$ if $|x | \leq \delta$, 
    $f(x) = 0$ if $\delta < |x | \leq \pi$, 
    and $f(x + 2\pi) = f(x)$ for all $x$.
    \begin{asparaenum}[(a)]
        \item Compute the Fourier coefficients of $f$.
        \item Conclude that 
        \begin{equation*}
            \sum_{n=1}^{\infty} \frac{\sin(n\delta)}{n} = \frac{\pi-\delta}{2}
            \quad 
            (0 < \delta < \pi) .
        \end{equation*}
        \item Deduce from Parseval's theorem that 
        \begin{equation*}
            \sum_{n=1}^{\infty} \frac{\sin^2(n\delta)}{n^2\delta} = \frac{\pi-\delta}{2} .
        \end{equation*}
        \item Let $\delta \rightarrow 0$ and prove that
        \begin{equation*}
            \int_{0}^{\infty} \left( \frac{\sin x}{x} \right)^2 \d x = \frac{\pi}{2}.
        \end{equation*}
        \item Put $\delta = \pi/2$ in (c). What do you get?
    \end{asparaenum}
\end{myExercise}


\begin{myExercise}
    \label{ex:8.13}
    Put $f(x) = x$ if $0 \leq x < 2\pi$, and apply Parseval's theorem to conclude that
    \begin{equation*}
        \sum_{n=1}^{\infty} \frac{1}{n^2} = \frac{\pi^2}{6} .
    \end{equation*}
\end{myExercise}


\begin{myExercise}
    \label{ex:8.14}
    If $f(x) = (\pi - |x|)^2$ on $[-\pi, \pi]$, prove that
    \begin{equation*}
        f(x) = \frac{\pi^2}{3} + \sum_{n=1}^{\infty}\frac{4}{n^2}\cos nx
    \end{equation*}
    and deduce that 
    \begin{equation*}
        \sum_{n=1}^{\infty} \frac{1}{n^2} = \frac{\pi^2}{6},
        \quad
        \sum_{n=1}^{\infty} \frac{1}{n^4} = \frac{\pi^4}{90}.
    \end{equation*}
    (A recent article by E. L. Stark contains many references to series of the form $\sum n^{-s}$,
    where $s$ is a positive integer. See \emph{Math. Mag.}, vol. 47, 1974, pp. 197-202.)
\end{myExercise}


\begin{myExercise}
    \label{ex:8.15}
    With $D_n$ as defined in (\ref{eq:8.77}), put
    \begin{equation*}
        K_N(x) = \frac{1}{N+1} \sum_{n=0}^{N} D_n(x).
    \end{equation*}
    Prove that 
    \begin{equation*}
        K_N(x) = \frac{1}{N+1} \cdot \frac{1-\cos(N+1)x}{1-\cos x}
    \end{equation*}
    and that 
    \begin{enumerate}[(a)]
        \item $K_N \geq 0$,
        \item $\frac{1}{2\pi} \int_{-\pi}^{\pi} K_N(x) \d x = 1$,
        \item $K_N(x) \leq \frac{1}{N+1}\cdot \frac{2}{1-\cos \delta}$ if $0<\delta\leq|x|\leq\pi$.
    \end{enumerate}
    If $s_N = s_N(f; x)$ is the $N$th partial sum of the Fourier series of $f$, consider the arithmetic means
    \begin{equation*}
        \sigma_N = \frac{s_0+s_1+\cdots+s_N}{N+1}.
    \end{equation*}
    Prove that 
    \begin{equation*}
        \sigma_N(f;x) = \frac{1}{2\pi} \int_{-\pi}^{\pi} f(x-t)K_N(t) \d t,
    \end{equation*}
    and hence prove Fej\'{e}r's theorem:

    \emph{If $f$ is continuous, with period $2\pi$, then $\sigma_N(f;x)\rightarrow f(x)$ uniformly on $[-\pi,\pi]$.}

    \emph{Hint:} Use properties (a), (b), (c) to proceed as in Theorem \ref{thm:7.26}.
\end{myExercise}


\begin{myExercise}
    \label{ex:8.16}
    Prove a pointwise version of Fej\'{e}r's theorem:

    \emph{If $f \in \mathscr{R}$ and $f(x+)$, $f(x-)$ exist for some $x$, then}
    \begin{equation*}
        \lim_{N \to \infty} \sigma_N (f;x) = \frac{1}{2}\left[ f(x+)-f(x-) \right].
    \end{equation*}
\end{myExercise}


\begin{myExercise}
    \label{ex:8.17}
    Assume $f$ is bounded and monotonic on $[-\pi, \pi)$, 
    with Fourier coefficients $c_n$, as given by (\ref{eq:8.62}).
    \begin{asparaenum}[(a)]
        \item Use Exercise 17 of Chap. 6 to prove that {ncn} is a bounded sequence.
        \item Combine (a) with Exercise \ref{ex:3.16} and with Exercise \ref{ex:3.14}(e), to conclude that 
        \begin{equation*}
            \lim_{N \to \infty} s_N(f;x) = \frac{1}{2}\left[ f(x+)-f(x-) \right].
        \end{equation*}
        for every $x$.
        \item Assume only that $f \in \mathscr{R}$ on $[-\pi,\pi]$ and that $f$ is monotonic in some segment
        $(\alpha, \beta)c [-\pi, \pi]$. 
        Prove that the conclusion of (b) holds for every $x \in (\alpha, \beta)$.
    \end{asparaenum}
    (This is an application of the localization theorem.)
\end{myExercise}


\begin{myExercise}
    \label{ex:8.18}
    Define
    \begin{align*}
        f(x) &= x^3 - \sin^2 x \tan x \\
        g(x) &= 2x^2 - \sin^2 x - x\tan x.
    \end{align*}
    Find out, for each of these two functions, whether it is positive or negative for all $x \in (0, \pi/2)$, or whether it changes sign. 
    Prove your answer.
\end{myExercise}


\begin{myExercise}
    \label{ex:8.19}
    Suppose $f$ is a continuous function on $\R^1$,
    $f(x + 2\pi) = f(x)$, and $\alpha/\pi$ is irrational.
    Prove that
    \begin{equation*}
        \lim_{N \to \infty} \frac{1}{N} \sum_{n=1}^{N} f(x+n\alpha) = \frac{1}{2\pi} \int_{-\pi}^{\pi} f(t) \d t
    \end{equation*}
    for every $x$.

    \emph{Hint:} Do it first for $f(x) = e^{ikx}$.
\end{myExercise}


\begin{myExercise}
    \label{ex:8.20}
    The following simple computation yields a good approximation to Stirling's formula.

    For $m=1,2,3,\dots$, define 
    \begin{equation*}
        f(x) = (m+1-x) \log m + (x-m) \log (m+1)
    \end{equation*}
    if $m \leq x \leq m+1$, and define 
    \begin{equation*}
        g(x) = \frac{x}{m} - 1 + \log m
    \end{equation*}
    if $m-\frac{1}{2} \leq x \leq m+\frac{1}{2}$.
    Draw the graphs of $f$ and $g$.
    Note that $f(x) \leq \log{x} \leq g(x)$ \\
    if $x \geq 1$ and that 
    \begin{equation*}
        \int_{1}^{n} f(x) \d x = \log (n!) - \frac{1}{2} \log n > -\frac{1}{8} + \int_{1}^{n} g(x) \d x.
    \end{equation*}
    Integrate $\log x$ over $[1,n]$.
    Conclude that 
    \begin{equation*}
        \frac{7}{8} < \log(n!) - (n+\frac{1}{2}) \log n + n <1
    \end{equation*}
    for $n=2,3,4,\dots$. (\emph{Note:} $\log \sqrt{2\pi}\sim 0.918\dots$.) Thus 
    \begin{equation*}
        e^{7/8}<\frac{n!}{(n/e)^n\sqrt{n}}<e.
    \end{equation*}
\end{myExercise}