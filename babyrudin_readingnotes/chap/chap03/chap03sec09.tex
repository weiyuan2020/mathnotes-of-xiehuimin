% chap03sec09
\section{The root and ratio tests}

\begin{thm}(Root test)
    \label{thm:3.33}
    Given $\sum a_n$ , put $\alpha = \limsup_{n \rightarrow \infty} \sqrt[n]{|a_n|} $ .

    Then \\
    (a) if $\alpha < 1$, $\sum a_n$ converges; \\
    (a) if $\alpha > 1$, $\sum a_n$ diverges; \\
    (a) if $\alpha = 1$, The test gives no information.
\end{thm}


\begin{thm}(Ratio test)
    \label{thm:3.34 ratio test}
    The series $\sum a_n$ \\
    (a) converges if $\limsup_{n \to \infty} \left|\frac{a_{n+1}}{a_n}\right| < 1$. \\
    (b) diverges if $\left|\frac{a_{n+1}}{a_n}\right| \geq 1$ for all $n \geq n_0$, where $n_0$ is some fixed integer.
\end{thm}

\mybox{3.33 为根值审敛法,  3.34 为比值审敛法}

Note: The knowledge that $\lim a_{n+1}/a_n = 1$ implies nothing about the convergence of $\sum a_n$.
The series $\sum 1/n$ and $\sum 1/n^2$ demonstrate this.

\begin{newexample}
    \label{newexample:3.35}
    (a) Consider the series
    \begin{equation*}
        \frac{1}{2} 
        + \frac{1}{3}
        + \frac{1}{2^2}
        + \frac{1}{3^2}
        + \frac{1}{2^3}
        + \frac{1}{3^3}
        + \frac{1}{2^4}
        + \frac{1}{3^4}
        + \cdots,
    \end{equation*}
    for which
    \begin{align*}
        \liminf_{n \to \infty} \frac{a_{n+1}}{a_n} 
        &= \lim_{n \to \infty} \left(\frac{2}{3}\right)^n = 0,\\
        \liminf_{n \to \infty} \sqrt[n]{a_n} 
        &= \lim_{n \to \infty} \sqrt[2n]{\frac{1}{3^n}} = \frac{1}{\sqrt{3}},\\
        \limsup_{n \to \infty} \frac{a_{n+1}}{a_n} 
        &= \lim_{n \to \infty} \frac{1}{2}\left(\frac{3}{3}\right)^n = +\infty,\\
        \limsup_{n \to \infty} \sqrt[n]{a_n} 
        &= \lim_{n \to \infty} \sqrt[2n]{\frac{1}{2^n}} = \frac{1}{\sqrt{2}}.
    \end{align*}
    The root test indicates convergence: the ratio test does not apply.

    (b) The same is true for the series
    \begin{equation*}
        \frac{1}{2} + 1 
        + \frac{1}{8}
        + \frac{1}{4}
        + \frac{1}{32}
        + \frac{1}{16}
        + \frac{1}{128}
        + \frac{1}{64}
        +\cdots,
    \end{equation*}
    where
    \begin{align*}
        \liminf_{n \to \infty} \frac{a_{n+1}}{a_n} &= \frac{1}{8},\\
        \limsup_{n \to \infty} \frac{a_{n+1}}{a_n} &= 2,\\
    \end{align*}
    but 
    \begin{equation*}
        \lim_{n \to \infty} \sqrt[n]{a_n} = \frac{1}{2}.
    \end{equation*}
\end{newexample}

\begin{myremark}
    \label{myremark:3.36}
    The ratio test is frequently easier to apply than the root test, since it is usually easier to compute ratios than nth roots. 
    However, the root test has wider scope. More precisely: 
    Whenever the ratio test shows convergence, the root test does too;
    whenever the root test is inconclusive, the ratio test is too. 
    This is a consequence of Theorem 3.37, and is illustrated by the above examples.

    Neither of the two tests is subtle with regard to divergence. Both deduce divergence from the fact that $a_n$ does not tend to zero as $n \rightarrow \infty$.
\end{myremark}

\begin{thm}
    \label{thm:3.37}
    For any sequence $\sequence{c_n}$ of positive numbers,
    \begin{align*}
        \liminf_{n \to \infty} \frac{c_{n+1}}{c_n} &\leq 
        \liminf_{n \to \infty} \sqrt[n]{c_n}, \\
        \limsup_{n \to \infty} \sqrt[n]{c_n} &\leq
        \limsup_{n \to \infty} \frac{c_{n+1}}{c_n}. \\
    \end{align*} 
\end{thm}

\begin{proof}
    We shall prove the second inequality;
    the proof of the first is quite similar.
    Put
    \begin{equation*}
        \alpha = \limsup_{n \to \infty} \frac{c_{n+1}}{c_n}.
    \end{equation*}
    If $\alpha = +\infty$ , there is nothing to prove.
    If $\alpha$ is finite , choose $\beta > \alpha$ .
    There is an integer $N$ such that
    \begin{equation*}
        \frac{c_{n+1}}{c_n} \leq \beta
    \end{equation*}
    for $n \geq N$. In particular, for any $p > 0$ ,
    \begin{equation*}
        c_{N+k+1} \leq \beta c_{N+k} \quad
        (k=0,1,\dots,p-1).
    \end{equation*} 
    Multiplying these inequalities, we obtain
    \begin{equation*}
        c_{N+p} \leq \beta^p c_N,
    \end{equation*}
    or
    \begin{equation*}
        c_n \leq c_N \beta^{-N}\cdot \beta^n \quad (n \geq N).
    \end{equation*}
    Hence
    \begin{equation*}
        \sqrt[n]{c_n} \leq
        \sqrt[n]{c_N \beta^{-N}}\cdot \beta,
    \end{equation*}
    so that
    \begin{equation}
        \label{eq:3.18}
        \limsup_{n \to \infty} \sqrt[n]{c_n} \leq \beta,
    \end{equation}
    by Theorem \ref{thm:3.20}(b).
    Since (\ref{eq:3.18}) is true for every $\beta > \alpha$, we have
    \begin{equation*}
        \limsup_{n \to \infty} 
        \sqrt[n]{c_n} \leq \alpha.
    \end{equation*}
\end{proof}