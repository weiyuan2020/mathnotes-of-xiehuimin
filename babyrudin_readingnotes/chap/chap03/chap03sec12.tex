% chap03sec12
\section{Absolute convergence}
\mybox{绝对收敛}
The series $\sum a_n$ i said to \emph{converge absolutely} if the series $\sum |a_n|$ converges.

\begin{thm}
    \label{thm:3.45}
    if $\sum a_n$ converges absolutely,
    then $\sum a_n$ converges.    
\end{thm}

\begin{proof}
    The assertion follow from the inequality
    \begin{equation*}
        \left|\sum_{k=n}^{m}a_k\right|
        \leq \sum_{k=n}^{m} |a_k|,
    \end{equation*}
    plus the Cauchy criterion.
\end{proof}

\begin{myremark}
    For series of positive terms, absolute convergence is the same as convergence.

    If $\sum a_n$ converges, but $\sum |a_n|$ diverges, we say that $\sum a_n$ converges \emph{nonabsolutely}.
    For instance, the series
    \begin{equation*}
        \sum \frac{(-1)^n}{n}
    \end{equation*}
    converges nonabsolutely (Theorem \ref{thm:3.43}).

    The comparison test, as well as the root and ratio tests, is really a test for absolute convergence, 
    and therefore cannot give any information about nonabsolutely convergent series.
    Summation by parts can sometimes be used to handle the latter.
    In particular, power series converge absolutely in the interior of the circle of convergence.

    We shall see that we may operate with absolutely convergent series very much as with finite sums.
    We may multiply thm term by term and we may change the order in which the additions are carried out, without affecting the sum of series.
    But for nonabsolutely convergent series this is no longer true, and more care has to be taken when dealing with them.
\end{myremark}
