% chap03exercise
\section{Exercises}

\begin{myexercise}
    \label{ex:3.1}
    Prove that convergence of $\{s_n\}$ implies convergence of $\{|s_n|\}$. 
    Is the converse true?
\end{myexercise}

\begin{myexercise}
    \label{ex:3.2}
    Calculate $\lim_{n \to \infty} (\sqrt{n^2+n}-n)$ 
\end{myexercise}

\mySolve
\begin{align*}
    \lim_{n \to \infty} n\left( \sqrt{1+\frac{1}{n}}-1 \right) 
    &= \lim_{n \to \infty} \frac{ \sqrt{1+\frac{1}{n}}-1 }{\frac{1}{n}} \\
    &= \lim_{n \to \infty} \frac{1+\frac{1}{2}\frac{1}{n}-1}{\frac{1}{n}} \\
    &= \frac{1}{2} .
\end{align*}

\begin{myexercise}
    \label{ex:3.3}
    If $s_1 = \sqrt{2}$ , and 
    \begin{equation*}
        s_{n+1} = \sqrt{2+\sqrt{s_n}} 
        \quad 
        (n = 1,2,3,\dots),
    \end{equation*}
    prove that $\{s_n\}$ converges, and that $s_n < 2$ for $n=1,2,3,...$ .
\end{myexercise}

\begin{myexercise}
    \label{ex:3.4}
    Find the upper and lower limits of the sequences $\{s_n\}$ defined by 
    \begin{equation*}
        s_1 = 0; \quad 
        s_{2m} = \frac{s_{2m-1}}{2}; \quad 
        s_{2m+1} = \frac{1}{2} + s_{2m} .
    \end{equation*}
\end{myexercise}


\begin{myexercise}
    \label{ex:3.5}
    For any two real sequences $\{a_n\}$, $\{b_n\}$, prove that
    \begin{equation*}
        \limsup_{n \to \infty} (a_n + b_n) \leq
        \limsup_{n \to \infty} a_n +
        \limsup_{n \to \infty} b_n ,
    \end{equation*}
    provided the sum on the right is not of the form $\infty - \infty$.
\end{myexercise}

\begin{myexercise}
    \label{ex:3.6}
    Investigate the behavior (convergence or divergence) of $\sum a_n$ if
    \begin{enumerate}[(a)]
        \item $a_n = \sqrt{n+1} - \sqrt{n}$ ;
        \item $a_n = \frac{\sqrt{n+1} - \sqrt{n}}{n}$ ;
        \item $a_n = (\sqrt[n]{n} - 1)^n$ ;
        \item $a_n = \frac{1}{1+z^n}$ , for complex values of $z$.
    \end{enumerate}
\end{myexercise}

\begin{myexercise}
    \label{ex:3.7}
    Prove that the convergence of $\sum a_n$ implies the convergence of
    \begin{equation*}
        \sum \frac{\sqrt{a_n}}{n},
    \end{equation*}
    if $a_n \geq 0$ .
\end{myexercise}


\begin{myexercise}
    \label{ex:3.8}
    If $\sum a_n$ converges, 
    and if $\{b_n\}$ is monotonic and bounded, 
    prove that $\sum a_n b_n$ converges.
\end{myexercise}


\begin{myexercise}
    \label{ex:3.9}
    Find the radius of convergence of each of the following power series:
    \begin{enumerate}[(a)]
        \item $\sum n^3z^n$ ,
        \item $\sum \frac{2^n}{n!}z^n$ ,
        \item $\sum \frac{2^n}{n^2}z^n$ ,
        \item $\sum \frac{n^3}{3^n}z^n$ ,
    \end{enumerate}
\end{myexercise}

\mySolve
\begin{enumerate}
    \item $\lim_{n \to \infty} \frac{1}{(1+1/n)^3} = 1$, $R = 1$.
    \item $\lim_{n \to \infty} \frac{n + 1}{2} = \infty$, $R = \infty$.
    \item $\lim_{n \to \infty} \frac{\left( 1+\frac{1}{n} \right)^2}{2} = \frac{1}{2}$, $R = \frac{1}{2}$.
    \item $\lim_{n \to \infty} \frac{3}{\left( 1+\frac{1}{n} \right)^3} = 3$, $R = 3$.
\end{enumerate}

\begin{myexercise}
    \label{ex:3.10}
    Suppose that the coefficients of the power series $\sum a_n z^n$ are integers, 
    infinitely many of which are distinct from zero. 
    Prove that the radius of convergence is at most 1.
\end{myexercise}

\begin{myexercise}
    \label{ex:3.11}
    Suppose $a_n > 0$, $s_n = a_1 + ... + a_n$ and $\sum a_n$ diverges.
    \begin{asparaenum}[(a)]
        \item Prove that $\sum \frac{a_n}{1+a_n}$ diverges.
        \item Prove that 
        \begin{equation*}
            \frac{a_{N+1}}{s_{N+1}} + \cdots +
            \frac{a_{N+k}}{s_{N+k}} \geq 
            1- \frac{s_N}{s_{N+k}}
        \end{equation*}
        and deduce that $\sum \frac{a_n}{s_n}$ diverges.
        \item Prove that 
        \begin{equation*}
            \frac{a_n}{s_n^2} \leq \frac{1}{s_{n-1}} - \frac{1}{s_n}
        \end{equation*}
        and deduce that $\sum \frac{a_n}{s_n^2}$ converges.
        \item What can be said about 
        \begin{equation*}
            \sum \frac{a_n}{1+n a_n} 
            \text{ and }
            \sum \frac{a_n}{1+n^2a_n}
        \end{equation*}
    \end{asparaenum}
\end{myexercise}

\begin{myexercise}
    \label{ex:3.12}
    Suppose $a_n > 0$ and $\sum a_n$ converges. Put
    \begin{equation*}
        r_n = \sum_{m=n}^{\infty} a_m .
    \end{equation*}
    \begin{asparaenum}[(a)]
        \item Prove that 
        \begin{equation*}
            \frac{a_m}{r_m} + \cdots +
            \frac{a_n}{r_n} > 
            1 - \frac{r_n}{r_m}
        \end{equation*}
        if $m<n$, and deduce that $\sum \frac{a_n}{r_n}$ diverges.
        \item Prove that 
        \begin{equation*}
            \frac{a_n}{\sqrt{r_n}} < 
            2\left( \sqrt{r_n} - \sqrt{r_{n+1}} \right)
        \end{equation*}
        and deduce that $\sum \frac{a_n}{\sqrt{r_n}}$ converges.
    \end{asparaenum}
\end{myexercise}


\begin{myexercise}
    \label{ex:3.13}
    Prove that the Cauchy product of two absolutely convergent series converges absolutely.
\end{myexercise}


\begin{myexercise}
    \label{ex:3.14}
    If $\{s_n\}$ is a complex sequence, define its arithmetic means $\sigma_n$ by 
    \begin{equation*}
        \sigma_n = \frac{s_0+s_1+\cdots+s_n}{n+1}
        \quad 
        (n=0,1,2,\dots).
    \end{equation*}
    \begin{asparaenum}[(a)]
        \item If $\lim s_n = s$, prove that $\lim \sigma_n = s$.
        \item Construct a sequence $\{s_n\}$ which does not converge, although $\lim \sigma_n= 0$.
        \item Can it happen that $s_n> 0$ for all $n$ and that $\limsup s_n = \infty$, although $\lim \sigma_n= 0$?
        \item Put $a_n = s_n - s_{n-1}$, for $n \geq 1$. 
        Show that
        \begin{equation*}
            s_n-\sigma_n = \frac{1}{n+1}\sum_{k=1}^{n}k a_k .
        \end{equation*}
        Assume that $\lim (n a_n)= 0$ and that $\{\sigma_n\}$ converges. 
        Prove that $\{s_n\}$ converges.
        [This gives a converse of (a), but under the additional assumption that $n a_n \rightarrow 0$.]
        \item Derive the last conclusion from a weaker hypothesis: 
        Assume $M < \infty$, $| n a_n | \leq M$ for all $n$, 
        and $\lim \sigma_n= \sigma$. 
        Prove that $\lim s_n = \sigma$, by completing the following outline:
        
        If $m < n$, then
        \begin{equation*}
            s_n - \sigma_n 
            = \frac{m+1}{n-m}(\sigma_n - \sigma_m)
            = \frac{  1}{n-m}\sum_{i=m+1}^{n}(s_n - s_i) .
        \end{equation*}
        For these $i$,
        \begin{equation*}
            \left| s_n - s_i \right| 
            \leq \frac{(n-i)M}{i+1} 
            \leq \frac{(n-m-1)M} {m+2} .
        \end{equation*}

        Fix $\varepsilon > 0$ and associate with each $n$ the integer $m$ that satisfies
        \begin{equation*}
            m \leq \frac{n - \varepsilon}{1 + \varepsilon}
            < m+1 .
        \end{equation*}
        Then $(m + 1)/(n - m) < 1/\varepsilon$ and $\left| s_n-s_i \right| < M\varepsilon$. Hence
        \begin{equation*}
            \limsup_{n \to \infty} \left| s_n - \sigma \right| \leq M \varepsilon .
        \end{equation*}
        Since $\varepsilon$ was arbitrary, $\lim s_n = \sigma$.
    \end{asparaenum}
\end{myexercise}

\begin{myexercise}
    \label{ex:3.15}
    Definition \ref{mydef:3.21} can be extended to the case
    in which the $a_n$ lie in some fixed $\R^k$.
    Absolute convergence is defined as convergence of $\sum \left| \mathbf{a_n} \right|$ , 
    Show that Theorems \ref{thm:3.22}, \ref{thm:3.23}, \ref{thm:3.25}(a), \ref{thm:3.33}, \ref{thm:3.34 ratio test}, \ref{thm:3.42}, \ref{thm:3.45}, \ref{thm:3.47}, and \ref{thm:3.55} are true in this more general setting. 
    (Only slight modifications are required in any of the proofs.)
\end{myexercise}

\begin{myexercise}
    \label{ex:3.16}
    Fix a positive number $\alpha$. 
    Choose $x_1 > \sqrt{\alpha}$, 
    and define $x_2, x_3, x_4, ...$ , 
    by the recursion formula
    \begin{equation*}
        x_{n+1} = \frac{1}{2}\left( x_n + \frac{\alpha}{x_n} \right).
    \end{equation*}
    \begin{asparaenum}[(a)]
        \item Prove that $\{x_n\}$ decreases monotonically and that $\lim x_n = \sqrt{\alpha}$.
        \item  Put $\varepsilon_n = x_n - \sqrt{\alpha}$, and show that
        \begin{equation*}
            \varepsilon_{n+1} 
            = \frac{\varepsilon_n^2}{2 x_n} 
            < \frac{\varepsilon_n^2}{2 \sqrt{\alpha}} 
        \end{equation*}
        so that, setting $\beta = 2 \sqrt{\alpha}$,
        \begin{equation*}
            \varepsilon_{n+1} 
            < \beta \left( \frac{\varepsilon_1}{\beta} \right)^{2^n}
            \quad 
            ( n = 1, 2, 3, ... ) .
        \end{equation*}
        \item  This is a good algorithm for computing square roots, 
        since the recursion formula is simple and the convergence is extremely rapid. 
        For example, if $\alpha = 3$ and $x_1 = 2$, 
        show that $\varepsilon_1/\beta < \frac{1}{10}$ and that therefore
        \begin{equation*}
            \varepsilon_5 < 4 \cdot 10^{-16}, 
            quad 
            \varepsilon_6 < 4 \cdot 10^{-32}
        \end{equation*}
    \end{asparaenum}
\end{myexercise}


\begin{myexercise}
    \label{ex:3.17}
    Fix $\alpha > 1$. Take $x_1 > \sqrt{\alpha}$, and define
    \begin{equation*}
        x_{n+1} = \frac{\alpha + x_n}{1 + x_n} = x_n + \frac{\alpha - x_n^2}{1 + x_n}.
    \end{equation*}
    \begin{enumerate}[(a)]
        \item Prove that $x_1 > x_3 > x_5 >  \cdots$,
        \item Prove that $x_2 < x_4 < x_6 <  \cdots$,
        \item Prove that lim  $x_n = \sqrt{\alpha}$.
        \item Compare the rapidity of convergence of this process with the one described in Exercise \ref{ex:3.16}.
    \end{enumerate}
\end{myexercise}


\begin{myexercise}
    \label{ex:3.18}
    Replace the recursion formula of Exercise \ref{ex:3.16} by 
    \begin{equation*}
        x_{n+1} = \frac{p-1}{p}x_n + \frac{\alpha}{p}x_n^{-p+1}
    \end{equation*}
    where $p$ is a fixed positive integer, and describe the behavior of the resulting sequences $\{x_n\}$ .
\end{myexercise}
\mybox{recursion 递归}

\begin{myexercise}
    \label{ex:3.19}
    Associate to each sequence $a = \{\alpha_n\}$, 
    in which $\alpha_n$ is 0 or 2,
    the real number 
    \begin{equation*}
        x(a) = \sum_{n=1}^{\infty} \frac{\alpha_n}{3^n} .
    \end{equation*}
    Prove that the set of all $x(a)$ is precisely the Cantor set described in Sec. \ref{mydef:2.44}.
\end{myexercise}


\begin{myexercise}
    \label{ex:3.20}
    Suppose $\{p_n\}$ is a Cauchy sequence in a metric space $X$,
    and some subsequence $\{p_{n_i}\}$ converges to a point $p \in X$ .
    Prove that the full sequence $\{p_n\}$ converges to $p$.
\end{myexercise}


\begin{myexercise}
    \label{ex:3.21}
    Prove the following analogue of Theorem \ref{thm:3.10}(b): 
    If $\{E_n\}$ is a sequence of closed nonempty and bounded sets in a \emph{complete} metric space $X$, 
    if $E_n \supset E_{n+1}$, and if $\diam$
    \begin{equation*}
        \lim_{n \to \infty} \diam E_n = 0 ,
    \end{equation*}
    then $\cap_1^{\infty} E_n$ consists of exactly one point.
\end{myexercise}


\begin{myexercise}
    \label{ex:3.22}
    Suppose $X$ is a nonempty complete metric space, 
    and $\{G_n\}$ is a sequence of dense open subsets of $X$. 
    Prove Baire's theorem, namely, that $\cap_1^{\infty} G_n$ is not
    empty. 
    (In fact, it is dense in $X$.) 
    
    \emph{Hint:} Find a shrinking sequence of neighborhoods $E_n$ such that 
    $\overline{E} \subset G_n$, and apply Exercise \ref{ex:3.21}.
\end{myexercise}


\begin{myexercise}
    \label{ex:3.23}
    Suppose $\{p_n\}$ and $\{q_n\}$ are Cauchy sequences in a metric space $X$. 
    Show that the sequence $\{d(p_n, q_n)\}$ converges. 
    
    \emph{Hint:} For any $m, n$,
    \begin{equation*}
        d(p_n, q_n) \leq d(p_n, p_m) + d(p_m, q_m) + d(q_m , q_n);
    \end{equation*}
    it follows that 
    \begin{equation*}
        \left| d(p_n, q_n) - d(p_m, q_m) \right| 
    \end{equation*}
    is small if $m$ and $n$ are large.
\end{myexercise}


\begin{myexercise}
    \label{ex:3.24}
    Let $X$ be a metric space.
    \begin{asparaenum}[(a)]
        \item Call two Cauchy sequences $\{p_n\}$, $\{q_n\}$ in $X$ \emph{equivalent} if
        \begin{equation*}
            \lim_{n \to \infty}  d(p_n, q_n) = 0.
        \end{equation*}
        Prove that this is an equivalence relation.
        \item Let $X^*$ be the set of all equivalence classes so obtained. If $P \in x^*$, $Q \in X^*$, $\{p_n\} \in P$, $\{q_n\} \in Q$, define
        \begin{equation*}
            \Delta(P, Q) = \lim_{n \to \infty} d (p_n, q_n) ;
        \end{equation*}
        by Exercise \ref{ex:3.23}, this limit exists. Show that the number $\Delta(P, Q)$ is unchanged if $\{p_n\}$ and $\{q_n\}$ are replaced by equivalent sequences, and hence that $\Delta$ is a distance
        function in $X^*$.
        \item Prove that the resulting metric space $X^*$ is complete.
        \item For each $p \in X$, there is a Cauchy sequence all of whose terms are $p$; let $P_p$ be the element of $X^*$ which contains this sequence. Prove that
        \begin{equation*}
            \Delta(P_p, P_q) = d(p, q)
        \end{equation*}
        for all $p, q \in X$. 
        In other words, the mapping $\phi$ defined by $\phi(p) = P_p$ is an isometry (i.e., a distance-preserving mapping) if $X$ into $X^*$ .
        \item Prove that $\phi(X)$ is dense in $X^*$, and that $\phi(X) = X^*$ if $X$ is complete. By (d), we may identify $X$ and $\phi(X)$ and thus regard $X$ as embedded in the complete metric space $X^*$. We call $X^*$ the \emph{completion} of $X$.
    \end{asparaenum}
\end{myexercise}


\begin{myexercise}
    \label{ex:3.25}
    Let $X$ be the metric space whose points are the rational numbers, with the metric $d(x, y) =|x - y|$, What is the completion of this space? (Compare Exercise \ref{ex:3.24}.)
\end{myexercise}