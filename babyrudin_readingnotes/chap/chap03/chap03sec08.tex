% chap03sec08
\section{The number $e$}

\begin{mydef}
    \label{mydef:3.30}
    $e = \sum_{n=0}^{\infty}\frac{1}{n!}.$

    Here $n! = 1 \cdot 2 \cdot 3 \cdots n$ if $n \geq 1$ , and $0! = 1$.
\end{mydef}

Since
\begin{align*}
    s_n
     & = 1 + 1
    + \frac{1}{1 \cdot 2}
    + \frac{1}{1 \cdot 2 \cdot 3}
    + \cdots
    + \frac{1}{1 \cdot 2 \cdots n} \\
     & < 1 + 1
    + \frac{1}{2}
    + \frac{1}{2^2}
    + \cdots
    + \frac{1}{2^{n-1}}
    < 3
\end{align*}
The series converges, and the definition makes sense. In fact, the series converges very rapidly and allows us to compute $e$ with great accuracy.

It is of interest to note that $e$ can also be difined by means of another limit process; the proof provides a good illustration of operations with limits:

\begin{thm}
    \label{thm:3.31}
    $\lim_{n \to \infty} (1+1/n)^n = e.$
\end{thm}
\mybox{Is this equation found by Bernoulli?}
\begin{proof}
    Let
    \begin{equation*}
        s_n = \sum_{k=0}^{n} \frac{1}{k!}, \quad
        t_n = \sum_{k=0}^{n} (1 + \frac{1}{n})^n.
    \end{equation*}
    by the binomial theorem,
    \begin{align*}
        t_n & = 1 + 1
        + \frac{1}{2!}\left(1 - \frac{1}{n}\right)
        + \frac{1}{3!}\left(1 - \frac{1}{n}\right)\left(1 - \frac{2}{n}\right)
        + \cdots                                                                                                        \\
            & + \frac{1}{n!}\left(1 - \frac{1}{n}\right)\left(1 - \frac{2}{n}\right)\cdots\left(1-\frac{n-1}{n}\right).
    \end{align*}
    Hence $t_n \leq s_n$ , so that
    \begin{equation}
        \label{eq:3.14}
        \limsup_{n \rightarrow \infty}  t_n \leq e,
    \end{equation}

    by Theorem \ref{thm:3.19}. Next, if $n \geq m$ ,
    \begin{equation*}
        t_n \geq 1 + 1
        + \frac{1}{2!}\left(1 - \frac{1}{n}\right)
        + \cdots
        + \frac{1}{m!}\left(1-\frac{1}{n}\right)
        + \cdots
        + \frac{1}{m!}\left(1-\frac{1}{n}\right)\cdots\left(1-\frac{m-1}{n}\right).
    \end{equation*}

    Let $n \rightarrow \infty$ , kepping $m$ fixed. We get
    \begin{equation*}
        \liminf_{n \to \infty} t_n \geq 1 + 1
        + \frac{1}{2!}
        + \cdots
        + \frac{1}{m!},
    \end{equation*}
    so that
    \begin{equation*}
        s_m \leq \liminf_{n \rightarrow \infty} t_n,
    \end{equation*}
    Letting $m \rightarrow \infty$, we finally get
    \begin{equation}
        \label{eq:3.15}
        e \leq \liminf_{n \rightarrow \infty} t_n.
    \end{equation}

    The Theorem follows from (\ref{eq:3.14}) and (\ref{eq:3.15}).
\end{proof}

The rapidly with which the series $\sum 1/n!$ converges can be estimated as follows: If $s_n$ has the same meaning as above, we have
\begin{align*}
    e - s_n
     & = \frac{1}{(n+1)!}
    + \frac{1}{(n+2)!}
    + \frac{1}{(n+3)!}
    + \cdots                     \\
     & < \frac{1}{(n+1)!}\left\{
    1
    + \frac{1}{n+1}
    + \frac{1}{(n+1)^2}
    + \cdots
    \right\} = \frac{1}{n!n}
\end{align*}
so that
\begin{equation}
    \label{eq:3.16}
    0 < e - s_n < \frac{1}{n!n}.
\end{equation}
Thus $s_{10}$, for instance, approximates $e$ with an error less than $10^{-7}$.
The inequality (\ref{eq:3.16}) is of theoretical interest as well, since it enables us to prove the irrationality of $e$ very easily.

\begin{thm}
    \label{thm:3.32}
    $e$ is irrational.
\end{thm}

\begin{proof}
    Suppose $e$ is rational. Then $e = p/q$, where $p$ and $q$ are positive integers.
    By (\ref{eq:3.16}),
    \begin{equation}
        \label{eq:3.17}
        0<q!(e-s_q)<\frac{1}{q}.
    \end{equation}
    By our assumption, $q!e$ is an integer. Since
    \begin{equation*}
        q!s_q =
        q!\left(
        1 + 1 + \frac{1}{2!} + \cdots + \frac{1}{q!}
        \right)
    \end{equation*}
    is an integer, we see that $q!(e-s_q)$ is an integer.

    Since $q \geq 1$, (\ref{eq:3.17}) implies the existence of an integer between $0$ and $1$. We have thus reached a contradiction.
\end{proof}

Actually, $e$ is not even an algebraic number.
For a simple proof of this, see page 25 of Niven's\cite{NIVEN1956} book, or page 176 of Herstein's\cite{HERSTEIN1964}, cited in the Bibliography.
