% chap09sec07

\section{Determinants}

Determinants are numbers associated to square matrices, and hence to the
operators represented by such matrices. They are 0 if and only if the 
corresponding operator fails to be invertible. 
They can therefore be used to decide
whether the hypotheses of some of the preceding theorems are satisfied. They
will play an even more important role in Chap. 10.

\begin{mydef}
    \label{mydef:9.33}
    If $(j_1, \dots, j_n)$ is an ordered $n$-tuple of integers, define
    \begin{equation}
        \label{eq:9.82}
        s(j_1, \dots, j_n) = 
        \prod_{p<q} \text{sgn } (j_q - j_p),
    \end{equation}
    where sgn $x = 1$ if $x > 0$, 
    sgn $x = -1$ if $x < 0$, 
    sgn $x = 0$ if $x = 0$. 
    Then $s(j_1, ... ,j_n) = 1, -1$, or $0$, and it changes sign if any two of the j's are interchanged.
    % todo add words
\end{mydef}

\begin{thm}
    \label{thm:9.34}
    \begin{asparaenum}[(a)]
        \item If $I$ is the identity operator on $\R^n$, then
        \begin{equation*}
            \det [I] = \det ( e_1, \dots , e_n) = 1.
        \end{equation*}
        \item $\det$ is a linear function of each of the column vectors $x_i$, if the others are held fixed.
        \item If $[A]_1$ is obtained from $[A]$ by interchanging two columns, then $\det [A]_1 = -\det [A]$.
        \item If $[A]$ has two equal columns, then $\det [A]= 0$.
    \end{asparaenum}
\end{thm}

% todo add proof

\begin{thm}
    \label{thm:9.35}
    If $[A]$ and $[B]$ are $n$ by $n$ matrices, then
    \begin{equation*}
        \det ([B][A]) = \det [B] \det [A].
    \end{equation*}
\end{thm}

% todo add proof

\begin{thm}
    \label{thm:9.36}
    A linear operator $A$ on $\R^n$ is invertible if and only if 
    $\det [A] \neq 0$.
\end{thm}

% todo add proof

\begin{myremark}
    Suppose $\{e_1, ... , e_n\}$ 
    and $\{u_1, ... , u_n\}$ are bases in $\R^n$.
    Every linear operator $A$ on $\R^n$ determines matrices $[A]$ and $[A]_U$, with entries $a_{ij}$ and $\alpha_{ij}$, given by
    \begin{equation*}
        A \mathbf{e}_j = \sum_{i} a_{ij} \mathbf{e}_i ,
        \quad
        A \mathbf{u}_j = \sum_{i} \alpha_{ij} \mathbf{u}_i .
    \end{equation*}
    % todo add words
\end{myremark}

\begin{mydef}
    \myKeyword{Jacobians}
    If $\mathbf{f}$ maps an open set $E \subset \R^n$ into $\R^n$, 
    and if $\mathbf{f}$ is differentiable at a point $\mathbf{x} \in E$, 
    the determinant of the linear operator $\mathbf{f'(x)}$ is called
    the \emph{Jacobian of} $\mathbf{f}$ at $\mathbf{x}$. 
    In symbols,
    \begin{equation}
        \label{eq:9.93}
        J_{\mathbf{f}}(\mathbf{x}) = 
        \det 
        \mathbf{f'(x)} .
    \end{equation}
    We shall also use the notation
    \begin{equation}
        \label{eq:9.94}
        \frac{\partial (y_1,...,y_n)}{\partial (x_1,...,x_n)}
    \end{equation}
    for $J_{\mathbf{f}} (\mathbf{x})$, if $(y_1, ... , y_n) = \mathbf{f} (x_1, ... , x_n)$.

    In terms of Jacobians, the crucial hypothesis in the inverse function theorem is that $J_{\mathbf{f}}(\mathbf{a}) \neq 0$ 
    (compare Theorem \ref{thm:9.36}). 
    If the implicit function theorem is stated in terms of the functions (\ref{eq:9.59}), 
    the assumption made there on A amounts to
    \begin{equation*}
        \frac{\partial (y_1,...,y_n)}{\partial (x_1,...,x_n)}
        \neq 0
    \end{equation*}
\end{mydef}

