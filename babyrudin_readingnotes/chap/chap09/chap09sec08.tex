% chap09sec08

\section{Derivatives of higher order}

\begin{mydef}
    \label{mydef:9.39}
    Suppose $f$ is a real function defined in an open set $E \subset \R^n$,
    with partial derivatives $D_1 f, \dots , D_n f$.
    If the functions $D_1 f$ are themselves differentiable,
    then the \emph{second-order partial derivatives} of $f$ are defined by
    \begin{equation*}
        D_{ij}f = D_i D_j f
        \quad
        (i,j=1, ... ,n) .
    \end{equation*}
    If all these functions $D_{ij} f$ are continuous in $E$,
    we say that $f$ is of class $\mathscr{C}''$ in $E$,
    or that $f \in \mathscr{C}''(E)$.

    A mapping $\mathbf{f}$ of $E$ into $\R^m$ is said to be of class $\mathscr{C}''$ if each component of $\mathbf{f}$ is of class $\mathscr{C}''$.

    It can happen that $D_{ij}f \neq D_{ji}f$ at some point, although both derivatives exist (see Exercise \ref{ex:9.27}).
    However, we shall see below that $D_{ij}f \neq D_{ji}f$ whenever these derivatives are continuous.

    For simplicity (and without loss of generality) we state our next two
    theorems for real functions of two variables.
    The first one is a mean value theorem.
\end{mydef}

\begin{thm}
    \label{thm:9.40}
    Suppose $f$ is defined in an open set $E \subset \R^2$ ,
    and $D_{1}f$ and $D_{21}f$ exist at every point of $E$.
    Suppose $Q \subset E$ is a closed rectangle with sides parallel to the coordinate axes, having $(a, b)$ and $(a +h, b + k)$ as opposite vertices ($h \neq 0$, $k \neq 0$). Put
    \begin{equation*}
        \Delta (f, Q) = f(a + h, b + k) - f(a + h, b) - f(a, b + k) + f(a, b).
    \end{equation*}

    Then there is a point $(x, y)$ in the interior of $Q$ such that
    \begin{equation}
        \label{eq:9.95}
        \Delta (f, Q) = hk(D_{21}f)(x, y).
    \end{equation}
\end{thm}

Note the analogy between (\ref{eq:9.95}) and Theorem \ref{thm:5.10};
the area of $Q$ is $hk$.

\begin{proof}
    Put $u(t) = f(t, b+k) - f(t, b)$.
    Two applications of Theorem \ref{thm:5.10} show that
    there is an $x$ between $a$ and $a+h$,
    and that there is a $y$ between $b$ and $b+k$,
    such that
    \begin{align*}
        \Delta(f, Q)
         & = u(a+h) - u(a)                                    \\
         & = h u'(x)                                          \\
         & = h \left[ (D_1 f)(x, b+k) - (D_1 f)(x, b) \right] \\
         & = hk (D_{21} f)(x, y) .
    \end{align*}
\end{proof}

\begin{thm}
    \label{thm:9.41}
    Suppose $f$ is defined in an open set $E \subset \R^2$,
    suppose that $D_1 f$, $D_{21} f$, and $D_2 f$ exist at every point of $E$,
    and $D_{21} f$ is continuous at some point $(a,b) \in E$.

    Then $D_{12} f$ exists at $(a,b)$ and
    \begin{equation}
        \label{eq:9.96}
        (D_{12} f)(a,b) =
        (D_{21} f)(a,b)
    \end{equation}
\end{thm}

\begin{myCorollary*}
    $D_{21} f = D_{12} f$ if $f \in \mathscr{C}''(E)$.
\end{myCorollary*}

\begin{proof}
    Put $A = (D_{21} f)(a,b)$.
    Choose $\varepsilon > 0$.
    If $Q$ is a rectangle as in Theorem \ref{thm:9.40},
    and of $h$ and $k$ are sufficiently small, we have
    \begin{equation*}
        \left| A - (D_{21} f)(x,y) \right| < \varepsilon
    \end{equation*}
    for all $(x,y) \in Q$.
    Thus
    \begin{equation*}
        \left| \frac{\Delta (f, Q)}{hk} - A \right| < \varepsilon ,
    \end{equation*}
    by (\ref{eq:9.95}).
    Fix $h$, and let $k \rightarrow 0$.
    Since $D_{2} f$ exists in $E$, the last inequality implies that
    \begin{equation}
        \label{eq:9.97}
        \left|
        \frac{(D_2 f)(a+h,b) - (D_2 f)(a,b)}{h} - A
        \right| \leq \varepsilon .
    \end{equation}
    Since $\varepsilon$ was arbitrary,
    and since (\ref{eq:9.97}) holds for all sufficiently small $h \neq 0$,
    it follows that $(D_{12} f)(a, b) = A$.
    This gives (\ref{eq:9.96}).
\end{proof}
