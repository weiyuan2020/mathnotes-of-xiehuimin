% chap09sec06

\section{The rank theorem}

Although this theorem is not as important as the inverse function theorem or
the implicit function theorem, we include it as another interesting illustration
of the general principle that the local behavior of a continuously differentiable
mapping $\mathbf{F}$ near a point $\mathbf{x}$ is similar to that of the linear transformation $\mathbf{F'(x)}$.

Before stating it, we need a few more facts about linear transformations.

\begin{mydef}
    \label{mydef:9.30}
    Suppose $X$ and $Y$ are vector spaces, and $A \in L( X, Y)$, as in Definition 9.6. 
    The \emph{null space} of $A$, $\mathscr{N}(A)$, 
    is the set of all $\mathbf{x} \in X$ at which $A \mathbf{x = 0}$.
    It is clear that $\mathscr{N}(A)$ is a vector space in $X$.
    
    Likewise, the \emph{range} of $A$, $\mathscr{R}(A)$, is a vector space in $Y$.

    The \emph{rank} of $A$ is defined to be the dimension of $\mathscr{R}(A)$.
    
    For example, the invertible elements of $L(\R^n)$ are precisely those whose rank is $n$. This follows from Theorem \ref{thm:9.5}.
    
    If $A \in L(X, Y)$ and $A$ has rank 0, 
    then $A \mathbf{x = 0}$ for all $x \in A$, 
    hence $\mathscr{N}(A) = X$.
    
    In this connection, see Exercise 25.
\end{mydef}

\begin{mydef}
    \myKeyword{Projections}
    Let $X$ be a vector space. An operator $P\in L(X)$ is said to be
    a \emph{projection} in $X$ if $P^2 = P$.

    More explicitly, the requirement is that $P(P \mathbf{x}) = P \mathbf{x}$ for every $\mathbf{x} \in X$. 
    In other words, $P$ fixes every vector in its range $\mathscr{R}(P)$.
\end{mydef}

Here are some elementary properties of projections:


% todo enumerate -> asparaenum

\begin{thm}
    \label{thm:9.32}
    Suppose $m, n, r$ are nonnegative integers, $m \geq r, n \geq r$, 
    $\mathbf{F}$ is a $\mathscr{C}'$-mapping of an open set $E \subset \R^n$ into $\R^m$, 
    and $\mathbf{F'(x)}$ has rank $r$ for every $\mathbf{x} \in E$.

    Fix $\mathbf{a} \in E$, put $A = \mathbf{F'(a)}$, 
    let $Y_1$ be the range of $A$, 
    and let $P$ be a projection in $\R^m$ whose range is $Y_1$. 
    Let $Y_2$ be the null space of $P$.

    Then there are open sets $U$ and $V$ in $\R^n$, 
    with $\mathbf{a} \in U$, $U \subset E$, 
    and there is a 1-1 $\mathscr{C'}$-mapping $\mathbf{H}$ of $V$ onto $U$ 
    (whose inverse is also of class $\mathscr{C'}$) such that
    \begin{equation}
        \label{eq:9.66}
        \mathbf{F(H(x))} = A\mathbf{x} + \phi(A\mathbf{x})
        \quad 
        (\mathbf{x} \in V)
    \end{equation}
    where $q$, is a $\mathscr{C'}$-mapping of the open set $A(V) \subset Y_1$ into $Y_2$.
\end{thm}

After the proof we shall give a more geometric description of the information that (\ref{eq:9.66}) contains.

% todo add proof


\begin{equation}
    \label{eq:9.67}
    S( c_1 \mathbf{y}_1 + \cdots + c_r \mathbf{y}_r ) =
    c_1 \mathbf{z}_1 + \cdots + c_r \mathbf{z}_r 
\end{equation}


\begin{equation}
    \label{eq:9.68}
    AS \mathbf{y} = \mathbf{y} 
    \quad 
    (\mathbf{y} \in Y_1).
\end{equation}

\begin{equation}
    \label{eq:9.69}
    \mathbf{G(x)} = 
    \mathbf{x} + SP[\mathbf{F(x)}-A\mathbf{x}]
    \quad 
    (\mathbf{x} \in E).
\end{equation}


\begin{equation}
    \label{eq:9.70}
    A \mathbf{G(x)} 
    P \mathbf{F(x)} 
    \quad 
    (\mathbf{x} \in E).
\end{equation}

\begin{equation}
    \label{eq:9.71}
    P \mathbf{F(H(x))} = A \mathbf{x} 
    \quad 
    (\mathbf{x} \in V).
\end{equation}

\begin{equation}
    \label{eq:9.72}
    \psi(\mathbf{x}) = \mathbf{F(H(x))} - A\mathbf{x}
    \quad 
    (\mathbf{x} \in V).
\end{equation}

\begin{equation}
    \label{eq:9.73}
    \phi(A \mathbf{x}) = \psi(\mathbf{x})
    \quad 
    (\mathbf{x} \in V).
\end{equation}

\begin{equation}
    \label{eq:9.74}
    \psi(\mathbf{x}_1) = 
    \psi(\mathbf{x}_2)
\end{equation}


\begin{equation}
    \label{eq:9.75}
    \rank \mathbf{\Phi'(x)} = 
    \rank \mathbf{F'(H(x))H'(x)} = r
    \quad 
    (\mathbf{x} \in V).
\end{equation}


\begin{equation}
    \label{eq:9.76}
    P \mathbf{\Phi'(x)} = A.
\end{equation}


\begin{equation}
    \label{eq:9.77}
    \mathbf{g}(t) = \psi(\mathbf{x}_1 + t \mathbf{h})
    \quad 
    (0 \leq t \leq 1).
\end{equation}

\begin{equation}
    \label{eq:9.78}
    \mathbf{g}'(t) = \psi'(\mathbf{x}_1 + t \mathbf{h})
    \quad 
    (0 \leq t \leq 1),
\end{equation}


\begin{equation}
    \label{eq:9.79}
    \mathbf{x} = \mathbf{x_0} + S(\mathbf{y-y_0})
\end{equation}

\begin{equation*}
    A \mathbf{x} =
    A \mathbf{x}_0 + \mathbf{y-y_0} = \mathbf{y} .
\end{equation*}

\begin{equation}
    \label{eq:9.80}
    \phi(\mathbf{y}) = \psi(\mathbf{x_0} - S\mathbf{y}_0 + S\mathbf{y})
    \quad 
    (\mathbf{y} \in W).
\end{equation}


% todo

\begin{equation}
    \label{eq:9.81}
    \mathbf{y} = P \mathbf{y} + \phi(P \mathbf{y})
    \quad 
    (\mathbf{y} \in \mathbf{F}(U)).
\end{equation}