% chap09exercise

\section{Exercises}


\begin{myExercise}
    \label{ex:9.1}
\end{myExercise}


\begin{myExercise}
    \label{ex:9.2}
\end{myExercise}


\begin{myExercise}
    \label{ex:9.3}
\end{myExercise}


\begin{myExercise}
    \label{ex:9.4}
\end{myExercise}


\begin{myExercise}
    \label{ex:9.5}
\end{myExercise}


\begin{myExercise}
    \label{ex:9.6}
\end{myExercise}


\begin{myExercise}
    \label{ex:9.7}
\end{myExercise}


\begin{myExercise}
    \label{ex:9.8}
\end{myExercise}


\begin{myExercise}
    \label{ex:9.9}
\end{myExercise}


\begin{myExercise}
    \label{ex:9.10}
\end{myExercise}



\begin{myExercise}
    \label{ex:9.11}
\end{myExercise}


\begin{myExercise}
    \label{ex:9.12}
\end{myExercise}


\begin{myExercise}
    \label{ex:9.13}
\end{myExercise}


\begin{myExercise}
    \label{ex:9.14}
\end{myExercise}


\begin{myExercise}
    \label{ex:9.15}
\end{myExercise}


\begin{myExercise}
    \label{ex:9.16}
\end{myExercise}


\begin{myExercise}
    \label{ex:9.17}
\end{myExercise}


\begin{myExercise}
    \label{ex:9.18}
\end{myExercise}


\begin{myExercise}
    \label{ex:9.19}
\end{myExercise}


\begin{myExercise}
    \label{ex:9.20}
\end{myExercise}


\begin{myExercise}
    \label{ex:9.21}
\end{myExercise}


\begin{myExercise}
    \label{ex:9.22}
\end{myExercise}


\begin{myExercise}
    \label{ex:9.23}
\end{myExercise}


\begin{myExercise}
    \label{ex:9.24}
\end{myExercise}


\begin{myExercise}
    \label{ex:9.25}
\end{myExercise}


\begin{myExercise}
    \label{ex:9.26}
\end{myExercise}


\begin{myExercise}
    \label{ex:9.27}
\end{myExercise}


\begin{myExercise}
    \label{ex:9.28}
\end{myExercise}


\begin{myExercise}
    \label{ex:9.29}
\end{myExercise}


\begin{myExercise}
    \label{ex:9.30}
\end{myExercise}


\begin{myExercise}
    \label{ex:9.31}
    Suppose $f \in \mathscr{C}^{(3)}$ in some neighborhood of a point $\mathbf{a} \in \R^2$, the gradient of $f$ is $\mathbf{0}$ at $\mathbf{a}$, but not all second-order derivatives of $f$ are 0 at $\mathbf{a}$. 
    Show how one can then determine from the Taylor polynomial of $f$ at $\mathbf{a}$ (of degree 2) whether $f$ has a local maximum, or a local minimum, or neither, at the point $\mathbf{a}$. 
    
    Extend this to $\R^n$ in place of $\R^2$.
\end{myExercise}