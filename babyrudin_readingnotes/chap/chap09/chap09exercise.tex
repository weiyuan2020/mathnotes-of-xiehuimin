% chap09exercise

\section{Exercises}


\begin{myexercise}
    \label{ex:9.1}
    If $S$ is a nonempty subset of a vector space $X$, 
    prove (as asserted in Sec. \ref{mydef:9.1}) that
    the span of $S$ is a vector space.
\end{myexercise}


\begin{myexercise}
    \label{ex:9.2}
    Prove (as asserted in Sec. \ref{mydef:9.6}) that $BA$ is linear if $A$ and $B$ are linear transformations.
    
    Prove also that $A^{- 1}$ is linear and invertible.
\end{myexercise}


\begin{myexercise}
    \label{ex:9.3}
    Assume $A \in  L(X, Y)$ and $A\mathbf{x}= \mathbf{0}$ only when $\mathbf{x}= \mathbf{0}$. 
    Prove that $A$ is then 1-1.
\end{myexercise}


\begin{myexercise}
    \label{ex:9.4}
    Prove (as asserted in Sec. \ref{mydef:9.30}) that null spaces and ranges of linear transformations are vector spaces.
\end{myexercise}


\begin{myexercise}
    \label{ex:9.5}
    Prove that to every $A \in L(\R^n, \R^1)$ corresponds a unique $\mathbf{y} \in \R^n$ such that $A\mathbf{x = x \cdot y}$.
    
    Prove also that $\left\| A \right\| = \left| \mathbf{y} \right| $.

    \emph{Hint:} Under certain conditions, equality holds in the Schwarz inequality.
\end{myexercise}


\begin{myexercise}
    \label{ex:9.6}
    If $f (0, 0) = 0$ and
    \begin{equation*}
        f(x, y) = \frac{xy}{x^2 + y^2} \quad\text{if } (x, y) \neq (0, 0),
    \end{equation*}
    prove that $(D_1f)(x, y)$ and $(D_2f)(x, y)$ exist at every point of $\R^2$, although $f$ is not continuous at $(0, 0)$.
\end{myexercise}


\begin{myexercise}
    \label{ex:9.7}
    Suppose that $f$ is a real-valued function defined in an open set $E \subset \R^n$, 
    and that the partial derivatives $D_1f, \dots , D_nf$ are bounded in $E$. 
    Prove that $f$ is continuous in $E$.

    \emph{Hint:} Proceed as in the proof of Theorem \ref{thm:9.21}.
\end{myexercise}


\begin{myexercise}
    \label{ex:9.8}
    Suppose that $f$ is a differentiable real function in an open set $E \subset \R^n$, 
    and that $f$ has a local maximum at a point $\mathbf{x} \in E$. 
    Prove that $f'(\mathbf{x}) = 0$.
\end{myexercise}


\begin{myexercise}
    \label{ex:9.9}
    If $\mathbf{f}$ is a differentiable mapping of a \myKeywordblue{connected} open set $E \subset \R^n$ into $\R^m$, 
    and if $\mathbf{f}'(\mathbf{x}) = 0$ for every $\mathbf{x} \in E$, 
    prove that $\mathbf{f}$ is constant in $E$.
\end{myexercise}


\begin{myexercise}
    \label{ex:9.10}
    If $f$ is a real function defined in a convex open set $E \subset \R^n$, 
    such that $(D_1f)(\mathbf{x}) = 0$ for every $\mathbf{x} \in E$, 
    prove that $f(\mathbf{x})$ depends only on $x_2, ... , x_n$.

    Show that the convexity of E can be replaced by a weaker condition, but that some condition is required. 
    For example, if $n = 2$ and $E$ is shaped like a horseshoe, the statement may be false.
\end{myexercise}



\begin{myexercise}
    \label{ex:9.11}
    If $f$ and $g$ are differentiable real functions in $\R^n$, 
    prove that
    \begin{equation*}
        \nabla (fg) = 
        f\nabla g + 
        g\nabla f 
    \end{equation*}
    and that $\nabla (1/f)=-f^{-2}\nabla f$ wherever $f \neq 0$.
\end{myexercise}


\begin{myexercise}
    \label{ex:9.12}
    Fix two real numbers $a$ and $b$, $0 <a< b$. 
    Define a mapping $\mathbf{f} = (f_1,f_2,f_3)$of $\R^2$ into $\R^3$ by
    \begin{align*}
        f_1(s,t)&=(b+a\cos s)\cos t \\
        f_2(s,t)&=(b+a\cos s)\sin t \\
        f_3(s,t)&=a\sin s.
    \end{align*}
    Describe the range $K$ of $\mathbf{f}$. 
    (It is a certain compact subset of $\R^3$.)
    \begin{asparaenum}[(a)]
        \item Show that there are exactly 4 points $\mathbf{p} \in K$ such that
        \begin{equation*}
            (\nabla f_1)(\mathbf{f}^{-1}(\mathbf{p})) = \mathbf{0}.
        \end{equation*}
        Find these points.
        \item Determine the set of all $\mathbf{q} \in K$ such that
        \begin{equation*}
            (\nabla f_3)(\mathbf{f}^{-1}(\mathbf{q})) = \mathbf{0}.
        \end{equation*}
        \item Show that one of the points $\mathbf{p}$ found in part (a) corresponds to a local maximum of $f_1$, 
        one corresponds to a local minimum, 
        and that the other two are neither 
        (they are so-called ``saddle points'').

        Which of the points $\mathbf{q}$ found in part (b) correspond to maxima or minima?
        \item Let $\lambda$  be an irrational real number, 
        and define $\mathbf{g}(t) = \mathbf{f}(t, \lambda t)$. 
        Prove that $\mathbf{g}$ is a 1-1 mapping of $\R^1$ onto a dense subset of $K$. 
        Prove that
        \begin{equation*}
            \left| \mathbf{g}'(t) \right|^2 =
            a^2 + \lambda^2(b + a \cos t)^2 .
        \end{equation*}
    \end{asparaenum}
\end{myexercise}


\begin{myexercise}
    \label{ex:9.13}
    Suppose $\mathbf{f}$ is a differentiable mapping of $\R^1$ into $\R^3$ 
    such that $|\mathbf{f}(t)|=1$ for every $t$.
    Prove that $\mathbf{f}'(t) \cdot \mathbf{f}(t) = 0$.

    Interpret this result geometrically.
\end{myexercise}


\begin{myexercise}
    \label{ex:9.14}
    Define $f(0, 0) = 0$ and
    \begin{equation*}
        f(x, y) = \frac{x^3}{x^2+y^2}
        \quad\text{if }(x, y) \neq (0, 0).
    \end{equation*}
    \begin{asparaenum}[(a)]
        \item Prove that $D_1 f$ and $D_2 g$ are bounded functions in $\R^2$. (Hence $f$ is continuous.)
        \item Let $\mathbf{u}$ be any unit vector in $\R^2$. Show that the directional derivative $(D_{\mathbf{u}}f)(0, 0)$ exists, and that its absolute value is at most 1.
        \item Let $\gamma$ be a differentiable mapping of $\R^1$ into $\R^2$ 
        (in other words, $\gamma$ is a differentiable curve in $\R^2$), 
        with $\gamma(0) = (0, 0)$ and $| \gamma'(0) |> 0$. 
        Put $g(t) =  f(\gamma(t))$ and prove that $g$ is differentiable for every $t \in \R^1$.

        If $\gamma \in \mathscr{C}'$, prove that $g \in \mathscr{C}'$.
        \item In spite of this, prove that $f$ is not differentiable at $(0, 0)$.
    \end{asparaenum}
    \emph{Hint:} Formula (\ref{eq:8.40}) fails.
\end{myexercise}


\begin{myexercise}
    \label{ex:9.15}
    Define $f(0, 0) = 0$, and put
    \begin{equation*}
        f(x,y)=x^2+y^2-2x^2y-\frac{4x^6y^2}{(x^4+y^2)^2}
    \end{equation*}
    if $(x,y)\neq (0,0)$.
    \begin{asparaenum}[(a)]
        \item Prove, for all $(x, y) \in \R^2$ , that
        \begin{equation*}
            4x^3y^2 \leq (x^4+y^2)^2.
        \end{equation*}
        Conclude that $f$ is continuous.
        \item For $0 \leq \theta \leq 2\pi$, $-\infty < t < \infty$, define
        \begin{equation*}
            g_{\theta}(t) = f(t\cos \theta, t\sin \theta).
        \end{equation*}
        Show that 
        $g_{\theta} (0) = 0$, 
        $g'_{\theta} (0) = 0$, 
        $g''_{\theta} (0) = 2$. 
        Each $g_{\theta}$ has therefore a strict local minimum at $t = 0$.

        In other words, the restriction of $f$ to each line through $(0, 0)$ has a strict local minimum at $(0, 0)$.
        \item Show that $(0, 0)$ is nevertheless not a local minimum for $f$, since $f(x, x^2) = -x^4$.
    \end{asparaenum}
\end{myexercise}


\begin{myexercise}
    \label{ex:9.16}
    Show that the continuity of $\mathbf{f}'$ at the point $\mathbf{a}$ is needed in the inverse function theorem, even in the case $n = 1$ : If
    \begin{equation*}
        f(t) = t+2t^2\sin \left( \frac{1}{t} \right)
    \end{equation*} 
    for $t \neq 0$, and $f(0) = 0$, then $f'(0) = 1$, $f'$ is bounded in $(-1, 1)$, 
    but $f$ is not one-to-one in any neighborhood of 0.
\end{myexercise}


\begin{myexercise}
    \label{ex:9.17}
    Let $f = (f_1,f_2)$ be the mapping of $\R^2$ into $\R^2$ given by
    \begin{equation*}
        f_1(x,y) = e^x \cos y, 
        \quad
        f_2(x,y) = e^x \sin y, 
    \end{equation*}
    \begin{enumerate}[(a)]
        \item What is the range of $f$?
        \item Show that the Jacobian of $f$ is not zero at any point of $\R^2$. 
        Thus every point of $\R^2$ has a neighborhood in which $f$ is one-to-one. 
        Nevertheless, $f$ is not one-to-one on $\R^2$.
        \item Put $\mathbf{a} = (0, \pi/3), \mathbf{b} = f(\mathbf{a})$, let $\mathbf{g}$ be the continuous inverse of $\mathbf{f}$, 
        defined in a neighborhood of $\mathbf{b}$, such that $\mathbf{g(b) = a}$. 
        Find an explicit formula for $\mathbf{g}$, compute $\mathbf{f'(a)}$ and $\mathbf{g'(b)}$, and verify the formula \eqref{eq:9.52}. 
        \item What are the images under $\mathbf{f}$ of lines parallel to the coordinate axes?
    \end{enumerate}
\end{myexercise}


\begin{myexercise}
    \label{ex:9.18}
    Answer analogous questions for the mapping defined by
    \begin{equation*}
        u = x^2-y^2, 
        \quad 
        v = 2xy.
    \end{equation*}
\end{myexercise}


\begin{myexercise}
    \label{ex:9.19}
    Show that the system of equations 
    \begin{align*}
        3x +  y -  z +  u^2 &= 0 \\
         x -  y + 2z +  u   &= 0 \\
        2x + 2y - 3z + 2u   &= 0 
    \end{align*}
    can be solved 
            for $x, y, u$ in terms of $z$;
            for $x, z, u$ in terms of $y$;
            for $y, z, u$ in terms of $x$;
    but not for $x, y, z$ in terms of $u$.
\end{myexercise}

\mySolve
\begin{equation*}
    \begin{pmatrix}
        3 &  1 & -1 \\
        1 & -1 &  2 \\
        2 &  2 & -3 \\
    \end{pmatrix}
    \begin{pmatrix}
        x \\ y \\ z \\
    \end{pmatrix} = 
    \begin{pmatrix}
        -u^2 \\ u \\ 2u \\
    \end{pmatrix}
\end{equation*}


\begin{myexercise}
    \label{ex:9.20}
    Take $n = m = 1$ in the implicit function theorem, and interpret the theorem (as well as its proof) graphically.
\end{myexercise}


\begin{myexercise}
    \label{ex:9.21}
    Define $f$ in $\R^2$ by
    \begin{equation*}
        f(x,y) = 2x^3-3x^2+2y^3+3y^2.
    \end{equation*}
    \begin{enumerate}[(a)]
        \item Find the four points in $\R^2$ at which the gradient of $f$ is zero. 
        Show that $f$ has exactly one local maximum and one local minimum in $\R^2$.
        \item Let $S$ be the set of all $(x, y) \in \R^2$ at which $f(x, y) = 0$. 
        Find those points of $S$ that have no neighborhoods in which the equation $f(x, y) = 0$ can be solved for $y$ in terms of $x$ 
        (or for $x$ in terms of $y$). 
        Describe $S$ as precisely as you can.
    \end{enumerate}
\end{myexercise}


\begin{myexercise}
    \label{ex:9.22}
    Give a similar discussion for
    \begin{equation*}
        f(x,y)=2x^3+6xy^2-3x^2+3y^2.
    \end{equation*}
\end{myexercise}


\begin{myexercise}
    \label{ex:9.23}
    Define $f$ in $\R^3$ by
    \begin{equation*}
        f(x,y_1,y_2) = x^2 y_1 + e^x + y_2 .
    \end{equation*}
    Show that $f(0, 1, -1) = 0$, $(D_1 f) (0, 1, -1) \neq 0$, 
    and that there exists therefore a differentiable function $g$ in some neighborhood of $(1, -1)$ in $\R^2$, 
    such that $g(1, -1) = 0$ and
    \begin{equation*}
        f(g(y_1, y_2), y_1, y_2) = 0.
    \end{equation*}
    Find $(D_1 g)(1, -1)$ and $(D_2 g)(1, -1)$.
\end{myexercise}


\begin{myexercise}
    \label{ex:9.24}
    For $(x, y) \neq (0, 0)$, define $\mathbf{f} = (f_1,f_2)$ by
    \begin{equation*}
        f_1(x,y) = \frac{x^2-y^2}{x^2+y^2}, 
        \quad 
        f_2(x,y) = \frac{xy}{x^2+y^2}.
    \end{equation*}
    Compute the rank of $\mathbf{f}'(x, y)$, and find the range of $\mathbf{f}$.
\end{myexercise}


\begin{myexercise}
    \label{ex:9.25}
    Suppose $A \in L(\R^n, \R^m)$, let $r$ be the rank of $A$.
    \begin{enumerate}[(a)]
        \item Define $S$ as in the proof of Theorem \ref{thm:9.32}. Show that $SA$ is a projection in $\R^n$ whose null space is $\mathscr{N}(A)$ and whose range is $\mathscr{R}(S)$. \emph{Hint:} By \eqref{eq:9.68}, $SASA = SA$.
        \item Use (a) to show that 
        \begin{equation*}
            \dim \mathscr{N}(A) + \dim \mathscr{R}(A) = n.
        \end{equation*}
    \end{enumerate}
\end{myexercise}


\begin{myexercise}
    \label{ex:9.26}
    Show that the existence (and even the continuity) of $D_{12}f$ does not imply the existence of $D_1 f$. 
    For example, let $f(x, y) = g(x)$, where $g$ is nowhere differentiable.
\end{myexercise}


\begin{myexercise}
    \label{ex:9.27}
    Put $f(0,0)=0$, and 
    \begin{equation*}
        f(x,y) = \frac{xy(x^2-y^2)}{x^2+y^2}
    \end{equation*}
    if $(x,y) \neq (0,0)$.
    Prove that 
    \begin{enumerate}[(a)]
        \item $f, D_1 f, D_2 f$ are continuous in $\R^2$;
        \item $D_{12}f$ and $D_{21}f$ exist at every point of $\R^2$, and are continuous except at $(0, 0)$;
        \item $(D_{12}f)(0,0)=1$, and $(D_{21}f)(0,0)=-1$.
    \end{enumerate}
\end{myexercise}


\begin{myexercise}
    \label{ex:9.28}
    For $t \geq 0$, put
    \begin{equation*}
        \phi(x,t) = \left\{ 
            \begin{array}{ll}
                x & (0 \leq x \leq \sqrt{t}) \\
                -x+2\sqrt{t} & (\sqrt{t} \leq x \leq \sqrt{t}) \\
                0 & (\text{otherwise}), \\
            \end{array}
         \right.
    \end{equation*}
    and put $\phi(x,t)=-\phi(x,|t|)$ if $t<0$.

    Show that $\phi$ is continuous on $\R^2$, and 
    \begin{equation*}
        (D_2 \phi)(x, 0) = 0
    \end{equation*}
    for all $x$. 
    Define 
    \begin{equation*}
        f(t) = \int_{-1}^{1} \phi (x,t) \d x.
    \end{equation*}
    Show that $f(t)=t$ if $|t|<\frac{1}{4}$.
    Hence 
    \begin{equation*}
        f'(0) \neq \int_{-1}^{1} (D_2 \phi) (x,0) \d x.
    \end{equation*}
\end{myexercise}


\begin{myexercise}
    \label{ex:9.29}
    Let $E$ be an open set in $\R^n$. 
    The classes $\mathscr{C}'(E)$ and $\mathscr{C}''(E)$ are defined in the text.
    By induction, $\mathscr{C}^{(k)} (E)$ can be defined as follows, for all positive integers $k$: 
    To say that $f \in \mathscr{C}^{(k)} (E)$ means that the partial derivatives $D_1 f, ... , D_n f$ belong to $\mathscr{C}^{(k-1)} (E)$.

    Assume $f \in \mathscr{C}^{(k)} (E)$. and show 
    (by repeated application of Theorem \ref{thm:9.41}) 
    that the $k$th-order derivative
    \begin{equation*}
        D_{i_1 i_2 \cdots i_k} f = D_{i_1} D_{i_2} \dots D_{i_k} f
    \end{equation*}
    is unchanged if the subscripts $i_1, ... , i_k$ are permuted.

    For instance, if $n \geq 3$, then
    \begin{equation*}
        D_{1213} f = D_{3112} f
    \end{equation*}
    for every $f \in \mathscr{C}^{(4)}$,
\end{myexercise}


\begin{myexercise}
    \label{ex:9.30}
    Let $f \in \mathscr{C}^{(m)} (E)$, 
    where $E$ is an open subset of $\R^n$. 
    Fix $\mathbf{a} \in E$, and suppose $x \in \R^n$ is so close to $\mathbf{0}$ that the points
    \begin{equation*}
        \mathbf{p}(t) = \mathbf{a} + t \mathbf{x}
    \end{equation*}
    lie in $E$ whenever $0 \leq t \leq 1$. 
    Define
    \begin{equation*}
        h(t) = f(\mathbf{p}(t))
    \end{equation*}
    for all $t \in \R^1$ for which $\mathbf{p}(t) \in E$.
    \begin{asparaenum}[(a)]
        \item For $1 \leq k \leq m$, show (by repeated application of the chain rule) that 
        \begin{equation*}
            h^{(k)}(t) = \sum 
            \left( D_{i_1 \cdots i_k} \right) \left( \mathbf{p}(t) \right) 
            x_{i_1} \dots x_{i_k} .
        \end{equation*}
        The sum extends over all ordered $k$-tuples $(i_1, ... , i_k)$ in which each $i_j$ is one of the integers $1, ... , n$.
        \item By Taylor's theorem \ref{thm:5.15},
        \begin{equation*}
            h(1) = \sum_{k=0}^{m-1} \frac{h^{(k)}(0)}{k!} + \frac{h^{(m)}(t)}{m!}
        \end{equation*}
        for some $t \in (0, 1)$. 
        Use this to prove Taylor's theorem in $n$ variables by showing that the formula
        \begin{equation*}
            f(\mathbf{a+x}) = \sum_{k=0}^{m-1} \frac{1}{k!}
            \sum \left( D_{i_1 \cdots i_k} f \right)(\mathbf{a})
            x_{i_1} \dots x_{i_k} 
            + r(\mathbf{x})
        \end{equation*}
        represents $f(\mathbf{a + x})$ as the sum of its so-called ``Taylor polynomial of degree $m - 1$,'' plus a remainder that satisfies
        \begin{equation*}
            \lim_{\mathbf{x} \to \mathbf{0}} \frac{r(\mathbf{x})}{|\mathbf{x}|^{m-1}} = 0.
        \end{equation*}
        Each of the inner sums extends over all ordered $k$-tuples $(i_1, ... , i_k)$, as in part (a); 
        as usual, the zero-order derivative of $f$ is simply $f$, so that the constant term of the Taylor polynomial off at $\mathbf{a}$ is $f(\mathbf{a})$.
        \item Exercise \ref{ex:9.29} shows that repetition occurs in the Taylor polynomial as written in part (b). 
        For instance, $D_{113}$ occurs three times, as $D_{113}, D_{131}, D_{311}$, 
        The sum of the corresponding three terms can be written in the form
        \begin{equation*}
            3\left( D_1^2 D_3 f \right)(\mathbf{a}) x_1^2 x_3 .
        \end{equation*}
        Prove (by calculating how often each derivative occurs) that the Taylor polynomial in (b) can be written in the form
        \begin{equation*}
            \sum \frac{(D_1^{s_1} \cdots D_n^{s_n} f)(\mathbf{a})}{s_1 ! \cdots s_n !} 
            x_1^{s_1} \cdots x_n^{s_n} .
        \end{equation*}
        Here the summation extends over all ordered $n$-tuples $(s_1, ... , s_n)$ such that each $s$, is a nonnegative integer, and $s_1 + \cdots + s_n \leq m - 1$.
    \end{asparaenum}
\end{myexercise}


\begin{myexercise}
    \label{ex:9.31}
    Suppose $f \in \mathscr{C}^{(3)}$ in some neighborhood of a point $\mathbf{a} \in \R^2$, the gradient of $f$ is $\mathbf{0}$ at $\mathbf{a}$, but not all second-order derivatives of $f$ are 0 at $\mathbf{a}$. 
    Show how one can then determine from the Taylor polynomial of $f$ at $\mathbf{a}$ (of degree 2) whether $f$ has a local maximum, or a local minimum, or neither, at the point $\mathbf{a}$. 
    
    Extend this to $\R^n$ in place of $\R^2$.
\end{myexercise}