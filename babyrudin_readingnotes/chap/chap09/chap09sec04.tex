% chap09sec04
\section{The inverse function theorem}

The inverse function theorem states, roughly speaking, that a continuously
differentiable mapping $\mathbf{f}$ is invertible in a neighborhood of any point $\mathbf{x}$ at which
the linear transformation $\mathbf{f'(x)}$ is invertible:

\begin{thm}
    \label{thm:9.24}
    Suppose $\mathbf{f}$ is a $\mathscr{C}'$-mapping of an open set $E \subset \R^n$ into $\R^n$, 
    $\mathbf{f'(a)}$ is invertible for some $\mathbf{a} \in E$, 
    and $\mathbf{b = f(a)}$. Then
    \begin{asparaenum}[(a)]
        \item there exist open sets $U$ and $V$ in $\R^n$ 
        such that $\mathbf{a} \in U$, $\mathbf{b} \in V$, 
        $\mathbf{f}$ is one-to-one on $U$, and $\mathbf{f}(U) = V$;
        \item if $\mathbf{g}$ is the inverse of $f$ [which exists, by (a)], defined in $V$ by
        \begin{equation*}
            \mathbf{g(f(x)) = x}
            \quad
            (\mathbf{x} \in U),
        \end{equation*}
        then $\mathbf{g} \in \mathscr{C}'(V)$.
    \end{asparaenum}
\end{thm}

% todo add proof


\begin{thm}
    \label{thm:9.25}
    If $\mathbf{f}$ is a $\mathscr{C}'$-mapping of an open set $E \subset \R^n$ into $\R^n$ 
    and if $\mathbf{f'(x)}$ is invertible for every $\mathbf{x} \in E$, 
    then $\mathbf{f}(W)$ is an open subset of $\R^n$ for every open set
    $W \subset E$
\end{thm}

In other words, $\mathbf{f}$ is an \emph{open mapping} of $E$ into $\R^n$

The hypotheses made in this theorem ensure that each point $\mathbf{x} \in E$ has a
neighborhood in which $\mathbf{f}$ is 1-1. 
This may be expressed by saying that $\mathbf{f}$ is
\emph{locally} one-to-one in $E$. 
But $\mathbf{f}$ need not be 1-1 in $E$ under these circumstances.
for an example, see Exercise 17.