% chap09sec04
\section{The inverse function theorem}

The inverse function theorem states, roughly speaking, that a continuously
differentiable mapping $\mathbf{f}$ is invertible in a neighborhood of any point $\mathbf{x}$ at which
the linear transformation $\mathbf{f'(x)}$ is invertible:

\begin{thm}
    \label{thm:9.24}
    Suppose $\mathbf{f}$ is a $\mathscr{C}'$-mapping of an open set $E \subset \R^n$ into $\R^n$, 
    $\mathbf{f'(a)}$ is invertible for some $\mathbf{a} \in E$, 
    and $\mathbf{b = f(a)}$. Then
    \begin{asparaenum}[(a)]
        \item there exist open sets $U$ and $V$ in $\R^n$ 
        such that $\mathbf{a} \in U$, $\mathbf{b} \in V$, 
        $\mathbf{f}$ is one-to-one on $U$, and $\mathbf{f}(U) = V$;
        \item if $\mathbf{g}$ is the inverse of $f$ [which exists, by (a)], defined in $V$ by
        \begin{equation*}
            \mathbf{g(f(x)) = x}
            \quad
            (\mathbf{x} \in U),
        \end{equation*}
        then $\mathbf{g} \in \mathscr{C}'(V)$.
    \end{asparaenum}
\end{thm}

Writing the equation $\mathbf{y = f(x)}$ in component form, we arrive at the following interpretation of the conclusion of the theorem: 
The system of $n$ equations
\begin{equation*}
    y_i = f_i(x_1,\dots,x_n)
    \quad
    (1 \leq i \leq n)
\end{equation*}
can be solved for $x_1, \dots , x_n$ in terms of $y_1, \dots , y_n$, if we restrict $\mathbf{x}$ and $\mathbf{y}$ to small enough neighborhoods of $\mathbf{a}$ and $\mathbf{b}$; 
the solutions are unique and continuously
differentiable.

% todo add proof

\begin{equation}
    \label{eq:9.46}
    2\lambda \left\| A^{-1} \right\| = 1.
\end{equation}

\begin{equation}
    \label{eq:9.47}
    \left\| \mathbf{f'(x)}-A \right\| < \lambda 
    \quad 
    (\mathbf{x} \in U).
\end{equation}

\begin{equation}
    \label{eq:9.48}
    \phi(\mathbf{x}) = \mathbf{x} + A^{-1} (\mathbf{y-f(x)})
    \quad 
    (\mathbf{x} \in E).
\end{equation}


\begin{equation}
    \label{eq:9.49}
    \left\| \phi'(\mathbf{x}) \right\| < \frac{1}{2}
    \quad 
    (\mathbf{x} \in U).
\end{equation}

\begin{equation}
    \label{eq:9.50}
    \left| 
        \phi(\mathbf{x}_1)-
        \phi(\mathbf{x}_2)
     \right| \leq 
     \frac{1}{2}
     \left| \mathbf{x}_1 - \mathbf{x}_2 \right| 
     \quad 
     (\mathbf{x}_1, \mathbf{x}_2 \in U).
\end{equation}


\begin{equation}
    \label{eq:9.51}
    \left| \mathbf{h} \right| \leq
    2 \left\| A^{-1} \right\| \left| \mathbf{k} \right| =
    \lambda^{-1} \left| \mathbf{k} \right| .
\end{equation}

\begin{equation*}
    \mathbf{g(y+k)-g(y)}-T\mathbf{k} =
    \mathbf{h} - T\mathbf{k} =
    -T\left[ 
        \mathbf{f(x+h)-f(x)-f'(x)h}
     \right],
\end{equation*}

\begin{equation*}
    \frac{|\mathbf{g(y+k)-g(y)}-T\mathbf{k}|}{|\mathbf{k}|} \leq
    \frac{\|T\|}{\lambda}\cdot \frac{|\mathbf{f(x+h)-f(x)-f'(x)h}|}{|\mathbf{h}|}.
\end{equation*}


\begin{equation}
    \label{eq:9.52}
    \mathbf{g'(y)} = 
    \left\{ \mathbf{f'(g(y))} \right\}^{-1}
    \quad 
    (\mathbf{y} \in V).
\end{equation}


\begin{thm}
    \label{thm:9.25}
    If $\mathbf{f}$ is a $\mathscr{C}'$-mapping of an open set $E \subset \R^n$ into $\R^n$ 
    and if $\mathbf{f'(x)}$ is invertible for every $\mathbf{x} \in E$, 
    then $\mathbf{f}(W)$ is an open subset of $\R^n$ for every open set
    $W \subset E$
\end{thm}

In other words, $\mathbf{f}$ is an \emph{open mapping} of $E$ into $\R^n$

The hypotheses made in this theorem ensure that each point $\mathbf{x} \in E$ has a
neighborhood in which $\mathbf{f}$ is 1-1. 
This may be expressed by saying that $\mathbf{f}$ is
\emph{locally} one-to-one in $E$. 
But $\mathbf{f}$ need not be 1-1 in $E$ under these circumstances.
for an example, see Exercise \ref{ex:9.17}.