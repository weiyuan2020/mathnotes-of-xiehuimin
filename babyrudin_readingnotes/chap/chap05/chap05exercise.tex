% chap05exercise

\section*{Exercise}


\begin{myExercise}
    \label{ex:5.1}
    Let $f$ be defined for all real $x$, and suppose that 
    \begin{equation*}
        \left| f(x) - f(y) \right| \leq (x-y)^2
    \end{equation*}
    for all real $x$ and $y$.
    Prove that $f$ is constant.
\end{myExercise}


\begin{myExercise}
    \label{ex:5.2}
    Suppose $f'(x)>0$ in $(a,b)$.
    Prove that $f$ is strictly increasing in $(a,b)$,
    and let $g$ be its inverse function.
    Prove that $g$ is differentiable, 
    and that 
    \begin{equation*}
        g'(f(x)) = \frac{1}{f'(x)}
        \quad 
        (a<x<b),
    \end{equation*}
\end{myExercise}


\begin{myExercise}
    \label{ex:5.3}
    Suppose $g$ is a real function on $\R^1$, 
    with bounded derivative (say $|g'|\leq M$).
    Fix $\varepsilon > 0$, and define $f(x) = x + \varepsilon g(x)$.
    Prove that $f$ is one-to-one if $\varepsilon$ is small enough.
\end{myExercise}


\begin{myExercise}
    \label{ex:5.4}
    If 
    \begin{equation*}
        C_0 + \frac{C_1}{2} + \cdots + \frac{C_{n-1}}{n} + \frac{C_n}{n+1} = 0,
    \end{equation*}
    where $C_0,...,C_n$ are real constants,
    prove that the equation 
    \begin{equation*}
        C_0 + C_1 x + \cdots + C_{n-1} x^{n-1} + C_n x^n = 0
    \end{equation*}
    has at least one real root between 0 and 1.
\end{myExercise}


\begin{myExercise}
    \label{ex:5.5}
    Suppose $f$ is defined and differentiable for every $x>0$,
    and $f' \rightarrow 0$ as $x\rightarrow + \infty$.
\end{myExercise}


\begin{myExercise}
    \label{ex:5.6}
    Suppose 
    \begin{enumerate}[(a)]
        \item $f$ is continuous for $x \geq 0$,
        \item $f'(x)$ exists for $x>0$,
        \item $f(0) = 0$,
        \item $f'$ is monotonically increasing.
    \end{enumerate}
    Put 
    \begin{equation*}
        g(x) = \frac{f(x)}{x}
        \quad 
        (x>0)
    \end{equation*}
    and prove that $g$ is monotonically increasing.
\end{myExercise}


\begin{myExercise}
    \label{ex:5.7}
    Suppose $f'(x), g'(x)$ exists, 
    $g'(x) \neq 0$, and $f(x) = g(x) = 0$.
    Prove that 
    \begin{equation*}
        \lim_{t \to x} \frac{f(t)}{g(t)} = \frac{f'(x)}{g'(x)},
    \end{equation*}
    (This holds also for complex functions.)
\end{myExercise}


\begin{myExercise}
    \label{ex:5.8}
    Suppose $f'$ is continuous on $[a,b]$ and $\varepsilon >0$ such that 
    \begin{equation*}
        \left| \frac{f(t)-f(x)}{t-x} -f'(x) \right| < \varepsilon
    \end{equation*}
    whenever $0 < |t - x| < \delta$, $a \leq x \leq b$, $a \leq t \leq b$. (This could be expressed by saying that $f$ is 
    \myKeywordblue{uniformly differentiable} on $[a, b]$ 
    if $f'$ is continuous on $[a, b]$.) 
    Does this hold for vector-valued functions too?
\end{myExercise}


\begin{myExercise}
    \label{ex:5.9}
    Let $f$ be a continuous real function on $\R^1$, 
    of which it is known that $f'(x)$ exists for all $x \neq 0$ 
    and that $f'(x) \rightarrow 3$  as $x \rightarrow 0$. 
    Does it follow that $f'(0)$ exists?
\end{myExercise}


\begin{myExercise}
    \label{ex:5.10}
    Suppose $f$ and $g$ are complex differentiable function on $(0,1)$,
    $f(x) \rightarrow 0$,
    $g(x) \rightarrow 0$,
    $f'(x) \rightarrow A$,
    $g'(x) \rightarrow B$ 
    as $x \rightarrow 0$,
    where $A$ and $B$ are complex numbers, $B \neq 0$.
    Prove that 
    \begin{equation*}
        \lim_{x \to 0} \frac{f(x)}{g(x)} = \frac{A}{B}.
    \end{equation*}
    Compare with Example \ref{ex:5.18}.

    \emph{Hint}: 
    \begin{equation*}
        \frac{f(x)}{g(x)} = \left\{ \frac{f(x)}{x} - A \right\}\cdot \frac{x}{g(x)} + A \cdot \frac{x}{g(x)}
    \end{equation*}
    Apply Theorem \ref{thm:5.13} to the real and imaginary parts of 
    $f(x)/x$ and $g(x)/x$.
\end{myExercise}


\begin{myExercise}
    \label{ex:5.11}
    Suppose $f$ is defined in a neighborhood of $x$, 
    and suppose $f''(x)$ does exists.
    Show that 
    \begin{equation*}
        \lim_{h \to 0} \frac{f(x+h)+f(x-h)-2f(x)}{h^2} = f''(x)
    \end{equation*}
    Show by an example that the limit may exist even if $f''(x)$ does not.
    
    \emph{Hint}: Use Theorem \ref{thm:5.13}.
\end{myExercise}

\begin{myExercise}
    \label{ex:5.12}
    If $f(x) = |x|^3$, compute $f'(x), f''(x)$ for all real $x$, 
    and show that $f^{(3)}(0)$ does not exist.
\end{myExercise}

\mySolve
$f(x) = |x|^3$.
\begin{align*}
    f(x)  &= 
    \left\{ 
        \begin{array}{ll}
            x^3 & (x\geq 0), \\
            -x^3 & (x < 0). \\
        \end{array}
     \right. \\
    f'(x) &= \left\{ 
        \begin{array}{ll}
            3x^2 & (x\geq 0), \\
            -3x^2 & (x < 0). \\
        \end{array}
     \right. \\
     f''(x) &= \left\{ 
        \begin{array}{ll}
            6x & (x\geq 0), \\
            -6x & (x < 0). \\
        \end{array}
     \right. 
\end{align*}
$f^{(3)}(x-) = -6$,
$f^{(3)}(x+) = +6$,
$f^{(3)}(x)$ doesn't exist.


\begin{myExercise}
    \label{ex:5.13}
    Suppose $a$ and $c$ are real numbers, $c>0$,
    and $f$ is defined on $[-1,1]$ by 
    \begin{equation*}
        f(x) = \left\{  
            \begin{array}{ll}
                x^a \sin (|x|^{-c}) & (\text{if } x \neq 0), \\
                0                   & (\text{if } x =    0). \\
            \end{array}
        \right.
    \end{equation*}
    Prove the following statements:
    \begin{enumerate}[(a)]
        \item $f$ is continuous if and only if $a>0$.
        \item $f'(0)$ exists if and only if $a>1$.
        \item $f'$ is bounded if and only of $a\geq 1+c$.
        \item $f'$ is continuous if and only if $a>1+c$.
        \item $f''(0)$ exists if and only if $a>2+c$.
        \item $f''$ is bounded if and only if $a\geq 2+2c$.
        \item $f''$ is continuous if and only if $a>2+2c$.
    \end{enumerate}
\end{myExercise}


\begin{myExercise}
    \label{ex:5.14}
    Let $f$ be a differentiable real function defined in $(a,b)$.
    Prove that $f$ is convex if and only if $f'$ is monotonically increasing.
    Assume next that $f''(x)$ exists for every $x \in (a,b)$,
    and prove that $f$ is convex if and only if $f''(x) \geq 0$ for all $x \in (a,b)$.
\end{myExercise}


\begin{myExercise}
    \label{ex:5.15}
    Suppose $a \in \R^1$, $f$ is twice-differentiable real function on $(a, \infty)$, and $M_0, M_1, M_2$ are the least upper bounds of $|f(x)|$, $|f'(x)|$, $|f''(x)|$, respectively, on $(a, \infty)$.
    Prove that 
    \begin{equation*}
        M_1^2 \leq 4 M_0 M_2 .
    \end{equation*}
    \emph{Hint}: If $h>0$, Taylor's theorem shows that 
    \begin{equation*}
        f'(x) = \frac{1}{2h}\left[ f(x+2h) - f(x) \right] - h f''(\xi)
    \end{equation*}
    for some $\xi \in (x, x+2h)$.
    Hence
    \begin{equation*}
        \left| f'(x) \right| \leq h M_2 + \frac{M_0}{h}.
    \end{equation*}
    To show that $M_1^2 = 4M_0 M_2$ can actually happen,
    take $a = -1$, define 
    \begin{equation*}
        f(x) = \left\{ 
            \begin{array}{ll}
                2x^2-1 & (-1<x<0), \\
                \frac{x^2-1}{x^2+1} & (0\leq x < \infty). \\
            \end{array}
        \right.
    \end{equation*}
    and show that 
    $M_0 = 1$,
    $M_1 = 4$,
    $M_2 = 4$.
    
    Does $M_1^2 \leq 4 M_0 M_2$ hold for vector-valued functions too?
\end{myExercise}


\begin{myExercise}
    \label{ex:5.16}
    Suppose $f$ is twice-differentiable on $(0, \infty)$, 
    $f''$ is bounded on $(0, \infty)$, 
    and $f(x) \rightarrow 0$
    as $x \rightarrow \infty$. 
    Prove that $f'(x) \rightarrow 0$ 
    as $x \rightarrow \infty$.

    \emph{Hint}: Let $a \rightarrow \infty$ in Exercise \ref{ex:5.15}.
\end{myExercise}


\begin{myExercise}
    \label{ex:5.17}
    Suppose $f$ is a real, three times differentiable function on $[-1, 1]$, such that
    \begin{equation*}
        f(-1) =0,   \quad
        f(0) =0,    \quad
        f(1) = 1,   \quad
        f'(0) = 0.
    \end{equation*}
    Prove that $f^{(3)} (x) \geq 3$ for some $x \in (-1, 1)$. 

    Note that equality holds for $\frac{1}{2}(x^3 + x^2)$. 
    
    \emph{Hint}: Use Theorem \ref{thm:5.15}, 
    with $\alpha = 0$ and $\beta = \pm 1$, 
    to show that there exist $s \in (0, 1)$ and $t \in (-1, 0)$ 
    such that 
    \begin{equation*}
        f^{(3)}(s) + f^{(3)}(t) = 6.
    \end{equation*}
\end{myExercise}


\begin{myExercise}
    \label{ex:5.18}
    Suppose $f$ is a real function on $[a, b]$, 
    $n$ is a positive integer, 
    and $f^{(n-1)}$ exists for every $t \in [a, b]$. 
    Let $\alpha, \beta$, and $P$ be as in Taylor's theorem (\ref{thm:5.15}). Define
    \begin{equation*}
        Q(t) = \frac{f(t)-f(\beta)}{t-\beta}
    \end{equation*}
    for $t \in [a,b]$, $t \neq \beta$, differentiate 
    \begin{equation*}
        f(t) - f(\beta) = (t - \beta) Q(t)
    \end{equation*}
    $n-1$ times at $t = \alpha$, and derive the following version of Taylor's theorem:
    \begin{equation*}
        f(\beta) = P(\beta) + \frac{Q^{(n-1)}(\alpha)}{(n-1)!}(\beta - \alpha)^n .
    \end{equation*}
\end{myExercise}


\begin{myExercise}
    \label{ex:5.19}
    Suppose $f$ is defined in $(-1,1)$ and $f'(0)$ exists. 
    Suppose $-1 < \alpha_n < \beta_n < 1$,
    $\alpha_n \rightarrow 0$, and 
    $\beta \rightarrow 0$ as 
    $n \rightarrow \infty$. 
    Define the difference quotients
    \begin{equation*}
        D_n = \frac{f(\beta_n)-f(\alpha_n)}{\beta_n-\alpha_n} .
    \end{equation*}
    Prove the following statements:
    \begin{enumerate}[(a)]
        \item If $\alpha_n < 0 < \beta_n$, then $\lim D_n = f'(0)$.
        \item If $0 < \alpha_n < \beta_n$ and $\{\beta_n/(\beta_n-\alpha_n)\}$ is bounded, then $\lim D_n = f'(0)$.
        \item If $f'$ is continuous in $(-1,1)$, then $\lim D_n = f'(0)$.
    \end{enumerate}
    Give an example in which $f$ is differentiable in $(-1, 1)$ 
    (but $f'$ is not continuous at 0) and in which $\alpha_n$ , 
    $\beta_n$ tend to 0 in such a way that $\lim D_n$ exists 
    but is different from $f'(0)$.
\end{myExercise}


\begin{myExercise}
    \label{ex:5.20}
    Formulate and prove an inequality which follows from Taylor's theorem and which remains valid for vector-valued functions.
\end{myExercise}


\begin{myExercise}
    \label{ex:5.21}
    Let $E$ be a closed subset of $\R^1$. 
    We saw in Exercise \ref{ex:4.22}, 
    that there is a real continuous function $f$ on $\R^1$ whose zero set is $E$. 
    Is it possible, for each closed set $E$, to find such an $f$ which is differentiable on $\R^1$, 
    or one which is $n$ times differentiable, 
    or even one which has derivatives of all orders on $\R^1$?
\end{myExercise}


\begin{myExercise}
    \label{ex:5.22}
\end{myExercise}


\begin{myExercise}
    \label{ex:5.23}
\end{myExercise}



\begin{myExercise}
    \label{ex:5.24}
\end{myExercise}


\begin{myExercise}
    \label{ex:5.25}
\end{myExercise}


\begin{myExercise}
    \label{ex:5.26}
\end{myExercise}


\begin{myExercise}
    \label{ex:5.27}
\end{myExercise}


\begin{myExercise}
    \label{ex:5.28}
\end{myExercise}


\begin{myExercise}
    \label{ex:5.29}
\end{myExercise}

