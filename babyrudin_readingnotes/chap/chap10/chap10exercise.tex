% chap10exercise

\section{Exercises}


\begin{myexercise}
    \label{ex:10.1}
    Let $H$ be a compact convex set in $\R^k$, with nonempty interior. 
    Let $f \in \mathscr{C}(H)$, put $f(\mathbf{x}) = 0$ in the complement of $H$, 
    and define $\int_H f$ as in Definition \ref{mydef:10.3}.

    Prove that $\int_H f$ is independent of the order in which the $k$ integrations are carried out.

    \emph{Hint:} Approximate $f$ by functions that are continuous on $\R^k$ and whose supports are in $H$, as was done in Example \ref{newexample:10.47}.
\end{myexercise}


\begin{myexercise}
    \label{ex:10.2}
    For $i = 1, 2, 3, ...$ , let $\phi_i \in \mathscr{C}(\R^1)$ have support in $(2^{-i} , 2^{1-i})$, such that $\int \phi_i = 1$.
    Put
    \begin{equation*}
        f(x,y) = \sum_{i=1}^{\infty}
        \left[ 
            \phi_{i}(x) -
            \phi_{i+1}(x)
         \right] \phi_i (y)
    \end{equation*}
    Then $f$ has compact support in $\R^2$, 
    $f$ is continuous except at $(0,0)$,
    and 
    \begin{equation*}
        \int \d y \int f(x,y) \d x = 0
        \quad \text{ but } \quad
        \int \d x \int f(x,y) \d y = 1.
    \end{equation*}
    Observe that $f$ is unbounded in every neighborhood of $(0, 0)$.
\end{myexercise}


\begin{myexercise}
    \label{ex:10.3}
    \begin{asparaenum}[(a)]
        \item If $F$ is as in Theorem \ref{thm:10.7}, put 
        $\mathbf{A} = \mathbf{F}'(0)$,
        $\mathbf{F_{1}(x)} = \mathbf{A^{-1}F(x)}$.
        Then $\mathbf{F'_1(0)}=I$.
        Show that 
        \begin{equation*}
            \mathbf{F_1(x) = G_n \circ G_{n-1} \circ \cdots \circ G_1(x)}
        \end{equation*}
        in some neighborhood of $\mathbf{0}$,
        for certain primitive mappings $\mathbf{G_n \circ \cdots \circ G_1(x)}$.
        This gives another version of Theorem \ref{thm:10.7}:
        \begin{equation*}
            \mathbf{F(x) = F'(0) G_n \circ G_{n-1} \circ \cdots \circ G_1(x)}.
        \end{equation*}
        \item Prove that the mapping $(x, y) > (y, x)$ of $\R^2$ onto $\R^2$ is not the composition of any two primitive mappings, in any neighborhood of the origin. 
        (This shows that the flips $B_1$ cannot be omitted from the statement of Theorem \ref{thm:10.7}.)
    \end{asparaenum}
\end{myexercise}