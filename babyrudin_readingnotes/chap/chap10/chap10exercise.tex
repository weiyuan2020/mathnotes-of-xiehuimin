% chap10exercise

\section{Exercises}


\begin{myexercise}
    \label{ex:10.1}
    Let $H$ be a compact convex set in $\R^k$, with nonempty interior. 
    Let $f \in \mathscr{C}(H)$, put $f(\mathbf{x}) = 0$ in the complement of $H$, 
    and define $\int_H f$ as in Definition \ref{mydef:10.3}.

    Prove that $\int_H f$ is independent of the order in which the $k$ integrations are carried out.

    \emph{Hint:} Approximate $f$ by functions that are continuous on $\R^k$ and whose supports are in $H$, as was done in Example \ref{newexample:10.47}.
\end{myexercise}


\begin{myexercise}
    \label{ex:10.2}
    For $i = 1, 2, 3, ...$ , let $\phi_i \in \mathscr{C}(\R^1)$ have support in $(2^{-i} , 2^{1-i})$, such that $\int \phi_i = 1$.
    Put
    \begin{equation*}
        f(x,y) = \sum_{i=1}^{\infty}
        \left[ 
            \phi_{i}(x) -
            \phi_{i+1}(x)
         \right] \phi_i (y)
    \end{equation*}
    Then $f$ has compact support in $\R^2$, 
    $f$ is continuous except at $(0,0)$,
    and 
    \begin{equation*}
        \int \d y \int f(x,y) \d x = 0
        \quad \text{ but } \quad
        \int \d x \int f(x,y) \d y = 1.
    \end{equation*}
    Observe that $f$ is unbounded in every neighborhood of $(0, 0)$.
\end{myexercise}


\begin{myexercise}
    \label{ex:10.3}
    \begin{asparaenum}[(a)]
        \item If $F$ is as in Theorem \ref{thm:10.7}, put 
        $\mathbf{A} = \mathbf{F}'(0)$,
        $\mathbf{F_{1}(x)} = \mathbf{A^{-1}F(x)}$.
        Then $\mathbf{F'_1(0)}=I$.
        Show that 
        \begin{equation*}
            \mathbf{F_1(x) = G_n \circ G_{n-1} \circ \cdots \circ G_1(x)}
        \end{equation*}
        in some neighborhood of $\mathbf{0}$,
        for certain primitive mappings $\mathbf{G_n \circ \cdots \circ G_1(x)}$.
        This gives another version of Theorem \ref{thm:10.7}:
        \begin{equation*}
            \mathbf{F(x) = F'(0) G_n \circ G_{n-1} \circ \cdots \circ G_1(x)}.
        \end{equation*}
        \item Prove that the mapping $(x, y) > (y, x)$ of $\R^2$ onto $\R^2$ is not the composition of any two primitive mappings, in any neighborhood of the origin. 
        (This shows that the flips $B_1$ cannot be omitted from the statement of Theorem \ref{thm:10.7}.)
    \end{asparaenum}
\end{myexercise}

\begin{myexercise}    
    \label{ex:10.4}
    For $(x,y) \in \R^2$, define
    \begin{equation*}
        \mathbf{F}(x,y) = (e^x \cos y - 1, e^x \sin y).
    \end{equation*}
    Prove that $\mathbf{F = G_2 \circ G_1}$, where 
    \begin{align*}
        \mathbf{G}_1 (x,y) &= (e^x \cos y - 1, y) \\
        \mathbf{G}_2 (u,v) &= (u, (1 + u) \tan v) 
    \end{align*}
    are primitive in some neighborhood of $(0, 0)$.

    Compute the Jacobians of $\mathbf{G_1, G_2, F}$ at $(0, 0)$. 
    Define
    \begin{equation*}
        \mathbf{H}_2 (x,y) = (x, e^x \sin y)
    \end{equation*}
    and find 
    \begin{equation*}
        \mathbf{H}_1 (u,v) = (h(u,v), v)
    \end{equation*}
    so that $\mathbf{F = H_1 \circ H_2}$ is some neighborhood of $(0,0)$.
\end{myexercise}


\begin{myexercise}    
    \label{ex:10.5}
    Formulate and prove an analogue of Theorem \ref{thm:10.8}, 
    in which $K$ is a compact subset of an arbitrary metric space. 
    (Replace the functions $\phi_i$ that occur in the
    proof of Theorem \ref{thm:10.8} 
    by functions of the type constructed in Exercise \ref{ex:4.22})
\end{myexercise}


\begin{myexercise}    
    \label{ex:10.6}
    Strengthen the conclusion of Theorem \ref{thm:10.8} by showing that the functions $\psi_i$ can be made differentiable, and even infinitely differentiable. 
    (Use Exercise \ref{ex:8.1} in the construction of the auxiliary functions $\phi_i$.)
\end{myexercise}


\begin{myexercise}    
    \label{ex:10.7}
    \begin{enumerate}[(a)]
        \item Show that the simplex $Q^k$ is the smallest convex subset of $\R^k$ that contains $\mathbf{0},\mathbf{e}_1,\dots,\mathbf{e}_k$.
        \item Show that affine mappings take convex sets to convex sets.
    \end{enumerate}
\end{myexercise}


\begin{myexercise}    
    \label{ex:10.8}
    Let $H$ be the parallelogram in $\R^2$ whose vertices are $(1, 1), (3, 2), (4, 5), (2, 4)$.
    Find the affine map $T$ which sends $(0, 0)$ to $(1, 1)$, $(1, 0)$ to $(3, 2)$, $(0, 1)$ to $(2, 4)$.
    Show that $J_T = 5$. 
    Use $T$ to convert the integral
    \begin{equation*}
        \alpha = \int_H e^{x-y} \d x \d y
    \end{equation*}
    to an integral over $I^2$ and thus compute $\alpha$.
\end{myexercise}


\begin{myexercise}    
    \label{ex:10.9}
    Define $(x, y) = T(r, \theta)$ on the rectangle
    \begin{equation*}
        0 \leq r \leq a, 
        \quad 
        0 \leq \theta \leq 2\pi
    \end{equation*}
    by the equations
    \begin{equation*}
        x = r \cos \theta , \quad 
        y = r \sin \theta .
    \end{equation*}
    Show that $T$ maps this rectangle onto the closed disc $D$ with center at $(0, 0)$ and radius $a$, 
    that $T$ is one-to-one in the interior of the rectangle, and that $J_T(r, \theta) = r$. 
    If $f \in \mathscr{C}(D)$, prove the formula for integration in polar coordinates:
    \begin{equation*}
        \int_D f(x,y) \d x \d y = 
        \int_{0}^{a} \int_{0}^{2\pi} f(T(r,\theta)) r \d r \d \theta .
    \end{equation*}
    \emph{Hint:} Let Do be the interior of $D$, minus the interval from $(0, 0)$ to $(0, a)$. 
    As it stands, Theorem \ref{thm:10.9} applies to continuous functions $f$ whose support lies in $D_0$. 
    To remove this restriction, proceed as in Example \ref{newexample:10.4}.
\end{myexercise}



\begin{myexercise}    
    \label{ex:10.10}
    Let $a \rightarrow \infty$ in \ref{ex:11.9} and prove that 
    \begin{equation*}
        \int_{\R^2} f(x,y) \d x \d y = 
        \int_{0}^{\infty} \int_{0}^{2\pi} f(T(r,\theta)) r \d r \d \theta ,
    \end{equation*}
    for continuous functions f that decrease sufficiently rapidly as $|x | + | y | \rightarrow \infty$.
    (Find a more precise formulation.) 
    Apply this to
    \begin{equation*}
        f(x, y) = \exp (-x^2 - y^2)
    \end{equation*}
    to derive formula \eqref{eq:8.101}.
\end{myexercise}


\begin{myexercise}    
    \label{ex:10.11}
    Define $(u,v)=T(s,t)$ on the strip
    \begin{equation*}
        0<s<\infty , \quad
        0<t<1
    \end{equation*}
    by setting $u = s - st$, $v = st$. 
    Show that $T$ is a 1-1 mapping of the strip onto the positive quadrant $Q$ in $\R^2$.
    Show that $J_T(s, t) = s$.

    For x > 0, y > 0, integrate
    \begin{equation*}
        u^{x-1} e^{-u} v^{y-1} e^{-v} 
    \end{equation*}
    over $Q$, use Theorem \ref{thm:10.9} to convert the integral to one over the strip, and derive formula \eqref{eq:8.96} in this way.
    (For this application, Theorem \ref{thm:10.9} has to be extended so as to cover certain improper integrals. 
    Provide this extension.)
\end{myexercise}


\begin{myexercise}    
    \label{ex:10.12}
    Let $I^k$ be the set of all $\mathbf{u} = (u_1, ... , u_k) \in \R^k$ with $0 \leq u_i \leq 1$ for all $i$; 
    let $Q^k$ be the set of all $\mathbf{x} = (x_1, ... , x_k) \in  \R^k$ with $x_i \geq 0, \sum x_i \leq 1$. 
    ($I^k$ is the unit cube; 
    $Q^k$ is the standard simplex in $\R^k$.) 
    Define $\mathbf{x} = T(\mathbf{u})$ by
    \begin{align*}
        x_1 &= u_1 \\
        x_2 &= (1-u_1)u_2 \\
        ... & \dots
        x_k &= (1-u_1)\cdots(1-u_{k-1})u_k. 
    \end{align*}
    Show that 
    \begin{equation*}
        \sum_{i=1}^{k} x_i = 1 - \prod_{i=1}^{k} (1-u_i) .
    \end{equation*}

    Show that $T$ maps $I^k$ onto $Q^k$, 
    that $T$ is 1-1 in the interior of $I^k$, 
    and that its inverse $S$ is defined in the interior of $Q^k$ by $u_1 = x_1$ and
    \begin{equation*}
        u_i = \frac{x_i}{1-x_1-\cdots-x_{i-1}}
    \end{equation*}
    for $i=2,\dots,k$.
    Show that 
    \begin{equation*}
        J_T(\mathbf{u}) = 
        (1-u_1)^{k-1}
        (1-u_2)^{k-2}
        \cdots 
        (1-u_{k-1}),
    \end{equation*}
    and 
    \begin{equation*}
        J_S(\mathbf{x}) = 
        \left[ 
            (1-x_1)
            (1-x_1-x_2)
            \cdots
            (1-x_1-\cdots-x_{k-1})
         \right]^{-1} .
    \end{equation*}
\end{myexercise}


\begin{myexercise}    
    \label{ex:10.13}
    Let $r_1,\dots,r_k$ be nonnegative integers, and prove that
    \begin{equation*}
        \int_{Q^k} 
        x_1^{r_1} 
        \cdots 
        x_k^{r_k}
        \d x = 
        \frac{r_1!\cdots r_k!}{(k+r_1+\cdots+r_k)!}
    \end{equation*}
    \emph{Hint:} Use Exercise \ref{ex:11.12}, Theorem \ref{thm:10.9} and \ref{thm:8.20}.

    Note that the special case $r_1 = \cdots = r_k = 0$ 
    shows that the volume of $Q^k$ is $1/k!$.
\end{myexercise}


\begin{myexercise}    
    \label{ex:10.14}
    Prove formula \eqref{eq:10.46}.
\end{myexercise}


\begin{myexercise}    
    \label{ex:10.15}
    If $\omega$ and $\lambda$ are $k$- and $m$-forms, respectively, prove that
    \begin{equation*}
        \omega \wedge \lambda = 
        (-1)^{km} \lambda \wedge \omega .
    \end{equation*}
\end{myexercise}


\begin{myexercise}    
    \label{ex:10.16}
    If $k \geq 2$ and $\delta = [\mathbf{p}_0, \mathbf{p}_1, ... , \mathbf{p}_t]$ is an oriented affine $k$-simplex, 
    prove that $\partial^2 \sigma = 0$, 
    directly from the definition of the boundary operator $\partial$. 
    Deduce from this that $\partial^2 \Psi = 0$ for every chain $\Psi$.
    
    \emph{Hint:} For orientation, do it first for $k= 2, k = 3$. 
    In general, if $i <j$, let $\sigma_{ij}$ be the $(k - 2)$-simplex obtained by deleting $\mathbf{p}_i$ and $\mathbf{p}_j$ from $u$. 
    Show that each $\sigma_{ij}$ occurs twice in $\partial^2 \sigma$, with opposite sign.
\end{myexercise}


\begin{myexercise}    
    \label{ex:10.17}
    Put $J^2 = \tau_1 + \tau_2$, where 
    \begin{equation*}
        \tau_1 =  \left[ \mathbf{0,e_1,e_1+e_2} \right], 
        \quad 
        \tau_2 = -\left[ \mathbf{0,e_2,e_2+e_1} \right].
    \end{equation*}
    Explain why it is reasonable to call $J^2$ the positively oriented unit square in $\R^2$ .
    Show that $\partial J^2$ is the sum of 4 oriented affine 1-simplexes. Find these. What is $\partial (\tau_1 - \tau_2)$?
\end{myexercise}


\begin{myexercise}    
    \label{ex:10.18}
    Consider the oriented affine 3-simplex 
    \begin{equation*}
        \sigma_1 =  \left[ \mathbf{0,e_1,e_1+e_2,e_1+e_2+e_3} \right]
    \end{equation*}
    in $\R^3$.
    Show that $\sigma_1$
    (regarded as a linear transformation) has determinant 1.
    Thus $\sigma_1$ is positively oriented.

    Let $\sigma_2 , ... , \sigma_6$ be five other oriented 3-simplexes, obtained as follows: 
    There are five permutations $(i_1, i_2, i_3)$ of $(1, 2, 3)$, distinct from $(1, 2, 3)$. 
    Associate with each $(i_1, i_2, i_3)$ the simplex 
    \begin{equation*}
        s(i_1, i_2, i_3) \left[ \mathbf{0,e_{i_1},e_1+e_2} \right]
    \end{equation*}
    where $s$ is the sign that occurs in the definition of the determinant. 
    (This is how $\tau_2$ was obtained from $\tau_1$ in Exercise \ref{ex:11.17}.)
    
    Show that $\sigma_2, \dots , \sigma_6$ are positively oriented.
    
    Put $J^3 = \sigma_1 + \dots + \sigma_6$. 
    Then $J^3$ may be called the positively oriented unit cube in $\R^3$. 

    Show that $\partial J^3$ is the sum of 12 oriented affine 2-simplexes. 
    (These 12 triangles cover the surface of the unit cube $I^3$,)

    Show that $\mathbf{x} = (\mathbf{x}_1, \mathbf{x}_2, \mathbf{x}_3)$ is in the range of $\sigma_1$ if and only if $0 \leq x_3 \leq x_2 \leq x_1 \leq 1$,

    Show that the ranges of $\sigma_1, ... , \sigma_6$ have disjoint interiors, and that their union covers $I^3$. 
    (Compare with Exercise \ref{ex:11.13}; note that $3! = 6$.)
\end{myexercise}


\begin{myexercise}    
    \label{ex:10.19}
    Let $J^2$ and $J^3$ be as in Exercise \ref{ex:11.17} and \ref{ex:11.18}.
    Define
    \begin{align*}
        B_{01}(u,v) = (0,u,v), & B_{11}(u,v) = (1,u,v), \\
        B_{02}(u,v) = (u,0,v), & B_{12}(u,v) = (u,1,v), \\
        B_{03}(u,v) = (u,v,0), & B_{13}(u,v) = (u,v,1), \\
    \end{align*}
    These are affine, and map $\R^2$ into $\R^3$.

    Put $\beta_{ri} = B_{ri}(J^2)$, for $r = 0,1$, $i=1,2,3$.
    Each $\beta_{ri}$ is an affine-oriented 2-chain.
    (See Sec. \ref{mydef:10.30}.)
    Verify that 
    \begin{equation*}
        \partial J^3 = \sum_{i=1}^{3}(-1)^i (\beta_{0i}-\beta_{1i}),
    \end{equation*}
    in agreement with Exercise \ref{ex:10.18}.
\end{myexercise}



\begin{myexercise}    
    \label{ex:10.20}
    State conditions under which the formula
    \begin{equation*}
        \int_{\Phi} f \d \omega = 
        \int_{\partial\Phi} f \omega -
        \int_{\Phi} (\d f) \wedge \omega 
    \end{equation*}
    is valid, and show that it generalizes the formula for integration by parts.

    \emph{Hint:} $\d(f \omega) = (\d f) \wedge \omega + f d\omega$.
\end{myexercise}


\begin{myexercise}    
    \label{ex:10.21}
    As in Example \ref{newexample:10.36}, consider the 1-form
    \begin{equation*}
        \eta = \frac{x \d y - y \d x}{x^2+y^2}
    \end{equation*}
    in $\R^2 - \{\mathbf{0}\}$.
    \begin{asparaenum}[(a)]
        \item Carry out the computation that leads to formula \eqref{eq:10.113}, and prove that $\d \eta = 0$.
        \item Let $\gamma(t) = (r \cos t, r \sin t)$, for some $r > 0$, 
        and let $r$ be a $\mathscr{C}''$-curve in $\R^2 - \{\mathbf{0}\}$, 
        with parameter interval $[0, 2\pi]$, with $\Gamma(0) = \Gamma(2\pi)$, such that the intervals $[\gamma(t), \Gamma(t)]$ do not contain $\mathbf{0}$ for any $t \in $$[0, 2\pi]$. Prove that
        \begin{equation*}
            \int_{\Gamma} \eta = 2 \pi .
        \end{equation*}
        
        \emph{Hint:} For $0 \leq t \leq 2\pi$, $0 \leq u \leq 1$, define 
        \begin{equation*}
            \Phi(t,u)=(1-u)\Gamma(t)+u\gamma(t).
        \end{equation*}
        Then $\Phi$ is a 2-surface in $\R^2 - \{\mathbf{0}\}$ whose parameter domain is the indicated rectangle. 
        Because of cancellations (as in Example \ref{newexample:10.32}),
        \begin{equation*}
            \partial \Phi = \Gamma - \gamma .
        \end{equation*}
        Use Stokes' theorem to deduce that 
        \begin{equation*}
            \int_{\Gamma} \eta = 
            \int_{\gamma} \eta 
        \end{equation*}
        because $\d \eta = 0$.
        \item Take $\Gamma(t)=(a \cos t, b \sin t)$ where $a>0,b>0$ are fixed. Use part (b) to show that 
        \begin{equation*}
            \int_{0}^{2\pi} \frac{ab}{a^2\cos^2 t + b^2 \sin^2 t} \d t = 2 \pi .
        \end{equation*}
        \item Show that 
        \begin{equation*}
            \eta = \d \left( \arctan\frac{y}{x} \right)
        \end{equation*}
        in any convex open set in which $x \neq 0$, and that
        \begin{equation*}
            \eta = \d \left( -\arctan\frac{x}{y} \right)
        \end{equation*}
        in any convex open set in which $y \neq 0$.
        \item Show that (b) can be derived from (d).
        \item If $\Gamma$ is any closed $\mathscr{C}'$-curve in $\R^2 - \{\mathbf{0}\}$, prove that
        \begin{equation*}
            \frac{1}{2\pi} \int_{\Gamma} \eta = \Ind (\Gamma).
        \end{equation*}
        (See Exercise \ref{ex:8.23} for the definition of the index of a curve.)
    \end{asparaenum}
\end{myexercise}


\begin{myexercise}   
    \label{ex:10.22}
    As in Example \ref{newexample:10.37}, define $\zeta$ in $\R^3 - \{\mathbf{0}\}$ by
    \begin{equation*}
        \zeta = 
        \frac{
            x \d y \wedge \d z + 
            y \d z \wedge \d x + 
            z \d x \wedge \d y 
        }{r^3}
    \end{equation*}
    where $r = \left( x^2+y^2+z^2 \right)^{1/2}$,
    let $D$ be the rectangle given by 
    $0 \leq u \leq \pi$,
    $0 \leq v \leq \pi$,
    and let $\sum$ be the 2-surface in $\R^3$,
    with parameter domain in $D$, given by 
    \begin{equation*}
        x = \sin u \cos v, \quad
        y = \sin u \sin v, \quad
        z = \cos u .
    \end{equation*}
    \begin{asparaenum}[(a)]
        \item Prove that $\d \zeta = 0$ in $\R^3 - \{\mathbf{0}\}$.
        \item Let $S$ denote the restriction of $\sum$ to a parameter domain $E \subset D$. 
        Prove that
        \begin{equation*}
            \int_S \zeta = \int_E \sin u \d u \d v = A(S),
        \end{equation*}
        where $A$ denotes area, as in Sec. \ref{thm:10.43}.
        Note that this contains \eqref{eq:10.115} as a special case.
        \item Suppose $g, h_1, h_2, h_3$, are $\mathscr{C}''$-functions on $[0, 1], g > 0$. 
        Let $(x, y, z) = \Phi(s, t)$
        define a 2-surface $\Phi$, with parameter domain $I^2$, by
        \begin{equation*}
            x = g(t)h_1(s) , \quad 
            y = g(t)h_2(s) , \quad 
            z = g(t)h_3(s) .
        \end{equation*}
        Prove that 
        \begin{equation*}
            \int_{\Phi} \zeta = 0 ,
        \end{equation*}
        directly from \eqref{eq:10.35}.

        Note the shape of the range of $\Phi$: 
        For fixed $s$, $\Phi(s, t)$ runs over an interval on a line through 0. 
        The range of $\Phi$ thus lies in a ``cone'' with vertex at the origin.
        \item Let $E$ be a closed rectangle in $D$, with edges parallel to those of $D$. 
        Suppose $f \in \mathscr{C}''(D), f> 0$. 
        Let $\Omega$ be the 2-surface with parameter domain $E$, defined by
        \begin{equation*}
            \Omega(u,v) = f(u,v)\sum(u,v).
        \end{equation*}
        Define $S$ as in (b) and prove that
        \begin{equation*}
            \int_{\Phi} \zeta =  \int_{S} \zeta = A(S).
        \end{equation*}
        (Since $S$ is the ``radial projection'' of $n$ into the unit sphere, this result makes it reasonable to call $\int_n \zeta$ the ``solid angle'' subtended by the range of $\Omega$ at the origin.) 
        
        \emph{Hint:} Consider the 3-surface $\Psi$ given by
        \begin{equation*}
            \Psi(t,u,v) = \left[ 1-t+t f(u,v) \right] \sum (u,v) ,
        \end{equation*}
        where $(u, v) \in E$, $0 \leq t \leq 1$. 
        For fixed $v$, the mapping $(t, u) \rightarrow \Psi(t, u, v)$ is a 2-surface $\Psi$ to which (c) can be applied to show that $\int_{\Phi} \zeta = 0$. 
        The same thing holds when $u$ is fixed. 
        By (a) and Stokes' theorem,
        \begin{equation*}
            \int_{\partial \Psi} \zeta = 
            \int_{\Psi} \d \zeta = 0.
        \end{equation*}
        \item Put $\lambda = -(z/r)\eta$, where 
        \begin{equation*}
            \eta = \frac{x \d y - y \d x}{x^2+y^2} ,
        \end{equation*}
        as in Exercise \ref{ex:10.21}.
        Then $\lambda$ is a 1-form in the open set $V \subset \R^3$ in which $x^2 + y^2 > 0$.
        Show that $\zeta$ is \myKeywordblue{exact in} $V$ by showing that
        \begin{equation*}
            \zeta = \d \lambda .
        \end{equation*}
        \item Derive (d) from (e), without using (c).
        \emph{Hint:} To begin with, assume $0 < u < \pi$ on $E$. 
        By (e), 
        \begin{equation*}
            \int_{\Omega} \zeta = \int_{\partial \Omega} \lambda 
            \quad \text{ and } \quad 
            \int_{S} \zeta = \int_{\partial S} \lambda .
        \end{equation*} 
        Show that the two integrals of $\lambda$ are equal, by using part (d) of Exercise \ref{ex:10.21}, 
        and by noting that $z/r$ is the same at $\sum(u, v)$ as at $\Omega(u, v)$.
        \item Is $\zeta$ exact in the complement of every line through the origin?
    \end{asparaenum}
\end{myexercise}


\begin{myexercise}    
    \label{ex:10.23}
    Fix $n$. 
    Define $r_k = (x_1^2 + \cdots + x_k^2)$ for $1 \leq k \leq n$, let $E_k$ be the set of all $\mathbf{x} \in \R^n$ at which $r_k > 0$, and let $\omega_k$ be the $(k - 1)$-form defined in $E_k$ by
    \begin{equation*}
        \omega_k = (r_k)^{-k} \sum_{i=1}^{k} (-1)^{i-1} x_i 
        \d x_1 \wedge \cdots \wedge
        \d x_{i-1} \wedge
        \d x_{i+1} \wedge \cdots \wedge
        \d x_k .
    \end{equation*}
    Note that $\omega_2 = \eta$, $\omega_3 = \zeta$, in the terminology of Exercises \ref{ex:10.21} and \ref{ex:10.22}. 
    Note also that
    \begin{equation*}
        E_1 \subset
        E_2 \subset
        \cdots \subset
        E_n = \R^n - \{\mathbf{0}\} .
    \end{equation*}
    \begin{asparaenum}[(a)]
        \item Prove that $\d \omega_k = 0$ in $E_k$.
        \item For $k=2,\dots,n$, prove that $\omega_k$ is exact in $E_{k-1}$, by showing that 
        \begin{equation*}
            \omega_k = \d (f_k \omega_{k-1}) 
            = (\d f_k) \wedge \omega_{k-1} ,
        \end{equation*}
        where $f_k(\mathbf{x})=(-1)^k g_k(x_k/r_k)$ and 
        \begin{equation*}
            g_k(t)=\int_{-1}^{t}(1-s^2)^{(k-3)/2} \d s
            \quad (-1<t<1).
        \end{equation*}
        \emph{Hint:} $f_k$ satisfies the differential equations 
        \begin{equation*}
            \mathbf{x} \cdot (\nabla f_k)(\mathbf{x}) = 0
        \end{equation*}
        and 
        \begin{equation*}
            (D_k f_k)(\mathbf{x}) = \frac{(-1)^k(r_{k-1})^{k-1}}{(r_k)^k}.
        \end{equation*}
        \item Is $\omega_n$ exact in $E_n$?
        \item Note that (b) is a generalization of part (e) of Exercise \ref{ex:10.22}. 
        Try to extend some of the other assertions of Exercises \ref{ex:10.21} and \ref{ex:10.22} to $\omega_n$, for arbitrary $n$.
    \end{asparaenum}
\end{myexercise}


\begin{myexercise}
    \label{ex:10.24}
    Let $\omega = \sum a_i(\mathbf{x}) \d x_i$ be a 1-form of class $\mathscr{C}''$ in a convex open set $E \subset \R^n$.
    Assume $\d \omega = 0$ and prove that $\omega$ is exact in $E$, by completing the following outline:

    Fix $\mathbf{p} \in E$. 
    Define 
    \begin{equation*}
        f(\mathbf{x}) = \int_{[\mathbf{p,x}]} \omega 
        \quad 
        (\mathbf{x} \in E).
    \end{equation*}
    Apply Stokes' theorem to affine-oriented 2-simplexes $[\mathbf{p, x, y}]$ in $E$. 
    Deduce that
    \begin{equation*}
        f(\mathbf{y}) -
        f(\mathbf{x}) =
        \sum_{i=1}^{n} (y_i-x_i)
        \int_{0}^{1} a_i ((1-t)\mathbf{x}+t\mathbf{y}) \d \mathbf{t}
    \end{equation*}
    for 
    $\mathbf{x} \in E$ ,
    $\mathbf{y} \in E$ .
    Hence $(D_i f)(\mathbf{x}) = a_i(\mathbf{x})$.
\end{myexercise}


\begin{myexercise}    
    \label{ex:10.25}
    Assume that $\omega$ is a 1-form in an open set $E \subset \R^n$ such that
    \begin{equation*}
        \int_{\gamma} \omega = 0
    \end{equation*}
    for every closed curve $\gamma$ in $E$, of class $\mathscr{C}'$. 
    Prove that $\omega$ is exact in $E$, by imitating part of the argument sketched in Exercise \ref{ex:10.24}.
\end{myexercise}


\begin{myexercise}    
    \label{ex:10.26}
    Assume $\omega$ is a 1-form in $\R^3-\{\mathbf{0}\}$, of class $\mathscr{C}'$ and $\d \omega =0$. 
    Prove that w is exact in $\R^3-\{\mathbf{0}\}$.

    \emph{Hint:} Every closed continuously differentiable curve in $\R^3-\{\mathbf{0}\}$ is the boundary of a 2-surface in $\R^3-\{\mathbf{0}\}$. 
    Apply Stokes' theorem and Exercise \ref{ex:10.25}.
\end{myexercise}


\begin{myexercise}    
    \label{ex:10.27}
    Let $E$ be an open 3-cell in $\R^3$, with edges parallel to the coordinate axes. 
    Suppose $(a, b, c) \in E$, $f_i \in \mathscr{C}'(E)$ for $i = 1, 2, 3$,
    \begin{equation*}
        \omega = 
        f_1 \d y \wedge \d z + 
        f_2 \d z \wedge \d x + 
        f_3 \d x \wedge \d y ,
    \end{equation*}
    and assume that $\d \omega = 0$ in $E$.
    Define 
    \begin{equation*}
        \lambda = g_1 \d x + g_2 \d y
    \end{equation*}
    where 
    \begin{align*}
        g_1(x,y,z) &= \int_{c}^{z} f_2(x,y,s) \d s - \int_{b}^{y} f_3(x,t,c) \d t \\
        g_2(x,y,z) &= -\int_{c}^{z} f_1(x,y,s) \d s ,
    \end{align*}
    for $(x, y, z) \in E$. 
    Prove that $\d \lambda = \omega$ in $E$.

    Evaluate these integrals when $\omega = \zeta$ and thus find the form $\lambda$ that occurs in part (e) of Exercise \ref{ex:10.22}.
\end{myexercise}


\begin{myexercise}    
    \label{ex:10.28}
    Fix $b > a > 0$, define
    \begin{equation*}
        \Phi(r, \theta) = (r \cos \theta, r \sin \theta)
    \end{equation*}
    for $a \leq r \leq b, 0 \leq \theta \leq 2\pi$. 
    (The range of $\Phi$ is an annulus in $\R^2$.) 
    Put $\omega = x^3 \d y$,
    and compute both 
    \begin{equation*}
        \int_{\Phi} \d \omega 
        \quad \text{and} \quad 
        \int_{\partial \Phi} \omega 
    \end{equation*}
    to verify that they are equal.
\end{myexercise}


\begin{myexercise}    
    \label{ex:10.29}
    Prove the existence of a function $\alpha$ with the properties needed in the proof of Theorem \ref{thm:10.38}, 
    and prove that the resulting function $F$ is of class $\mathscr{C}'$. 
    (Both assertions become trivial if $E$ is an open cell or an open ball, 
    since $\alpha$ can then be taken to be a constant. 
    Refer to Theorem \ref{thm:9.42}.)
\end{myexercise}



\begin{myexercise}
    \label{ex:10.30}
    If $N$ is the vector given by \eqref{eq:10.135}, 
    prove that
    \begin{equation*}
        \det 
        \begin{bmatrix}
            \alpha_1 & \beta_1 & \alpha_2 \beta_3 - \alpha_3 \beta_2 \\
            \alpha_2 & \beta_2 & \alpha_3 \beta_1 - \alpha_1 \beta_3 \\
            \alpha_3 & \beta_3 & \alpha_1 \beta_2 - \alpha_2 \beta_1 \\
        \end{bmatrix} = 
        \left| \mathbf{N} \right|^2 .
    \end{equation*}
    Also, verify Eq. \eqref{eq:10.137}.
\end{myexercise}


\begin{myexercise}
    \label{ex:10.31}
    Let $E \subset \R^3$ be open, 
    suppose $g \in  \mathscr{C}''(E)$, $h \in \mathscr{C}''(E)$, and consider the vector field
    \begin{equation*}
        \mathbf{F} = g \nabla h .
    \end{equation*}
    \begin{asparaenum}[(a)]
        \item Prove that 
        \item If $\Omega$ is a closed subset of E with positively oriented boundary en (as in Theorem \ref{thm:10.51}), prove that
        \item Assume that $h$ is \myKeywordblue{harmonic} in $E$; this means that $\nabla^2 h = 0$. Take $g = 1$ and conclude that
        \item Show that Green's identities are also valid in $\R^2$.
    \end{asparaenum}
\end{myexercise}


\begin{myexercise}
    \label{ex:10.32}
\end{myexercise}


