% chap10exercise

\section{Exercises}


\begin{myexercise}
    \label{ex:10.1}
    Let $H$ be a compact convex set in $\R^k$, with nonempty interior. 
    Let $f \in \mathscr{C}(H)$, put $f(\mathbf{x}) = 0$ in the complement of $H$, 
    and define $\int_H f$ as in Definition \ref{mydef:10.3}.

    Prove that $\int_H f$ is independent of the order in which the $k$ integrations are carried out.

    \emph{Hint:} Approximate $f$ by functions that are continuous on $\R^k$ and whose supports are in $H$, as was done in Example \ref{newexample:10.47}.
\end{myexercise}


\begin{myexercise}
    \label{ex:10.2}
    For $i = 1, 2, 3, ...$ , let $\phi_i \in \mathscr{C}(\R^1)$ have support in $(2^{-i} , 2^{1-i})$, such that $\int \phi_i = 1$.
    Put
    \begin{equation*}
        f(x,y) = \sum_{i=1}^{\infty}
        \left[ 
            \phi_{i}(x) -
            \phi_{i+1}(x)
         \right] \phi_i (y)
    \end{equation*}
    Then $f$ has compact support in $\R^2$, 
    $f$ is continuous except at $(0,0)$,
    and 
    \begin{equation*}
        \int \d y \int f(x,y) \d x = 0
        \quad \text{ but } \quad
        \int \d x \int f(x,y) \d y = 1.
    \end{equation*}
    Observe that $f$ is unbounded in every neighborhood of $(0, 0)$.
\end{myexercise}


\begin{myexercise}
    \label{ex:10.3}
    \begin{asparaenum}[(a)]
        \item If $F$ is as in Theorem \ref{thm:10.7}, put 
        $\mathbf{A} = \mathbf{F}'(0)$,
        $\mathbf{F_{1}(x)} = \mathbf{A^{-1}F(x)}$.
        Then $\mathbf{F'_1(0)}=I$.
        Show that 
        \begin{equation*}
            \mathbf{F_1(x) = G_n \circ G_{n-1} \circ \cdots \circ G_1(x)}
        \end{equation*}
        in some neighborhood of $\mathbf{0}$,
        for certain primitive mappings $\mathbf{G_n \circ \cdots \circ G_1(x)}$.
        This gives another version of Theorem \ref{thm:10.7}:
        \begin{equation*}
            \mathbf{F(x) = F'(0) G_n \circ G_{n-1} \circ \cdots \circ G_1(x)}.
        \end{equation*}
        \item Prove that the mapping $(x, y) > (y, x)$ of $\R^2$ onto $\R^2$ is not the composition of any two primitive mappings, in any neighborhood of the origin. 
        (This shows that the flips $B_1$ cannot be omitted from the statement of Theorem \ref{thm:10.7}.)
    \end{asparaenum}
\end{myexercise}

\begin{myexercise}    
    \label{ex:10.4}
    For $(x,y) \in \R^2$, define
    \begin{equation*}
        \mathbf{F}(x,y) = (e^x \cos y - 1, e^x \sin y).
    \end{equation*}
    Prove that $\mathbf{F = G_2 \circ G_1}$, where 
    \begin{align*}
        \mathbf{G}_1 (x,y) &= (e^x \cos y - 1, y) \\
        \mathbf{G}_2 (u,v) &= (u, (1 + u) \tan v) 
    \end{align*}
    are primitive in some neighborhood of $(0, 0)$.

    Compute the Jacobians of $\mathbf{G_1, G_2, F}$ at $(0, 0)$. 
    Define
    \begin{equation*}
        \mathbf{H}_2 (x,y) = (x, e^x \sin y)
    \end{equation*}
    and find 
    \begin{equation*}
        \mathbf{H}_1 (u,v) = (h(u,v), v)
    \end{equation*}
    so that $\mathbf{F = H_1 \circ H_2}$ is some neighborhood of $(0,0)$.
\end{myexercise}


\begin{myexercise}    
    \label{ex:10.5}
    Formulate and prove an analogue of Theorem \ref{thm:10.8}, 
    in which $K$ is a compact subset of an arbitrary metric space. 
    (Replace the functions $\phi_i$ that occur in the
    proof of Theorem \ref{thm:10.8} 
    by functions of the type constructed in Exercise \ref{ex:4.22})
\end{myexercise}


\begin{myexercise}    
    \label{ex:10.6}
    Strengthen the conclusion of Theorem \ref{thm:10.8} by showing that the functions $\psi_i$ can be made differentiable, and even infinitely differentiable. 
    (Use Exercise \ref{ex:8.1} in the construction of the auxiliary functions $\phi_i$.)
\end{myexercise}


\begin{myexercise}    
    \label{ex:10.7}
    \begin{enumerate}[(a)]
        \item Show that the simplex $Q^k$ is the smallest convex subset of $\R^k$ that contains $\mathbf{0},\mathbf{e}_1,\dots,\mathbf{e}_k$.
        \item Show that affine mappings take convex sets to convex sets.
    \end{enumerate}
\end{myexercise}


\begin{myexercise}    
    \label{ex:10.8}
    Let $H$ be the parallelogram in $\R^2$ whose vertices are $(1, 1), (3, 2), (4, 5), (2, 4)$.
    Find the affine map $T$ which sends $(0, 0)$ to $(1, 1)$, $(1, 0)$ to $(3, 2)$, $(0, 1)$ to $(2, 4)$.
    Show that $J_T = 5$. 
    Use $T$ to convert the integral
    \begin{equation*}
        \alpha = \int_H e^{x-y} \d x \d y
    \end{equation*}
    to an integral over $I^2$ and thus compute $\alpha$.
\end{myexercise}


\begin{myexercise}    
    \label{ex:10.9}
    Define $(x, y) = T(r, \theta)$ on the rectangle
    \begin{equation*}
        0 \leq r \leq a, 
        \quad 
        0 \leq \theta \leq 2\pi
    \end{equation*}
    by the equations
    \begin{equation*}
        x = r \cos \theta , \quad 
        y = r \sin \theta .
    \end{equation*}
    Show that $T$ maps this rectangle onto the closed disc $D$ with center at $(0, 0)$ and radius $a$, 
    that $T$ is one-to-one in the interior of the rectangle, and that $J_T(r, \theta) = r$. 
    If $f \in \mathscr{C}(D)$, prove the formula for integration in polar coordinates:
    \begin{equation*}
        \int_D f(x,y) \d x \d y = 
        \int_{0}^{a} \int_{0}^{2\pi} f(T(r,\theta)) r \d r \d \theta .
    \end{equation*}
    \emph{Hint:} Let Do be the interior of $D$, minus the interval from $(0, 0)$ to $(0, a)$. 
    As it stands, Theorem \ref{thm:10.9} applies to continuous functions $f$ whose support lies in $D_0$. 
    To remove this restriction, proceed as in Example \ref{newexample:10.4}.
\end{myexercise}



\begin{myexercise}    
    \label{ex:10.10}
    Let $a \rightarrow \infty$ in \ref{ex:11.9} and prove that 
    \begin{equation*}
        \int_{\R^2} f(x,y) \d x \d y = 
        \int_{0}^{\infty} \int_{0}^{2\pi} f(T(r,\theta)) r \d r \d \theta ,
    \end{equation*}
    for continuous functions f that decrease sufficiently rapidly as $|x | + | y | \rightarrow \infty$.
    (Find a more precise formulation.) 
    Apply this to
    \begin{equation*}
        f(x, y) = \exp (-x^2 - y^2)
    \end{equation*}
    to derive formula \eqref{eq:8.101}.
\end{myexercise}


\begin{myexercise}    
    \label{ex:10.11}
    Define $(u,v)=T(s,t)$ on the strip
    \begin{equation*}
        0<s<\infty , \quad
        0<t<1
    \end{equation*}
    by setting $u = s - st$, $v = st$. 
    Show that $T$ is a 1-1 mapping of the strip onto the positive quadrant $Q$ in $\R^2$.
    Show that $J_T(s, t) = s$.

    For x > 0, y > 0, integrate
    \begin{equation*}
        u^{x-1} e^{-u} v^{y-1} e^{-v} 
    \end{equation*}
    over $Q$, use Theorem \ref{thm:10.9} to convert the integral to one over the strip, and derive formula \eqref{eq:8.96} in this way.
    (For this application, Theorem \ref{thm:10.9} has to be extended so as to cover certain improper integrals. 
    Provide this extension.)
\end{myexercise}


\begin{myexercise}    
    \label{ex:10.12}
    Let $I^k$ be the set of all $\mathbf{u} = (u_1, ... , u_k) \in \R^k$ with $0 \leq u_i \leq 1$ for all $i$; 
    let $Q^k$ be the set of all $\mathbf{x} = (x_1, ... , x_k) \in  \R^k$ with $x_i \geq 0, \sum x_i \leq 1$. 
    ($I^k$ is the unit cube; 
    $Q^k$ is the standard simplex in $\R^k$.) 
    Define $\mathbf{x} = T(\mathbf{u})$ by
    \begin{align*}
        x_1 &= u_1 \\
        x_2 &= (1-u_1)u_2 \\
        ... & \dots
        x_k &= (1-u_1)\cdots(1-u_{k-1})u_k. 
    \end{align*}
    Show that 
    \begin{equation*}
        \sum_{i=1}^{k} x_i = 1 - \prod_{i=1}^{k} (1-u_i) .
    \end{equation*}

    Show that $T$ maps $I^k$ onto $Q^k$, 
    that $T$ is 1-1 in the interior of $I^k$, 
    and that its inverse $S$ is defined in the interior of $Q^k$ by $u_1 = x_1$ and
    \begin{equation*}
        u_i = \frac{x_i}{1-x_1-\cdots-x_{i-1}}
    \end{equation*}
    for $i=2,\dots,k$.
    Show that 
    \begin{equation*}
        J_T(\mathbf{u}) = 
        (1-u_1)^{k-1}
        (1-u_2)^{k-2}
        \cdots 
        (1-u_{k-1}),
    \end{equation*}
    and 
    \begin{equation*}
        J_S(\mathbf{x}) = 
        \left[ 
            (1-x_1)
            (1-x_1-x_2)
            \cdots
            (1-x_1-\cdots-x_{k-1})
         \right]^{-1} .
    \end{equation*}
\end{myexercise}


\begin{myexercise}    
    \label{ex:10.13}
    Let $r_1,\dots,r_k$ be nonnegative integers, and prove that
    \begin{equation*}
        \int_{Q^k} 
        x_1^{r_1} 
        \cdots 
        x_k^{r_k}
        \d x = 
        \frac{r_1!\cdots r_k!}{(k+r_1+\cdots+r_k)!}
    \end{equation*}
    \emph{Hint:} Use Exercise \ref{ex:11.12}, Theorem \ref{thm:10.9} and \ref{thm:8.20}.

    Note that the special case $r_1 = \cdots = r_k = 0$ 
    shows that the volume of $Q^k$ is $1/k!$.
\end{myexercise}


\begin{myexercise}    
    \label{ex:10.14}
    Prove formula \eqref{eq:10.46}.
\end{myexercise}


\begin{myexercise}    
    \label{ex:10.15}
    If $\omega$ and $\lambda$ are $k$- and $m$-forms, respectively, prove that
    \begin{equation*}
        \omega \wedge \lambda = 
        (-1)^{km} \lambda \wedge \omega .
    \end{equation*}
\end{myexercise}


\begin{myexercise}    
    \label{ex:10.16}
    If $k \geq 2$ and $\delta = [\mathbf{p}_0, \mathbf{p}_1, ... , \mathbf{p}_t]$ is an oriented affine $k$-simplex, 
    prove that $\partial^2 \sigma = 0$, 
    directly from the definition of the boundary operator $\partial$. 
    Deduce from this that $\partial^2 \Psi = 0$ for every chain $\Psi$.
    
    \emph{Hint:} For orientation, do it first for $k= 2, k = 3$. 
    In general, if $i <j$, let $\sigma_{ij}$ be the $(k - 2)$-simplex obtained by deleting $\mathbf{p}_i$ and $\mathbf{p}_j$ from $u$. 
    Show that each $\sigma_{ij}$ occurs twice in $\partial^2 \sigma$, with opposite sign.
\end{myexercise}


\begin{myexercise}    
    \label{ex:10.17}
    Put $J^2 = \tau_1 + \tau_2$, where 
    \begin{equation*}
        \tau_1 =  \left[ \mathbf{0,e_1,e_1+e_2} \right], 
        \quad 
        \tau_2 = -\left[ \mathbf{0,e_2,e_2+e_1} \right].
    \end{equation*}
    Explain why it is reasonable to call $J^2$ the positively oriented unit square in $\R^2$ .
    Show that $\partial J^2$ is the sum of 4 oriented affine 1-simplexes. Find these. What is $\partial (\tau_1 - \tau_2)$?
\end{myexercise}


\begin{myexercise}    
    \label{ex:10.18}
    Consider the oriented affine 3-simplex 
    \begin{equation*}
        \sigma_1 =  \left[ \mathbf{0,e_1,e_1+e_2,e_1+e_2+e_3} \right]
    \end{equation*}
    in $\R^3$.
    Show that $\sigma_1$
    (regarded as a linear transformation) has determinant 1.
    Thus $\sigma_1$ is positively oriented.

    Let $\sigma_2 , ... , \sigma_6$ be five other oriented 3-simplexes, obtained as follows: 
    There are five permutations $(i_1, i_2, i_3)$ of $(1, 2, 3)$, distinct from $(1, 2, 3)$. 
    Associate with each $(i_1, i_2, i_3)$ the simplex 
    \begin{equation*}
        s(i_1, i_2, i_3) \left[ \mathbf{0,e_{i_1},e_1+e_2} \right]
    \end{equation*}
    where $s$ is the sign that occurs in the definition of the determinant. 
    (This is how $\tau_2$ was obtained from $\tau_1$ in Exercise \ref{ex:11.17}.)
    
    Show that $\sigma_2, \dots , \sigma_6$ are positively oriented.
    
    Put $J^3 = \sigma_1 + \dots + \sigma_6$. 
    Then $J^3$ may be called the positively oriented unit cube in $\R^3$. 

    Show that $\partial J^3$ is the sum of 12 oriented affine 2-simplexes. 
    (These 12 triangles cover the surface of the unit cube $I^3$,)

    Show that $\mathbf{x} = (\mathbf{x}_1, \mathbf{x}_2, \mathbf{x}_3)$ is in the range of $\sigma_1$ if and only if $0 \leq x_3 \leq x_2 \leq x_1 \leq 1$,

    Show that the ranges of $\sigma_1, ... , \sigma_6$ have disjoint interiors, and that their union covers $I^3$. 
    (Compare with Exercise \ref{ex:11.13}; note that $3! = 6$.)
\end{myexercise}


\begin{myexercise}    
    \label{ex:10.19}
\end{myexercise}



\begin{myexercise}    
    \label{ex:10.20}
\end{myexercise}


\begin{myexercise}    
    \label{ex:10.21}
\end{myexercise}


\begin{myexercise}    
    \label{ex:10.22}
\end{myexercise}


\begin{myexercise}    
    \label{ex:10.23}
\end{myexercise}


\begin{myexercise}    
    \label{ex:10.24}
\end{myexercise}


\begin{myexercise}    
    \label{ex:10.25}
\end{myexercise}


\begin{myexercise}    
    \label{ex:10.26}
\end{myexercise}


\begin{myexercise}    
    \label{ex:10.27}
\end{myexercise}


\begin{myexercise}    
    \label{ex:10.28}
\end{myexercise}


\begin{myexercise}    
    \label{ex:10.29}
\end{myexercise}



\begin{myexercise}
    \label{ex:10.30}
\end{myexercise}


\begin{myexercise}
    \label{ex:10.31}
\end{myexercise}


\begin{myexercise}
    \label{ex:10.32}
\end{myexercise}


