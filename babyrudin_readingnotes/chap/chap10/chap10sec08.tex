% chap10sec08

\section{Closed forms and exact forms}

\begin{mydef}
    Let $\omega$ be a $k$-form in an open set $E \subset \R^n$. 
    If there is a $(k - 1)$- form $\lambda$ in $E$ such that $\omega = \d \lambda$, then $\omega$ is said to be exact in $E$.
    
    If $\omega$ is of class $\mathscr{C}'$ and $\d \omega = 0$, 
    then $\omega$ is said to be closed.
    
    Theorem \ref{thm:10.20}(b) shows that every exact form of class $\mathscr{C}'$ is closed.

    In certain sets $E$, for example in convex ones, the converse is true; 
    this is the content of Theorem \ref{thm:10.39} 
    (usually known as Poincare's lemma) and Theorem \ref{thm:10.40}. 
    However, Examples \ref{newexample:10.36} and \ref{newexample:10.37} will exhibit closed forms that are not exact.
\end{mydef}

\begin{myremark}
    \label{myremark:10.35}
\end{myremark}

\begin{newexample}
    \label{newexample:10.36}
    Let $E = \R^2 - \{\mathbf{0}\}$, the plane with the origin removed. 
    The 1-form 
    \begin{equation}
        \label{eq:10.110}
        \eta = \frac{x \d y - y \d x}{x^2+y^2}
    \end{equation}
    is \myKeywordblue{closed} in $\R^2 - \{\mathbf{0}\}$.
    This is easily verified by differentiation.
    Fix $r>0$, and define 
    \begin{equation}
        \label{eq:10.11}
        \gamma(t) = (r \cos t, r \sin t)
        \quad 
        (0 \leq t \leq 2\pi).
    \end{equation}
    Then $\gamma$ is a curve (an ``oriented I-simplex'') in $\R^2 - \{\mathbf{0}\}$. 
    Since $\gamma(0) = \gamma(2n)$,
    we have
    \begin{equation}
        \label{eq:10.112}
        \partial \gamma = 0 .
    \end{equation}

    Direct computation shows that
    \begin{equation}
        \label{eq:10.113}
        \int_{\gamma} \eta = 2\pi \neq 0 .
    \end{equation}

    The discussion in Remarks \ref{myremark:10.35}(b) and (c) shows that we can draw two conclusions from \eqref{eq:10.113}:

    First, $\eta$ is not exact in $\R^2 - \{\mathbf{0}\}$, for otherwise \eqref{eq:10.112} would force the integral \eqref{eq:10.113} to be 0. 
    
    Secondly, $\gamma$ is not the boundary of any 2-chain in $\R^2 - \{\mathbf{0}\}$ ( of class $\mathscr{C}''$), for otherwise the fact that $\eta$ is closed would force the integral \eqref{eq:10.113} to be 0.
\end{newexample}

\begin{newexample}
    \label{newexample:10.37}
    Let $E = \R^3 - \{\mathbf{0}\}$, 3-space with the origin removed.
    Define 
    \begin{equation}
        \label{eq:10.114}
        \zeta = 
        \frac{
            x \d y \wedge \d z + 
            y \d z \wedge \d x + 
            z \d x \wedge \d y 
        }{\left( x^2+y^2+z^2 \right)^{3/2}}
    \end{equation}
    where we have written $(x, y, z)$ in place of $(x_1, x_2 , x_3)$. Differentiation shows that $\d \zeta = 0$, so that $\zeta$ is a closed 2-form in $\R^3 - \{\mathbf{0}\}$. 
    
    Let $\sum$ be the 2-chain in $\R^3 - \{\mathbf{0}\}$ that was constructed in Example \ref{newexample:10.32}; 
    recall that $\sum$ is a parametrization of the unit sphere in $\R^3$. 
    Using the rectangle $D$ of Example \ref{newexample:10.32} as parameter domain, it is easy to compute that
    \begin{equation}
        \label{eq:10.115}
        \int_{\sum} \zeta = 
        \int_{D}    \sin u \d u \d v =
        4 \pi \neq 0 .
    \end{equation}
    As in the preceding example, we can now conclude that $\zeta$ is not exact in $\R^3 - \{\mathbf{0}\}$ 
    (since $\partial \sum = 0$, as was shown in Example \ref{newexample:10.32}) 
    and that the sphere $\sum$ is not the boundary of any 3-chain in $\R^3 - \{\mathbf{0}\}$ ( of class $\mathscr{C}''$), although $\partial \sum = 0$. 
    
    The following result will be used in the proof of Theorem \ref{thm:10.39}.
\end{newexample}

\begin{thm}
    \label{thm:10.38}
    Suppose $E$ is a convex open set in $\R^n$,
    $f \in \mathscr{C}'(E)$, $p$ is an integer, 
    $1 \leq p \leq n$, and
    \begin{equation}
        \label{eq:10.116}
        (D_j f)(\mathbf{x}) = 0
        \quad 
        (p < j \leq n, \mathbf{x} \in E).
    \end{equation}
    Then there exists an $F \in \mathscr{C}'(E)$ such that
\end{thm}

\begin{thm}
    \label{thm:10.39}
    If $E \in \R^n$ is convex and open, 
    if $k \geq 1$, 
    if $\omega$ is a $k$-form of class $\mathscr{C}'$ in $E$, a
    nd if $\d \omega = 0$, 
    then there is a $(k - 1)$-form $\lambda$ in $E$ 
    such that $\omega = \d \lambda$.
\end{thm}

\begin{thm}
    \label{thm:10.40}
    Fix $k$, $1 \leq k \leq n$. 
    Let $E \subset \R^n$ be an open set in which every closed $k$-form is exact. 
    Let $T$ be a 1-1 $\mathscr{C}''$-mapping of $E$ onto an open set $U \subset \R^n$ 
    whose inverse $S$ is also of class $\mathscr{C}''$.
    
    Then every closed $k$-form in $V$ is exact in $V$.
\end{thm}

Note that every convex open set $E$ satisfies the present hypothesis, by
Theorem \ref{thm:10.39}. 
The relation between $E$ and $V$ may be expressed by saying
that they are $\mathscr{C}''$-\emph{equivalent}.

\emph{Thus every closed form is exact in any set which is $\mathscr{C}''$-equivalent to a convex open set.}

\begin{myremark}
    \label{myremark:10.40}
    
\end{myremark}