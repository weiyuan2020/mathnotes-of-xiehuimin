% chap10sec01

\section{Integration}
\begin{mydef}
    \label{mydef:10.1}
    Suppose $I^k$ is a $k$-cell in $\R^k$, consisting of all
    \begin{equation*}
        \mathbf{x} = (x_1,\dots,x_k)
    \end{equation*}
    such that 
    \begin{equation}
        \label{eq:10.1}
        a_i \leq x_i \leq b_i 
        \quad 
        (i = 1,\dots, k) ,
    \end{equation}
    $I^j$ is the $j$-cell in $\R^j$ defined by the first $j$ inequalities (\ref{eq:10.1}), and f is a real continuous function on $I^k$.

    Put $f = f_k$, and define $f_{k-1}$ on $I^{k-1}$ by
    \begin{equation*}
        f_{k-1}(x_1,\dots,x_{k-1}) = 
        \int_{a_k}^{b_k} f_k (x_1,\dots,x_{k-1},x_k) \d x_k .
    \end{equation*}
    The uniform continuity of $f_k$ on $I^k$ shows that $f_{k-1}$ is continuous on $I^{k-1}$.
    Hence we can repeat this process and obtain functions $f_j$, continuous on $I^j$, 
    such that $f_{j-1}$ is the integral of $f_j$, with respect to $x_j$, over $[a_j, b_j]$. 
    After $k$ steps we arrive at a \emph{number} $f_0$, 
    which we call the \emph{integral of $f$ over $I^k$}; 
    we write it in the form
    \begin{equation}
        \label{eq:10.2}
        \int_{I^k} f(\mathbf{x}) \d \mathbf{x}
        \text{  or  }
        \int_{I^k} f.
    \end{equation}
    
    A priori, this definition of the integral depends on the order in which the $k$ integrations are carried out. 
    However, this dependence is only apparent. 
    To prove this, let us introduce the temporary notation $L(f)$ for the integral (\ref{eq:10.2}) and $L'(f)$ for the result obtained by carrying out the $k$ integrations in some other order.
\end{mydef}
\mybox{
    $L(f)$ 积分 (\ref{eq:10.2}) 暂时的记号
    
    $L'(f)$ 用另外的次序求这 $k$ 个积分的结果
}

\begin{thm}
    \label{thm:10.2}
    For every $f \in \mathscr{C}(I^k)$, $L(f) = L'(f)$.
\end{thm}
\mybox{对区间上的连续函数, 积分结果与积分顺序无关}
% todo add proof
\mybox{Stone-Weierstrass 定理能够用到这些函数上}

\begin{mydef}
    \label{mydef:10.3}
    The \emph{support} of a (real or complex) function $f$ on $\R^k$ is the
    closure of the set of all points $\mathbf{x} \in R^k$ 
    at which $f(\mathbf{x}) \neq 0$. 
    If $f$ is a continuous function with compact support, 
    let $I^k$ be any $k$-cell which contains the support of $f$, 
    and define
    \begin{equation}
        \label{eq:10.3}
        \int_{R^k} f =
        \int_{I^k} f .
    \end{equation}
    The integral so defined is evidently independent of the choice of $I^k$, provided only that $I^k$ contains the support of $f$.

\end{mydef}

\mybox{support 支集}

It is now tempting to extend the definition of the integral over $R^k$ to
functions which are limits (in some sense) of continuous functions with compact support. 
We do not want to discuss the conditions under which this can be done; 
the proper setting for this question is the Lebesgue integral. 

\begin{newexample}
    \label{newexample:10.4}
    Let $Q^k$ be the $k$-simplex which consists of all points $\mathbf{x} = (x_1, \dots , x_k)$ in $\R^k$ for which $x_1 + \dots + x_k \leq 1$ and $x_i \geq 0$ for $i = 1, ... , k$. 
    If $k = 3$, for example, $Q^k$ is a tetrahedron, with vertices at $\mathbf{0, e_1, e_2, e_3}$. 
    If $f \in \mathscr{C}(Q^k)$,
    extend $f$ to a function on $I^k$ by setting $f(\mathbf{x}) = \mathbf{0}$ off $Q^k$, and define
    \begin{equation}
        \label{eq:10.4}
        \int_{Q^k} f =
        \int_{I^k} f .
    \end{equation}
    Here $I^k$ is the ``unit cube'' defined by
    \begin{equation*}
        0 \leq x_i \leq 1
        \quad 
        (1 \leq i \leq k).
    \end{equation*}

    Since $f$ may be discontinuous on $I^k$, the existence of the integral on the right of \eqref{eq:10.4} needs proof. 
    We also wish to show that this integral is independent of the order in which the $k$ single integrations are carried out.

    To do this, suppose $0 < \delta < 1$, put
    \begin{equation}
        \label{eq:10.5}
        \phi(t) = \left\{ 
            \begin{array}{ll}
                1 & (t\leq 1-\delta) \\
                \frac{(1-t)}{\delta} & (1-\delta < t \leq 1) \\
                0 & (1<t), \\
            \end{array}
         \right.
    \end{equation}
    and define 
    \begin{equation}
        \label{eq:10.6}
        F(\mathbf{x}) = 
        \phi(x_1+\cdots+x_k) f(\mathbf{x})
        \quad 
        (\mathbf{x} \in I^k).
    \end{equation}
    Then $F \in \mathscr{C}(I^k)$.

    Put $\mathbf{y} = (x_1, \dots , x_{k-1})$, $\mathbf{x} = (\mathbf{y}, x_k)$. 
    For each $\mathbf{y} \in I^{k-1}$, the set of all $x_k$ such that $F(\mathbf{y}, x_k) \neq f(\mathbf{y}; x_k)$ is either empty or is a segment whose length does not exceed $\delta$. 
    Since $0 \leq \phi \leq 1$, it follows that
    \begin{equation}
        \label{eq:10.7}
        \left| F_{k-1}(\mathbf{y})-f_{k-1}(\mathbf{y}) \right| \leq \delta \left\| f \right\| 
        \quad 
        (\mathbf{y} \in I^{k-1}),
    \end{equation}
    where $\left\| f \right\|$ has the same meaning as in the proof of Theorem \ref{thm:10.2}, 
    and $F_{k-1}$, $f_{k-1}$ are as in Definition \ref{mydef:10.1}.

    As $\delta \rightarrow 0$, \eqref{eq:10.7} exhibits $f_{k-1}$ as a uniform limit of a sequence of continuous functions. 
    Thus $f \in \mathscr{C}(I^{k-1})$, and the further integrations present no problem.

    This proves the existence of the integral \eqref{eq:10.4}. 
    Moreover, \eqref{eq:10.7} shows that
    \begin{equation}
        \label{eq:10.8}
        \left| 
            \int_{I^k} F(\mathbf{x}) \d \mathbf{x} -
            \int_{I^k} f(\mathbf{x}) \d \mathbf{x} 
         \right| \leq 
        \delta \left\| f \right\| .
    \end{equation}
    Note that \eqref{eq:10.8} is true, regardless of the order in which the $k$ single integrations are carried out. 
    Since $F \in \mathscr{C}(I^k)$, $\int F$ is unaffected by any change in this order.
    Hence \eqref{eq:10.8} shows that the same is true of $\int f$

    This completes the proof.

    Our next goal is the change of variables formula stated in Theorem \ref{thm:10.9}. 
    To facilitate its proof, we first discuss so-called primitive mappings, and partitions of unity.  
    Primitive mappings will enable us to get a clearer picture of the local action of $\mathscr{C}'$-mapping with invertible derivative, 
    and partitions of unity are a very useful device that makes it possible to use local information in a global setting.
\end{newexample}

