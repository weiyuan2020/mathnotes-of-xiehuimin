% chap10sec09

\section{Vector analysis}

We conclude this chapter with a few applications of the preceding material to theorems concerning vector analysis in $\R^3$. 
These are special cases of theorems about differential forms, but are usually stated in different terminology. 
We are thus faced with the job of translating from one language to another.

\begin{mydef}
    \myKeyword{Vector fileds}

\end{mydef}

\begin{thm}
    \label{thm:10.43}
    Suppose $E$ is an open set in $\R^3$, $u \in \mathscr{C}''(E)$, and $\mathbf{G}$ is a vector field in $E$, of class $\mathscr{C}''$.
    \begin{enumerate}[(a)]
        \item If $\mathbf{F} = \nabla u$, then $\nabla \times \mathbf{F} = \mathbf{0}$.
        \item If $\mathbf{F} = \nabla \times \mathbf{G}$, then $\nabla \cdot \mathbf{F} = 0$.
    \end{enumerate}

    Furthermore, if $E$ is $\mathscr{C}''$-equivalent to a convex set, 
    then (a) and (b) have converses, 
    in which we assume that $\mathbf{F}$ is a vector field in $E$, of class $\mathscr{C}'$:
    \begin{enumerate}[(a')]
        \item If $\nabla \cdot \mathbf{F} = \mathbf{0}$, then $\mathbf{F} = \nabla u$ for some $u \in \mathscr{C}''(E)$.
        \item If $\nabla \times \mathbf{F} = 0$, then $\mathbf{F} = \nabla \times \mathbf{G}$ for some vector field $\mathscr{G}$. in $E$, of class $\mathscr{C}''$
    \end{enumerate}
\end{thm}

\begin{mydef}
    \myKeyword{Volume elements}

\end{mydef}

\begin{mydef}
    \myKeyword{Green's theorem}
    
\end{mydef}


\begin{mydef}
    \myKeyword{Area elements in $\R^3$}
    
\end{mydef}

\begin{newexample}
    \label{newexample:10.47}
\end{newexample}

\begin{mydef}
    \myKeyword{Integrals of 1-forms in $\R^3$}
    
\end{mydef}

\begin{mydef}
    \myKeyword{Integrals of 2-forms in $\R^3$}
    
\end{mydef}

\begin{mydef}
    \myKeyword{Stokes' formula}
    
    \begin{equation}
        \label{eq:10.145}
        \int_{\Phi} \left( \nabla \times \mathbf{F} \right) \cdot \mathbf{n} \d A = 
        \int_{\partial \Phi} \left( \mathbf{F \cdot t} \right)  \d s
    \end{equation}
\end{mydef}

\begin{mydef}
    \myKeyword{The divergence theorem}
    
\end{mydef}

