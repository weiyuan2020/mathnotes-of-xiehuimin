% chap10sec06

\section{Simplexes and chains}

\begin{mydef}
    \label{mydef:10.26}
    \myKeyword{Affine simplexes}
    A mapping $\mathbf{f}$ that carries a vector space $X$ into a
    vector space $Y$ is said to be \emph{affine} if $\mathbf{f - f(0)}$ is linear. 
    In other words, the requirement is that
    \begin{equation}
        \label{eq:10.73}
        \mathbf{f(x)} = 
        \mathbf{f(0)} + A \mathbf{x}
    \end{equation}
    for some $A \in L(X, Y)$.

    An affine mapping of $\R^k$ into $\R^n$ is thus determined if we know $\mathbf{f}(0)$ and $\mathbf{f(e_i)}$ for $1 \leq i \leq k$; 
    as usual, $\{\mathbf{e}_1, ... , \mathbf{e}_k\}$ is the standard basis of $\R^k$.
    
    We define the \emph{standard simplex} $Q^k$ to be the set of all $\mathbf{u} \in \R^k$ of the form
    \begin{equation}
        \label{eq;10.74}
        \mathbf{u} = \sum_{i=1}^{k} \alpha_i \mathbf{e}_i
    \end{equation}
    such that $\alpha \geq 0$ for $i = 1, ... , k$ and $\sum \alpha_i \leq 1$.
    % todo
\end{mydef}

\begin{thm}
    \label{thm:10.27}
    If $\delta$ is an oriented rectilinear $k$-simplex in an open set $E \subset \R^n$ 
    and if $\overline{\Delta} = \varepsilon \delta$ then
    \begin{equation}
        \label{eq:10.81}
        \int_{\overline{\delta}} \omega = 
        \varepsilon \int_{\delta} \omega 
    \end{equation}
    for every $k$-form $\omega$ in $E$.
\end{thm}

% todo add proof

\begin{mydef}
    \label{mydef:10.28}
    \myKeyword{Affine chains}
    An \emph{affine $k$-chain} $\Gamma$ in an open set $E \subset \R^n$ is a collection of finitely many oriented affine $k$-simplexes $\delta_1, \dots, \delta_r$ in $E$. 
    These need not be distinct; 
    a simplex may thus occur in $\Gamma$ with a certain multiplicity.

    If $\Gamma$ is as above, and if $\omega$ is a $k$-form in $E$,
    we define 
    \begin{equation}
        \label{eq:10.82}
        \int_{\Gamma} \omega =
        \sum_{i=1}^{r} \int_{\sigma_i} \omega .
    \end{equation}
    % todo add word
\end{mydef}

\begin{mydef}
    \label{mydef:10.29}
    \myKeyword{Boundaires}
    For $k \geq 1$, the \emph{boundary} of the oriented affine $k$-simplex
    \begin{equation*}
        \delta = \left[ \mathbf{p_0,p_1,\dots,p_k} \right]
    \end{equation*}
    is defined to be the affine $(k - 1)$-chain
    \begin{equation}
        \label{eq:10.85}
        \partial \delta = \sum_{j=0}^{k} (-1)^j
        \left[ \mathbf{p_0,p_1,\dots,p_{j-1},p_{j+1},\dots,p_k} \right]
    \end{equation}
\end{mydef}

For example, if $\sigma = [\mathbf{p}_0 , \mathbf{p}_1, \mathbf{p}_2 ]$, then
\begin{equation*}
    \partial\sigma = 
    [\mathbf{p}_1, \mathbf{p}_2] - 
    [\mathbf{p}_0, \mathbf{p}_2] + 
    [\mathbf{p}_0, \mathbf{p}_1] = 
    [\mathbf{p}_0, \mathbf{p}_1] + 
    [\mathbf{p}_1, \mathbf{p}_2] + 
    [\mathbf{p}_2, \mathbf{p}_0] ,
\end{equation*}
which coincides with the usual notion of the oriented boundary of a triangle.
For $1 \leq j \leq k$, observe that the simplex $\sigma_j = [\mathbf{p}_0 , ... , \mathbf{p}_{i- 1} , \mathbf{p}_{i+ 1}, ... , \mathbf{p}_k]$
which occurs in \eqref{eq:10.85} has $Q^{k- 1}$ as its parameter domain and that it is defined by
\begin{equation}
    \label{eq:10.86}
    \sigma_j(\mathbf{u}) = \mathbf{p}_0 + B\mathbf{u} 
    \quad 
    (\mathbf{u} \in Q^{k-1}),
\end{equation}
where $B$ is the linear mapping from $\R^{k-1}$ to $\R^n$ determined by 
\begin{align*}
    B\mathbf{e}_i &= \mathbf{p}_i - \mathbf{p}_0 \quad (\text{if } 1 \leq i \leq j-1) , \\
    B\mathbf{e}_i &= \mathbf{p}_{i+1} - \mathbf{p}_0 \quad (\text{if } j \leq i \leq k-1) .
\end{align*}

The simplex 
\begin{equation*}
    \sigma_0 = [\mathbf{p}_1,\mathbf{p}_2,\dots,\mathbf{p}_k]
\end{equation*}
which also occurs in \eqref{eq:10.85}, is given by the mapping
\begin{equation*}
    \sigma_0(\mathbf{u}) = \mathbf{p}_1 + B \mathbf{u},
\end{equation*}
where $B\mathbf{e}_i = \mathbf{p}_{i+1} - \mathbf{p}_1$ for $1 \leq i \leq k-1$.



\begin{mydef}
    \label{mydef:10.30}
    \myKeyword{Differentiable simplexes and chains}
    Let $T$ be a $\mathscr{C}''$-mapping of an open set $E \subset \R^n$ into an open set $V \subset \R^m$; 
    $T$ need not be one-to-one. 
    If $\sigma$ is an oriented affine $k$-simplex in $E$, then the composite mapping $\Phi = T \circ \sigma$
    (which we shall sometimes write in the simpler form $T\sigma$) 
    is a $k$-surface in $V$, with parameter domain $Q^k$. 
    We call $\Phi$ an \myKeywordblue{oriented $k$-simplex of class $\mathscr{C}''$}.

    A finite collection $\Psi$ of oriented $k$-simplexes $\Phi_1, ... , \Phi_r$ of class $\mathscr{C}''$ in $V$
    is called a \myKeywordblue{$k$-chain of class $\mathscr{C}''$} in $V$. 
    If $\omega$ is a $k$-form in $V$, we define
    \begin{equation}
        \label{eq:10.87}
        \int_{\Psi} \omega = 
        \sum_{i=1}^{r} \int_{\Phi i} \omega
    \end{equation}
    and use the corresponding notation $\Psi = \sum \Phi_i$.

    If $\Gamma = \sum \sigma_i$ is an affine chain 
    and if $\Phi_i = T \circ \sigma_i$, 
    we also write $\Psi = T \circ \Gamma$, 
    or 
    \begin{equation}
        \label{eq:10.88}
        T\left( \sum \sigma_i \right) = \sum T \sigma_i .
    \end{equation}
    The boundary $\partial \Phi$ of the oriented $k$-simplex $\Phi = T \circ \sigma$ a is defined to be the $(k - 1)$ chain
    \begin{equation}
        \label{eq:10.89}
        \partial \Phi = T \left( \partial \sigma \right) .
    \end{equation}

    In justification of \eqref{eq:10.89}, observe that if $T$ is affine, 
    then $\Phi = T \circ \sigma$ is an oriented affine $k$-simplex, 
    in which case \eqref{eq:10.89} is not a matter of definition, 
    but is seen to be a consequence of \eqref{eq:10.85}. 
    Thus \eqref{eq:10.89} generalizes this special case.

    It is immediate that $\partial \Phi$ is of class $\mathscr{C}''$ if this is true of $\Phi$.

    Finally, we define the boundary $\partial \Phi$ of the $k$-chain $\Psi = \sum \Phi_i$ to be the $(k - 1)$ chain
    \begin{equation}
        \label{eq:10.90}
        \partial \Psi = \sum \partial \Phi_i ,
    \end{equation}
\end{mydef}

\begin{mydef}
    \label{mydef:10.31}
    \myKeyword{Positively oriented boundaries}
    So far we have associated boundaries to chains, not to subsets of $\R^n$. 
    This notion of boundary is exactly the one that is most suitable for the statement and proof of Stokes' theorem.
    However, in applications, especially in $\R^2$ or $\R^3$, it is customary and convenient to talk about ``oriented boundaries'' of certain sets as well. 
    We shall now describe this briefly. 
    
    Let $Q^n$ be the standard simplex in $\R^n$, 
    let $\sigma_0$ be the identity mapping with domain $Q^n$. 
    As we saw in Sec. \ref{mydef:10.26}, $\sigma_0$ may be regarded as a positively oriented n-simplex in $\R^n$. 
    Its boundary $\partial\sigma_0$ is an affine $(n - 1)$-chain. 
    This chain is called the positively oriented boundary of the set $Q^n$. 
    
    For example, the positively oriented boundary of $Q^3$ is
    \begin{equation*}
        [\mathbf{e}_1, \mathbf{e}_2, \mathbf{e}_3] -
        [           0, \mathbf{e}_2, \mathbf{e}_3] +
        [           0, \mathbf{e}_1, \mathbf{e}_3] -
        [           0, \mathbf{e}_1, \mathbf{e}_2] .
    \end{equation*}
    Now let $T$ be a 1-1 mapping of $Q^n$ into $\R^n$, of class $\mathscr{C}''$, 
    whose Jacobian is positive 
    (at least in the interior of $Q^n$). 
    Let $E = T(Q^n)$. 
    By the inverse function theorem, $E$ is the closure of an open subset of $\R^n$. 
    We define the positively oriented boundary of the set $E$ to be the $(n - 1)$-chain
    \begin{equation*}
        \partial T = T (\partial \sigma_0) ,
    \end{equation*}
    and we may denote this $(n - 1)$-chain by $\partial E$.
    % todo
\end{mydef}

\begin{newexample}
    \label{newexample:10.32}
    For $0 \leq u \leq \pi, 0 \leq v \leq 2\pi$, define
    \begin{equation*}
        \sum(u,v) = \left( 
            \sin u \cos v,
            \sin u \sin v,
            \cos u
         \right).
    \end{equation*}
    % todo
\end{newexample}
