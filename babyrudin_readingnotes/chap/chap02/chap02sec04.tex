% chap02sec04
\section{Perfect sets}
\mybox{perfect set 完全集 区别于 complete set 完备集}

% 243 Theorem 
\begin{thm}\label{thm:2.43}
    Let $P$ be a nonempty perfect set in $\R^k$. Then P is uncountable.    
\end{thm}
\begin{proof}
    Since $P$ has limit points, $P$ must be infinite. 
    Suppose $P$ is countable, and denote the points of $P$ by $\mathbf{x}_{1}, \mathbf{x}_{2}, \mathbf{x}_{3},\dots$. 
    We shall construct a sequence $\sequence{V_n}$ of neighborhoods, as follows.
    
    Let $V_{1}$, be any neighborhood of $\mathbf{x}_1$. 
    If $V_{1}$ consists of all $\mathbf{y} \in \R^k$ 
    such that $\left| \mathbf{y} - \mathbf{x}_1 \right| < r$, 
    the closure $\overline{V}_{1}$ of $V_1$ is the set of all $\mathbf{y} \in \R^k$ 
    such that $\left| \mathbf{y} - \mathbf{x}_1 \right| \leq r$.

    Suppose $V_n$ has been constructed, 
    so that $V_n \cap P$ is not empty. 
    Since every point of $P$ is a limit point of $P$, 
    there is a neighborhood $V_{n+1}$ such that 
    \begin{inparaenum}[(i)]
        \item $\overline{V}_{n+1} \subset V_n$,
        \item $x_n \not\in \overline{V}_{n+1}$,
        \item $V_{n+1} \cap P$ is not empty. 
    \end{inparaenum}
    By (iii), $V_{n+1}$ satisfies our induction hypothesis, 
    and the construction can proceed.
    
    Put $K_n = \overline{V}_n \cap P$. 
    Since $\overline{V}_n$ is closed and bounded, 
    $\overline{V}_n$ is compact.
    Since $x_n \not\in K_{n+1}$, 
    no point of $P$ lies in $\cap_1^{\infty} K_n$.
    Since $K_n \subset P$, 
    this implies that $\cap_1^{\infty} K_n$ is empty. 
    But each $K_n$ is nonempty, 
    by (iii), and $K_n \supset K_{n+1}$,
    by (i); this contradicts the Corollary to Theorem \ref{thm:2.36}.
\end{proof}

\begin{myCorollary*}
	Every interval $[a, b] (a <b)$ is uncountable. In particular, the set
	all real numbers is uncountable.
\end{myCorollary*}


\begin{mydef}
    \label{mydef:2.44}
	\myKeyword{The Cantor set} The set which we are now going to construct shows
	that there exist perfect sets in $\R^{1}$ which contain no segment.
\end{mydef}

\mybox{
    Cantor 构建的一个经典例子, $\R^{1}$ 中的一个无处稠密的完备集

    康托三分集中有无穷多个点, 所有的点处于非均匀分布状态. 此点集具有自相似性, 其局部与整体是相似的, 所以是一个分形系统. 

    康托三分集具有
    \begin{enumerate}[(1)]
        \item 自相似性;
        \item 精细结构;
        \item 无穷操作或迭代过程;
    \end{enumerate}

    康托尔集P具有的性质:
    \begin{enumerate} [1]
        \item P是完备集;
        \item P没有内点;
        \item P的基数为$c$;
        \item P是不可数集;
    \end{enumerate}
    康托尔集是一个基数为$c$的疏朗完备集. 
}

Let $E_0$ be the interval $[0, 1]$. Remove the segment $(\frac{1}{3}, \frac{2}{3})$, and let $E_1$ be the union of the intervals
\begin{equation*}
    \left[0, \frac{1}{3}\right] \quad 
    \left[\frac{2}{3}, 1\right]
\end{equation*}

Remove the middle thirds of these intervals, and let $E_2$ be the union of the intervals
\begin{equation*}
    \left[0, \frac{1}{9}\right] \quad 
    \left[\frac{2}{9}, \frac{3}{9}\right] \quad 
    \left[\frac{6}{9}, \frac{7}{9}\right] \quad 
    \left[\frac{8}{9}, 1\right]
\end{equation*}

Continuing in this way, we obtain a sequence of compact sets $E_n$, such that

(a) $E_1 \supset E_2 \supset E_3 \supset \dots$;

(b) $E_n$ is the union of $2^n$ intervals, each of length $3^{-n}$.

The set
\begin{equation*}
    P = \bigcap_{n=1}^{\infty} E_n
\end{equation*}

is called the \emph{Cantor set}. $P$ is clearly compact, and Theorem \ref{thm:2.36} shows that $P$ is not empty.
% 42 PRINCIPLES OF MATHEMATICAL ANALYSIS

No segment of the form
\begin{equation}
    \left(
        \frac{3k+1}{3^m},
        \frac{3k+2}{3^m}
    \right)
\end{equation}
where $k$ and $m$ are positive integers, has a point in common with $P$. Since every segment $(\alpha, \beta)$ contains a segment of the form (24), if
\begin{equation*}
    3^{-m} < \frac{\beta - \alpha}{6},
\end{equation*}
$P$ contains no segment.

To show that $P$ is perfect, it is enough to show that $P$ contains no isolated point. Let $x \in P$, and let S be any segment containing $x$. Let $I_n$ be that interval of $E_n$ which contains $x$. Choose n large enough, so that $I_n \subset S$. Let $x_n$ be an endpoint of $I_n$, such that $x_n \neq x$.

It follows from the construction of $P$ that $x_n \in P$. Hence $x$ is a limit point of $P$, and $P$ is perfect.

One of the most interesting properties of the Cantor set is that it provides
us with an example of an uncountable set of measure zero (the concept of
measure will be discussed in Chap. 11).
