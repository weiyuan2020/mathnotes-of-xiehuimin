% chap07exercise
\section{Exercises}

\begin{myexercise}
    \label{ex:7.1}
    Prove that every uniformly convergent sequence of bounded functions is uniformly bounded.
\end{myexercise}


\begin{myexercise}
    \label{ex:7.2}
    If $\{f_n\}$ and $\{g_n\}$ converge uniformly on a set $E$, prove that $\{f_n+g_n\}$ converges uniformly on $E$.
    If, in addition, $\{f_n\}$ and $\{g_n\}$ are sequences of bounded functions, prove that $\{f_n g_n\}$ converges uniformly on $E$.
\end{myexercise}


\begin{myexercise}
    \label{ex:7.3}
    Construct sequences $\{f_n\}$ , $\{g_n\}$ which converge uniformly on some set $E$, but such that $\{f_n g_n\}$ 
    does not converge uniformly on $E$ 
    (of course, $\{f_n g_n\}$ must converge on $E$).
\end{myexercise}


\begin{myexercise}
    \label{ex:7.4}
    Consider 
    \begin{equation*}
        f(x) = \sum_{n=1}^{\infty} \frac{1}{1+n^2x} .
    \end{equation*}
    For what values of $x$ does the series converge absolutely?
    On what intervals does it converge uniformly?
    On what intervals does it fail to converge uniformly?
    Is $f$ continuous wherever the series converges?
    Is $f$ bounded?
\end{myexercise}


\begin{myexercise}
    \label{ex:7.5}
    % \begin{numcases}{f_n(x)}
    %     0 & \left( x<\frac{1}{n+1} \right), \\
    %     \sin^2 \frac{\pi}{x} & \left( \frac{1}{n+1}\leq x \leq \frac{1}{n} \right), \\
    % \end{numcases}
    \begin{equation*}
        f_n(x) = \left\{ 
            \begin{array}{ll}
                0 & \left( x<\frac{1}{n+1} \right), \\
                \sin^2 \frac{\pi}{x} & \left( \frac{1}{n+1}\leq x \leq \frac{1}{n} \right), \\
                0 & \left( \frac{1}{n} < x \right). \\ 
            \end{array}
         \right.
    \end{equation*}
    Show that $\{f_n\}$ converges to a continuous function, but not uniformly. 
    Use the series $\sum f_n$ to show that absolute convergence, even for all $x$, does not imply uniform convergence.
\end{myexercise}


\begin{myexercise}
    \label{ex:7.6}
    Prove that the series 
    \begin{equation*}
        \sum_{n=1}^{\infty} (-1)^n \frac{x^2+n}{n^2}
    \end{equation*}
    converges uniformly in every bounded interval, but does not converge absolutely for any value of $x$.
\end{myexercise}


\begin{myexercise}
    \label{ex:7.7}
    For $n=1,2,3,...,x$ real, put 
    \begin{equation*}
        f_n(x) = \frac{x}{1+nx^2}.
    \end{equation*}
    Show that $\{f_n\}$ converges uniformly to a function $f$, and that the equation
    \begin{equation*}
        f'(x) = \lim_{n \to \infty} f'_n(x)
    \end{equation*}
    is correct if $x \neq 0$, but false if $x=0$.
\end{myexercise}


\begin{myexercise}
    \label{ex:7.8}
    If 
    \begin{equation*}
        I(x) = \left\{ 
            \begin{array}{ll}
                0 & (x \leq 0), \\
                1 & (x >    0), \\
            \end{array}
         \right.
    \end{equation*}
    If $\{x_n\}$ is a sequence of distinct points of $(a,b)$, and of $\sum |c_n|$ converges, prove that the series 
    \begin{equation*}
        f(x) = \sum_{n=1}^{\infty} c_n I(x-x_n) \quad 
        (a \leq x \leq b)
    \end{equation*}
    converges uniformly, and that $f$ is continuous for every $x \neq x_n$.
\end{myexercise}


\begin{myexercise}
    \label{ex:7.9}
    Let $\{f_n\}$ be a sequence of continuous functions which converges uniformly to a function $f$ on a set $E$.
    Prove that 
    \begin{equation*}
        \lim_{n \to \infty} f_n (x_n) = f(x)
    \end{equation*}
    for every sequence of points $x_n \in E$ such that $x_n \rightarrow x$, and $x \in E$.
    Is the converse of this true?
\end{myexercise}


\begin{myexercise}
    \label{ex:7.10}
    Letting $(x)$ denote the fractional part of the real number $x$
    (see Exercise \ref{ex:4.16} for the definition),
    consider the function
    \begin{equation*}
        f(x) = \sum_{n=1}^{\infty} \frac{(nx)}{n^2} \quad 
        (x \text{ real}).
    \end{equation*}
    Find all discontinuities of $f$,
    and show that they form a countable dense set.
    Show that $f$ is nevertheless Riemann-integrable on every bounded interval.
\end{myexercise}


\begin{myexercise}
    \label{ex:7.11}
    Suppose $\{f_n\}$ , $\{g_n\}$ are defined on $E$, and 
    \begin{enumerate}[(a)]
        \item $\sum f_n$ has uniformly bounded partial sums;
        \item $g_n \rightarrow 0$ uniformly on $E$;
        \item $g_1(x) \geq g_2(x) \geq g_3(x) \geq \cdots$ for every $x \in E$.
    \end{enumerate}
    
    Prove that $\sum f_n g_n$ converges uniformly on $E$.

    \emph{Hint:} Compare with Theorem \ref{thm:3.42}.
\end{myexercise}


\begin{myexercise}
    \label{ex:7.12}
    Suppose $g$ and $f_n(n=1,2,3,\dots)$ are defined on $(0,\infty)$, are Riemann-integrable on $[t,T]$ whenever $0 < t < T < \infty$, $|f_n| \leq g$, $f_n \rightarrow f$ uniformly on every compact subset of $(0, \infty)$, and 
    \begin{equation*}
        \int_{0}^{\infty} g(x) \d x < \infty .
    \end{equation*}
    Prove that 
    \begin{equation*}
        \lim_{n \to \infty} \int_{0}^{\infty} f_n (x) \d x = 
        \int_{0}^{\infty} f(x) \d x .
    \end{equation*}
    (See Exercises \ref{ex:6.7} and \ref{ex:6.8} for the relevant definitions.)

    This is a rather weak form of Lebesgue's dominated convergence theorem (Theorem \ref{thm:11.32}).
    Even in the context of the Riemann integral, uniform convergence can be replaced by pointwise convergence if it is assumed that $f \in \mathscr{R}$.
    (See the articles by F. Cunningham in \emph{Math. Mag.}, col.40, 1967, pp. 179-186, and by H. Kestelamn in \emph{Amer. Math. Monthly}, vol. 77, 1970, pp. 182-187.)
\end{myexercise}


\begin{myexercise}
    \label{ex:7.13}
    Assume that $\{f_n\}$ is a sequence of monotonically increasing functions on $\R^1$ with $0 \leq f_n(x) \leq 1$ for all $x$ and all $n$.
    \begin{asparaenum}[(a)]
        \item Prove that there is a function $f$ and a sequence $\{n_k\}$ such that 
        \begin{equation*}
            f(x) = \lim_{k \to \infty} f_{n_k}(x)
        \end{equation*}
        for every $x \in \R^1$.
        (The existence of such a pointwise convergent subsequence is usually called \myKeywordblue{Helly's selection theorem})
        \item If, moreover, $f$ is continuous, prove that $f_{n_k} \rightarrow f$ uniformly on compact sets.
    \end{asparaenum}

    \emph{Hint:} \begin{inparaenum}[(i)]
        \item Some subsequence $\{f_{n_i}\}$ converges at all rational points $r$, say, to $f(r)$. 
        \item Define $f(x)$, for any $x \in \R^1$, to be $\sup f(r)$, the sup being taken over all $r \leq x$.
        \item Show that $f_{n_i}(x) \rightarrow f(x)$ at every $x$ at which $f$ is continuous. (This is where monotonicity is strongly used.) 
        \item A subsequence of $\{f_{n_i}\}$ converges at every point of discontinuity of $f$ since there are at most countably many such points. This proves (a). To prove (b), modify your proof of (iii) appropriately.
    \end{inparaenum}
\end{myexercise}


\begin{myexercise}
    \label{ex:7.14}
    Let $f$ be a continuous real function on $\R^1$ with the following properties:
    $0 \leq f(t) \leq 1$, $f(t + 2) = f(t)$ for every $t$, and
    \begin{equation*}
        f(t) = \left\{ 
            \begin{array}{ll}
                0 & \left( 0 \leq t \leq \frac{1}{3} \right) \\
                1 & \left( \frac{2}{3} \leq t \leq 1 \right) . \\
            \end{array}
         \right.
    \end{equation*}
    Put $\Phi(t) = (x(t), y(t))$, where 
    \begin{equation*}
        x(t) = \sum_{n=1}^{\infty} 2^{-n} f(3^{2n-1} t), \quad 
        y(t) = \sum_{n=1}^{\infty} 2^{-n} f(3^{2n} t).
    \end{equation*}
    Prove that $\Phi$ is \myKeywordblue{continuous} and that $\Phi$ maps $I=[0,1]$ \myKeywordblue{onto} the unit square $I^2 \subset \R^2$.
    In fact, show that $\Phi$ maps the Cantor set onto $I^2$.

    \emph{Hint:} Each $(x_0, y_0) \in I^2$ has the form 
    \begin{equation*}
        x_0 = \sum_{n=1}^{\infty} 2^{-n} a_{2n-1}, \quad 
        y_0 = \sum_{n=1}^{\infty} 2^{-n} a_{2n}.
    \end{equation*}
    where each $a_i$ is 0 or 1. 
    If 
    \begin{equation*}
        t_0 = \sum_{i=1}^{\infty} 3^{-i-1}(2a_i)
    \end{equation*}
    show that $f(3^k t_0) = a_k$, and hence that $x(t_0)=x_0$, $y(t_0)=y_0$.

    (This simple example of a so-called ``space-filling curve'' is due to I. J. Schoenberg, \emph{Bull. A.M.S.}, vol. 44, 1938, pp. 519.)
\end{myexercise}


\begin{myexercise}
    \label{ex:7.15}
    Suppose $f$ is a real continuous function on $\R^1$,$f_n(t) =f(nt)$ for $n =1, 2, 3, ...$ , and
    $\{f_n\}$ is equicontinuous on $[0, 1]$. 
    What conclusion can you draw about $f$?
\end{myexercise}

\begin{myexercise}
    \label{ex:7.16}
    Suppose $\{f_n\}$ is an equicontinuous sequence of functions on a compact set $K$, and $\{f_n\}$ converges pointwise on $K$. 
    Prove that $\{f_n\}$  converges uniformly on K.
\end{myexercise}


\begin{myexercise}
    \label{ex:7.17}
    Define the notions of uniform convergence and equicontinuity for mappings into any metric space. 
    Show that Theorems \ref{thm:7.9} and \ref{thm:7.12} are valid for mappings into any metric space, 
    that Theorems \ref{thm:7.8} and \ref{thm:7.11} are valid for mappings into any complete metric space, 
    and that Theorems \ref{thm:7.10}, \ref{thm:7.16}, \ref{thm:7.17}, \ref{thm:7.24}, and \ref{thm:7.25} hold for vector-valued functions, that is, for mappings into any $\R^k$.
\end{myexercise}


\begin{myexercise}
    \label{ex:7.18}
    Let $\{f_n\}$ be a uniformly bounded sequence of functions which are Riemann-integrable on $[a, b]$, and put
    \begin{equation*}
        F_n (x) = \int_{a}^{x} f_n(t) \d t \quad 
        (a \leq x \leq b).
    \end{equation*}
    Prove that there exists a subsequence $\{F_{n_k}\}$ which converges uniformly on $[a,b]$.
\end{myexercise}


\begin{myexercise}
    \label{ex:7.19}
    Let $K$ be a compact metric space, let $S$ be a subset of $\mathscr{C}(K)$. 
    Prove that $S$ is compact 
    (with respect to the metric defined in Section \ref{mydef:7.14}) 
    if and only if $S$ is uniformly closed, pointwise bounded, and equicontinuous. 
    (If $S$ is not equicontinuous, then $S$ contains a sequence which has no equicontinuous subsequence, hence has no subsequence that converges uniformly on $K$.)
\end{myexercise}


\begin{myexercise}
    \label{ex:7.20}
    If $f$ is continuous on $[0,1]$ and if 
    \begin{equation*}
        \int_{0}^{1} f(x) x^n \d x = 0 \quad 
        (n = 0,1,2,\dots),
    \end{equation*}
    prove that $f(x)=0$ on $[0,1]$ .
    
    \emph{Hint:} The integral of the product of $f$ with any polynomial is zero.
    Use the Weierstrass theorem to show that $\int_{0}^{1}f^2(x) \d x = 0$.
\end{myexercise}


\begin{myexercise}
    \label{ex:7.21}
    Let $K$ be the unit circle in the complex plane 
    (i.e., the set of all $z$ with $| z | = 1$ ), 
    and let $\mathscr{A}$ be the algebra of all functions of the form
    \begin{equation*}
        f(e^{i \theta}) = \sum_{n=0}^{N} c_n e^{i n \theta} \quad 
        (\theta \text{ real}).
    \end{equation*}
    Then $\mathscr{A}$ separates points on $K$ and $\mathscr{A}$ vanishes at no point of $K$, but nevertheless there are continuous functions on $K$ which are not in the uniform closure of $\mathscr{A}$. 

    \emph{Hint:} For every $f \in \mathscr{A}$
    \begin{equation*}
        \int_{0}^{2\pi} f(e^{i\theta})e^{i\theta} \d \theta = 0 ,
    \end{equation*}
    and this is also true for every $f$ in the closure of $\mathscr{A}$.
\end{myexercise}


\begin{myexercise}
    \label{ex:7.22}
    Assume $f \in \mathscr{R}(\alpha)$ on $[a, b]$, and prove that there are polynomials $P_n$ such that
    \begin{equation*}
        \lim_{n \to \infty} \int_{a}^{b} \left| f - P_n \right|^2 \d \alpha = 0 . 
    \end{equation*}
    (Compare with Exercise \ref{ex:6.12}.)
\end{myexercise}


\begin{myexercise}
    \label{ex:7.23}
    Put $P_0 = 0$, and define, for $n=0,1,2,\dots$,
    \begin{equation*}
        P_{n+1}(x) = P_n(x) + \frac{x^2 - P_n^2(x)}{2}.
    \end{equation*}
    Prove that 
    \begin{equation*}
        \lim_{n \to \infty} P_n(x) = \left| x \right| ,
    \end{equation*}
    \myKeywordblue{uniformly} on $[-1,1]$.

    (This makes it possible to prove the Stone-Weierstrass theorem without first proving Theorem \ref{thm:7.26})

    \emph{Hint:} Use the identity 
    \begin{equation*}
        |x| - P_{n+1}(x) = \left[ |x| - P_n(x) \right]\left[ 1-\frac{|x|+P_n(x)}{2} \right]
    \end{equation*}
    to prove that $0 \leq P_n(x) \leq P_{n+1}(x) \leq |x|$ if $|x| \leq 1$, and that 
    \begin{equation*}
        |x| - P_n(x) \leq |x| \left( 1-\frac{|x|}{2} \right)^n < \frac{2}{n+1}
    \end{equation*}
    if $|x| \leq 1$.
\end{myexercise}


\begin{myexercise}
    \label{ex:7.24}
    Let $X$ be a metric space, with metric $d$. 
    Fix a point $a \in X$. 
    Assign to each $p \in X$ the function $f_p$ defined by
    \begin{equation*}
        f_p(x) = d(x, p) - d(x, a) \quad (x \in X).
    \end{equation*} 
    Prove that $|f_p (x)| \leq d(a,p)$ for all $x \in X$, 
    and that therefore $f_p \in \mathscr{C}(X)$.
    
    Prove that
    \begin{equation*}
        \left\| f_p-f_q \right\| = d(p, q)
    \end{equation*}
    for all $p, q \in X$. 
    
    If $\Phi(p) = f_p$ it follows that $\Phi$ is an \myKeywordblue{isometry} (a distance-preserving mapping) of $X$ onto $\Phi(X) \subset \mathscr{C}(X)$.
    
    Let $Y$ be the closure of $\Phi(X)$ in $\mathscr{C}(X)$. 
    Show that $Y$ is complete.
    
    \emph{Conclusion: $X$ is isometric to a dense subset of a complete metric space $Y$.}
    (Exercise \ref{ex:3.24}, contains a different proof of this.)
\end{myexercise}


\begin{myexercise}
    \label{ex:7.25}
    Suppose $\phi$ is a continuous bounded real function in the strip defined by $0 \leq x \leq 1$, $- \infty < y < \infty$. 
    Prove that the initial-value problem
    \begin{equation*}
        y' = \phi(x,y), \quad 
        y(0) = c
    \end{equation*}
    has a solution. 
    (Note that the hypotheses of this existence theorem are less stringent than those of the corresponding uniqueness theorem; see Exercise \ref{ex:5.27}.)

    \emph{Hint:} Fix $n$. For $i = 0, ... , n$, put $x_i = i/n$. Let $f_n$ be a continuous function on $[0, 1]$ such that $f_n(0) = c$,
    \begin{equation*}
        f'_n(t) = \phi(x_i, f_n(x_i)) \quad 
        \text{ if } 
        x_i < t < x_{i+1} ,
    \end{equation*}
    and put 
    \begin{equation*}
        \Delta_n(t) = f'_n(t) - \phi(t, f_n(t)),
    \end{equation*}
    except at he points $x_i$, where $\Delta_n(t) = 0$.
    Then 
    \begin{equation*}
        f_n(x) = c + \int_{0}^{x} \left[ \phi(t,f_n(t)) + \Delta_n(t) \right] \d t.
    \end{equation*}
    Choose $M < \infty$ so that $|\phi| \leq M$.
    Verify the following assertions.
    \begin{asparaenum}[(a)]
        \item $|f'_n| \leq M$, $|\Delta_n| \leq 2M$, $\Delta_n \in \mathscr{R}$ , and $|f_n| \leq |c| + M = M_1$, say, on $[0,1]$, for all $n$.
        \item $\{f_n\}$ is equicontinuous on $[0,1]$, since $|f'_n| \leq M$.
        \item Some $\{f_{n_k}\}$ converges to some $f$, uniformly on $[0,1]$.
        \item Since $\phi$ is uniformly continuous on the rectangle $0 \leq x \leq 1$, $|y| \leq M_1$,
        \begin{equation*}
            \phi(t,f_{n_k}(t)) \rightarrow
            \phi(t,f(t)) 
        \end{equation*}
        uniformly on $[0,1]$ .
        \item $\Delta_n(t) \rightarrow 0$ uniformly on $[0,1]$, since
        \begin{equation*}
            \Delta_n(t) = \phi(x_i,f_{n}(x_i)) - \phi(t,f_{n}(t))
        \end{equation*}
        int $(x_i, x_{i+1})$ .
        \item Hence 
        \begin{equation*}
            f(x) = c + \int_{0}^{x} \phi (t, f(t)) \d t .
        \end{equation*}
        This $f$ is a solution of the given problem.
    \end{asparaenum}
\end{myexercise}


\begin{myexercise}
    \label{ex:7.26}
    Prove an analogous existence theorem for the initial-value problem
    \begin{equation*}
        \mathbf{y' = \Phi(x, y)}, \quad    \mathbf{y(0) = c},
    \end{equation*}
    where now $\mathbf{c} \in \R^{k}$, $\mathbf{y} \in \R^{k}$, and $\mathbf{\Phi}$ is a continuous bounded mapping of the part of $\R^{k+1}$ defined by $0 \leq x \leq 1$, $\mathbf{y} \in \R^k$ into $\R^k$. 
    (Compare Exercise \ref{ex:5.28}.) 
    \emph{Hint:} Use the vector-valued version of Theorem \ref{thm:7.25}.
\end{myexercise}