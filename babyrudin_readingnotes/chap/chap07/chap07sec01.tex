% chap07sec01
\section{Discussion of main problem}

\begin{mydef}
    \label{mydef:7.1}
    Suppose $\sequence{f_n}$, $n = 1,2,3,\dots,$ is a sequence of functions
    defined on a set $E$,
    and suppose that the sequence of numbers $\sequence{f_n\{x\}}$ converges
    for every $x \in E$.
    We can then define a function $f$ by
    \begin{equation}
        \label{eq:7.1}
        f(x) = \lim_{n \to \infty} f_n (x)
        \quad
        (x \in E).
    \end{equation}

    Under these circumstances we say that $\sequence{f_n}$ converges on $E$ and that $f$ is the \emph{limit},
    or the \emph{limit function}, of $\sequence{f_n}$.
    Sometimes we shall use a more descriptive terminology and shall say that ``$\sequence{f_n}$ converges to $f$ \emph{pointwise} on $E$'' if (\ref{eq:7.1}) holds.
    \mybox{pointwise convergence 逐点收敛}

    Similarly, if $\sum f_n(x)$ converges for every $x \in E$, and if we define
    \begin{equation}
        \label{eq:7.2}
        f(x) = \sum_{n=1}^{\infty} f_n (x)
        \quad
        (x \in E).
    \end{equation}
    the function $f$ is called the \emph{sum} of the series $\sum f_n$.

    The main problem which arises is to determine whether important
    properties of functions are preserved under the limit operations (\ref{eq:7.1}) and (\ref{eq:7.2}).
    For instance, if the functions $f_n$. are continuous, or differentiable, or integrable,
    is the same true of the limit function? What are the relations between $f'_n$ and $f'$,
    say, or between the integrals of $f_n$. and that of $f$?

    To say that $f$ is continuous at a limit point $x$ means
    \begin{equation*}
        \lim_{t \to x} f(t) = f(x).
    \end{equation*}

    Hence, to ask whether the limit of a sequence of continuous functions is continuous is the same as to ask whether
    \begin{equation}
        \label{eq:7.3}
        \lim_{t \to x}  \lim_{n \to \infty}  f_n(t) =
        \lim_{n \to \infty}  \lim_{t \to x}  f_n(t),
    \end{equation}

    i.e., whether the order in which limit processes are carried out is immaterial.
    On the left side of (\ref{eq:7.3}), we first let $n \rightarrow \infty$, then $t \rightarrow x$;
    on the right side, $t \rightarrow x$ first, then $n \rightarrow \infty$.
    \mybox{求极限的顺序可否改变}

    We shall now show by means of several examples that limit processes
    cannot in general be interchanged without affecting the result.
    Afterward, we shall prove that under certain conditions the order in which limit operations are carried out is immaterial.

    Our first example, and the simplest one, concerns a ``double sequence.''
\end{mydef}

\begin{newexample}
    For $m = 1,2,3,\dots,n = 1,2,3,...$, let
    \begin{equation*}
        s_{m,n} = \frac{m}{m+n}.
    \end{equation*}
    Then, for every fixed $n$,
    \begin{equation*}
        \lim_{m \to \infty} s_{m,n} = 1,
    \end{equation*}
    so that
    \begin{equation}
        \label{eq:7.4}
        \lim_{m \to \infty} \lim_{n \to \infty} s_{m,n} = 1.
    \end{equation}
    On the other hand, for every fixed $m$,
    \begin{equation*}
        \lim_{n \to \infty} s_{m,n} = 0,
    \end{equation*}
    so that
    \begin{equation}
        \label{eq:7.5}
        \lim_{m \to \infty} \lim_{n \to \infty} s_{m,n} = 0.
    \end{equation}
\end{newexample}
\mybox{本例是一个二元函数数列(double sequence)收敛至不同值的例子}

\begin{newexample}
    Let
    \begin{equation*}
        f_n (x) = \frac{x^2}{(1+x^2)^n}
        \quad
        (x \text{ real }; n = 0,1,2,...),
    \end{equation*}
    and consider
    \begin{equation}
        \label{eq:7.6}
        f(x) =
        \sum_{n=0}^{\infty} f_n (x) =
        \sum_{n=0}^{\infty} \frac{x^2}{(1+x^2)^n} .
    \end{equation}
    Since $f_n(0)=0$, we have $f(0) = 0$.
    For $x \neq 0$, the last series in (\ref{eq:7.6}) is a convergent
    geometric series with sum $1+x^2$ (Theorem \ref{thm:3.26}).
    Hence
    \begin{equation}
        \label{eq:7.7}
        f(x) = \left\{
        \begin{array}{ll}
            0       & (x \neq 0) \\
            1 + x^2 & (x =    0) \\
        \end{array}
        \right.
    \end{equation}
    so that a convergent series of continuous functions may have a discontinuous sum.
\end{newexample}
\mybox{一个连续函数的收敛级数可能有一个不连续的和}

\begin{newexample}
    For $m = 1,2,3,\dots$, put
    \begin{equation*}
        f_m(x) = \lim_{n \to \infty} (\cos m! \pi x)^{2n}.
    \end{equation*}
    When $m!x$ is an integer, $f_m(x) = 1$.
    For all other values of $x$, $f_m(x) = 0$.
    Now let
    \begin{equation*}
        f(x) = \lim_{m \to \infty} f_m (x).
    \end{equation*}
    For irrational $x$, $f_m(x) = 0$ for every $m$; hence $f(x) = 0$.
    For rational $x$, say $x = p/q$, where $p$ and $q$ are integers,
    we see that $m!x$ is an integer if $m \geq q$, so that $f(x) = 1$.
    Hence
    \begin{equation}
        \label{eq:7.8}
        \lim_{m \to \infty} \lim_{n \to \infty} (\cos m!\pi x)^{2n} =
        \left\{
        \begin{array}{ll}
            0 & (x \text{irrational}), \\
            1 & (x \text{rational}).
        \end{array}
        \right.
    \end{equation}
    We have thus obtained an everywhere discontinuous limit function, which
    is not Riemann-integrable (Exercise \ref{ex:6.4}).
\end{newexample}
\mybox{这个函数的有理数点与无理数点收敛至不同极限

    本例为狄利克雷Direchlet函数, 该函数不是黎曼可积的.
}

\begin{newexample}
    \label{newexample:7.5}
    Let
    \begin{equation}
        \label{eq:7.9}
        f_n(x) = \frac{\sin nx}{\sqrt{n}}
        \quad
        (x \text{ real}, n = 1,2,3,\dots),
    \end{equation}
    and
    \begin{equation*}
        f(x) = \lim_{n \to \infty} f_n (x) = 0.
    \end{equation*}
    Then $f'(x) = 0$, and
    \begin{equation*}
        f'_n (x) = \sqrt{n} \cos nx,
    \end{equation*}
    so that $\sequence{f'_n}$ does not converge to $f'$.
    For instance
    \begin{equation*}
        f'_n(0) = \sqrt{n} \rightarrow +\infty
    \end{equation*}
    as $n \rightarrow \infty$, whereas $f'(0) = 0$.
\end{newexample}

\begin{newexample}
    \label{newexample:7.6}
    Let
    \begin{equation}
        \label{eq:7.10}
        f_n(x) = n^2 x(1-x^2)^n
        \quad
        (0 \leq x \leq 1, n = 1,2,3,\dots).
    \end{equation}
    For $0 < x \leq 1$, we have
    \begin{equation*}
        \lim_{n \to \infty} f_n (x) = 0,
    \end{equation*}
    by Theorem \ref{thm:3.20}(d). Since $f_n(0) = 0$, we see that
    \begin{equation}
        \label{eq:7.11}
        \lim_{n \to \infty} f_n (x) = 0
        \quad
        (0 \leq x \leq 1).
    \end{equation}
    A simple calculation shows that
    \begin{equation*}
        \int_{0}^{1} x(1-x^2)^n \d x = \frac{1}{2n+2}.
    \end{equation*}
    Thus, in spite of (\ref{eq:7.11}),
    \begin{equation*}
        \int_{0}^{1} f_n(x) \d x = \frac{n^2}{2n+2} \rightarrow +\infty
    \end{equation*}
    as $n \rightarrow \infty$.

    If, in (\ref{eq:7.10}), we replace $n^2$ by $n$,
    (\ref{eq:7.11}) still holds, but we now have
    \begin{equation*}
        \lim_{n \to \infty} \int_{0}^{1} f_n (x) \d x =
        \lim_{n \to \infty} \frac{n}{2n+2} = \frac{1}{2},
    \end{equation*}
    whereas
    \begin{equation*}
        \int_{0}^{1} \left\{ \lim_{n \to \infty} f_n (x) \right\} \d x = 0.
    \end{equation*}
    Thus the limit of the integral need not be equal to the integral of the limit, even if both are finite.
\end{newexample}
\mybox{积分的极限不必等于极限的积分}
After these examples, which show what can go wrong if limit processes
are interchanged carelessly,
we now define a new mode of convergence,
stronger than pointwise convergence as defined in Definition {\ref{mydef:7.1}},
which will enable us to arrive at positive results.
\mybox{由于现有的逐点收敛存在不足, 因此需要提出一种更强的收敛. }
