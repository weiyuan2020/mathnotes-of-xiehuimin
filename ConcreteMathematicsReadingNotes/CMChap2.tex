\documentclass[mode=geye]{elegantnote}
%!TEX program=xelatex
\usepackage{amsmath}
\usepackage{amssymb}
\usepackage{amsthm}
\usepackage{hyperref}

\newtheorem{exercise}{练习}
\newtheorem{answer}{题目解答}
\newtheorem{solve}{解}

\title{具体数学阅读笔记-chap2}
\author{weiyuan}
\date{2022-07-04}
\begin{document}
\maketitle
\section{求和}
\subsection{求和符号}
\subsection{求和与递归式 Sums and recurrences}
和式
\begin{equation}
    S_n = \sum_{k=0}^{n}a_k
\end{equation}
等价于递归式
\begin{equation}
    \left\{
        \begin{array}{l}
            S_0 = a_0 \\
            S_n = S_{n-1}+a_n,\quad n>0.
        \end{array}
    \right.
\end{equation}
若$ a_n = const.+ k\cdot n $ , 则有
\begin{equation}\label{ReceEasy}
    \left\{
    \begin{aligned}
        R_0 &= \alpha \\
        R_n &= R_{n-1} + \beta + \gamma n, \quad n>0
    \end{aligned}
    \right.
\end{equation}
\begin{align*}
    R_1 &= R_0 + \beta + \gamma \\
    R_2 &= R_0 + 2\beta + 3\gamma \\
        &\vdots \\
    R_n &= A(n)\alpha + B(n)\beta + C(n)\gamma
\end{align*}
\begin{equation}\label{RecuEasyABC}
    R_n = A(n)\alpha + B(n)\beta + C(n)\gamma
\end{equation}
repertoire method
令$ R_n=1 $ , 则$ \alpha=1, \beta=0, \gamma=0 $ , 
\begin{equation*}
    A(n)=1
\end{equation*}
令$ R_n=n $ , 则$ \alpha=0, \beta=1, \gamma=0 $ , 
\begin{equation*}
    B(n)=n
\end{equation*}
令$ R_n=n^2 $ , 则$ \alpha=0, \beta=-1, \gamma=2 $ , 
\begin{equation*}
    C(n)= \frac{n(n+1)}{2}
\end{equation*}

\begin{example}
    \begin{equation*}
        \sum_{k=0}^{n}(a+b k)
    \end{equation*}
\end{example}
\begin{solve}
    \begin{equation*}
        \left\{
            \begin{array}{l}
                R_0 = a \\
                R_n = R_{n-1} + a+bn\\
            \end{array}
        \right.
    \end{equation*}
    \begin{equation*}
        \left\{
            \begin{array}{l}
                R_0 = \alpha \\
                R_n = R_{n-1} + \beta + \gamma n\\
            \end{array}
        \right.
    \end{equation*}
    $ \alpha = \beta =a, \gamma = b $ 
    \begin{align*}
        A(n)\alpha + B(n)\beta + C(n)\gamma 
        &= a A(n) + a B(n) + b C(n)\\
        &= a + a n + b \frac{n(n+1)}{2}\\
        &= a(n+1)+\frac{bn(n+1)}{2}
    \end{align*}
\end{solve}

对上述递归情况进行推广
\begin{equation}\label{RecuHard}
    \left\{
    \begin{aligned}
        R_0 &= \alpha \\
        R_n &= R_{n-1} + \beta + \gamma n + \delta n^2, \quad n>0
    \end{aligned}
    \right.
\end{equation}
\begin{equation}\label{RecuHardABC}
    R_n = A(n)\alpha + B(n)\beta + C(n)\gamma +D(n) \delta
\end{equation}
$ \delta=0 $ 时 (\ref{RecuHard})与(\ref{ReceEasy})一致, 说明
$ A(n), B(n), C(n) $ 不变\\
$ R_n = n^3 $ 
\begin{align*}
    R_n = R_{n-1} &= n^3 - (n-1)^3 \\
    &= 3n^2 - 3n + 1
\end{align*}
解得$ \alpha=0, \beta=1, \gamma=-3, \delta=3 $ 
\begin{align*}
    n^3 &= B(n)-3C(n)+3D(n)\\
    &= n-3 \frac{n(n+1)}{2}+3D(n)
\end{align*}
\begin{align*}
    3D(n) &= n^3 - n + 3 \frac{n(n+1)}{2} \\
    &= n(n+1) \left[(n-1)+\frac{3}{2}\right] \\
    &= n(n+1)(n+\frac{1}{2})\\
    D(n) &= \frac{1}{3}\left((n+1)(n+\frac{1}{2})n\right)
\end{align*}

\subsection{求和式处理}
\subsection{多重求和}

\subsubsection{Exercise 1}
\begin{equation}
    A = \begin{bmatrix}
        a_1 a_1 & a_1 a_2 & \cdots & a_1 a_n \\
        a_2 a_1 & a_2 a_2 & \cdots & a_2 a_n \\
        \vdots  & \vdots  &        & \vdots  \\
        a_n a_1 & a_n a_2 & \cdots & a_n a_n \\
    \end{bmatrix}
\end{equation}
求 $ S_{\triangleleft} = \sum_{1\leqslant j \leqslant k \leqslant n} a_j a_k $ \footnote{下三角形矩阵的符号是一个右上部分的直角三角形, 目前我还不会输入}

\begin{solve}
    $ \because a_j a_k = a_k a_j $ , $ \therefore  $ 矩阵A沿主对角线对称, $ S_{\triangleleft} = S_{\triangleright} $ .
    \begin{equation*}
        [1\leqslant j \leqslant k \leqslant n] + [1\leqslant k \leqslant j \leqslant n] =[1\leqslant j , k \leqslant n] + [1\leqslant j = k \leqslant n]
    \end{equation*}
    \begin{align*}
        2S_{\triangleleft} &= S_{\triangleleft}+S_{\triangleright} = S_{A}+S_{diag(A)} \\
        &= \sum_{1\leqslant j , k \leqslant n} a_j a_k + \sum_{1\leqslant j = k \leqslant n} a_j a_k\\
        &= \left(\sum_{j=1}^{n} a_j\right)\left(\sum_{k=1}^{n} a_k\right) + \sum_{k=1}^{n} a_k^2\\
        &= \left(\sum_{k=1}^{n} a_k\right)^2 + \sum_{k=1}^{n} a_k^2
    \end{align*}
    $ \therefore S_{\triangleleft} = \frac{1}{2}[\left(\sum_{k=1}^{n} a_k\right)^2 + \sum_{k=1}^{n} a_k^2] $ 
\end{solve}

\subsubsection{Exercise 2}

\begin{equation}
    S = \sum_{1\leqslant j < k \leqslant n} (a_k-a_j)(b_k-b_j)
\end{equation}

\begin{solve}
    交换$ j,k $ 仍有对称性.\\
    \begin{equation*}
        S = \sum_{1\leqslant j < k \leqslant n} (a_k-a_j)(b_k-b_j)
          = \sum_{1\leqslant j < k \leqslant n} (a_j-a_k)(b_j-b_k)
    \end{equation*}
    \begin{equation*}
        [1\leqslant j < k \leqslant n] + [1\leqslant k < j \leqslant n] =[1\leqslant j , k \leqslant n] - [1\leqslant j = k \leqslant n]
    \end{equation*}
    \begin{align*}
        2S  &= 2\sum_{1\leqslant j < k \leqslant n} (a_k-a_j)(b_k-b_j)\\
            &= \sum_{1\leqslant j < k \leqslant n} (a_k-a_j)(b_k-b_j) + \sum_{1\leqslant k < j \leqslant n} (a_k-a_j)(b_k-b_j)\\
            &= \sum_{1\leqslant j , k \leqslant n} (a_k-a_j)(b_k-b_j) - \sum_{1\leqslant j = k \leqslant n} (a_k-a_j)(b_k-b_j)\\
            & (a_j-a_k=0, b_j-b_k=0, [j=k])
            &= \sum_{1\leqslant j , k \leqslant n} (a_k b_k - a_j b_k - a_k b_j + a_j b_j) \\
            &= \sum_{j=1}^{n} \sum_{k=1}^{n} a_k b_k - \sum_{j=1}^{n} \sum_{k=1}^{n} a_j b_k - \sum_{j=1}^{n} \sum_{k=1}^{n} a_k b_j + \sum_{j=1}^{n} \sum_{k=1}^{n} a_j b_j \\
            &= n\sum_{k=1}^{n} a_k b_k - \sum_{j=1}^{n} \sum_{k=1}^{n} a_j b_k - \sum_{j=1}^{n} \sum_{k=1}^{n} a_k b_j + n \sum_{j=1}^{n}  a_j b_j \\
            &= 2 n \sum_{k=1}^{n} a_k b_k - 2 \sum_{j=1}^{n} a_j  \sum_{k=1}^{n} b_k 
    \end{align*}
    \begin{equation*}
        S = n \sum_{k=1}^{n} a_k b_k - \left(\sum_{k=1}^{n} a_k \right) \left(\sum_{k=1}^{n} b_k \right)
    \end{equation*}
\end{solve}
对上式结果重新排序得
\begin{equation*}
    \left(\sum_{k=1}^{n} a_k \right) \left(\sum_{k=1}^{n} b_k \right) = n \sum_{k=1}^{n} a_k b_k - \sum_{1\leqslant j < k \leqslant n} (a_k-a_j)(b_k-b_j)
\end{equation*}

\begin{theorem}
    切比雪夫单调不等式 (Chebyshec's monotonic inequality)

\begin{equation*}
    \begin{array}{lll}
        \left(\sum_{k=1}^{n} a_k \right) \left(\sum_{k=1}^{n} b_k \right)   &\leqslant n \sum_{k=1}^{n} a_k b_k & \quad a_1\leqslant a_2\leqslant \dots\leqslant a_n, \text{and } b_1\leqslant b_2\leqslant \dots\leqslant b_n \\        
        & & \quad a_1\geqslant a_2\geqslant \dots\geqslant a_n, \text{and } b_1\geqslant b_2\geqslant \dots\geqslant b_n\\
        \left(\sum_{k=1}^{n} a_k \right) \left(\sum_{k=1}^{n} b_k \right)   &\geqslant n \sum_{k=1}^{n} a_k b_k & \quad a_1\leqslant a_2\leqslant \dots\leqslant a_n, \text{and } 
        b_1\geqslant b_2\geqslant \dots\geqslant b_n\\
        & & \quad a_1\geqslant a_2\geqslant \dots\geqslant a_n, \text{and }
        b_1\leqslant b_2\leqslant \dots\leqslant b_n \\        
    \end{array}
\end{equation*}
\end{theorem}
    一般来说,如果$ a_1 \leqslant a_2 \leqslant \dots \leqslant a_n $ 且 $ p $ 是$ \{1,\dots,n \} $ 的一个排列。\\
    那么不难证明:\\
    当$ b_{p(1)}\leqslant \dots \leqslant b_{p(n)} $ 时$ \sum_{k=1}^{n} a_k b_{P(k)} $ 最大.\\
    当$ b_{p(1)}\geqslant \dots \geqslant b_{p(n)} $ 时$ \sum_{k=1}^{n} a_k b_{P(k)} $ 最小.

    \begin{equation*}
        \sum_{k\in K} a_k = \sum_{P(k)\in K} a_{P(k)} 
    \end{equation*}
    $ P(k) $ 为这些整数的任意一个排列。
    \begin{equation*}
        f: J\rightarrow K, \quad j\in J \quad f(j) \in K
    \end{equation*}
    \begin{equation*}
        \sum_{j\in J}a_{f(j)} = \sum_{k\in K} a_k \quad \#f^-(k)
    \end{equation*}
    式中 $ \#f^-(k) $ 表示集合$ f^-(k) = \{j | f(j)=k\} $ 中元素的个数
    \begin{equation*}
        \sum_{j\in J}[f(j)=k] = \#f^-(k)
    \end{equation*}
    \begin{equation*}
        \sum_{j\in J}a_{f(j)} = \sum_{\begin{array}{l}j\in J\\ k\in K\\ \end{array}}a_k[f(j)=k] = \sum_{k\in K}a_k \sum_{j\in J} [f(j)=k]
    \end{equation*}
    若有 $ \#f^-(k) = 1 $ (一一对应)\footnote{这里还不太理解}
    \begin{equation*}
        \sum_{j\in J} a_{f(j)} = \sum_{f(j)\in K}a_{f(j)} = \sum_{k\in K} a_k
    \end{equation*}

\subsubsection{Exercise 3}
\begin{equation*}
    S_n = \sum_{1\leqslant j < k\leqslant n}\frac{1}{k-j}
\end{equation*}
首先写出前几项,尝试寻找规律:
\begin{align*}
    S_1 &= 0\\
    S_2 &= \frac{1}{2-1} = 1\\
    S_3 &= \frac{1}{2-1}+\frac{1}{3-1}+\frac{1}{3-2} = \frac{5}{2}\\
    S_4 &= \frac{1}{2-1}+\frac{1}{3-1}+\frac{1}{4-1}+\frac{1}{3-2}+\frac{1}{4-2}+\frac{1}{4-3} = \frac{13}{3}
\end{align*}
\begin{solve}
    1. 先对$ j $ 求和
    \begin{align*}
        S_n 
        &= \sum_{1\leqslant k \leqslant n} \sum_{1\leqslant j < k}\frac{1}{k-j}\\
        &= \sum_{1\leqslant k \leqslant n} \sum_{1\leqslant (k-j) < k}\frac{1}{k-(k-j)} \quad j\rightarrow (k-j) \\
        &= \sum_{1\leqslant k \leqslant n} \sum_{0<j\leqslant k-1}\frac{1}{j} \\
        &= \sum_{1\leqslant k \leqslant n} H_{k-1} \quad(H_k\text{为调和级数})\\
        &= \sum_{1\leqslant k+1 \leqslant n} H_{k}\quad k\rightarrow k+1 \\
        &= \sum_{0\leqslant k < n} H_{k}
    \end{align*}
    2. 先对$ k $ 求和
    \begin{align*}
        S_n 
        &= \sum_{1\leqslant j \leqslant n} \sum_{j < k\leqslant n}\frac{1}{k-j}\\
        &= \sum_{1\leqslant j \leqslant n} \sum_{j < (k+j)\leqslant n}\frac{1}{(k+j)-j} \quad k\rightarrow (k+j) \\
        &= \sum_{1\leqslant j \leqslant n} \sum_{0<k\leqslant n-j}\frac{1}{k} \\
        &= \sum_{1\leqslant j \leqslant n} H_{n-j} \quad(H_k\text{为调和级数})\\
        &= \sum_{1\leqslant n-j \leqslant n} H_{k}\quad j\rightarrow n-j \\
        &= \sum_{0\leqslant j < n} H_{j}
    \end{align*}
    以上两种常用的求和顺序都无法得到这个多重求和的结果,我们需要转换思路.

    3. 先用$ k+j $ 替换$ k $ (先换元,再求和)
    \begin{align*}
        S_n 
        &= \sum_{1\leqslant j < (k+j)\leqslant n}\frac{1}{(k+j)-j} \quad k\rightarrow k+j \\
        &= \sum_{1\leqslant j < (k+j)\leqslant n}\frac{1}{k} \\
        &= \sum_{1\leqslant k \leqslant n} \sum_{1\leqslant j \leqslant n-k}\frac{1}{k} \quad\text{首先对}j\text{求和} \\
        &= \sum_{1\leqslant k \leqslant n} \frac{n-k}{k}\\
        &= \sum_{1\leqslant k \leqslant n} \left( \frac{n}{k}-1 \right) = n H_n - n
    \end{align*}
    综上可得$ \sum_{1\leqslant k\leqslant n}H_k = n H_n - n $ 
\end{solve}
代数:
$ k+f(j) $ , $ f $ 为任意函数.\\
用$ k-f(j) $ 替换$ k $ ,并对$ j $ 先求和较好。\\
几何:
$ S_n \;(n=4) $ 
\begin{equation*}
    \begin{array}{ccccc}
            & k=1   & k=2   & k=3   & k=4   \\
        j=1 & & \frac{1}{1} & +\frac{1}{2} & +\frac{1}{3} \\
        j=2 & &             & +\frac{1}{1} & +\frac{1}{2} \\
        j=3 & &             &              & +\frac{1}{1} \\
        j=4 & &             &              &              \\
    \end{array}
\end{equation*}
先对$ j $ 求和(按列) $ H_1 + H_2 + H_3 $ 
先对$ k $ 求和(按行) $ H_3 + H_2 + H_1 $ 
$ k\rightarrow k+j $ 按对角线求和 
% \begin{align*}
%     \frac{3}{1}+\frac{2}{2}+\frac{1}{3} &= \sum_{k=1}^{3}\frac{3-k}{k} \\ 
%     &= 3\sum_{k=1}^{3}\frac{1}{k}-\sum_{k=1}^{3} 1 \\
%     &= 3H_3 - 3
% \end{align*}
\begin{equation*}
    \sum_{k=1}^{n}\frac{n-k}{k} = n \sum_{k=1}^{n}\frac{1}{k}-\sum_{k=1}^{n} 1
\end{equation*}
$ nH_n-n $ ,$ n=4 $ 
\begin{align*}
    \frac{4}{1}+\frac{3}{2}+\frac{2}{3}+\frac{1}{4} &= \sum_{k=1}^{4}\frac{4-k}{k} \\ 
    &= 4\sum_{k=1}^{4}\frac{1}{k}-\sum_{k=1}^{4} 1 \\
    &= 4H_4 - 4
\end{align*}
\begin{equation*}
    4\left(1+\frac{1}{2}+\frac{1}{3}+\frac{1}{4}\right)-4 = \frac{4}{2}+\frac{4}{3}+\frac{4}{4}
\end{equation*}
\begin{equation*}
    \begin{array}{ccccc}
            & k-j=0   & k-j=1   & k-j=2   & k-j=3   \\
        j=1 & & \frac{1}{1} & +\frac{1}{2} & +\frac{1}{3} \\
        j=2 & & \frac{1}{1} & +\frac{1}{2} &              \\
        j=3 & & \frac{1}{1} &              &              \\
        j=4 & &             &              &              \\
    \end{array}
\end{equation*}

\subsection{General methods}
\subsection{Exercise 4}
求 $ \square_n = \sum_{0\leqslant k \leqslant n}k^2 $ , $ n\geqslant 0 $ 的封闭形式
\begin{align*}
    \sum_{k=0}^{n}k^2 
    &= \sum_{k=0}^{n}[(k+1)^2-2k-1] \\
    &= \sum_{k=1}^{n+1}k^2 - 2 \sum_{k=0}^{n}k - \sum_{k=0}^{n}1
\end{align*}
\begin{align*}
    0^2-(n+1)^2 &= -2 \sum_{k=0}^{n}k - (n+1) \\
    2\sum_{k=0}^{n} k &= (n+1)^2 - (n+1)\\
    \sum_{k=0}^{n} k &= \frac{(n+1)n}{2}\\
\end{align*}
上述运算没有告诉我们$ \square_n $ 的值,但却能推导出$ \sum_{k=0}^{n}k $ 的值。我们可以利用这种思路求解$ \square_n $ 。
\begin{align*}
    \sum_{k=0}^{n}\left[(k+1)^3-k^3\right] &= \sum_{k=0}^{n}\left[ 3k^2+3k+1 \right] \\
    (n+1)^3 - 0^3 &= 3 \sum_{k=0}^{n}k^2 + 3 \sum_{k=0}^{n}k +  \sum_{k=0}^{n}1 \\
    (n+1)^3 &= 3 \sum_{k=0}^{n}k^2 + 3 \frac{n(n+1)}{2} + (n+1)
\end{align*}
\begin{align*}
    3\sum_{k=0}^{n}k^2 &= (n+1)^3 - 3\frac{n(n+1)}{2}-(n+1) \\
    \sum_{k=0}^{n}k^2 &= \frac{1}{3}(n+1)\left((n+1)^2-\frac{3}{2}n-1\right)\\
    \sum_{k=0}^{n}k^2 &= \frac{1}{3}(n+1)(n+\frac{1}{2})n
\end{align*}

reference book list:\\
1. (CRC Tables) CRC Standard Mathematical Tables\\
2. Handbook of Mathematical Functions\\
3. Sloane. Handbook of Integer Sequences\\
software: \\
Axiom MACSYMA Maple Mathematica\\
my: Octave maxima
熟悉标准的信息源

方法3:建立成套方法

参考第二节的内容

方法4:用积分替换和式 $ \sum \rightarrow \int $ 
\begin{equation*}
    \square_n = 1\times1+1\times4+1\times9 +\dots+1\times n^2
\end{equation*}
该式近似等于0到$ n $ 之间曲线$ f(x)=x^2 $ 下的面积
\begin{align*}
    S &= \int_{0}^{n}x^2 dx \\
    &=\frac{n^3}{3}
\end{align*}
$ \square_n $ 近似等于 $ \frac{n^3}{3} $ 。
近似的误差$ E_n = \square_n - \frac{n^3}{3} $ 
\end{document}