\documentclass[mode=geye]{elegantnote}
%!TEX program=xelatex
\usepackage{amsmath}
\usepackage{amssymb}
\usepackage{amsthm}

\newtheorem{exercise}{练习}
\newtheorem{answer}{题目解答}
\newtheorem{solve}{解}

\title{具体数学阅读笔记-chap2}
\author{weiyuan}
\date{2022-07-04}
\begin{document}
\maketitle
\begin{equation}
    A = \begin{bmatrix}
        a_1 a_1 & a_1 a_2 & \cdots & a_1 a_n \\
        a_2 a_1 & a_2 a_2 & \cdots & a_2 a_n \\
        \vdots  & \vdots  &        & \vdots  \\
        a_n a_1 & a_n a_2 & \cdots & a_n a_n \\
    \end{bmatrix}
\end{equation}
求 $ S_{\triangleleft} = \sum_{1\leqslant j \leqslant k \leqslant n} a_j a_k $ \footnote{下三角形矩阵的符号是一个右上部分的直角三角形, 目前我还不会输入}

\begin{solve}
    $ \because a_j a_k = a_k a_j $ , $ \therefore  $ 矩阵A沿主对角线对称, $ S_{\triangleleft} = S_{\triangleright} $ .
    \begin{equation*}
        [1\leqslant j \leqslant k \leqslant n] + [1\leqslant k \leqslant j \leqslant n] =[1\leqslant j , k \leqslant n] + [1\leqslant j = k \leqslant n]
    \end{equation*}
    \begin{align*}
        2S_{\triangleleft} &= S_{\triangleleft}+S_{\triangleright} = S_{A}+S_{diag(A)} \\
        &= \sum_{1\leqslant j , k \leqslant n} a_j a_k + \sum_{1\leqslant j = k \leqslant n} a_j a_k\\
        &= \left(\sum_{j=1}^{n} a_j\right)\left(\sum_{k=1}^{n} a_k\right) + \sum_{k=1}^{n} a_k^2\\
        &= \left(\sum_{k=1}^{n} a_k\right)^2 + \sum_{k=1}^{n} a_k^2
    \end{align*}
    $ \therefore S_{\triangleleft} = \frac{1}{2}[\left(\sum_{k=1}^{n} a_k\right)^2 + \sum_{k=1}^{n} a_k^2] $ 
\end{solve}

\begin{equation}
    S = \sum_{1\leqslant j < k \leqslant n} (a_k-a_j)(b_k-b_j)
\end{equation}

\begin{solve}
    交换$ j,k $ 仍有对称性.\\
    \begin{equation*}
        S = \sum_{1\leqslant j < k \leqslant n} (a_k-a_j)(b_k-b_j)
          = \sum_{1\leqslant j < k \leqslant n} (a_j-a_k)(b_j-b_k)
    \end{equation*}
    \begin{equation*}
        [1\leqslant j < k \leqslant n] + [1\leqslant k < j \leqslant n] =[1\leqslant j , k \leqslant n] - [1\leqslant j = k \leqslant n]
    \end{equation*}
    \begin{align*}
        2S  &= 2\sum_{1\leqslant j < k \leqslant n} (a_k-a_j)(b_k-b_j)\\
            &= \sum_{1\leqslant j < k \leqslant n} (a_k-a_j)(b_k-b_j) + \sum_{1\leqslant k < j \leqslant n} (a_k-a_j)(b_k-b_j)\\
            &= \sum_{1\leqslant j , k \leqslant n} (a_k-a_j)(b_k-b_j) - \sum_{1\leqslant j = k \leqslant n} (a_k-a_j)(b_k-b_j)\\
            & (a_j-a_k=0, b_j-b_k=0, [j=k])
            &= \sum_{1\leqslant j , k \leqslant n} (a_k b_k - a_j b_k - a_k b_j + a_j b_j) \\
            &= \sum_{j=1}^{n} \sum_{k=1}^{n} a_k b_k - \sum_{j=1}^{n} \sum_{k=1}^{n} a_j b_k - \sum_{j=1}^{n} \sum_{k=1}^{n} a_k b_j + \sum_{j=1}^{n} \sum_{k=1}^{n} a_j b_j \\
            &= n\sum_{k=1}^{n} a_k b_k - \sum_{j=1}^{n} \sum_{k=1}^{n} a_j b_k - \sum_{j=1}^{n} \sum_{k=1}^{n} a_k b_j + n \sum_{j=1}^{n}  a_j b_j \\
            &= 2 n \sum_{k=1}^{n} a_k b_k - 2 \sum_{j=1}^{n} a_j  \sum_{k=1}^{n} b_k 
    \end{align*}
    \begin{equation*}
        S = n \sum_{k=1}^{n} a_k b_k - \left(\sum_{k=1}^{n} a_k \right) \left(\sum_{k=1}^{n} b_k \right)
    \end{equation*}
\end{solve}
对上式结果重新排序得
\begin{equation*}
    \left(\sum_{k=1}^{n} a_k \right) \left(\sum_{k=1}^{n} b_k \right) = n \sum_{k=1}^{n} a_k b_k - \sum_{1\leqslant j < k \leqslant n} (a_k-a_j)(b_k-b_j)
\end{equation*}

\begin{theorem}
    切比雪夫单调不等式 (Chebyshec's monotonic inequality)

\begin{equation*}
    \begin{array}{lll}
        \left(\sum_{k=1}^{n} a_k \right) \left(\sum_{k=1}^{n} b_k \right)   &\leqslant n \sum_{k=1}^{n} a_k b_k & \quad a_1\leqslant a_2\leqslant \dots\leqslant a_n, \text{and } b_1\leqslant b_2\leqslant \dots\leqslant b_n \\        
        & & \quad a_1\geqslant a_2\geqslant \dots\geqslant a_n, \text{and } b_1\geqslant b_2\geqslant \dots\geqslant b_n\\
        \left(\sum_{k=1}^{n} a_k \right) \left(\sum_{k=1}^{n} b_k \right)   &\geqslant n \sum_{k=1}^{n} a_k b_k & \quad a_1\leqslant a_2\leqslant \dots\leqslant a_n, \text{and } 
        b_1\geqslant b_2\geqslant \dots\geqslant b_n\\
        & & \quad a_1\geqslant a_2\geqslant \dots\geqslant a_n, \text{and }
        b_1\leqslant b_2\leqslant \dots\leqslant b_n \\        
    \end{array}
\end{equation*}

\end{theorem}

\end{document}