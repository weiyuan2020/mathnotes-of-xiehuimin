\documentclass[openany,twoside,scheme=chinese,fontset=none]{ctexrep}
\usepackage{geometry}
\geometry{
    paperheight=260mm,
    paperwidth=185mm,
    top=25mm,
    bottom=15mm,
    left=25mm, % 左侧留 5mm 装订线距离
    right=15mm
}

\setmainfont{XITS}  % 英文字体, Times 风格

\setCJKmainfont{Source Han Serif SC}[         % 方正书宋_GBK
    BoldFont=Source Han Serif SC Bold,  % 思源宋体粗体
    ItalicFont=FZKai-Z03                % 方正楷体_GBK
    ]
\setCJKsansfont{Source Han Sans SC}[             % 方正黑体_GBK
    BoldFont=Source Han Sans SC Bold    % 思源黑体粗体
    ]
\setCJKmonofont{FZFangSong-Z02}         % 方正仿宋_GBK

\setCJKfamilyfont{zhsong}{FZShuSong-Z01}
\setCJKfamilyfont{zhxbs}{Source Han Serif SC Bold}
\setCJKfamilyfont{zhdbs}{Source Han Serif SC Heavy}
\setCJKfamilyfont{zhhei}{FZHei-B01}
\setCJKfamilyfont{zhdh}{Source Han Sans SC Bold}
\setCJKfamilyfont{zhfs}{FZFangSong-Z02}
\setCJKfamilyfont{zhkai}{FZKai-Z03}

\newcommand{\songti}{\CJKfamily{zhsong}}
\newcommand{\xbsong}{\CJKfamily{zhxbs}}
\newcommand{\dbsong}{\CJKfamily{zhdbs}}
\newcommand{\heiti}{\CJKfamily{zhhei}}
\newcommand{\dahei}{\CJKfamily{zhdh}}
\newcommand{\fangsong}{\CJKfamily{zhfs}}
\newcommand{\kaishu}{\CJKfamily{zhkai}}


\usepackage{amsmath}
\usepackage{amsthm} % amsthm 与 ntheorem 冲突
\usepackage{amssymb}
\usepackage{hyperref} % \url
\usepackage{graphicx} % \includegraphics


\usepackage{enumerate} % 罗列专用宏包
\usepackage{fancybox} % 使用盒子
 


\usepackage{xcolor} % 使用颜色 color
\usepackage{booktabs} % 三线表宏包 toprule midrule bottomrule

% \usepackage{ntheorem} % 调整定理格式
% \theorembodyfont{\upshape} %正体
% \theorembodyfont{\itshape} %斜体
% 定理类环境默认的字体为斜体,可以通过 \theorembodyfont 来设置字体,其中 \upshape 为正体,\itshape 为斜体。\theorembodyfont 必须放在 newthorem 的上方。



\theoremstyle{plain} % default
\newtheorem{thm}{Theorem}[chapter] % 如果不采用章节号做前缀, 则不用[section]
\newtheorem{myLemma}[thm]{Lemma}% 这句定义使得 lem 环境和 thm 共享编号

\theoremstyle{definition} % definition
\newtheorem{myDefinition}[thm]{Definition} % 这句定义使得 defn 环境和 thm 共享编号
\newtheorem{myExample}[thm]{Example}% 这句定义使得 example 环境和thm 共享编号
\newtheorem{myRemark}[thm]{Remark}
\newtheorem{myProposition}[thm]{Proposition}
\newtheorem{myNotation}[thm]{Notation}
\newtheorem{exercise}{练习}
\newtheorem{answer}{题目解答}
\newtheorem{remark}{Remark}
\newtheorem{case}{\normalfont\bfseries 案例} %
\newtheorem{solve}{解}
\newtheorem{Example}{例}
\newtheorem{example}{例}
\newtheorem{theorem}{定理}
\newtheorem{definition}{定义}




% 我的笔记环境
\newcommand{\mybox}[1]{
    % \fbox{mynotes:\\#1}
    \vskip 2.5mm
    \fbox{\parbox{120mm}{mynotes:\\#1}}
    \vskip 2.5mm
}



\usepackage{mathtools} % xrightarrow

\usepackage{tikz}       % tikz
% \usepackage{colortbl}   % tikznode color 块上色



% falling factorial power
\newcommand{\fallingfactorial}[1]{%
  ^{\underline{#1}}%
}
% rising factorial power
\newcommand{\risingfactorial}[1]{%
  ^{\overline{#1}}%
}
% tikz pic draw
% \newcommand\y{\cellcolor{clight2}}
\definecolor{clight2}{RGB}{212, 237, 244}%
\newcommand\tikznode[3][]%
{\tikz[remember picture,baseline=(#2.base)]
	\node[minimum size=0pt,inner sep=0pt,#1](#2){#3};%
}
\tikzset{>=stealth}

\title{具体数学阅读笔记-chap1}
\author{weiyuan}
\date{2022-06-26}
\begin{document}
	\maketitle
	\section{递归问题}
	\subsection{河内塔}
	使用递归方法解决河内塔问题。先从n为1的情况考虑该问题。为了递归式的完整?将n等于0也纳入考虑
	\begin{align*}
		T_0 = 0\\
		T_1 = 1\\
		T_2 = 3\\
		T_3 = 7
	\end{align*}
	从圆盘的移动规律可以看出,要将n个圆盘从A柱移动到C柱上,首先需要将n-1个圆盘移动到B柱上,再移动最大的圆盘到C柱上,最后将n-1个圆盘从B柱移动到A柱上。
	
	书中使用
		\begin{align*}
			&T_n \leqslant 2T_{n-1}+1\\
			&T_n \geqslant 2T_{n-1}+1
		\end{align*}
	得到 $ T_n = 2T_{n-1}+1 $ 。 将n等于0的情况作为初始条件,得到完整的递归式 (recurrence)
	\begin{equation}\label{Henei1}
		\begin{aligned}
			&T_0 = 0\\
			&T_n = 2T_{n-1}+1, \quad n > 0
		\end{aligned}
	\end{equation}
	
	计算得到
	\begin{equation}\label{HeneiRecur}
		T_n = 2^n - 1 , \quad n\geqslant 0
	\end{equation}

\textbf{数学归纳法} (mathematical induction)\footnote{数学归纳法的难点并不在于证明本身,而是如何得到关系式}
\begin{enumerate}
	\item 基础 basis, n取最小值 $ n_0 $证明该命题
	\item 归纳 induction, 假设$ k=n-1 $时归纳结果成立,证明$ k=n $时该结果也成立
\end{enumerate}

使用数学归纳法证明河内塔问题递归式
\begin{proof}证明河内塔问题递归式
	1. $ k=0 $时, $ T_0 = 2^0-1 = 0 $\\
	2. 假设 $ k=n-1 $时 $ T_{n-1} = 2^{n-1} - 1 $成立。\\
	3. $ k=n $时, $ T_{n} = 2*T_{n-1}+1 = 2*(2^{n-1} - 1)+1 = 2^n-1 $
\end{proof}

\begin{remark}
	\begin{enumerate}
		\item  研究小的情形
		\item  求解递归式(\ref{Henei1})
		\item  求解递归式(\ref{HeneiRecur})
	\end{enumerate}
\end{remark}

Q: 递归式$ T_n= 2^n -1 $是怎样得到的?
\begin{align*}
	&T_0 + 1 = 1\\
	&T_n+1 = 2T_{n-1}+2 = 2(T_{n-1}+1)
\end{align*}
令$ U_n = T_n+1 $
\begin{align*}
	&U_0 = 1\\
	&U_n = 2U_{n-1}, \quad n>0.
\end{align*}
容易推出$ U_n  = 2^n, T_n = 2^n-1 $

\subsection{平面上的直线}
平面上n条直线所界定的区域最大个数$ L_n $是多少?
\begin{align*}
	&L_0 = 1\\
	&L_1 = 2\\
	&L_2 = 4\\
	&L_3 = 7\\
	&\ldots\\
	&L_n \leqslant L_{n-1}+n, \quad n>0	
\end{align*}

\begin{align*}
	L_n &= L_{n-1} + n\\
	& = L_{n-2}+(n-1)+n\\
	&=\cdots\\
	&=L_0+1+2+\dots+n\\
	&=1+S_n
\end{align*}
其中$ S_n = 1+2+\dots+n $被称为\textbf{三角形数}。
\begin{equation}\label{sanjiaoxingshu}
	S_n = \frac{n(n+1)}{2}
\end{equation}
由此得到平面分割数$ L_n = \frac{n(n+1)}{2}+1 $.

使用数学归纳法证明该公式
\begin{proof}
	1. $ k=0 $, $ L_0 = 1 $.\\
	2. 设$ k=n-1 $, $ L_{n-1} = \frac{(n-1)((n-1)+1)}{2}+1=\frac{(n-1)n}{2}+1 $成立.\\
	3. $ k=n $, $ L_n = L_{n-1}+n = \frac{(n-1)n}{2}+1+n = \frac{n(n+1)}{2}+1 $
\end{proof}

\begin{remark}
	将直线的情况拓展到折线。
	\begin{align*}
		&Z_1 = 2\\
		&Z_2 = 7
	\end{align*}
\end{remark}
做法,将折线补齐成两条直线\footnote{$ L_n \sim \frac{1}{2}n^2\\ Z_n \sim 2n^2 $}
\begin{align*}
	Z_n& = L_2n-2n \qquad\text{锯齿点不在交点}\\
	&=\frac{2n(2n+1)}{2}+1 - 2n\\
	&=2n^2-n+1, \quad n\geqslant 0
\end{align*}

\subsection{约瑟夫问题}
n个人围成一圈,从第一个人开始,每隔一个删除一个。

\begin{table}[htbp]
	\centering
	\small
	\caption{约瑟夫问题最终剩余数字 J(n) 与全体数字 n 之间的关系}
	\begin{tabular}{c|ccccccccc}
		\toprule
		n & 1 & 2 & 3 & 4 & 5 & 6 & 7 & 8 & 9 \\  
		\midrule
		J(n) & 1 & 1 & 3 & 1 & 3 & 5 & 7 & 1 & 3\\
		\bottomrule
	\end{tabular}%
	\label{tab:reg}%
\end{table}%

人数总数为偶数 $ J(2n)   = 2J(n)-1, n\geqslant 1 $.
人数总数为奇数 $ J(2n+1) = 2J(n)+1, n\geqslant 1 $.

递归式
\begin{align*}
	J(1)  &= 1&\\
	J(2n) &= 2J(n)-1,&  n\geqslant 1 \\
	J(2n+1) &= 2J(n)+1,&  n\geqslant 1 
\end{align*}

计算得到 $ n = 2^m+l $,封闭形式$ J(2^m+l) = 2l+1, m\geqslant 0, 0\leqslant l <2^m  $.
\begin{proof}
	1. l is even.
	\begin{align*}
		J(2^m+l) &= 2J(2^m+\frac{l}{2})-1\\
		&=2(2*\frac{l}{2}+1)-1\\
		&=2l+1
	\end{align*}
	2. l is odd.
	\begin{align*}
		J(2^m+l) &= 2J(2^m+\frac{l-1}{2})-1\\
		&=2(2*\frac{l-1}{2}+1)-1\\
		&=2l+1
	\end{align*}
\end{proof}

\begin{equation}\label{JosephRecur1}
	J(2n+1)-J(2n)=2
\end{equation}

\begin{remark}
	将n和J(n)以2为基数表示 (表示为二进制). 假设:\\
	\begin{align*}
		n&=(b_mb_{m-1} \dots b_1 b_0)_2\\
		&=b_m2^m+b_{m-1}2^{m-1}+\dots+b_1 2^1+b_0 2^0
	\end{align*}
其中$ b_m = 1, b_i = 0\text{ 或 }1  \quad(0\leqslant i <m, i\in\mathbb{N}^+)$
\end{remark}

\begin{align*}
	n&=2^m+l\\
	n&=(1 b_{m-1} b_{m-2}\dots b_1 b_0)_2\\
	l&=(0 b_{m-1} b_{m-2}\dots b_1 b_0)_2\\
	2l&=(b_{m-1} b_{m-2}\dots b_1 b_0 0)_2\\
	2l+1&=(b_{m-1} b_{m-2}\dots b_1 b_0 1)_2\\
	J(n)&=(b_{m-1} b_{m-2}\dots b_1 b_0 1)_2
\end{align*}

因此我们得到 $ J((1 b_{m-1} b_{m-2}\dots b_1 b_0)_2)=(b_{m-1} b_{m-2}\dots b_1 b_0 1)_2 $.\\
n向左循环移动一位得到J(n)!\\

\begin{case}
	$ J((1011)_2) = (0111)_2 = (111)_2 $, 该式即 $ J(11) = 7 $
\end{case}
注意:0移动至首位会消失,而不需要保留空位。

$ 2^{\nu(n)}-1 $, 其中$ \nu(n) $为n转换成的二进制数中1的个数

\begin{case}
	n=13, $ (13)_{10} = (1101)_2 $, $ \mu(13) = 3 $.\\
	$ J(J(\dots(J(13))\dots)) = 2^3-1 = 7 $
\end{case}
\begin{case}
	n=23403, $ \mu(23403)=10 $, therefore 	$ J(J(\dots(J(23403))\dots)) = 2^{10}-1 = 1023 $
\end{case}

%pp10
回到第一个猜测 $ J(n) = \frac{n}{2} $ n为偶数。
这个猜想在什么情况下成立?

目前已知 $ n = 2^m + l $, 

\begin{align*}
	J(n) &= \frac{n}{2}\\
	2l+1 &= (2^m+l)/2\\
	\frac{3}{2} l &=2^{m-1}-1\\
	l&=\frac{1}{3}(2^{m-1}-2)
\end{align*}

m为奇数, $ 2^m - 2 $是3的倍数

m为偶数, $ 2^m - 2 $不是3的倍数

$ J(n) = \frac{n}{2} $ 有无穷多组解


\begin{table}[htbp]
	\centering
	\small
	\caption{J(n)关系式之间的关系}
	\begin{tabular}{ccccr}
		\toprule
		m & l & $ n = 2^m +l $ & $ J(n) = 2l+1 = \frac{n}{2} $ & $ (n)_2 $ \\
		\midrule
		1 & 0  & 2   & 1  & 10\\
		3 & 2  & 10  & 5  & 1010\\
		5 & 10 & 42  & 21 & 101010\\
		7 & 42 & 170 & 85 & 10101010\\
		\bottomrule
	\end{tabular}%
	\label{tab:J(n)关系式之间的关系}%
\end{table}%

\begin{remark}
	再推广,引入常数 $ \alpha, \beta, \gamma $.
\end{remark}

\begin{equation}\label{J(n)forConst}
	\left\{
	\begin{aligned}
		f(1)    &= \alpha\\
		f(2n)   &= 2f(n)+\beta,  \quad n\geqslant 1\\
		f(2n+1) &= 2f(n)+\gamma, \quad n\geqslant 1
	\end{aligned}
	\right.
\end{equation}


\begin{table}[htbp]
	\centering
	\small
	\caption{f(n)常数的变化规律}
%	\begin{tabular}{c|l}
%		\toprule
%		n & f(n)\\
%		\midrule
%		1 & $ \;\alpha $\\
%		2 & $ 2\alpha + \;\beta $\\
%		3 & $ 2\alpha \qquad + \gamma $\\ 
%		4 & $ 4\alpha + 3\beta $\\ 
%		5 & $ 4\alpha + 2\beta  + \gamma $\\ 
%		6 & $ 4\alpha + \;\beta + 2\gamma $\\ 
%		7 & $ 4\alpha \qquad + 3\gamma $\\ 
%		8 & $ 8\alpha + 7\beta $\\ 
%		9 & $ 8\alpha + 6\beta + \gamma $\\ 
%		\bottomrule
%	\end{tabular}%
	\begin{tabular}{c|lcccc}
	\toprule
	n & f(n) & $ \qquad $ & $ \alpha $ & $ \beta $ & $ \gamma $ \\
	\midrule
	1 & $ \;\alpha $                    && 1& 0 &0 \\
	2 & $ 2\alpha + \;\beta $           && 2& 1 &0 \\
	3 & $ 2\alpha \qquad + \gamma $     && 2& 0 &1 \\ 
	4 & $ 4\alpha + 3\beta $            && 4& 3 &0 \\ 
	5 & $ 4\alpha + 2\beta  + \gamma $  && 4& 2 &1 \\ 
	6 & $ 4\alpha + \;\beta + 2\gamma $ && 4& 1 &2 \\ 
	7 & $ 4\alpha \qquad + 3\gamma $    && 4& 0 &3 \\ 
	8 & $ 8\alpha + 7\beta $            && 8& 7 &0 \\ 
	9 & $ 8\alpha + 6\beta + \gamma $   && 8& 6 &1 \\ 
	\bottomrule
\end{tabular}%
	\label{tab:fnLists}%
\end{table}%

假设$ f(n) $具有如下迭代形式
\begin{equation}\label{fnReccu}
	f(n) = A(n)\alpha + B(n)\beta + C(n)\gamma
\end{equation}

看起来似乎有
%\begin{align*}
%	A(n) &= 2^m
%\end{align*}

\begin{equation}
	\left\{
	\begin{aligned}
		A(n) &= 2^m\\
		B(n) &= 2^m-1-l\\
		C(n) &= l
	\end{aligned}
	\right.\quad
	\begin{aligned}
    	n=2^m+l, \\ 
		0\leqslant l \leqslant 2^m \\
		(n\geqslant 1)
	\end{aligned}
\end{equation}

使用归纳法证明过程较为繁琐,可选用特殊值组合

1. $ \alpha =1, \beta=\gamma = 0 $, 此时 $ f(n) = A(n) $, 式\ref{J(n)forConst} 变为
\begin{equation*}
	\left\{
	\begin{aligned}
		A(1)&=1\\
		A(2n)   &= 2A(n),  \quad n\geqslant 1\\
		A(2n+1) &= 2A(n),  \quad n\geqslant 1
	\end{aligned}
	\right.
\end{equation*}

\begin{table}[htbp]
	\centering
	\small
	\caption{f(n)在$ \alpha =1, \beta=\gamma = 0 $时的变化情况}
	\begin{tabular}{c|cccccccccccccccccc}
		\toprule
		n   & 1 & 2 & 3 & 4 & 5 & 6 & 7 & 8 & 9 & 10 & 11 & 12 & 13 & 14 & 15 & 16 & 17 & 18 \\
		\midrule
		f(n)& 1 & 2 & 2 & 4 & 4 & 4 & 4 & 8 & 8 &  8 &  8 &  8 &  8 &  8 &  8 & 16 & 16 & 16 \\
		\bottomrule
	\end{tabular}%
	\label{tab:fncase1}%
\end{table}%

此时由归纳法可得 $ A(2^m+l)=2^m $

2. $ f(n) = 1 $
\begin{equation*}
	\left\{
	\begin{aligned}
		1 &= \alpha             \\
		1 &= 2\times 1 + \beta  \\
		1 &= 2\times 1 + \gamma 
	\end{aligned}
	\right.
	\rightarrow
	\left\{
	\begin{aligned}
		\alpha &= 1  \\
		\beta  &= -1 \\ 
		\gamma &= -1
	\end{aligned}
	\right.	
\end{equation*}

\begin{equation*}
	A(n)-B(n)-C(n) = f(n) = 1
\end{equation*}

3. $ f(n) = n $
\begin{equation*}
	\left\{
	\begin{aligned}
		1    &= \alpha             \\
		2n   &= 2\times n + \beta  \\
		2n+1 &= 2\times n + \gamma 
	\end{aligned}	
	\right.
	\rightarrow
	\left\{
	\begin{aligned}
		\alpha &= 1  \\
		\beta  &= 0  \\ 
		\gamma &= 1
	\end{aligned}
	\right.	
\end{equation*}

\begin{equation*}
	A(n) + C(n) = f(n) = n
\end{equation*}

当 $ n = 2^m+l, 0\leqslant l< 2^m $时,
由上述三种情况可以得到
\begin{equation*}
	\left\{
	\begin{aligned}
		A(n) &       &          &= 2^m        \\
		A(n) &- B(n) &-  C(n)   &= 1          \\
		A(n) &       &+  C(n)   &= n = 2^m+l
	\end{aligned}
	\right.
\end{equation*}
解得
\begin{equation*}
	\left\{
	\begin{aligned}
		A(n) &= 2^m        \\
		B(n) &= 2^m-1-1    \\
		C(n) &= l
	\end{aligned}
	\right.
\end{equation*}
以上为求解递归式的\textbf{成套方法}(repertorire method)

\begin{remark}
	约瑟夫递归式 二进制解的推广
\end{remark}
已知约瑟夫递归式的解为
\begin{equation*}
	J((b_m b_{m-1}\dots b_1 b_0)_2) = (b_{m-1} b_{m-2}\dots b_1 b_0 b_m)_2, \qquad b_m = 1
\end{equation*}
推广的约瑟夫递归式有无这种形式的解?

令$ \beta_0 = \beta, \beta_1 = \gamma $
\begin{equation*}
	\left\{
	\begin{aligned}
		f(1)    &= \alpha  \\
		f(2n+j) &= 2f(n) + \beta_j, \quad j=0,1, \quad n\leqslant 1  
	\end{aligned}
	\right.
\end{equation*}
\begin{align*}
	    f((b_m b_{m-1}\dots b_3 b_2 b_1 b_0)_2) 
	&= 2f((b_m b_{m-1}\dots b_3 b_2 b_1)_2)     +  \beta_{b_0}  \\
	&= 2f((b_m b_{m-1}\dots b_3 b_2)_2)         + 2\beta_{b_1}+ \beta_{b_0}  \\
	&=\qquad\qquad\vdots\\
	&= 2^{m} f((b_{m})_2) + 2^{m-1} f((b_{m-1})_2) + \dots + 2\beta_{b_1}+ \beta_{b_0}  \\
	&= 2^{m} \alpha + 2^{m-1} \beta_{b_{m-1}} + \dots + 2\beta_{b_1}+ \beta_{b_0}  
\end{align*}

\begin{equation*}
    f((b_m b_{m-1}\dots b_1 b_0)_2) = (\alpha b_{m-1}\dots b_1 b_0)_2
\end{equation*}
使用成套方法再次求解约瑟夫问题
\begin{table}[htbp]
	\centering
	\small
	\caption{成套方法各项系数}
		\begin{tabular}{c|l c}
			\toprule
			n & f(n)                             & J(n)  \\
			\midrule
			1 & $ \;\alpha $                     & 1     \\
			2 & $ 2\alpha + \beta  $             & 1     \\
			3 & $ 2\alpha + \gamma $             & 3     \\ 
			4 & $ 4\alpha + 2\beta  + \beta $    & 1     \\ 
			5 & $ 4\alpha + 2\beta  + \gamma $   & 3     \\ 
			6 & $ 4\alpha + \;\beta + 2\gamma $  & 5     \\ 
			7 & $ 4\alpha \qquad + 3\gamma $     & 7     \\ 
			8 & $ 8\alpha 7\beta $               & 1     \\ 
			7 & $ 4\alpha 6\beta + \;\gamma $    & 3     \\ 
			\bottomrule
		\end{tabular}%
	\label{tab:fnLists002}%
\end{table}%

$ n = 100 = (1100100)_2 $, $ (\alpha, \beta, \gamma) = (1, -1, 1) $
\begin{equation*}
	\begin{array}{rlrrrrrrrll}
		n    &= ( &   1  &  1  &  0  &  0  &  1 &  0 &  0&)_2   &= 100  \\
		f(n) &= ( &   1  &  1  & -1  &  -1 &  1 & -1 & -1&)_2   &       \\
		     &=   & +64  & +32 & -16 &  -8 & +4 & -2 & -1&      &= 73   \\
	\end{array}
\end{equation*}

总结:
\begin{equation*}
	\begin{array}{rlrrrrrrrll}
		n    &= ( &   1     &  1       & 0     &  0     &  1      &  0    &  0    &)_2   &= 100  \\
		f(n) &= ( & \alpha  &  \gamma  & \beta &  \beta &  \gamma & \beta & \beta &)_2   &       \\
		     &= ( &   1     &  1       & -1    &  -1    &  1      & -1    & -1    &)_2   &       \\
		     &=   & +2^6    & +2^5     & -2^4  &  -2^3  & +2^2    & -2^1  & -2^0  &      &= 73   \\
	\end{array}
\end{equation*}

由于在n的二进制表示中每一块二进制数字 $ (1000\dots00)_{2} $ 都被变换成 
\begin{equation}\label{JnCurrBina}
	(1 {-1} {-1} {-1}\dots {-1} {-1})_{2} = (000\dots01)_{2}
\end{equation}
 因而推出循环移位性质

\begin{align*}
	f(j)    &= \alpha_{j}          & 1 \leqslant j < d. &                     \\
	f(dn+j) &= c f(n) + \beta_{j}  & 0 \leqslant j < d, &\quad n \leqslant 1  
\end{align*}
有变动基数的解
\begin{equation}\label{JnForDiffBase}
	f((b_m b_{m-1}\dots b_1 b_0)_{d}) = (\alpha_{b_{m}} \beta_{b_{m-1}}\dots \beta_{b_1} \beta_{b_0})_{c}
\end{equation}
前一式基数为d, 后一式基数为c \footnote{之后会用tikz绘图标注}


example:
\begin{align*}
	f(   1) &= 34              &               \\
	f(   2) &= 5               &               \\
	f(3n  ) &= 10f(n) + 76,    & n \geqslant 1 \\
	f(3n+1) &= 10f(n) -  2,    & n \geqslant 1 \\
	f(3n+2) &= 10f(n) +  8,    & n \geqslant 1 \\
\end{align*}
计算$ f(19) $.

Solve:
通过基数方法可知 $ d = 3, c = 10 $.
\begin{align*}
	\alpha_{1} &=  34  \\
	\alpha_{2} &=  5   \\
	\beta_{1}  &=  76  \\
	\beta_{2}  &= -2  \\
	\beta_{3}  &=  8  
\end{align*}

\begin{equation*}
	(19)_{10}   = 2 \times 3^2 + 0 \times 3^1 + 1 \times 3^0 = (201)_{3}
\end{equation*}
\begin{align*}
	f((19)_{10}) 
	&= f((201)_3) \\
	&= (5 \quad 76 \quad -2)_{10}\\
	&=5\times 10^2 + 76 \times 10^1 + (-2) \times 10^0\\
	&=500+760-2\\
	&=1258
\end{align*}


\end{document}