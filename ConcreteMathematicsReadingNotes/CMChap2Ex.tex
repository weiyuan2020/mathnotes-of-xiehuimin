\documentclass[mode=geye]{elegantnote}
%!TEX program=xelatex
\usepackage{amsmath}
\usepackage{amssymb}
\usepackage{amsthm}
\usepackage{hyperref}
\usepackage{mathtools} % xrightarrow

\usepackage{tikz}       % tikz
% \usepackage{colortbl}   % tikznode color 块上色


\newtheorem{exercise}{练习}
\newtheorem{answer}{题目解答}
\newtheorem{solve}{解}

% falling factorial power
\newcommand{\fallingfactorial}[1]{%
  ^{\underline{#1}}%
}
% rising factorial power
\newcommand{\risingfactorial}[1]{%
  ^{\overline{#1}}%
}
% tikz pic draw
% \newcommand\y{\cellcolor{clight2}}
\definecolor{clight2}{RGB}{212, 237, 244}%
\newcommand\tikznode[3][]%
{\tikz[remember picture,baseline=(#2.base)]
	\node[minimum size=0pt,inner sep=0pt,#1](#2){#3};%
}
\tikzset{>=stealth}

\title{具体数学阅读笔记-chap2Ex}
\author{weiyuan}
\date{2022-07-08}
\begin{document}
\maketitle
\section{Exercises}
\subsection{Warmups}
\begin{exercise}
1
\end{exercise}

\begin{answer}
    1. $ \sum_{k=m}^{n} q_k $ is always equivolent to $ \sum_{m \leqslant k \leqslant n} q_k $ then the stated sum is zero.\\
    2. $ q_4+q_3+q_2+q_1+q_0 $ .\\
    3. We can say that $ \sum_{m \leqslant k \leqslant n} q_k = \sum_{k \leqslant n} q_k  -\sum_{k<m} q_k $ . then $ \sum_{k=4}^{0}q_k = -q_1-q_2-q_3 $ .\\
    $ [ $ obey the law 
        $ \sum_{k=a}^{b}+
        \sum_{k=b}^{c}=
        \sum_{k=a}^{c}, 
        \quad \forall a, b, c $  $ ] $ .\\
    It's best to use the notation $ \sum_{k=m}^{n} $ only when $ n-m \geqslant 1 $ ; then both conventions 1 and 3 agree.
\end{answer}

\begin{exercise}
    2
\end{exercise}

\begin{answer}
    the quantity ([x>0]\_[x<0]) is often called sign(x) or signum(x).
    \begin{equation*}
        sign(x) = \left\{
            \begin{array}{ll}
                1,  & x>0; \\
                0,  & x=0; \\
                -1, & x<0; \\
            \end{array}
        \right.
    \end{equation*}
    $ x\; sign(x) = |x| $ , $ [] $ 判定命题真假
\end{answer}

\begin{exercise}
    3
\end{exercise}

\begin{answer}
    \begin{equation*}
        \sum_{0\leqslant k\leqslant 5}a_k = a_0+a_1+a_2+a_3+a_4+a_5
    \end{equation*}
    \begin{equation*}
        \begin{array}{rlrrrr}
            \sum_{0\leqslant k^2\leqslant 5}a_k 
            &= {\color{red} \;a_4}&{\color{red}+a_1}&+a_0&+a_1&+a_4 \\
            {\color{blue}(k}&{\color{blue}= -2,} &{\color{blue}-1,} &{\color{blue}0, }&{\color{blue}1, }&{\color{blue}2)}\\
        \end{array}
    \end{equation*}
\end{answer}

\begin{exercise}
    $ \sum_{1\leqslant i< j<k\leqslant k} a_{ijk}$ 
\end{exercise}

\begin{answer}
    \begin{equation}
        I = 
        \sum_{1\leqslant i\leqslant 4}
        \sum_{1\leqslant j\leqslant 4}
        \sum_{1\leqslant k\leqslant 4}
        a_{ijk}
        \quad (k\rightarrow j\rightarrow i)
    \end{equation}
    \begin{equation}
        I = 
        \sum_{1\leqslant k\leqslant 4}
        \sum_{1\leqslant j\leqslant 4}
        \sum_{1\leqslant i\leqslant 4}
        a_{ijk}
        \quad (i\rightarrow j\rightarrow k)
    \end{equation}
    1.
    \begin{equation*}
        \left\{
            \begin{array}{ll}
                 & a_{1jk} \\
                +& a_{2jk} \\
                +& a_{3jk} \\
                +& a_{4jk} \\
            \end{array}
        \right.
    \end{equation*}
    \begin{equation*}
        \left\{
            \begin{array}{lll}
                a_{12k} &+ a_{13k} &+ a_{14k} \\
                &+ a_{23k} &+ a_{24k} \\
                &          &+ a_{34k} \\
            \end{array}
        \right.
    \end{equation*}
    \begin{equation*}
        \left\{
            \begin{array}{lll}
                a_{123} &+a_{124} &+a_{134} \\
                &+a_{234}& \\
            \end{array}
        \right.
    \end{equation*}

    (1). $ \Bigl(\bigl(a_{123}+a_{124}\bigr)+a_{134}\Bigr)+a_{234} $ .
    
    $ k\rightarrow j\rightarrow i $ 

    (2). $ a_{123}+\Bigl(a_{124}+\bigl(a_{134}+a_{234}\bigr)\Bigr) $ .
    
    $ i\rightarrow j\rightarrow k $ 
\end{answer}

\begin{exercise}
    $\left( \sum_{j=1}^{n}a_j \right) \left( \sum_{k=1}^{n}\frac{1}{a_k}\right) = \sum_{j=1}^{n}\sum_{k=1}^{n}\frac{a_j}{a_k} = \sum_{k=1}^{n}\sum_{k=1}^{n}\frac{a_k}{a_k}=n$.

    where is wrong?
\end{exercise}

\end{document}