\documentclass[mode=geye]{elegantnote}
%!TEX program=xelatex
\usepackage{amsmath}
\usepackage{amssymb}
\usepackage{amsthm}

\newtheorem{exercise}{练习}
\newtheorem{answer}{题目解答}

\title{具体数学阅读笔记-chap1 repertoire method 参考}
\author{weiyuan}
\date{2022-07-03}
\begin{document}
\maketitle
Solve 
\begin{equation*}
    \left\{
        \begin{array}{ll}
            r_0 &= 1\\
            r_n &= r_{n-1}+3n+5\\ 
        \end{array}
    \right.
\end{equation*}

First, get some cases
\begin{align*}
    r_0 &= 1 & \\
    r_1 &= 1+3\times 1 + 5 &= 9  \\
    r_2 &= 9+3\times 2 + 5 &= 20 \\
    r_3 &= 20+3\times 3 + 5 &= 34 
\end{align*}

Unsimplified cases
\begin{align*}
    r_0 &= 1 & \\
    r_1 &= r_0+3\times 1 + 5 &= 9  \\
    r_2 &= r_1+3\times 2 + 5 &= 20 \\
    r_3 &= r_2+3\times 3 + 5 &= 34 
\end{align*}

A pattern in unsimplified cases
\begin{equation*}
    r_n = 1 A(n) + 3 B(n) +5 C(n)
\end{equation*}
where $ A(n), B(n), C(n) $ are simple functions of n
\begin{equation*}
    \left\{
        \begin{array}{ll}
            A(n) &= 1\\
            B(n) &= \frac{n(n+1)}{2}\\
            C(n) &= n\\
        \end{array}
    \right.
\end{equation*}
\begin{align*}
    r_n &= 1 \times 3 \times \frac{n(n+1)}{2} +5 \times n\\
        &= \frac{3}{2}n^2 +\frac{13}{2}+1
\end{align*}

Summarizing 
\begin{equation*}
    \left\{
        \begin{array}{ll}
            r_0 &= 1\\
            r_n &= r_{n-1}+3n+5\\
        \end{array}
    \right.
\end{equation*}
is $ r_n=\frac{3}{2}n^2+\frac{13}{2}n+1 $ .

Testing
\begin{table}[htbp]
	\centering
	\small
	\caption{r(n) 与 n 之间的关系}
	\begin{tabular}{c|ccc ccc}
		\toprule
		n & 0 & 1 & 2 & 3 & 4 & 5 \\  
		\midrule
		$ r_n $ & 1 & 9 & 20 & 34 & 51 & 71\\
        $ \frac{3}{2}n^2+\frac{13}{2}n+1 $ & 1 & 9 & 20 & 34 & 51 & 71\\
		\bottomrule
	\end{tabular}%
	\label{tab:rnCompare}%
\end{table}%

\end{document}