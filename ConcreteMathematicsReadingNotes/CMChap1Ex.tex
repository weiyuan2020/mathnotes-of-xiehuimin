\documentclass[mode=geye]{elegantnote}
%\usepackage{amsmath}
\usepackage{amssymb}
\usepackage{amsthm}

\newtheorem{exercise}{练习}
\newtheorem{answer}{题目解答}

\title{具体数学阅读笔记-chap1 exercise}
\author{weiyuan}
\date{2022-06-27}
\begin{document}
\maketitle

\section{Exercises}
\subsection{Warmups}

\begin{exercise}1
	All horses are the same color; we can prove this by induction on the
	number of horses in a given set. Here's how: "If there's just one horse
	then it's the same color as itself, so the basis is trivial. For the induction
	step, assume that there are n horses numbered 1 to n. By the induc-
	tion hypothesis, horses 1 through n − 1 are the same color, and similarly
	horses 2 through n are the same color. But the middle horses, 2 through
	n − 1, can't change color when they're in different groups; these are
	horses, not chameleons. So horses 1 and n must be the same color as
	well, by transitivity. Thus all n horses are the same color; QED." What,
	if anything, is wrong with this reasoning?
\end{exercise}

\begin{answer}
	n=1 情况下马有相同颜色
	
	但 n=2 时	该假设不一定成立
\end{answer}


\begin{exercise}2
\end{exercise}

\begin{answer}
	不允许在A B之间直接移动, 求最短的移动序列
	\begin{equation*}
	\begin{array}{cll}
		k=1     & 1\quad A \rightarrowtail C, C \rightarrowtail B   & 2 \quad sum=2\\
		k=2     & 1\quad A \rightarrowtail C, C \rightarrowtail B,  & 2	\\
				& 2\quad A \rightarrowtail C                 		& 1	\\
			 	& 1\quad B \rightarrowtail C, C \rightarrowtail A,  & 2	\\
			 	& 2\quad C \rightarrowtail B						& 1	\\
			 	& 1\quad A \rightarrowtail C, C \rightarrowtail B   & 2 \quad sum=8 \\
		k=3     & 1\quad A \rightarrowtail C, C \rightarrowtail B,  & 2	\\
				& 2\quad A \rightarrowtail C                 		& 1	\\
				& 1\quad B \rightarrowtail C, C \rightarrowtail A,  & 2	\\
				& 2\quad C \rightarrowtail B						& 1	\\
				& 1\quad A \rightarrowtail C, C \rightarrowtail B   & 2 \quad sum=8\\
				& 3\quad A \rightarrowtail C                 		& 1	\\
				& 1\quad B \rightarrowtail C, C \rightarrowtail A,  & 2	\\
				& 2\quad B \rightarrowtail C						& 1	\\
				& 1\quad A \rightarrowtail C, C \rightarrowtail B,  & 2	\\
				& 2\quad C \rightarrowtail A						& 1	\\
				& 1\quad B \rightarrowtail C, C \rightarrowtail A,  & 2	\\
				& 3\quad C \rightarrowtail B						& 1	\quad sum=18\\
				& 1\quad A \rightarrowtail C, C \rightarrowtail B,  & 2	\\
				& 2\quad A \rightarrowtail C                 		& 1	\\
				& 1\quad B \rightarrowtail C, C \rightarrowtail A,  & 2	\\
				& 2\quad C \rightarrowtail B						& 1	\\
				& 1\quad A \rightarrowtail C, C \rightarrowtail B   & 2 \quad sum=26\\
		\vdots  &													& \\
		k=n     & 1\quad A \rightarrowtail C, C \rightarrowtail B	&  \\
	\end{array}
	\end{equation*}
	从前面的移动可以看出 $ f(n) = 3*f(n-1)+2 $, 设 $ g(n) = f(n)+1 $, $ g(1)=f(1)+1 = 3 $, $ g(n) = 3g(n-1) $. $ g(n) = 3^{n} $, $ f(n) = 3^{n}-1 $.
\end{answer}

\begin{exercise}3
\end{exercise}

\begin{answer}
	是的,以n个圆盘为例
	正确的叠放方法有$ 3^n $种 将ABC视为3个序列,将所有圆盘从大到小依次放置在3个序列中,每个圆盘放置时有3种选择,所共有$ 3^n $种正确的叠放方法。第二题移动$ 3^n-1 $次,再加上移动前所有圆盘都在A柱上的情况,共有$ 3^n $种情况,所以所有正确的叠放方法均会出现。
	
	我的思考,n个圆盘在3根柱子上任意放的方法有多少种?
\end{answer}

\begin{exercise}4
\end{exercise}

\begin{answer}
	Are there any starting and ending configurations of n disks on three pegs
	that are more than $ 2^n − 1 $ moves apart, under Lucas's original rules?
	
	是否存在 $ m > 2^n-1 $
	
%		I don't know...
	不存在。根据卢卡斯的规则,将可能出现的移动情况分为两种:\\		
	1. 最大的圆盘不需要移动,根据归纳法,最多需要移动$ 2^{n-1}-1 $次。\\	
	2. 最大的圆盘需要移动,根据归纳法,最多需要移动 $ 2^{n-1}-1 + 1 + 2^{n-1}-1 $ 即 $  2^{n}-1  $次
\end{answer}

\begin{exercise}5
\end{exercise}

\begin{answer}
	3个给定集合,共有8个可能子集。使用 Venn 图表示
	
	\footnote{Venn 图之后会补上}
%		空集 $ \emptyset $ 或 $ \varnothing $ 
%		并集$ \cup $, 交集 $\cap$
	$ A, B, C, $ 三个集合的所有子集为 $ \{\varnothing, A, B, C, A\cap B, A\cap C, B\cap C, A \cap B \cap C, \} $ , $\{A\setminus B, A\setminus C, B\setminus A, B\setminus C, C\setminus A, C\setminus B \}$, $ \{A\setminus (B\cup C), B\setminus (C\cup A), C\setminus (A\cup B)\} $, $ \{A\cup B, A\cup C, B\cup C, A\cup B\cup C\} $. $ \dots $
	
	我认为这里所将的八个子集应当是$ \{\varnothing\}, \{A\setminus (B\cup C), B\setminus (C\cup A), C\setminus (A\cup B)\} $, $ \{(A\cap B)\setminus C, (C\cap A)\setminus B, (B\cap C)\setminus A \} $, $ \{A\cap B\cap C\} $. 空集和7个互不相交的真子集。
	
	对于4个集合,Venn图不能给出可能的16个子集,因为不同的圆至多交于两点。 参考答案中说的卵形 (ovals) 是什么意思?
\end{answer}

\begin{exercise}6
\end{exercise}

\begin{answer}
	无界区域个数$ 2n $\\
	所有区域个数$ \frac{n(n+1)}{2}+1 $ \\
	二者相减得到有界区域个数 $ \frac{(n-1)(n-2)}{2} $
\end{answer}

\begin{exercise}7
\end{exercise}

\begin{answer}
	设 $ H(n) = J(n+1) - J(n) $.\\
	$ H(2n) = 2 $, 对$ n\geqslant 1 $有\\
	\begin{align*}
		 H(2n+1)
		 &= J(2n+2) - J(2n+1) \\
		 &= (2J(n+1)-1) - (2J(n)+1)\\
		 &= 2H(n) - 2
	\end{align*}
	但在$ n=0 $	时,由此推出
	\begin{equation*}
		H(1) = J(2)-J(1) = 1 - 1 = 0 \neq 2
	\end{equation*}
\end{answer}

\subsection{作业题}
\begin{exercise}	
	\begin{align*}
		Q_0 &= \alpha\\
		Q_1 &= \beta \\
		Q_n &= \frac{1+Q_{n-1}}{Q_{n-2}} ,\quad n>1
	\end{align*}
	(hint: $ Q_4 = \frac{1+\alpha}{\beta} $)
\end{exercise}

\begin{answer}
	\begin{equation*}
		\begin{array}{lllll}
			Q_0 &= \alpha & & &= \alpha\\
			Q_1 &= \beta  & & &= \beta\\
			Q_2 &= \cfrac{1+Q_1}{Q_0} & &&= \cfrac{1+\beta}{\alpha} \\
			Q_3 &= \cfrac{1+Q_2}{Q_1} &= \cfrac{1+\cfrac{1+\beta}{\alpha}}{\beta} 
			&&= \cfrac{1+\alpha+\beta}{\alpha\beta} \\
			Q_4 &= \cfrac{1+Q_3}{Q_2} 
			&= \cfrac{1+\cfrac{1+\alpha+\beta}{\alpha\beta} }{\cfrac{1+\beta}{\alpha}}
			&= \cfrac{\alpha\beta+1+\alpha+\beta}{\beta(1+\beta)} 
			&= \cfrac{1+\alpha}{\beta} \\
			Q_5 &= \cfrac{1+Q_4}{Q_3} 
			&= \cfrac{1+\cfrac{1+\alpha}{\beta}}{ \cfrac{1+\alpha+\beta}{\alpha\beta}}
			&= \cfrac{\alpha\beta+\alpha(1+\alpha)}{1+\alpha+beta} &= \alpha \\
			Q_6 &= \cfrac{1+Q_5}{Q_4} &= \cfrac{1+\alpha}{\cfrac{1+\alpha}{\beta}} & &= \beta\\
		\end{array}
	\end{equation*}
	因此解得 
	\begin{equation*}
		\begin{array}{clcccccc}
			Q_i &= \{ &\alpha ,& \beta ,& \cfrac{1+\beta}{\alpha} ,& \cfrac{1+\alpha+\beta}{\alpha\beta} ,& \cfrac{1+\alpha}{\beta} & \}\\
			(i\%n) &= \{ & 0 ,& 1 ,& 2 ,& 3 ,& 4 ,& \}\\
		\end{array}
	\end{equation*}
\end{answer}	

\begin{exercise}
		反向归纳法,从 $ n $ 到 $  n-1 $ 证明命题
	\begin{equation*}
		P(n) : \; x_{1} \dots x_{n} \leqslant \left( \frac{x_{1}+\cdots+x_{n}}{n}\right) ^n, \quad x_i \geqslant 0, i=1,\dots,n
	\end{equation*}
	$ n=2 $时为真 
	\begin{equation*}
		(x_{1}+x_{2})^2-4x_{1}x_{2} = (x_{1}-x_{2})^2 \geqslant 0
	\end{equation*}
	\begin{enumerate}[a)]
		\item $ x_n = \frac{x_{1}+\cdots+x_{n-1}}{n-1} $, 证明只要 $ n>1 $ 时 $ P(n) $ 蕴含 $ P(n-1) $.
		\item 证明 $ P(n) $和 $ P(2) $蕴含 $ P(2n) $
		\item 由 a), b) 说明这就蕴含了$ P(n) $对所有$ n $为真
	\end{enumerate}
\end{{exercise}

\begin{answer}
	a) $ P(n) $成立,$ \forall n>1 $\\
	给定 $ x_n = \frac{x_{1}+\cdots+x_{n-1}}{n-1} $,则有
	\begin{equation*}
		\begin{array}{ll}
			x_1 \dots x_{n-1} \cdot \cfrac{x_{1}+\dots+x_{n-1}}{n-1} 
			&\leqslant \left( \frac{	x_1 + x_{n-1} + \cfrac{x_{1}+\dots+x_{n-1}}{n-1} }{n} \right) ^n \\
			x_1 \dots x_{n-1} \cdot \cfrac{x_{1}+\dots+x_{n-1}}{n-1}
			&\leqslant \left(  \cfrac{x_{1}+\dots+x_{n-1}}{n-1} \right) ^n \\
			x_1 \dots x_{n-1}
			&\leqslant \left(  \cfrac{x_{1}+\dots+x_{n-1}}{n-1} \right) ^{n-1}\\
		\end{array}
	\end{equation*}
	$ P(n) $成立
\end{answer}

\end{document}