\documentclass[mode=geye]{elegantnote}
%\usepackage{amsmath}
\usepackage{amssymb}
\usepackage{amsthm}

\newtheorem{exercise}{练习}
\newtheorem{answer}{题目解答}

\title{具体数学阅读笔记-chap1 exercise}
\author{weiyuan}
\date{2022-06-27}
\begin{document}
\maketitle

\section{Exercises}
\subsection{Warmups}

\begin{exercise}1
	All horses are the same color; we can prove this by induction on the
	number of horses in a given set. Here's how: "If there's just one horse
	then it's the same color as itself, so the basis is trivial. For the induction
	step, assume that there are n horses numbered 1 to n. By the induc-
	tion hypothesis, horses 1 through n − 1 are the same color, and similarly
	horses 2 through n are the same color. But the middle horses, 2 through
	n − 1, can't change color when they're in different groups; these are
	horses, not chameleons. So horses 1 and n must be the same color as
	well, by transitivity. Thus all n horses are the same color; QED." What,
	if anything, is wrong with this reasoning?
\end{exercise}

\begin{answer}
	n=1 情况下马有相同颜色
	
	但 n=2 时	该假设不一定成立
\end{answer}


\begin{exercise}2
\end{exercise}

\begin{answer}
	不允许在A B之间直接移动, 求最短的移动序列
	\begin{equation*}
	\begin{array}{cll}
		k=1     & 1\quad A \rightarrowtail C, C \rightarrowtail B   & 2 \quad sum=2\\
		k=2     & 1\quad A \rightarrowtail C, C \rightarrowtail B,  & 2	\\
				& 2\quad A \rightarrowtail C                 		& 1	\\
			 	& 1\quad B \rightarrowtail C, C \rightarrowtail A,  & 2	\\
			 	& 2\quad C \rightarrowtail B						& 1	\\
			 	& 1\quad A \rightarrowtail C, C \rightarrowtail B   & 2 \quad sum=8 \\
		k=3     & 1\quad A \rightarrowtail C, C \rightarrowtail B,  & 2	\\
				& 2\quad A \rightarrowtail C                 		& 1	\\
				& 1\quad B \rightarrowtail C, C \rightarrowtail A,  & 2	\\
				& 2\quad C \rightarrowtail B						& 1	\\
				& 1\quad A \rightarrowtail C, C \rightarrowtail B   & 2 \quad sum=8\\
				& 3\quad A \rightarrowtail C                 		& 1	\\
				& 1\quad B \rightarrowtail C, C \rightarrowtail A,  & 2	\\
				& 2\quad B \rightarrowtail C						& 1	\\
				& 1\quad A \rightarrowtail C, C \rightarrowtail B,  & 2	\\
				& 2\quad C \rightarrowtail A						& 1	\\
				& 1\quad B \rightarrowtail C, C \rightarrowtail A,  & 2	\\
				& 3\quad C \rightarrowtail B						& 1	\quad sum=18\\
				& 1\quad A \rightarrowtail C, C \rightarrowtail B,  & 2	\\
				& 2\quad A \rightarrowtail C                 		& 1	\\
				& 1\quad B \rightarrowtail C, C \rightarrowtail A,  & 2	\\
				& 2\quad C \rightarrowtail B						& 1	\\
				& 1\quad A \rightarrowtail C, C \rightarrowtail B   & 2 \quad sum=26\\
		\vdots  &													& \\
		k=n     & 1\quad A \rightarrowtail C, C \rightarrowtail B	&  \\
	\end{array}
	\end{equation*}
	从前面的移动可以看出 $ f(n) = 3*f(n-1)+2 $, 设 $ g(n) = f(n)+1 $, $ g(1)=f(1)+1 = 3 $, $ g(n) = 3g(n-1) $. $ g(n) = 3^{n} $, $ f(n) = 3^{n}-1 $.
\end{answer}

\begin{exercise}3
\end{exercise}

\begin{answer}
	是的,以n个圆盘为例
	正确的叠放方法有$ 3^n $种 将ABC视为3个序列,将所有圆盘从大到小依次放置在3个序列中,每个圆盘放置时有3种选择,所共有$ 3^n $种正确的叠放方法。第二题移动$ 3^n-1 $次,再加上移动前所有圆盘都在A柱上的情况,共有$ 3^n $种情况,所以所有正确的叠放方法均会出现。
	
	我的思考,n个圆盘在3根柱子上任意放的方法有多少种?
\end{answer}

\begin{exercise}4
\end{exercise}

\begin{answer}
	Are there any starting and ending configurations of n disks on three pegs
	that are more than $ 2^n − 1 $ moves apart, under Lucas's original rules?
	
	是否存在 $ m > 2^n-1 $
	
%		I don't know...
	不存在。根据卢卡斯的规则,将可能出现的移动情况分为两种:\\		
	1. 最大的圆盘不需要移动,根据归纳法,最多需要移动$ 2^{n-1}-1 $次。\\	
	2. 最大的圆盘需要移动,根据归纳法,最多需要移动 $ 2^{n-1}-1 + 1 + 2^{n-1}-1 $ 即 $  2^{n}-1  $次
\end{answer}

\begin{exercise}5
\end{exercise}

\begin{answer}
	3个给定集合,共有8个可能子集。使用 Venn 图表示
	
	\footnote{Venn 图之后会补上}
%		空集 $ \emptyset $ 或 $ \varnothing $ 
%		并集$ \cup $, 交集 $\cap$
	$ A, B, C, $ 三个集合的所有子集为 $ \{\varnothing, A, B, C, A\cap B, A\cap C, B\cap C, A \cap B \cap C, \} $ , $\{A\setminus B, A\setminus C, B\setminus A, B\setminus C, C\setminus A, C\setminus B \}$, $ \{A\setminus (B\cup C), B\setminus (C\cup A), C\setminus (A\cup B)\} $, $ \{A\cup B, A\cup C, B\cup C, A\cup B\cup C\} $. $ \dots $
	
	我认为这里所将的八个子集应当是$ \{\varnothing\}, \{A\setminus (B\cup C), B\setminus (C\cup A), C\setminus (A\cup B)\} $, $ \{(A\cap B)\setminus C, (C\cap A)\setminus B, (B\cap C)\setminus A \} $, $ \{A\cap B\cap C\} $. 空集和7个互不相交的真子集。
	
	对于4个集合,Venn图不能给出可能的16个子集,因为不同的圆至多交于两点。 参考答案中说的卵形 (ovals) 是什么意思?
\end{answer}

\begin{exercise}6
\end{exercise}

\begin{answer}
	无界区域个数$ 2n $\\
	所有区域个数$ \frac{n(n+1)}{2}+1 $ \\
	二者相减得到有界区域个数 $ \frac{(n-1)(n-2)}{2} $
\end{answer}

\begin{exercise}7
\end{exercise}

\begin{answer}
	设 $ H(n) = J(n+1) - J(n) $.\\
	$ H(2n) = 2 $, 对$ n\geqslant 1 $有\\
	\begin{align*}
		 H(2n+1)
		 &= J(2n+2) - J(2n+1) \\
		 &= (2J(n+1)-1) - (2J(n)+1)\\
		 &= 2H(n) - 2
	\end{align*}
	但在$ n=0 $	时,由此推出
	\begin{equation*}
		H(1) = J(2)-J(1) = 1 - 1 = 0 \neq 2
	\end{equation*}
\end{answer}

\subsection{作业题}
\begin{exercise}	
	\begin{align*}
		Q_0 &= \alpha\\
		Q_1 &= \beta \\
		Q_n &= \frac{1+Q_{n-1}}{Q_{n-2}} ,\quad n>1
	\end{align*}
	(hint: $ Q_4 = \frac{1+\alpha}{\beta} $)
\end{exercise}

\begin{answer}
	\begin{equation*}
		\begin{array}{lllll}
			Q_0 &= \alpha & & &= \alpha\\
			Q_1 &= \beta  & & &= \beta\\
			Q_2 &= \cfrac{1+Q_1}{Q_0} & &&= \cfrac{1+\beta}{\alpha} \\
			Q_3 &= \cfrac{1+Q_2}{Q_1} &= \cfrac{1+\cfrac{1+\beta}{\alpha}}{\beta} 
			&&= \cfrac{1+\alpha+\beta}{\alpha\beta} \\
			Q_4 &= \cfrac{1+Q_3}{Q_2} 
			&= \cfrac{1+\cfrac{1+\alpha+\beta}{\alpha\beta} }{\cfrac{1+\beta}{\alpha}}
			&= \cfrac{\alpha\beta+1+\alpha+\beta}{\beta(1+\beta)} 
			&= \cfrac{1+\alpha}{\beta} \\
			Q_5 &= \cfrac{1+Q_4}{Q_3} 
			&= \cfrac{1+\cfrac{1+\alpha}{\beta}}{ \cfrac{1+\alpha+\beta}{\alpha\beta}}
			&= \cfrac{\alpha\beta+\alpha(1+\alpha)}{1+\alpha+beta} &= \alpha \\
			Q_6 &= \cfrac{1+Q_5}{Q_4} &= \cfrac{1+\alpha}{\cfrac{1+\alpha}{\beta}} & &= \beta\\
		\end{array}
	\end{equation*}
	因此解得 
	\begin{equation*}
		\begin{array}{clcccccc}
			Q_i &= \{ &\alpha ,& \beta ,& \cfrac{1+\beta}{\alpha} ,& \cfrac{1+\alpha+\beta}{\alpha\beta} ,& \cfrac{1+\alpha}{\beta} & \}\\
			(i\%n) &= \{ & 0 ,& 1 ,& 2 ,& 3 ,& 4 ,& \}\\
		\end{array}
	\end{equation*}
\end{answer}	

\begin{exercise}
		反向归纳法,从 $ n $ 到 $  n-1 $ 证明命题
	\begin{equation*}
		P(n) : \; x_{1} \dots x_{n} \leqslant \left( \frac{x_{1}+\cdots+x_{n}}{n}\right) ^n, \quad x_i \geqslant 0, i=1,\dots,n
	\end{equation*}
	$ n=2 $时为真 
	\begin{equation*}
		(x_{1}+x_{2})^2-4x_{1}x_{2} = (x_{1}-x_{2})^2 \geqslant 0
	\end{equation*}
	\begin{enumerate}[a)]
		\item $ x_n = \frac{x_{1}+\cdots+x_{n-1}}{n-1} $, 证明只要 $ n>1 $ 时 $ P(n) $ 蕴含 $ P(n-1) $.
		\item 证明 $ P(n) $和 $ P(2) $蕴含 $ P(2n) $
		\item 由 a), b) 说明这就蕴含了$ P(n) $对所有$ n $为真
	\end{enumerate}
\end{exercise}

\begin{answer}
	a) $ P(n) $成立,$ \forall n>1 $\\
	给定 $ x_n = \frac{x_{1}+\cdots+x_{n-1}}{n-1} $,则有
	\begin{equation*}
		\begin{array}{ll}
			x_1 \dots x_{n-1} \cdot \cfrac{x_{1}+\dots+x_{n-1}}{n-1} 
			&\leqslant \left( \frac{	x_1 + x_{n-1} + \cfrac{x_{1}+\dots+x_{n-1}}{n-1} }{n} \right) ^n \\
			x_1 \dots x_{n-1} \cdot \cfrac{x_{1}+\dots+x_{n-1}}{n-1}
			&\leqslant \left(  \cfrac{x_{1}+\dots+x_{n-1}}{n-1} \right) ^n \\
			x_1 \dots x_{n-1}
			&\leqslant \left(  \cfrac{x_{1}+\dots+x_{n-1}}{n-1} \right) ^{n-1}\\
		\end{array}
	\end{equation*}
	$ P(n) $成立
	
	b) 由 $ P(n) $可得
	\begin{equation*}
		x_{1} \dots x_{n} \cdot x_{n+1} \dots x_{2n} \leqslant \left(\cfrac{x_{1} + \dots + x_{n}}{n}\right)^n \cdots \left(\cfrac{x_{n+1} + \dots + x_{2n}}{n}\right)^n
	\end{equation*}
	记$ A = \left(\cfrac{x_{1} + \dots + x_{n}}{n}\right) $,  $ B = \left(\cfrac{x_{n+1} + \dots + x_{2n}}{n}\right) $\\
	由$ P(2) $可得
	\begin{equation*}
		\begin{array}{rl}
			AB &\leqslant \left(\cfrac{A+B}{2}\right)^{2}  \\
			A^{n} B^{n} = (AB)^{n} &\leqslant \left(\cfrac{A+B}{2}\right)^{2n} \\
			x_{1}\dots x_{2n} &\leqslant \left(\cfrac{x_{1}+\dots +x_{2n}}{2n}\right)^{2n} \\
		\end{array}
	\end{equation*}
	由此推知$ P(2n) $成立。
	
	c) Cauchy 向前-向后方法。\\
	1. $ P(2)\rightarrow P(4)\rightarrow\cdots P(2^n) $. \\
	2. $ P(n)\rightarrow P(n-1) $.\\
	$ \therefore \forall n\geqslant 1 $, $ P(n) $成立
\end{answer}

\begin{exercise}
	圆盘只能在ABC三根柱子上 按照顺时针方向移动。记:\\
	$ Q_{n} $为$ n $个盘从A到B最少移动的次数。\\
	$ R_{n} $为$ n $个盘从B到A最少移动的次数。	
\end{exercise}

\begin{answer}
	先列出两种移动方式各自的迭代式:
	\begin{equation*}
		Q_{n} = \left\{
		\begin{array}{ll}
			0 ,& n = 0 \\
			2R_{n-1}+1 ,& n>0\\
		\end{array}
		\right. \quad
		R_{n} = \left\{
		\begin{array}{ll}
			0 ,& n = 0 \\
			Q_{n}+Q_{n-1}+1 ,& n>0\\
		\end{array}
		\right.
	\end{equation*}
	这两个公式是如何得到的?
	\begin{equation*}
		\begin{array}{lll}
			k=0 & Q_{0} = 0		 					& R_{0} = 0 \\
			k=1 & A \rightarrow B	 				& B \rightarrow C \rightarrow A \\
				& Q_1=1 							& R_1 = 2\\
			k=2 & 1: A \rightarrow B \rightarrow C 	& 1: B \rightarrow C \rightarrow A \\
				& 2: A \rightarrow B 				& 2: B \rightarrow C \\
				& 1: C \rightarrow A \rightarrow B  & 1: A \rightarrow B \\
				& 									& 2:C \rightarrow A \\
				&									& 1: B \rightarrow C \rightarrow A \\
				& Q_{2} = 5							& R_{2} = 7 \\
			k=n & A \rightarrow B	\; Q_n			& B \rightarrow A\; R_n\\
				& (n-1) A \rightarrow C	\;R_{n-1}	& (n-1) B \rightarrow A \; R_{n-1} \\
				& n: A \rightarrow B	\;1			& n: B \rightarrow C 	\; 1 \\
				& (n-1) C \rightarrow B \;R_{n-1}	& (n-1) A \rightarrow B \; Q_{n-1} \\
				& 									& n: C \rightarrow A 	\; 1 \\
				& 									& (n-1) B \rightarrow A \; R_{n-1} \\
				& Q_n = 2R_{n-1}+1					& R_{n} = Q_{n-1}+2R_{n-1}+2 = Q_{n}+Q_{n-1}+1\\
		\end{array}
	\end{equation*}
\end{answer}

\begin{exercise}
	双重河内塔 2n个圆盘,第$ 2k-1 $个与第$ 2k $个大小相同。
\end{exercise}

\begin{answer}
	a) 不区分相同尺寸
	\begin{equation*}
		\begin{aligned}
			n=0 & S_0=0	& \\
			n=1 & S_1=2	& A \rightarrow B, A \rightarrow B\\
			n=2 & S_2=6	& A \rightarrow C, A \rightarrow C\\
						& A \rightarrow B, A \rightarrow B\\
						& C \rightarrow B, C \rightarrow B
		\end{aligned}
	\end{equation*}
	解得$ S_n = 2T_n = 2(2^n-1)=2^{n+1}-2 $
	
	b) 在最后排列中将圆盘恢复次序需要移动几次?
	\begin{equation*}
		\begin{array}{lll}
			k=0	& R_0=0		&	\\
			k=1	& R_1=3		& 1.1:\; A \rightarrow C\\
				&			& 1.2:\; A \rightarrow B\\
				&			& 1.1:\; C \rightarrow B\\
			k=2	& R_2=11	& 1.1:\; A \rightarrow B\\
				&			& 1.2:\; A \rightarrow B\\
				&			& 2.1:\; A \rightarrow C\\
				&			& 1.2:\; B \rightarrow C\\
				&			& 1.1:\; B \rightarrow C\\
				&			& 2.2:\; A \rightarrow B\\
				&			& 1.1:\; C \rightarrow A\\
				&			& 1.2:\; C \rightarrow A\\
				&			& 2.1:\; C \rightarrow B\\
				&			& 1.2:\; A \rightarrow B\\
				&			& 1.1:\; A \rightarrow B\\
%			k=n	& R_n 		& (1\sim n-1) \;  	A \rightarrow B\quad S_{n-1}\\
%				&			& n.1:\; 			A \rightarrow C\quad 1		\\	
%				&			& (1\sim n-1)\; 	B \rightarrow C\quad S_{n-1}\\	
%				&			& n.2:\; 			A \rightarrow B\quad 1		\\	
%				&			& (1\sim n-1)\; 	C \rightarrow A\quad S_{n-1}\\	
%				&			& n.1:\; 			C \rightarrow B\quad 1		\\
%				&			& (1\sim n-1) \;  	A \rightarrow B\quad S_{n-1}\\
			k=n	& R_n 		& n-1\;  			A \rightarrow B\quad S_{n-1}\\
				&			& n.1:\; 			A \rightarrow C\quad 1		\\	
				&			& n-1\; 			B \rightarrow C\quad S_{n-1}\\	
				&			& n.2:\; 			A \rightarrow B\quad 1		\\	
				&			& n-1\; 			C \rightarrow A\quad S_{n-1}\\	
				&			& n.1:\; 			C \rightarrow B\quad 1		\\
				&			& n-1\;  			A \rightarrow B\quad S_{n-1}\\
		\end{array}
	\end{equation*}
\begin{equation*}
	R_n = 4S_{n-1}+3 = 2^{n+2}-5 \quad(n\geqslant 1)
\end{equation*}
\end{answer}

\begin{exercise}
	12 11推广,	$ m_k $个尺寸为$ k $的圆盘,不区分相同尺寸的圆盘
	移动一个塔最少次数 $ A(m_1, \dots, m_n) $
\end{exercise}

\begin{answer}
	\begin{align*}
		F(0) 	&= 0	\\
		F(1) 	&= m_1	\\
		F(2) 	&= 2F(1)+m_2 = 2m_1+m_2	\\
				&\vdots	\\
		F(n)	&= 2F(n-1)+m_n
	\end{align*}
	\begin{align*}
		A(m_1, \dots, m_n) 
		&= F(n) = 2F(n-1)+m_n\\
		&= 2^{n-1}m_1+2^{n-2}m_2+\cdots+m_n\\
		&= \sum_{k=1}^{n} 2^{n-k} m_{k}
	\end{align*}
\end{answer}

\begin{exercise}
	13
\end{exercise}

\begin{answer}
	\begin{equation*}
		\begin{array}{ll}
			k=1 & ZZ_1 = 2+0 = 2\\
			k=2	& ZZ_2 = 4+8 = 12\\
			k=3	& ZZ_3 = 6+25 = 31\\
		\end{array}
	\end{equation*}
对于定义了$ L_n $个区域的$ n $条直线,
可以用极狭窄的Z形线来代替。

例如,每一对Z形线间有9个交点
\begin{equation}
	\begin{array}{ll}
		ZZ_n 	&= ZZ_{n-1}+9n-8,\quad (n>0)\\
		ZZ_n	&= 9S_n - 8n+1\\
				&= 9\cfrac{n(n+1)}{2} - 8n+1\\
				&= \frac{9}{2}n^2 - \frac{7}{2}n + 1\\
	\end{array}
\end{equation}
\end{answer}

\begin{exercise}
	14
\end{exercise}

\begin{answer}
	\begin{equation*}
		\begin{array}{ll}
			n=0	& P_0 = 1	\\
			n=1	& P_1 = 2	\\
			n=2 & P_2 = 4	\\
			n=3 & P_3 = 8	\\
			n=4 & P_4 = 8+6=14	\\
		\end{array}
	\end{equation*}
\begin{equation*}
	P_n = P_{n-1}+L_{n-1}
\end{equation*}
其中
\begin{equation*}
	L_n = 1+S_n,\quad S_n = \frac{n(n+1)}{2}
\end{equation*}
$ \therefore  P_n = P_{n-1}+1+\frac{n(n+1)}{2}$
\begin{equation*}
	\begin{array}{llll}
		P_0 &= 1 		& 							&		\\
		P_1 &= P_0+L_0 	&= 1+1+\frac{0\cdot 1 }{2} 	&= 2	\\ 
		P_2 &= P_1+L_1 	&= 2+1+\frac{1\cdot 2 }{2} 	&= 4	\\ 
		P_3 &= P_2+L_2 	&= 4+1+\frac{2\cdot 3 }{2} 	&= 8	\\ 
		P_4 &= P_3+L_3 	&= 8+1+\frac{3\cdot 4 }{2} 	&= 15	\\ 
		P_5 &= P_4+L_4 	&=15+1+\frac{4\cdot 5 }{2} 	&= 26	\\ 
	\end{array}
\end{equation*}
\begin{align*}
	P_n 
	&= P_{n-1}+L_{n-1} 		\\
	&= 0 + \sum_{k=0}^{n-1}\left(1+\frac{k(k+1)}{2}\right)	\\
	&= n + \frac{(n-1)n(n+1)}{6} \\
	&= \frac{n(n^2+5)}{6}
\end{align*}
\end{answer}

\begin{exercise}
	15 约瑟夫问题, 倒数第二个 $ I(n) $
\end{exercise}

\begin{answer}
	\begin{table}[htbp]
		\centering
		\small
		\caption{约瑟夫问题J(n)与I(n)}
%		\begin{tabular}{c|c|cc|cccc|cccccccc|ccccc}
		\begin{tabular}{c|c cc cccc cccccccc ccc}
			\toprule
			n 
			& 1 
			& 2 & 3 
			& 4 & 5 & 6 & 7 
			& 8 & 9 & 10 & 11 & 12 & 13 & 14 & 15 
%			& 16 & 17 & 18 & 19 & 20 \\  
			& 16 & 17 & 18 \\
			\midrule
			J(n) 
			& 1 
			& 1 & 3 
			& 1 & 3 & 5 & 7 
			& 1 & 3	& 5 & 7 & 9 & 11 & 13 & 15
%			& 1 & 3 & 5 & 7 & 9\\			
			& 1 & 3 & 5 \\
			\midrule
			I(n) 
			& $\sim$ 
			& 2  
			& 1 & 3 & 5  
			& 1 & 3	& 5 & 7 & 9 & 11 
%			& 1 & 3	& 5 & 7 & 9 & 11 &13 &15 &17 \\
			& 1 & 3	& 5 & 7 & 9 & 11 &13 \\
			\bottomrule
		\end{tabular}%
		\label{tab:JNIN}%
	\end{table}%

$ n >1 $时, $ J(n), I(n) $有相同递归式
\begin{align*}
	& I(2) = 2, I(1) = 1\\
	& n=2^m + 2^{m-1}+k, \quad 0\leqslant k \leqslant 2^m + 2^{m-1}\\
	& I(n) = 2k+1 
\end{align*}
\begin{equation*}
	n = 2^m + l ,\quad I(n) =
	\left\{
	\begin{array}{ll}
		J(n)+2^{m-1}, 	& 0 		\leqslant l < 2^{m-1} 	\\
		J(n)-2^{m}, 	& 2^{m-1}	\leqslant l < 2^{m} 	\\
	\end{array}
	\right.
\end{equation*}

\begin{exercise}
	\begin{equation*}
		\left\{
		\begin{array}{rl}
			g(1) 	&= \alpha	\\
			g(2n+j)	&= 3g(n)+\gamma n + \beta_j, \quad j=0,1, n\leqslant 1\\
		\end{array}
		\right.
	\end{equation*}
(提示,用$ g(n)=n $)
\end{exercise}

\begin{answer}
	Suppose $ g(n) = n $
	\begin{equation*}
		\begin{aligned}
			g(1) 	&= 1 = \alpha,\\
			g(2n+j)	&= 2n+j = 3n+\gamma n+\beta_j.
		\end{aligned}
	\end{equation*}
解得 $   \alpha = 1, \gamma = -1, \beta_j = \left\{ \begin{array}{l}
	0,\quad j=0\\
	1,\quad j=1\\
\end{array}\right.$

%(题解)
%\begin{equation*}
%	g(n) = a(n)\alpha + b(n)\beta_0 + c(n)\beta_1+d(n)\gamma
%\end{equation*}
%$ n = (1 b_{m-1}\dots b_1b_0)_2 $ 将$ n $以基数2展开(写成二进制)。\\
%$ a(n)\alpha + b(n)\beta_0 + c(n)\beta_1 =  $
\end{answer}

\end{answer}
\end{document}