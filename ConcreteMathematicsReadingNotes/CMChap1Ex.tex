\documentclass[mode=geye]{elegantnote}
\usepackage{amssymb}
\usepackage{amsthm}

\newtheorem{exercise}{练习}
\newtheorem{answer}{解}

\title{具体数学阅读笔记-chap1 exercise}
\author{weiyuan}
\date{2022-06-27}
\begin{document}
	\maketitle
	
	\section{Exercises}
	\subsection{Warmups}
	
	\begin{exercise}
		All horses are the same color; we can prove this by induction on the
		number of horses in a given set. Here's how: "If there's just one horse
		then it's the same color as itself, so the basis is trivial. For the induction
		step, assume that there are n horses numbered 1 to n. By the induc-
		tion hypothesis, horses 1 through n − 1 are the same color, and similarly
		horses 2 through n are the same color. But the middle horses, 2 through
		n − 1, can't change color when they're in different groups; these are
		horses, not chameleons. So horses 1 and n must be the same color as
		well, by transitivity. Thus all n horses are the same color; QED." What,
		if anything, is wrong with this reasoning?
	\end{exercise}
	\begin{answer}
		n=1 情况下马有相同颜色
		
		但 n=2 时	该假设不一定成立
	\end{answer}
	
	\begin{answer}
		不允许在A B之间直接移动, 求最短的移动序列
		\begin{align*}
			k=1     & 1\quad A \rightarrowtail C, C \rightarrowtail B   & 2 \quad sum=2\\
			k=2     & 1\quad A \rightarrowtail C, C \rightarrowtail B,  & 2	\\
					& 2\quad A \rightarrowtail C                 		& 1	\\
				 	& 1\quad B \rightarrowtail C, C \rightarrowtail A,  & 2	\\
				 	& 2\quad C \rightarrowtail B						& 1	\\
				 	& 1\quad A \rightarrowtail C, C \rightarrowtail B   & 2 \quad sum=8 \\
			k=3     & 1\quad A \rightarrowtail C, C \rightarrowtail B,  & 2	\\
					& 2\quad A \rightarrowtail C                 		& 1	\\
					& 1\quad B \rightarrowtail C, C \rightarrowtail A,  & 2	\\
					& 2\quad C \rightarrowtail B						& 1	\\
					& 1\quad A \rightarrowtail C, C \rightarrowtail B   & 2 \quad sum=8\\
					& 3\quad A \rightarrowtail C                 		& 1	\\
					& 1\quad B \rightarrowtail C, C \rightarrowtail A,  & 2	\\
					& 2\quad B \rightarrowtail C						& 1	\\
					& 1\quad A \rightarrowtail C, C \rightarrowtail B,  & 2	\\
					& 2\quad C \rightarrowtail A						& 1	\\
					& 1\quad B \rightarrowtail C, C \rightarrowtail A,  & 2	\\
					& 3\quad C \rightarrowtail B						& 1	\quad sum=18\\
					& 1\quad A \rightarrowtail C, C \rightarrowtail B,  & 2	\\
					& 2\quad A \rightarrowtail C                 		& 1	\\
					& 1\quad B \rightarrowtail C, C \rightarrowtail A,  & 2	\\
					& 2\quad C \rightarrowtail B						& 1	\\
					& 1\quad A \rightarrowtail C, C \rightarrowtail B   & 2 \quad sum=26\\
			\vdots  &													&
			k=n     & 1\quad A \rightarrowtail C, C \rightarrowtail B	& ?\\
		\end{align*}
		从前面的移动可以看出 $ f(n) = 3*f(n-1)+2 $, 设 $ g(n) = f(n)+1 $, $ g(1)=f(1)+1 = 3 $, $ g(n) = 3g(n-1) $. $ g(n) = 3^{n} $, $ f(n) = 3^{n}-1 $.
	\end{answer}
	
	\begin{answer}
		是的,以n个圆盘为例
		正确的叠放方法有$ 3^n $种 将ABC视为3个序列,将所有圆盘从大到小依次放置在3个序列中,每个圆盘放置时有3种选择,所共有$ 3^n $种正确的叠放方法。第二题移动$ 3^n-1 $次,再加上移动前所有圆盘都在A柱上的情况,共有$ 3^n $种情况,所以所有正确的叠放方法均会出现。
		
		我的思考,n个圆盘在3根柱子上任意放的方法有多少种?
	\end{answer}

	\begin{answer}
		Are there any starting and ending configurations of n disks on three pegs
		that are more than $ 2^n − 1 $ moves apart, under Lucas's original rules?
		
		是否存在 $ m > 2^n-1 $
		
%		I don't know...
		不存在。根据卢卡斯的规则,将可能出现的移动情况分为两种:\\		
		1. 最大的圆盘不需要移动,根据归纳法,最多需要移动$ 2^{n-1}-1 $次。\\	
		2. 最大的圆盘需要移动,根据归纳法,最多需要移动 $ 2^{n-1}-1 + 1 + 2^{n-1}-1 $ 即 $  2^{n}-1  $次
	\end{answer}
	
	\begin{answer}
		3个给定集合,共有8个可能子集。使用 Venn 图表示
		
		\footnote{Venn 图之后会补上}
%		空集 $ \emptyset $ 或 $ \varnothing $ 
%		并集$ \cup $, 交集 $\cap$
		$ A, B, C, $ 三个集合的所有子集为 $ \{\varnothing, A, B, C, A\cap B, A\cap C, B\cap C, A \cap B \cap C, \} $ , $\{A\setminus B, A\setminus C, B\setminus A, B\setminus C, C\setminus A, C\setminus B \}$, $ \{A\setminus (B\cup C), B\setminus (C\cup A), C\setminus (A\cup B)\} $, $ \{A\cup B, A\cup C, B\cup C, A\cup B\cup C\} $. $ \dots $
		
		我认为这里所将的八个子集应当是$ \{\varnothing\}, \{A\setminus (B\cup C), B\setminus (C\cup A), C\setminus (A\cup B)\} $, $ \{(A\cap B)\setminus C, (C\cap A)\setminus B, (B\cap C)\setminus A \} $, $ \{A\cap B\cap C\} $. 空集和7个互不相交的真子集
	\end{answer}
	
\end{document}