\documentclass[mode=geye]{elegantnote}
\usepackage{amssymb}
\title{具体数学阅读笔记}
\author{weiyuan}
\date{\today}
\begin{document}
	\section{递归问题}
	\subsection{河内塔}
	使用递归方法解决河内塔问题。先从n为1的情况考虑该问题。为了递归式的完整?将n等于0也纳入考虑
	\begin{align*}
		T_0 = 0\\
		T_1 = 1\\
		T_2 = 3\\
		T_3 = 7
	\end{align*}
	从圆盘的移动规律可以看出,要将n个圆盘从A柱移动到C柱上,首先需要将n-1个圆盘移动到B柱上,再移动最大的圆盘到C柱上,最后将n-1个圆盘从B柱移动到A柱上。
	
	书中使用
		\begin{align*}
			&T_n \leqslant 2T_{n-1}+1\\
			&T_n \geqslant 2T_{n-1}+1
		\end{align*}
	得到 $ T_n = 2T_{n-1}+1 $ 。 将n等于0的情况作为初始条件,得到完整的递归式 (recurrence)
	\begin{equation}\label{Henei1}
		\begin{aligned}
			&T_0 = 0\\
			&T_n = 2T_{n-1}+1, \quad n > 0
		\end{aligned}
	\end{equation}
	
	计算得到
	\begin{equation}\label{HeneiRecur}
		T_n = 2^n - 1 , \quad n\geqslant 0
	\end{equation}

\textbf{数学归纳法} (mathematical induction)\footnote{数学归纳法的难点并不在于证明本身,而是如何得到关系式}
\begin{enumerate}
	\item 基础 basis, n取最小值 $ n_0 $证明该命题
	\item 归纳 induction, 假设$ k=n-1 $时归纳结果成立,证明$ k=n $时该结果也成立
\end{enumerate}

使用数学归纳法证明河内塔问题递归式
\begin{proof}证明河内塔问题递归式
	1. $ k=0 $时, $ T_0 = 2^0-1 = 0 $\\
	2. 假设 $ k=n-1 $时 $ T_{n-1} = 2^{n-1} - 1 $成立。\\
	3. $ k=n $时, $ T_{n} = 2*T_{n-1}+1 = 2*(2^{n-1} - 1)+1 = 2^n-1 $
\end{proof}

\begin{remark}
	\begin{enumerate}
		\item  研究小的情形
		\item  求解递归式(\ref{Henei1})
		\item  求解递归式(\ref{HeneiRecur})
	\end{enumerate}
\end{remark}

Q: 递归式$ T_n= 2^n -1 $是怎样得到的?
\begin{align*}
	&T_0 + 1 = 1\\
	&T_n+1 = 2T_{n-1}+2 = 2(T_{n-1}+1)
\end{align*}
令$ U_n = T_n+1 $
\begin{align*}
	&U_0 = 1\\
	&U_n = 2U_{n-1}, \quad n>0.
\end{align*}
容易推出$ U_n  = 2^n, T_n = 2^n-1 $

\subsection{平面上的直线}
平面上n条直线所界定的区域最大个数$ L_n $是多少?
\begin{align*}
	&L_0 = 1\\
	&L_1 = 2\\
	&L_2 = 4\\
	&L_3 = 7\\
	&\ldots\\
	&L_n \leqslant L_{n-1}+n, \quad n>0	
\end{align*}

\begin{align*}
	L_n &= L_{n-1} + n\\
	& = L_{n-2}+(n-1)+n\\
	&=\cdots\\
	&=L_0+1+2+\dots+n\\
	&=1+S_n
\end{align*}
其中$ S_n = 1+2+\dots+n $被称为\textbf{三角形数}。
\begin{equation}\label{sanjiaoxingshu}
	S_n = \frac{n(n+1)}{2}
\end{equation}
由此得到平面分割数$ L_n = \frac{n(n+1)}{2}+1 $.

使用数学归纳法证明该公式
\begin{proof}
	1. $ k=0 $, $ L_0 = 1 $.\\
	2. 设$ k=n-1 $, $ L_{n-1} = \frac{(n-1)((n-1)+1)}{2}+1=\frac{(n-1)n}{2}+1 $成立.\\
	3. $ k=n $, $ L_n = L_{n-1}+n = \frac{(n-1)n}{2}+1+n = \frac{n(n+1)}{2}+1 $
\end{proof}

\begin{remark}
	将直线的情况拓展到折线。
	\begin{align*}
		&Z_1 = 2\\
		&Z_2 = 7
	\end{align*}
\end{remark}
做法,将折线补齐成两条直线\footnote{$ L_n \sim \frac{1}{2}n^2\\ Z_n \sim 2n^2 $}
\begin{align*}
	Z_n& = L_2n-2n \qquad\text{锯齿点不在交点}\\
	&=\frac{2n(2n+1)}{2}+1 - 2n\\
	&=2n^2-n+1, \quad n\geqslant 0
\end{align*}

\section{约瑟夫问题}
n个人围城一圈,从第一个人开始,每隔一个删除一个。

\begin{table}[htbp]
	\centering
	\small
	\caption{约瑟夫问题最终剩余数字 J(n) 与全体数字 n 之间的关系}
	\begin{tabular}{c|cccccc}
		\toprule
		n & 1 & 2 & 3 & 4 & 5 & 6 \\  
		\midrule
		J(n) & 1 & 1 & 3 & 1 & 3 & 5 \\
		\bottomrule
	\end{tabular}%
	\label{tab:reg}%
\end{table}%

人数总数为偶数 $ J(2n)   = 2J(n)-1, n\geqslant 1 $.
人数总数为奇数 $ J(2n+1) = 2J(n)+1, n\geqslant 1 $.

递归式
\begin{align*}
	J(1)  &= 1&\\
	J(2n) &= 2J(n)-1,&  n\geqslant 1 \\
	J(2n+1) &= 2J(n)+1,&  n\geqslant 1 
\end{align*}

计算得到 $ n = 2^m+l $,封闭形式$ J(2^m+l) = 2l+1, m\geqslant 0, 0\leqslant l <2^m  $.
\begin{proof}
	1. l is even.
	\begin{align*}
		J(2^m+l) &= 2J(2^m+\frac{l}{2})-1\\
		&=2(2*\frac{l}{2}+1)-1\\
		&=2l+1
	\end{align*}
	2. l is odd.
	\begin{align*}
		J(2^m+l) &= 2J(2^m+\frac{l-1}{2})-1\\
		&=2(2*\frac{l-1}{2}+1)-1\\
		&=2l+1
	\end{align*}
\end{proof}

\begin{equation}\label{JosephRecur1}
	J(2n+1)-J(2n)=2
\end{equation}

\begin{remark}
	将n和J(n)以2为基数表示 (表示为二进制). 假设:\\
	\begin{align*}
		n&=(b_mb_{m-1} \dots b_1 b_0)_2\\
		&=b_m2^m+b_{m-1}2^{m-1}+\dots+b_1 2^1+b_0 2^0
	\end{align*}
其中$ b_m = 1, b_i = 0\text{ 或 }1  \quad(0\leqslant i <m, i\in\mathbb{N}^+)$
\end{remark}

\begin{align*}
	n&=2^m+l\\
	n&=(1 b_{m-1} b_{m-2}\dots b_1 b_0)_2\\
	l&=(0 b_{m-1} b_{m-2}\dots b_1 b_0)_2\\
	2l&=(b_{m-1} b_{m-2}\dots b_1 b_0 0)_2\\
	2l+1&=(b_{m-1} b_{m-2}\dots b_1 b_0 1)_2\\
	J(n)&=(b_{m-1} b_{m-2}\dots b_1 b_0 1)_2
\end{align*}

因此我们得到 $ J((1 b_{m-1} b_{m-2}\dots b_1 b_0)_2)=(b_{m-1} b_{m-2}\dots b_1 b_0 1)_2 $.\\
n向左循环移动一位得到J(n)!\\

\begin{case}
	$ J((1011)_2) = (0111)_2 = (111)_2 $, 该式即 $ J(11) = 7 $
\end{case}
注意:0移动至首位会消失,而不需要保留空位。

$ 2^{\nu(n)}-1 $, 其中$ \nu(n) $为n转换成的二进制数中1的个数

\begin{case}
	n=13, $ (13)_{10} = (1101)_2 $, $ \mu(13) = 3 $.\\
	$ J(J(\dots(J(13))\dots)) = 2^3-1 = 7 $
\end{case}
\begin{case}
	n=23403, $ \mu(23403)=10 $, therefore 	$ J(J(\dots(J(23403))\dots)) = 2^{10}-1 = 1023 $
\end{case}

pp10
\end{document}